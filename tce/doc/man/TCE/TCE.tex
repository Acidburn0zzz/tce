% Copyright 2002-2008 Tampere University of Technology.  All Rights Reserved.
%
% This file is part of TTA-Based Codesign Environment (TCE).
%
% TCE is free software; you can redistribute it and/or modify it under the
% terms of the GNU General Public License version 2 as published by the Free
% Software Foundation.
%
% TCE is distributed in the hope that it will be useful, but WITHOUT ANY
% WARRANTY; without even the implied warranty of MERCHANTABILITY or FITNESS
% FOR A PARTICULAR PURPOSE.  See the GNU General Public License for more
% details.
%
% You should have received a copy of the GNU General Public License along with
% TCE; if not, write to the Free Software Foundation, Inc., 51 Franklin St,
% Fifth Floor, Boston, MA  02110-1301  USA
%
% As a special exception, you may use this file as part of a free software
% library without restriction.  Specifically, if other files instantiate
% templates or use macros or inline functions from this file, or you compile
% this file and link it with other files to produce an executable, this file
% does not by itself cause the resulting executable to be covered by the GNU
% General Public License.  This exception does not however invalidate any
% other reasons why the executable file might be covered by the GNU General
% Public License.
\documentclass[twoside]{tceusermanual}
\usepackage{pslatex}

% for index
\makeindex

\begin{document}
\title{TTA Codesign Environment v1.0}
\ver{9}
\firstday{25.10.2006}
\lastday{10.10.2008}
% id number in P- sequence
\docnum{1000}
% draft/complete/committedt
\state{review}
\maketitle

% Table of contents
\tableofcontents

% Document text
\chapter{INTRODUCTION}

\section{Document Overview}

This is the user manual for TTA Codesign Environment (TCE). The document
describes the usage of all the tools in the toolset, and the most common
design flows in the form of tutorials. 

Chapter~\ref{chapter:tceFlow} provides an overview to the TCE
processor design flow, and Chapter~\ref{chapter:tutorials} contains
tutorials for several common TCE use cases. These should be sufficient
for starting to use the TCE toolset.  The rest of the chapters
describe the use of each tool separately, and they can be referred to
for information on more advanced usage the tools.

\section{Acronyms, Abbreviations and Definitions}
\begin{tabular}[h]{p{0.15\textwidth}p{0.80\textwidth}}
ADF & (Processor/Machine) Architecture Definition File.\\
BEM & Binary Encoding Map. Describes the encoding of instructions.\\
CLI & Command Line Interface\\
ExpResDB  & Exploration Result Database.\\
GUI & Graphical User Interface \\
GPR & General Purpose Register \\
HDB & Hardware Database \\
HDL & Hardware Description Language. \\
HLL & High Level (Programming) Language. \\
IDF & (Machine/Processor) Implementation Definition File.\\
ILP & Instruction Level Parallelism.\\
LLVM & Low Level Virtual Machine \\
MAU & Minimum Addressable Unit\\
PIG & Program Image Generator\\
SQL & Structured Query Language.\\
TCE & TTA Codesign Environment.\\
TPEF & TTA Program Exchange Format \\
TraceDB & Execution Trace Database. \\
TTA & Transport Triggered Architecture.\\
VHDL & VHSIC Hardware Description Language. \\
XML & Extensible Markup Language. \\
\end{tabular}

\section{Typographic Conventions Used in the Document}

\begin{center}
\begin{tabular}{|p{0.30\textwidth}|p{0.63\textwidth}|}
\hline
\textbf{Style} &\textbf{Purpose}\\
\hline
\emph{italic}     & parameter names in running text\\
\hline
[brackets]       & bibliographic references\\
\hline
`single quoted'   & keywords or literal strings in running
                    text\\
\hline
\file{file name}   & file and directory names in running text\\
\hline
\textbf{bold}     & Shorthand name of a TCE application.\\
\hline
\end{tabular}
\end{center}


\chapter{TCE DESIGN FLOW}
\label{chapter:tceFlow}

\section{Overview}

The main goal for the TTA Codesign Environment (TCE) is to provide a reliable
and effective toolset for designing programmable application specific
processors, and generate machine code for them from applications written 
in high-level languages.

In addition, TCE provides an extensible research platform for 
experimenting with new ideas for Transport Triggered Architectures (TTAs), 
retargetable ILP code generation, and application specific processor 
design methodology, among others.

The TCE design flow starts from an application described in a high level
language (currently the C language). The LLVM compiler framework~\cite{llvm-home-page} is used to
compile the application to 'bitcode', the intermediate representation of
LLVM. The resulting bitcode is then compiled and scheduled to a particular
TTA processor  by TCE. Traditional compiler optimizations are done in LLVM
before bitcode generation, so it is possible to utilize the same bitcode file
when exploring different TTA processors for running an application.

The initial software development phase is intended 
to be separate from the actual codesign flow of TCE. That is, the program is 
expected to be implemented and tested natively (on a workstation PC) before 
``porting'' it to the TTA/TCE platform. The porting includes ensuring
that TTA/TCE runs the program correctly, and optimizing the hardware together
with the software by modifying the resources and architecture of the processor
to fit  the application
at hand -- a process called hardware/software 
codesign.

% ISO C99 if LLVM/TCE is used

The main phases in the design flow of TCE are illustrated in the following
figures. Figure~\ref{fig:design_flow_initial} depicts the initial
inputs to TCE, Figure~\ref{fig:design_flow_exploration} the design space 
exploration phase, Figure~\ref{fig:design_flow_selection} the processor 
configuration selection phase, Figure~\ref{fig:design_flow_codegen} the 
code generation and analysis phase, and Figure~\ref{fig:design_flow_generation} 
the generation of the final outputs: the processor description and the 
program bit image.

\begin{figure}[p]
\centerline{\psfig{figure=eps/design_flow_initial.eps,width=0.2\textwidth}}
\caption{The Initial Inputs for TCE Design Flow.}
\label{fig:design_flow_initial}
\end{figure}

\begin{figure}[p]
\centerline{\psfig{figure=eps/design_flow_exploration.eps,
width=1.0\textwidth}}
\caption{Design Space Exploration.}
\label{fig:design_flow_exploration}
\end{figure}

\begin{figure}[p]
\centerline{\psfig{figure=eps/design_flow_selection.eps,width=0.35\textwidth
}}
\caption{Processor Configuration Selection.}
\label{fig:design_flow_selection}
\end{figure}

\begin{figure}[p]
\centerline{\psfig{figure=eps/design_flow_codegen.eps,width=0.6\textwidth}}
\caption{Code Generation and Analysis.}
\label{fig:design_flow_codegen}
\end{figure}

\begin{figure}[p]
\centerline{\psfig{figure=eps/design_flow_generation.eps,
width=0.85\textwidth}}
\caption{Processor and Program Image Generation.}
\label{fig:design_flow_generation}
\end{figure}


\section{Main File Formats}
\label{section:fileFormats}

This chapter gives an overview of files and databases manipulated by TCE
applications and accessible to users of the toolset.  

\subsection{Architecture Definition File (ADF)}
\label{sec:adf}

Filename extension: .adf

Machine Architecture Definition File (ADF) is a file format
for defining target processor architectures.  ADF is a minimal specification
of the target processor architecture, meaning that only the information
needed  to generate valid programs for the processor is stored, nothing
else. 

\subsection{Hardware Database (HDB)}
\label{section:hdb}
% Describe what is shipped by default in TCE
Filename extension: .hdb

Hardware Database (HDB) is the main database used by the
Processor Generator and the Cost Estimator.  The data stored in HDB consist
of hardware description language definitions (HDL) of TTA components
(function units, register files, buses and sockets) and metadata that
describe certain parameters used in implementations. In addition, HDB may
include data of each implementation needed by the cost estimation
algorithms.

TCE ships with an example HDB that includes implementations for several
function units and register files, and cost data for the default
interpolating cost estimation plugin.

\subsection{Implementation Definition File (IDF)}
\label{sec:idf}

Filename extension:  .idf

Describes which implementations to use for each component in the
architecture (defined in an ADF file). Using this information it is possible
to fetch correct hardware description language
(HDL) files from the hardware block library for cost estimation and processor 
generation. 

\subsection{Binary Encoding Map}
\label{sec:bem}

Filename extension: .bem 

Provides enough information to produce an executable uncompressed bit
image (Section~\ref{sec:db_bim}) from TPEF program data.

\subsection{TTA Program Exchange Format (TPEF)}
\label{section:TPEF}

Filename extension: .tpef

TTA Program Exchange Format (TPEF) is a file format for storing unscheduled,
partially scheduled, and scheduled TTA programs. TPEF supports auxiliary
sections for storing additional information related to
the program, such as execution profiles, machine resource data, and target
address space definitions.

\subsection{Operation Set Abstraction Layer (OSAL) Files}
\label{section:osal}
% Describe what is shipped by default in TCE
Filename extension: .opp, .cc, .opb

OSAL stores the simulation behavior and static properties of operations in
function units.

Simulation behavior of function unit operations is described by implementing
simulation functions which can be plugged in to the simulator run time.

The .opp file is an XML file for defining the static properties of
operations (for example, how many inputs and outputs an operation has). The
.cc is the C++ source file that defines the behavior model for a set of
operations. The .opb is the plugin module compiled from the .cc.

Operations are divided in ``operation modules''. For example, `base' module,
included in the TCE distribution, contains all the operations available 
to the front end compiler's code generation.

An operation in OSAL is defined by its properties
and its behavior.  The properties defined by OSAL do not restrict in any
way the hardware implementation.  For example, latency, bit width or the
fact that reading the result of an operation can lock the processor are
properties of the implementation, and are not defined in OSAL module.

\subsubsection{Operation Properties}
\label{ssec:operation-properties}

The following properties define a TTA operation:
\begin{itemize}
\item name (unique string)
\item description (string)
\item number of inputs (integer)
\item number of outputs (integer)
\item accesses memory (yes/no)
%\item can cause trap (yes/no)
\item has side effects (yes/no)
\item clocked (yes/no)
\item affected-by (set of operations)
\item affects (set of operations)
\end{itemize}

\paragraph{operation name}
The operation name is a string of characters starting with a character in
set [A-Z\_] and followed by one or more character in set [0-9A-Z\_] 
(i.e.\ lowercase letters are not allowed).  All
names of operations of the database must be unique. Different data bases 
can contain operations with equal names. 

\paragraph{operation description}
Optional description of the operation.

\paragraph{inputs}
Number of inputs of the operation. The number of inputs is a
nonnegative integer. It must be positive if the number of outputs is
zero.

\paragraph{outputs}
Number of outputs of the operation. The number of outputs is a
nonnegative integer. It must be positive if the number of inputs is
zero.

\paragraph{reads/writes-memory}
Indicates that this operation can access memory.  Normally, memory access is
also implied from the properties `mem-address' and `mem-data' of operation
inputs (or the `mem-data' property of operation outputs).  However, it is
possible to define operations that perform \emph{invisible} accesses to
memory, whereby no input or output is related to the memory access itself.
That is, neither the address nor the data moved into or out of memory is
explicitly specified by the operation.  In these operations, memory accesses
occur as a side effect, and none of the inputs or outputs have
memory-related properties.  To avoid potential errors, the memory property
must be specified explicitly even when it is implied by some of the inputs
or outputs of the operation.  See sections on input and output declarations,
below.

%Not supported in TCE v1.0
%\paragraph{trap}
%Indicates that this operation can raise an exception when executed on some
%implementations.

\paragraph{side-effect}
Indicates that two subsequent executions of this operation with the same
input values may generate different output values.  An operation with
side-effect may be an operation with state or an operation whose results
depend on other operations or on a context that is external to the
processor.  Side-effect is implied if the `affected-by' or the `affects'
lists are not empty. Side-effect is also implied by the fact that any of the
inputs is optional, see section on input declaration.

\paragraph{clocked}
Clocked attribute indicates that the operation can change its state
synchronously with clock signal and independently from its input.

\paragraph{affected-by}
Optional. A list of operation names. If an operation is listed, it means
that it may affect the outcome of this operation.  The order in which the
two operations are executed cannot be swapped.  An operation where
`affected-by' list is not empty is also a side-effect operation.

\paragraph{affects}
Optional. A list of operation names.  If an operation is listed, it means
that it may be affected by the execution of this operation. The order in
which the two operations are executed cannot be swapped.  An operation where
`affects' list is not empty is also a side-effect operation.

Note: it is not necessary that, if operation A `affects' operation B, then B
must contain A in its `affected-by' list.  Vice versa, if A is `affected-by'
B, it is not needed that B must contain A in its `affects' list. This
allows a user to add new operations that share state with the base
operations shipped with TCE, without needing to modify the base operations.

\subsubsection{Operation Input Properties}
\label{ssec:opinput-properties}

Each input of an operation requires an independent declaration of its
properties.  An operation input is completely defined by the following
properties:
\begin{itemize}
\item identification number (integer)
\item memory address (yes/no)
\item memory data (yes/no)
\item can be swapped (set of integers)
\end{itemize}

\paragraph{identification number}
Integer number in the range [1,N] where N is the number of inputs as defined
in section~\ref{ssec:operation-properties} of operation declaration.  If N
is zero, then no input declarations can be specified. TCE does not currently
allow operations with zero inputs to be defined.

\paragraph{can-swap}
A list of identification numbers.  All the inputs listed can be swapped with
this input.  The identification number of this input definition is not
allowed in the list.  The can-swap property is commutative, thus any of the
listed inputs is implicitly `can-swap' with this input and all the other
inputs listed.  The can-swap declaration need not be symmetrical, but it is
not an error if the implied declaration is also specified.

\paragraph{mem-address}
Optional.  Indicates that the input is used to compute (affects the value
of) the memory address accessed by this operation.

\paragraph{mem-data}
Optional.  Indicates that the input is used to compute (affects the value
of) the data word written to memory by this operation.  This property
implies that the operation writes to memory.

\subsubsection{Operation Output Properties}
\label{ssec:opoutput-properties}

Note: it is not an error if a program, before instruction scheduling,
contains an operation where one of the output moves is missing. If all
output moves of an operation are missing, then the only useful work that can
be performed by the operation is state change.

\paragraph{mem-data}
Optional.  Indicates that the output contains a value that depends on a
data word read from memory by this operation.  This property implies that
the operation reads data from memory.

\subsubsection{Operation Behavior}
\label{ssec:operation-behavior}

To be complete, the model of an operation needs to describe the
behavior of the operation. The behavior is specified in a restricted
form of C++, the source language of the TCE toolset, augmented with
macro definitions.  This definition is used for simulating the
operation in the instruction set simulator of TCE.

\paragraph{Definition of Operation Behavior}

Operation behavior simulation functions are entered inside an operation
behavior definition block. There are two kinds of such blocks: one for
operations with no state, and one for operations with state.

\begin{description}
\item[OPERATION(\emph{operationName})] %
  Starts an operation behavior definition block for an operation with name 
  \emph{operationName}. Operations defined with this statement do not 
  contain state. Operation names must be written in upper case letters!
\item[END\_OPERATION(\emph{operationName})] %
  End an operation behavior definition block for an operation with no state.
  \emph{operationName} has to be exactly the same as it was entered in the
  block start statement \verb|OPERATION()|.
\item[OPERATION\_WITH\_STATE(\emph{operationName}, \emph{stateName})] %
  Starts an operation behavior definition block for an operation with state.
  \emph{operationName} contains the name of the operation, \emph{stateName}
  name of the state. DEFINE\_STATE() definition for the \emph{stateName}
must 
  occur before this statement in the definition file. Operation and state
  names must be written in upper case letters!
\item[END\_OPERATION\_WITH\_STATE(\emph{operationName})] %
  Ends an operation behavior definition block for an operation with state.
  \emph{operationName} has to be exactly the same as it was entered in the
  block start statement \verb|OPERATION_WITH_STATE()|.
\end{description}

The main emulation  function definition block is given as:
\begin{description}
\item[TRIGGER ... END\_TRIGGER;] %
  Main emulation function.

% \item[LATE\_RESULT ... END\_LATE\_RESULT;] %
%   Function that emulates late-coming results.
\end{description}

% Only the first definition is mandatory.  The LATE\_RESULT definition allows 
% the user to define operations with complex, nondeterministic behavior.

The bodies of the function definitions are written in the operation behavior
language, described in Section~\ref{ssec:behaviour-commands}.

\paragraph{Operations with state.}

To define the behavior of an operation with state it is necessary to
declare the state object of the operation.  An operation state declaration
is introduced by the special statement \verb#DEFINE_STATE()#.  See
Section~\ref{ssec:behaviour-commands} for a description of this and related
statements. State must be declared before it is used in operation behavior
definition. 

A target processor may contain several \emph{implementations} of the
same operation.  These implementations are called Hardware Operations and
are described in~\cite{ADF-specs}.
%
Each Hardware Operation instance belongs to a different function unit and is
independent from other instances.
%
When an operation has state, each of its Hardware Operations uses a
different, independent instance of the state class (one for each function
unit that implements that operation).

An operation state object is unambiguously associated with an operation (or
a
group of operations, in case the state is shared among several) by means of
its name, which should be unique across all the operation definitions.%
\note{see issue \ref{sec:state-name}}

Operation state can be accessed in the code that implements the behavior of
the operation by means of a \verb#STATE# expression.  The fields of the
state object are accessed with the dot operator, as in C++.  See
Section~\ref{ssec:behaviour-commands} for a complete description of this
statement.

\paragraph{Main emulation function.}
The behavior model of an operation must include a function that, given a
set of input operand values and, optionally, an operation state instance,
produces one or more output values that the operation would produce.

The definition of an emulation function is introduced by the statement
\verb#TRIGGER# and is terminated by the statement \verb#END_TRIGGER;#.  

An emulation function is expected to read all the inputs of its operation
and to update the operation outputs with any new result value that can be
computed before returning.

\subsubsection{Behavior Description language}
\label{ssec:behaviour-commands}

The behavior of operations and the information contents of operation state
objects are defined by means of the behavior description language.

The emulation functions that model operation behavior are written in C++
with some restrictions.
The  OSAL behavior definition
language augments the C++ language with a number of statements.  For
example, 
control may exit the definition body at any moment by using a special 
statement, a set of statements is provided to refer to operation inputs
and 
outputs, and a statement is provided to access the memory model.

\paragraph{Base data types.}
The behavior description language defines a number of base data
types.  These types should be used to implement the operation behavior
instead of the C base data types, because they guarantee the bit width and
the format.
\begin{description}
\item[IntWord] %
  Unsigned integer 32-bit word.
\item[FloatWord] %
  Single-precision (32-bit) floating-point word in IEEE-754 format.
\item[DoubleWord] %
  Double-precision (64-bit) floating-point word in IEEE-754 format.
\end{description}

\paragraph{Access to operation inputs and outputs.}
Inputs and outputs of operations (henceforth referred to as
\emph{terminals}, when a distinction is not needed) are referred to by a
unique number.  The inputs are assigned a number starting from 1 for the
first input.  The first output is assigned the number $n+1$, where $n$ is
the number of inputs of the operations, the second $n+2$, and so on.

Two sets of expressions are used when accessing terminals. The value of
an input terminal
can be read as an unsigned integer, a signed integer, a single precision 
floating point number, or
a double precision floating point number using the following expressions:
\begin{description}
\item[UINT(\emph{number})] %
  Treats the input terminal denoted by \emph{number} as a number of type
  IntWord, which is an unsigned integer of 32 bits maximum length.
\item[INT(\emph{number})] %
  Treats the input terminal denoted by \emph{number} as a number of type
  SIntWord, which is a signed integer of 32 bits maximum length.
\item[FLT(\emph{number})] %
  Treats the input terminal denoted by \emph{number} as a number of type
  \emph{FloatWord}.
\item[DBL(\emph{number})] %
  Treats the input terminal denoted by \emph{number} as a number of type
  \emph{DoubleWord}.
\end{description}

Output terminals can be written using the following expression:
\begin{description}
\item[IO(\emph{number})] %
   Treats the terminal denoted by \emph{number} as an output terminal.
   The actual bit pattern (signed, unsigned or floating point) written to
   the output terminal is determined by the right hand expression
   assigned to the IO() expression.
\end{description}

Since the behavior of certain operations may depend in non-trivial ways on
the bit width of the terminals of a given implementation, it is sometimes
necessary to know the bit width of every terminal.  The expression
\begin{description}
\item[BWIDTH(\emph{number})]
\end{description}
returns the bit width of the terminal denoted by \emph{number} in the
implementation of the calling client.

Bit width of the operands can be extended using two different expressions.

\begin{description}
\item[SIGN\_EXTEND(\emph{integer}, \emph{sourceWidth})]%
  Sign extends the given integer from \emph{sourceWidth} to 32 bits.
 
  Sign extension means that the sign bit of the source word is duplicated
  to the extra bits provided by the wider target destination word.

  For example a sign extension from 1001b (4 bits) to 8 bits provides
  the result 1111 1001b.

\item[ZERO\_EXTEND(\emph{integer}, \emph{sourceWidth})]%
  Zero extends the given integer from \emph{sourceWidth} to 32 bits.

  Zero extension means that the extra bits of the wider target destination
  word are set to zero. 

  For example a zero extension from 1001b (4 bits) to 8 bits provides
  the result 0000 1001b.

\end{description}

\emph{Example.} The following code implements the behavior of an accumulate
operation with one input and one output, where the result value is saturated
to the ``all 1's'' bit pattern if it exceeds the range that can be expressed
by the output:
\begin{verbatim}
  STATE.accumulator += INT(1);
  IntWord maxVal = (1 << BWIDTH(2)) - 1;
  IO(2) = (STATE.accumulator <= maxVal ? STATE.accumulator : maxVal);
\end{verbatim}

\paragraph{Definition of operation state.}
Operation state consists of a data structure.  Its value
is shared by one or more operations, and it is introduced by the statement
\begin{description}
\item[DEFINE\_STATE(\emph{name})]
\end{description}
where \emph{name} is a string that identifies this type of operation state.
This statement is followed by a list of data type fields. The state name
string must be generated with the following regular expression:
\begin{quote}
  [A-Z][0-9A-Z\_]*
\end{quote}
Note that only upper case letters are allowed.

A state definition block is terminated by the statement
\begin{description}
\item[END\_DEFINE\_STATE]
\end{description}

\emph{Example.} The following declaration defines an operation state class
identified by the name string ``BLISS'', consisting of one integer word, one
floating-point word and a flag:
\begin{verbatim}
DEFINE_STATE(BLISS) 
  IntWord data1;
  FloatWord floatData;
  bool errorOccurred;
END_DEFINE_STATE;
\end{verbatim}

%% These were already defined a couple a pages before:
% The data type IntWord represents a 32-bit integer number; the data type
% FloatWord represents a 32-bit floating-point IEEE-754 number.

Some operation state definitions may require that the data is initialized to
a predefined state, or even that dynamic data structures are allocated when
the operation state object is created. In these cases, the user is required
to provide an initialization definition inside the state definition block.
\begin{description}
\item[INIT\_STATE(\emph{name})]%
  Introduces the code that initializes the operation state.
\item[END\_INIT\_STATE]%
  Terminates the block that contains the initialization code.
\end{description}

Some state definitions may contain resources that need to be released when the
state model is destroyed. For example, state may contain dynamically
allocated data or files that need to be closed. In these cases, the user
must define a function that is called when the state is deallocated. This
function is defined by a finalization definition block, which must be
defined inside the state definition block.
\begin{description}
\item[FINALIZE\_STATE(\emph{name})]%
  Introduces the code that finalizes the operation state, that is,
  deallocates the dynamic data contained in an operation state object.
\item[END\_FINALIZE\_STATE]%
  Terminates the block that contains finalisation code.
\end{description}

The state model provides two special definition blocks to support emulation
of operation behaviour.

\begin{description}
\item[ADVANCE\_CLOCK \ldots END\_ADVANCE\_CLOCK]%
  In case the model of operations state is synchronous, this definition can
  be used to specify activity that occurs ``in the raising edge of the clock
  signal'', that is, at the end of a simulation cycle. The C `return'
  statement can used to return from this function.
\end{description}

\paragraph{Access to operation state.}
Operation state is denoted by a unique name string and is accessed by means
of the statement
\begin{description}
\item[STATE]
\end{description}

Typically, an operation state data structure consists of several fields,
which are accessed using the dot operator of C++.  For example, the
expression \verb#STATE.floatData# in a simulation function refers to the
field \verb#floatData# of the state object assigned to the operation
being defined.  The precise format of an
operation state structure is defined by means of the \verb#DEFINE_STATE()#
statement, and it is specific for an operation. State must be defined before it
can be used in an operation definition.

\paragraph{Access to control registers.}
Operations that can modify the program control flow can access the program
counter register and the return address of the target processor by means of
the following expressions:
\begin{description}
\item[PROGRAM\_COUNTER] %
\item[RETURN\_ADDRESS] %
\end{description}
Not all operations can access the control registers.  Only operations
implemented on a Global Control Unit (see~\cite{ADF-specs}) provide the
necessary data.  It is an error to use \verb#PROGRAM_COUNTER# or
\verb#RETURN_ADDRESS# in operations that are not implemented on a Global
Control Unit.

\paragraph{Context Identification}

Each operation context can be identified with a single integer which
can be accessed with \verb#CONTEXT_ID#.

\paragraph{Returning from operation behavior emulation functions.}
Normally, an emulation function returns control to the caller when control
flows out of the definition body.  To return immediately, from an arbitrary
point in the definition body, the following normal C++ return 
statement can be used. The return value is boolean indicating the success
of the operation execution. In pratice, 'true' is always returned:
\verb|return true;|.

\paragraph{Memory Interface.}

The memory interface of OSAL is very simplified allowing easy modeling of
data memory accessing operations. The following keywords are used
to define the memory access behavior:

\begin{description}
\item[MEMORY.read(address, count, target)]%
  Reads \emph{count} units of data from the \emph{address} to the variable
  \emph{target}.
\item[MEMORY.write(address, count, data)]%
  Writes \emph{count} units of data in variable \emph{data} to 
  the \emph{address}.
\end{description}


\subsection{Simulation Trace Database}
\label{sec:tracedb}

Filename extension: .tracedb

Stores data collected from simulations and used by instruction scheduler
(profiling data) and cost estimator (utilization statistics, etc.).

\subsection{Exploration Result Database}
\label{sec:expresdb}

Exploration Result Database (ExpResDB) contains the configurations that have
been evaluated during exploration (manual or automatic) and a summary of
their characteristics. Files that define each tested configuration (ADF and
IDF) are stored in the database as well. 


\section{Notes About the Processor Template of TCE}
\label{section:template}

The processor template from which the application specific processors
designed with TCE are defined from is called Transport Triggered 
Architecture (TTA). For a detailed description behind the TTA philosophy, 
refer to \cite{HCorp97}.
This section describes certain aspects of the TTA template used in the TCE
toolset that might not be very clear.

\subsection{Immediates/Constants}

The TTA template supports two ways of transporting program constants in
instructions. \textit{Short immediates} are encoded in the move slot's
source field, and thus consume a part of a single move slot. The constants
transported in the source field are usually relatively small in size. 
Wider constants can be transported by means of so called \textit{long 
immediates}. Long immediates can be defined using a 
parameter called \textit{instruction template}. The idea is that
each TTA instruction is connected to a single instruction template which 
defines the move slots that contain pieces of a long immediate, if any. 
The slots cannot be used for
regular data transports when they are used for transporting pieces of
a long immediate. An instruction containing a long immediate also provides
a target to which the long immediate must be transported. The target is so 
called \textit{immediate unit} which is written directly from the control 
unit, not through the transport buses. The immediate unit is like
a register file expect that it contains only read ports and is written
only by the instruction decoder in the control unit when it detects an 
instruction with a long immediate.

\subsection{Operations, Function Units, and Operand Bindings}

Due to the way TCE abstracts operations and function units, an
additional concept of \textit{operand binding} is needed to 
connect the two in processor designs.

Operations in TCE are defined in a separate database 
(OSAL, Section~\ref{section:osal}) in order to allow defining a reusable
database of ``operation semantics''. The operations are
used in processor designs by adding \textit{function units} (FU) that implement
the wanted operations. Operands of the operations can be
mapped to different ports of the implementing FU, which affects 
programming of the  processor. Mapping of operation operands
to the FU ports must be therefore described by the processor designer
explicitly.

\textbf{Example.} Designer adds an FU called 'ALU' which implements
operations 'ADD', 'SUB', and 'NOT'. ALU has two input ports called 'in1' and 
'in2t' (triggering), and an output port called 'out'. A logical binding of 
the 'ADD' and 'SUB' operands to ALU ports is the following:

\begin{verbatim}
 ADD.1 (the first input operand) bound to ALU.in1
 ADD.2 (the second input operand) bound to ALU.in2t
 ADD.3 (the output operand) bound to ALU.out

 SUB.1 (the first input operand) bound to ALU.in1
 SUB.2 (the second input operand) bound to ALU.in2t
 SUB.3 (the output operand) bound to ALU.out
\end{verbatim}

However, operation 'NOT', that is, the bitwise negation has only one input
thus it must be bound to port 'FU.in2t' so it can be triggered:

\begin{verbatim}
 NOT.1 bound to ALU.in2t
 NOT.2 (the output operand) bound to ALU.out
\end{verbatim}

Because we have a choice in how we bind the 'ADD' and 'SUB' input operands,
the binding has to be explicit in the architecture definition. The operand
binding described above defines architecturally different TTA function unit
from the following:

\begin{verbatim}
 SUB.2 bound to ALU.in1
 SUB.1 bound to ALU.in2t
 SUB.3 bound to ALU.out
\end{verbatim}

With the rest of the operands bound similarly as in the first example.

Due to the differing 'SUB' input bindings one cannot run code scheduled
for the previous processor on a machine with an ALU with the latter 
operand bindings. This small detail is important to understand when 
designing more complex FUs, with multiple operations with different number of 
operands of varying size, but is usually transparent to the basic user of 
TCE.

Reasons for wanting to fine tune the operand bindings might include using
input ports of a smaller width for some operation operands. For example, the 
width of the address operands in memory accessing operations of a load store 
unit is often smaller than the data width. Similarly, the second operand of
a shift operation that defines the number of bits to shift requires less
bits than the shifted data operand.

% !!!!!!! FOR EACH TOOL DESCRIBE:
% The main purpose of existence. The main use cases.
% ALL features and how to use them in case of command line tools.
% In GUIs, describe only the most unclear parts (esp. in ProDe).


\chapter{TUTORIALS}
\label{chapter:tutorials}


\section{TCE Tour}
\label{sec:tcetour}

This tutorial goes through most of the tools in TCE using a fairly simple
example application. It starts from C code and ends up with VHDL of the
processor and a bit image of the parallel program. This tutorial will also
explain how to accelerate an algorithm by customizing the instruction
set, i.e.\ by using custom operations. The total time  to go through
the tutorial is about 2 to 3 hours.

The tutorial file package is available at:

\url{http://tce.cs.tut.fi/tutorial_files/tce_tutorials.tar.gz}\\

Unpack it to a working directory and cd to tce\_tutorials/tce\_tour.

\subsection{The Sample Application}

The test application counts a 32-bit CRC (Cyclic Redundant Check). The C code
implementation is written by Michael Barr and it is published under Public
Domain. The implementation consists of two different version of crc, but we will
be using the fast version only.

The program consists of two separate files: \file{main.c} contains the simple main
function and \file{crc.c} contains the actual implementation. Open \file{crc.c} in
your preferred editor and take a look at the code. The
main difference between the crcSlow and crcFast implementations is that 
crcFast exploits precalculated table values. This is a quite usual method of
algorithm optimization.

\subsection{Generating LLVM Bitcode}

We first generate LLVM bitcode from the C code. The bitcode
will be later compiled to an architecture dependent TTA program.
Compile the C code with the 'tcecc' compiler:

\shellcmd{tcecc -O2 -o crc.bc -k main,result crc.c main.c}

This produces an architecture independent bitcode program \file{crc.bc}
of the C code. The switch -k was used to tell the
compiler to keep the \textit{main} and \textit{result} symbols in the
generated program in order to access them by name in the simulator.

\subsection{Starting Point Processor Architecture}

We use the {\em minimal architecture} as the starting point. File
\file{minimal.adf} describes a minimalistic architecture containing just enough
resources that the TCE compiler can still compile programs for it. Function
units in \file{minimal.adf} are selected from the hardware database (HDB,
Section~\ref{section:hdb}) so we are able to generate a VHDL implementation
of the processor automatically later in the tutorial. 
% It is also possible to
% design all the components from scratch in order to evaluate the architecture
% without actually implementing the components in VHDL.

Copy the \file{minimal.adf} included in TCE distribution to a new ADF file which
is your starting point architecture:

\shellcmd{cp \$(tce-config -{}-prefix)/share/tce/data/mach/minimal.adf start.adf}

Take a look at the starting point architecture using the graphical Processor
Designer (ProDe, Section~\ref{sec:prode}) tool. Start ProDe with:

\shellcmd{prode start.adf \&}

If you have GHDL installed in your system and you want to simulate the
generated processor I suggest you to decrease the amount of memory in the
processor. Otherwise the GHDL generated testbench might consume tremendous
amount of memory on your computer. To do this select Edit -> Address Spaces
in ProDe. Then edit the bit widths of data and instruction address spaces and
set them to 12 bits which should be plenty for our case.


\subsection{Evaluating the Starting Point Architecture}

Now we want to know how well the starting point architecture executes our
program. First we need to map (``schedule'') the generic bitcode program
to the parallel architecture we examined in the previous section. This can be
done by compiling the bitcode program against the processor architecture
with the command:

\shellcmd{tcecc -a start.adf -o crc.tpef -k main,result crc.bc -O3}

This will produce a parallel program called \file{crc.tpef} that can be
executed with the processor design \file{start.adf}.  The parallel
program is now tied to a specified architecture, so it is no longer
architecture independent like the bitcode program.

After successfully compiling the program we can now simulate it. Lets use
the command line based simulator called ttasim. Start the simulator with:

\shellcmd{ttasim -a start.adf -p crc.tpef}

The simulator will load the architecture definition and the program and
execute it. After this you should see ttasim prompt: \textit{(ttasim)}.
The simulator can show various pieces of information about the execution of the program and
utilization of processor resources.

To see the cycle count use the following command in ttasim:

\shellcmd{info proc cycles}

Lets take a look at the processor utilization. Enter:

\shellcmd{info proc stats}

This will output a lot of information about the execution, like
utilization of transport busses, register files and function units.

Also check the result with:

\shellcmd{x /u w \_result}

The checksum should be \textbf{0x62488e82}.
Next take a look at the graphical user interface of the simulator called Proxim
(Section~\ref{section:proxim}). You can start it with:

\shellcmd{proxim start.adf crc.tpef \&}

Run the simulation and check the final cycle count it took to execute the
program. The cycle count is written in the bottom bar of the simulator.

Proxim can also show various pieces of information about a program's execution and
resource utilization. To check the utilization of the resources of our
architecture, select \textit{View>Machine Window} from the top menu. 
Then enable the utilization visualization: \textit{Right click the machine
window > Display utilizations}. The parts of the processor that are utilized
the most are visualized with darker red color.

Next, check the profiling data to see which instructions were executed the
most. Select: \textit{Source>Profile data>Highlight Top Execution Counts}.
Now the disassembly window has the most executed instructions highlighted.
You should be able to spot the main loop of the program easily.


As you probably noticed from the processor utilisation, the dividers
(divs, divu) were not used. Currently the dividers have been implemented
using Design Ware IP, which isn't freely distributed. In order to generate
functioning processor you will have to remove these two FUs from the machine.
Open ProDe and delete the divs and divu FUs and their sockets and remember to
save.

\subsection{Initial optimization}

As you noticed the cycle count is tremendous. The reason for this is that the
crcTable array is evaluated at runtime in crcInit function. In an actual
system the evaluation would only be done once at startup, so the impact on cycle
count would be miniscule in the long run. Nevertheless, we will remove this
overhead and precalculate the values so we can consentrate on the optimization
of the main CRC algorithm instead of the initialization routines.

A simple method to precalculate the array values is to modify the
crcInit function so that it prints the array after it has been
calculated. The values obtained are included in the tutorial as the
file \file{crcTable.dat}. Open \file{crc.c} and find the line:

\begin{verbatim}
 crc crcTable[256];
\end{verbatim}

To load the precalculated values, modify the line to read:

\begin{verbatim}
 crc crcTable[256] = {
 #include "crcTable.dat"
 };
\end{verbatim}

Next open \file{main.c} and remove or comment out the crcInit() function call because
we now use the precalculated crcTable. Then
recompile the code and simulate it again. Verify the result and check cycle
count as done earlier. This time you should see a huge drop in the cycle
count. Write down the cycle count as a baseline for future comparison.

\subsection{Accelerating the Algorithm}

Custom operations implement application specific functionality in TTA
processors. In this part of the tutorial we accelerate the CRC computation
by adding a custom operation to the processor design.

\subsubsection{Evaluating Custom Operation Candidates}

First of all, it is quite simple and efficient to implement CRC calculation
entirely on hardware. Naturally, using the whole CRC function as a custom
operation would be quite pointless and the benefits of using a processor
based implementation would get smaller. Instead, we will consentrate
on trying to accelerate smaller parts of the algorithm, picking a custom
operation that is potentially useful also for other algorithms than CRC.

\subsubsection{Finding the Bottlenecks}

The first thing to do when trying to optimize code is to profile the execution.
In Proxim you can trace the most executed instructions (see section
\ref{sec:ProfileProxim} for more information). Another way is to enable
``procedure transfer tracking'',  which traces function calls and
corresponding cycle counts. A call profile helps locating the functions
that are executed the most. Refer to section \ref{sec:traces} for further
information.

In this case finding the operation to be optimized is quite obvious if you
look at function crcFast(). It consists of a for-loop in which the 
function reflect() is called through the macro REFLECT\_DATA. If you look at 
the actual function you can see that it is quite simple to implement on 
hardware, but requires many instructions if done with basic operations in
software. The function ``reflects'' the bitpattern around the middle point like
a mirror. For example,
the bit pattern \textbf{0101 0100} would look like this after reflection:
\textbf{0010 1010}. The main complexity of the function is that the bit pattern width is not
fixed. Fortunately, the width cannot be arbitrary. If you examine the
crcFast()-function and the reflect macros you can spot that function reflect() is
only called with 8 and 32 bit widths (unsigned char and 'crc' which is an unsigned long).

\subsection{Analyzing the Custom Operation}

A great advantage of TCE is that the operation semantics, 
processor architecture and implementation are separate abstractions. 
How this affects designing custom operations is that you can
simulate your design by simply defining the simulation behaviour of 
the operation and setting the latency of the operation to the processor
architecture definition. This is nice as you
do not need an actual hardware implementation of the operation at this point
of the design, but
can evaluate different custom operation possibilities at the architectural
level. However, this brings up an awkward question: how to determine 
the latency of the operation? Unrealistic or too pessimistic 
latency estimates can give inaccurate results or bias the analysis.

One approach to the problem is to take an educated guess and simulate
some test cases with different custom operation latencies. This way you can
determine a latency range in which the custom operation would accelerate your
application to the satisfactory level. After this you can scetch how the
operation could be implemented in hardware, or consult someone knowledgeable 
in hardware design to figure out whether the custom operation is 
implementable within the latency constraint.

Another approach is to try and determine the latency by examining the
operation itself and considering how it could be implemented. This approach
requires some insight in digital design.

Besides latency you should also consider the size of the custom function unit.
It will consume extra die area, but the size limit is always case-specific. For
accurate size estimation you need to have the actual implementation and synthesis.

Let us consider the reflect function. If we had fixed width we could implement
the reflect by hard wiring (and registering the output) because the operation
only moves bits about. This could be done easily in one clock cycle. But we
need two different bit widths so things would be a bit more complicated. We
could design the implementation to be 32 bits wide. This way we can also use
it to reflect 8 bit patterns by aligning the input around the middle position
and align the output back to the right position. The alignment could be done
with multiplexers etc. When a lower clock frequency is used, this could still be a one
cycle operation, but let us play it safe and allow for faster clock frequencies,
and use a latency of two clock cycles.

\subsection{Using the Custom Operation}
\label{section:customOperations}
% includes adding a FU that implements the custom operation to HDB and
% adding the operation to OSAL

Now we have decided the operation to be accelerated and its latency. Next we
will create a function unit implementing the operation and add it to our
processor design. First, a description of
the semantics of the new operation must be added at least to Operation Set
Abstraction Layer (Section~\ref{section:osal}). OSAL stores the semantic
properties of the operation, which includes the simulation behavior, operand
count etc., but not the latency. OSAL definitions can be added by using the
OSAL GUI, \emph{OSEd} (Section~\ref{sec:osed}).

If processors that use the custom operation are to be synthesized or simulated
at the VHDL level, at least one function unit implementing
the operation should be added to the
Hardware Database (Section~\ref{section:hdb}). Cost data of the function unit
needs to be added to the cost database if cost estimates of a processor
containing the custom function unit are wanted. In this tutorial we add the
FU implementation for our custom operation so the processor implementation can
be generated, but omit the cost data required for the cost estimation.

\paragraph{Using Operation Set Editor (OSEd) to add the operation data.}

OSEd is started with the command 'osed'.

Create a new operation module, which is a container for a set of operations.
You can add a new module in any of the predefined search paths, provided
that you have sufficient file system access on the chosen directory.

For example, choose directory `/home/\emph{user}/.tce/opset/custom', where
\emph{user} is the name of the user account being used for the tutorial.
This directory is intended for the custom operations defined by the
current user, and should always have sufficient access rights.

\begin{enumerate}
\item%
  Right-click on a path name in the left area of the main window.  A
  drop-down menu appears below the mouse pointer.
\item%
  Select \textbf{Add module} menu item. 
\item%
  Type in the name of the module (for example, `tutorial') and press \emph{OK}.
  The module is now added under the selected path.
\end{enumerate} 

\paragraph{Adding a new operation.} We will now add an operation
definition to the newly created operation module.

\begin{enumerate}
\item%
  Select the module that you just added by right-clicking on its name,
  displayed in the left area of the main window. A drop down menu appears.
\item%
  Select \textbf{Add operation} menu item.
\item%
  Type `REFLECT' as the name of the operation.
\item%
  Add two inputs by pressing the \emph{Add} button under the operation input
  list repeatedly. Select \textit{UIntWord} as type.
\item%
  Add one output by pressing the \emph{Add} button under the operation output
  list. Select \textit{UIntWord} as type.
\item%
  After the inputs and the output of the operation have been added, close
  the dialog by pressing the \emph{OK} button. A confirmation dialog will pop
  up. Press \emph{Yes} to confirm the action. The operation definition is
  now added to the module.
\end{enumerate}

\paragraph{Adding simulation behavior to the operation.} The new operation
REFLECT does not yet have a simulation behavior model, so we cannot simulate
programs that use the operation with the TCE processor simulator. Open again the
operation property dialog by right-clicking REFLECT, then choosing \emph{Modify
properties}. Now press the \emph{Open} button to open an empty behavior source
file for the module. Type in the following code in the editor window:

\begin{verbatim}
#include "OSAL.hh"
OPERATION(REFLECT)
 TRIGGER

 unsigned long data = UINT(1);
 unsigned char nBits = UINT(2);

 unsigned long  reflection = 0x00000000;
 unsigned char  bit;

 /*
  * Reflect the data about the center bit.
  */
 for (bit = 0; bit < nBits; ++bit)
 {
     /*
      * If the LSB bit is set, set the reflection of it.
      */
     if (data & 0x01)
     {
         reflection |= (1 << ((nBits - 1) - bit));
     }

     data = (data >> 1);
 }

 IO(3) = static_cast<unsigned> (reflection);

 return true;
 END_TRIGGER;
 END_OPERATION(REFLECT)
\end{verbatim}

This behavior definition simply sums up the input operand integers (with ids
1 and 2) and writes the result to the ''output operand`` with id 3 which is the first output
and signals the simulator that all results are computed successfully.

Open file \file{crc.c} in your preferred editor. Compare the definition file
version and the original reflect-function. The function is mostly similar
except for parameter passing. On hardware the function unit's data is read
from input ports and written to output ports. So, we must declare the input
parameters data and nBits as normal variables. Then we assign input values to
them: value of data is read from (input) operand one (INT(1)) and nBits is read
from operand two (INT(2)).  This declaration binds the input ports,
data is to be written to port one and nBits to port two.

Also the function's return value must be handled differently because the 
result is read from the output port. Here is an important notice: the operation
operands are \textbf{running numbered} so that the number of first
output port is not 1 but (number of input operands) + 1! In addition, the output
operands are presented as IO(x).

Save the code and close the editor. REFLECT operation now has a TCE simulator
behavior model.

\paragraph{Compiling operation behavior.} Operation  REFLECT has been
added to the test module. Before we can simulate the behavior of our
operation, the module must be compiled.

\begin{enumerate}
\item%
  Right-click on the module name displayed in the left area to bring up the
  drop down menu.
\item%
  Select \textbf{Build} menu item. 
\item%
  Hopefully, no errors were found during the compilation! Otherwise, re-open
  the behaviour source file and try to locate the errors with the
  help of the diagnostic information displayed in the build dialog.
\end{enumerate}

\paragraph{Simulating the operation behavior.} Now the operation simulation
behavior definition can be tested.

\begin{enumerate}
\item%
  Right-click on the operation name displayed in the left area of the
  main window, under the operation module that contains it.
\item%
  Select \textbf{Simulate} menu item from the drop-down menu. 
\item%
  Edit the input values of REFLECT by selecting the input operand from the
  list and typing the values in the text input field and pressing the
  \emph{Update} button next to the field. In this case it is handy to choose
  the format to be binary.
\item%
  Press the \emph{Trigger} button. The value of the output should be updated.
\item%
  The format of the displayed input and output values can be modified by
  selecting different formats from the choice list next to the label
  `Format:'. It might be more illustrating if you change input and output
   format to binary.  This way we can enter binary patterns, 
  for example '11001100' (as binary) for the first operand and '8' (as decimal)
  as the second. The trigger should produce the reflected
  value '00110011'

\item%
  The state of REFLECT can be reset by pressing the \emph{Reset} button.
\item%
  Repeat the test with different input values. After you are convinced
  the behavior definition works, close the dialog by clicking on the \emph{OK} button.
\end{enumerate}

Now the operation definition of the custom operation has been added to the
Operation Set Abstraction Layer (OSAL) database. Next, we need to add at
least one functional unit (FU) that implements the operation which can be
used in the processor design. Note the separation between ''operation`` and
an ''function unit`` that implements the operation which allows using the
same OSAL operation definitions in multiple FUs with different latencies.

First, add the architecture of the FU that implements the custom operation
to the processor design we defined in the previous step. Let us first backup
our processor design to another file so we can later more easily compare the
architecture with and without the custom operation support:

\shellcmd{cp start.adf custom.adf}

Open the copy in ProDe:

\shellcmd{prode custom.adf \&}

Then:
\begin{enumerate}
\item%
Add a new function unit to the design, right click the canvas and select:
\textit{Add>Function Unit}. Name the FU ''reflecter``. Add two input
ports (named as input1 and trigger) and an output port (output1) to the FU in
the Function unit dialog. Set the second input port (trigger)
triggering (\textit{Click the port named trigger->Edit->Check dialog 
''triggers``}). This port starts the execution of the operation when
it is written to.
\item%
Add the operation ''REFLECT`` we defined to the FU: \textit{Add from
opset>REFLECT>OK} and set the latency to 2. Click on the REFLECT operation and
ensure that the operation inputs are bound to the input ports and the output
is bound to the output port. Check that the operand usage is in such a way that
all inputs are read at cycle 0 and the result is written at the end of the 
next cycle. Thus, the latency of the operation is 2 clock cycles.
\item%
Now an FU that supports the custom operation has been added to the architecture. 
Finally, fully connect the machine to connect the
FU to the rest of the architecture. This can be done by selecting
\textit{Tools->Fully Connect IC}. Finally save the architecture.
\end{enumerate}


\subsection{Use the custom operation in C code.}

To get some benefits from the added custom hardware,
we must use it from the C code. This is done by replacing
a C statement with a custom operation invocation. 

Let us first backup the original C code.

\shellcmd{cp crc.c crc\_with\_custom\_op.c}

Then open \file{crc\_with\_custom\_op.c} in your preferred text editor.

\begin{enumerate}
\item%
Add \#include ``tceops.h'' to the top of the file. This includes 
automatically generated macros which allow us to use specific 
operations from C code without getting our hands dirty with
inline assembly.

Usage of these macros is as follows:

\begin{verbatim}
 _TCE_<name>(input1, ... , inputN, output1, ... , outputN);
\end{verbatim}

where <name> is the name of the operation in OSAL. Number of input and output
operands depends on the operation. Input operands are given first and they are
followed by output operands if any. 

In our case we need to write operands into the reflecter and read the
result from it. We named the operation ``REFLECT'' so the macro we are going
to use is:

\begin{verbatim}
 _TCE_REFLECT(input1, input2, output);
\end{verbatim}

Now we will modify the crcFast function to use the custom op.
First declare 2 new variables at the beginning of the function:

\begin{verbatim}
 crc input;
 crc output;
\end{verbatim}

These will help in using the reflect FU macro.

Take a look at the REFLECT\_DATA and REFLECT\_REMAINDER macros. The first one
has got a magic number 8 and ``variable'' X is the data. This is used in the
for-loop.

In the for-loop the input data of reflect function is read from message[].
Let us modify this so that at the beginning of the loop the input data is
read to the input variable. Then we will use the \_TCE\_REFLECT macro to run
the custom operations, and finally replace the REFLECT\_DATA macro with the
output variable. After these modifications the body of the for-loop should
look like this:

\begin{verbatim}
 input = message[byte];
 _TCE_REFLECT(input, 8, output);
 data = (unsigned char) output ^ (remainder >> (WIDTH - 8));
 remainder = crcTable[data] ^ (remainder << 8);
\end{verbatim}

Next we will modify the return statement. Originally it uses
REFLECT\_REMAINDER macro where nBits is defined as WIDTH and data is
remainder. Simply use \_TCE\_REFLECT macro before return statement and replace
the original macro with the variable output:

\begin{verbatim}
 _TCE_REFLECT(remainder, WIDTH, output);
 return (output ^ FINAL_XOR_VALUE);
\end{verbatim}

And now we are ready. Remember to save the file.

\item%
This time we skip the bitcode phase and compile directly from C to 
a parallel TTA program using the new architecture which includes
a FU with the custom operation:

\shellcmd{tcecc -O3 -a custom.adf -o crc\_with\_custom\_op.tpef -k main,result }\verb#\#\\
\shellcmd{\ \ crc\_with\_custom\_op.c main.c}

\item%
Simulate the parallel program. You can do it quickly with the command line
simulator:

\shellcmd{ttasim -a custom.adf -p crc\_with\_custom\_op.tpef}.

Verify that the result is the same as before. Check the cycle count
\textit{info proc cycles} and compare it to the cycle count of the version 
which does not use a custom operation. You should see a significant drop.

In addition to custom operations, other ways to improve the cycle count 
further is to add more resources such as registers and transport buses. 
You can play around with these parameters easily with ProDe. After 
modifying the architecture, recompile and simulate the code to see 
the effect to the cycle count.
\end{enumerate}

\subsection{Adding an implementation of the FU to the hardware database
(HDB).}
\label{par:AddToHDB}

Now we have seen that the custom operation accelerates our application. Next
we'll add a VHDL implementation of the custom FU to Hardware Database (hdb).
This way we will be able to generate a VHDL implementation of our processor.

Start HDBEditor (see Section~\ref{sec:hdbedit}):

\shellcmd{hdbeditor \&}

TCE needs some data of the FU implementation in order to be able to 
automatically generate processors that include the FU. \\
\begin{enumerate}
 \item%
Create a new hdb file (name it \file{tour.hdb}) with hdbeditor and add the 
''reflecter`` function unit from \file{custom.adf} file 
(\textit{edit->add->FU architecture from ADF}). You can leave
the ''parametrized width`` and ''guard support`` unchecked.
Then define implementation for the added function unit entry \textit{right 
click reflect -> Add implementation....}\\

\item%
Open file \file{reflect.vhdl} that was provided in the tutorial package with the editor 
you prefer, and take a look. This is an example implementation of 
a TTA function unit performing the custom 'reflect' operation.

\item%
The HDB implementation dialog needs the following information from the VHDL:\\

\textbf{1. Name of the entity and the naming of the FU interface ports.}

Name the implemention after the top level entity: ``fu\_reflect''.

By examining the VHDL code you can easily spot the clock port (clk),
reset (rstx) and global lock port (glock). Write these into the appropriate
text boxes. You do not have to fill the Opcode port or Global lock req. port
fields because the function unit has only one opcode and does not need to 
cause a global lock to the processor during its execution.

\textbf{2. Parameters.}

Parameters can be found from the VHDL file. On top of the file there are two
parameters: dataw and busw. The dataw tells the width of input ports and
busw tells us the width of the transport buses.

Thus, add two 32-bit integer parameters (dataw and busw) to the Parameter
dialog.

\textbf{3. Architecture ports.}

This settings defines the connection between the ports in the
architectural description of the FU and the VHDL implementation.
Each input data port in the FU is accompanied with a load port that
controls the updating of the FU input port registers.

Choose a port in the Architecture ports dialog and click edit.
Set the name of architecture port p1 to o1data and the load port to o1load.
Width formula is the parameter dataw.

Name of the architecture port p2 is t1data and load port is t1load.
Width formula is the parameter dataw.

Name the output port (p3) to r1data and the width formula is now busw
because the output port writes to the bus.
The output port does not have a load port.

\textbf{4. Add VHDL source file.}

Add the VHDL source file into the Source code dialog. Notice that the HDL
files must be added in the compilation order (see section \ref{sec:hdbedit}).
But now we have only one source file so we can simply add it without
considering the compilation order (Add -> Browse -> reflect.vhdl).

Now you are done with adding the FU implementation. Click OK.

\end{enumerate}

\subsection{Generating the Final Products}

In this step we generate the VHDL implementation of the processor, and the
bit image of the parallel program.

First create an encoding for the instructions in the architecture:

\shellcmd{createbem custom.adf}

This should generate \file{custom.bem} which is a description on how
instructions should be encoded in the architecture.

You can print the encoding info in a more human readable format with:

\shellcmd{viewbem custom.bem}

The program should print details on how each type of source/destination, etc.
is encoded. You do not have to care about this, it is just for curiosity.
Maybe the only important part in the printout is the length of the
instruction word. In this case, even though the processor has only
one move slot, the instruction word width is more than 40 bits. Memorize the
instruction word width in bits. The size can be reduced with instruction
compression, which is not used in the tutorial.
% TODO: add the instruction compression part here also (TEEMU)

Next, we must set correct instruction memory MAU and then 
select implementations for all components in the architecture. Each
architecture component can be implemented in multiple ways, so we must
choose one implementation for each component in order to be able to
generate the implementation for the processor.

This can be done in the ProDe tool:

\shellcmd{prode custom.adf}

Now the architecture is loaded. First select change the MAU of instruction
memory \textit{Edit>Address Spaces...} to the value you checked earlier in
viewbem. See the FAQ (chapter \ref{chapter:faq}) for more information about
instruction memory.

You should also change the widths of GCU ports and LSU trigger port to avoid
error messages from the Processor Generator. So open the Address Spaces
dialog and check the bitwidth fields. Then set the GCU port widths according
the instruction memorys bitwidth field. Next change the LSU trigger port's
width. Set it to the same value as the data memory's bitwidth.

Create the binary encoding map again just in case the port width changes
caused some changes in encoding:

\shellcmd{createbem custom.adf}

Then we'll select implementations for the FUs which can be done in
\textit{Tools>Processor Implementation...}. Note that the selection window
is not currently very informative about the different implementations, so a
safe bet is to select an implementation with parametrizable width/size.

\begin{enumerate}
\item%
Select implementation for RF: Click the RF name, 'Select RF implementation',
find the TCE's default HDB file from your tce installation path 
(PREFIX/share/tce/hdb/asic\_130nm\_1.5V.hdb) and select an
implementation for the RF from there.

\item%
Next select implementation for the boolean RF like above. But this time
select an implementation which is guarded i.e. select an implementation which
has word ``guarded\_0'' in its name.

\item%
Similarly, select implementations for the function units from TCE's default
HDB. Then select implementation for the custom\_adder but this time you have to
use the \file{tour.hdb} created earlier to find the FU we added that supports the
REFLECT custom operation.

\item%
Finally select the IC/Decoder generator plugin used to generate the
decoder in the control unit and interconnection network:
\textit{Browse... (installation\_path)/share/tce/icdecoder\_plugins/base/
DefaultICDecoderPlugin.so>OK}.
You do not have to care about the HDB file text box because we are not going 
to use cost estimation data.
\end{enumerate}

\paragraph{Generate the VHDL for the processor using Processor Generator
(ProGe).} 

You can start processor generation from the implementation selection dialog:
Click ``Generate Processor''. For Binary Encoding Map: Select the
previously generated .bem (Load from file...). Create and set target directory
\file{proge-output}. Or alternatively, save the .idf file and execute ProGe from
command line:

\shellcmd{generateprocessor -b custom.bem -i custom.idf custom.adf}

Now directory \file{proge-output} includes the VHDL implementation of the
designed processor. You can take a look what the directory includes, how the RF
and FU implementations are collected up under \file{vhdl} subdir and 
the interconnection network has been generated to connect the units (the \file{gcu\_ic}
subdir).

\paragraph{Generate instruction memory bit image using Program Image Generator.}

Finally, to get our shiny new processor some bits to chew on,
we use generatebits to create an instruction memory image:

\shellcmd{generatebits -b custom.bem -d -w 4 -p crc\_with\_custom\_op.tpef
custom.adf} 

Now the file \file{crc\_with\_custom\_op.img} includes the instruction memory
image in ``ascii 0/1'' format. Each line in that file represents a single
instruction. Thus, you can get the count of instructions by counting the
lines in that file:

\begin{verbatim}
 wc -l crc_with_custom_op.img
\end{verbatim}

By multiplying the count with the instruction width we found earlier you
can get the required size of instruction memory for the program.

\subsection{Final Words}

This tutorial is now finished. Now you should know
how to make and use your own custom operations and generate the
processor implementation along with its instruction memory bit image.

If given a functional testbench you could simulate the generated 
VHDL implementation of the processor with a VHDL
simulator. You could also synthesize the processor implementation with
appropriate application. There are some scripts in the produced
\file{proge-output} directory you can use as a starting point for synthesis
and VHDL simulation.

You can go to the proge\_output directory and compile the testbench with if
you have GHDL installed in your system. After that you must move the data
images to the tb directory:

\shellcmd{cd proge\_output}

\shellcmd{./ghdl\_compile.sh} 

\shellcmd{cp ../crc\_with\_custom\_op\_data.img tb/dmem\_init.img}

\shellcmd{cp ../crc\_with\_custom\_op.img tb/imem\_init.img}

Then you can try and simulate the design:

\shellcmd{./ghdl\_simulate.sh}

Please notice that GHDL is a work in progress VHDL compiler/simulator so it
is recommended to use some other VHDL tool.

If you want a more detailed information about what happens in the testbench
simulation you can enable bustracing from the Processor
Implementation-dialog's IC / Decoder Plugin tab in ProDE. To enable this, set
the bustrace plugin parameter to ``yes'' and the bustracestartingcycle to
``1''. Then save the .idf and generate the processor again. Notice that you'll
have to remove the existing proge-output dircetory or generate the new
processor in another directory. Now the testbench will create a file called
\file{bus.dump} which holds the values of every bus on every cycle.

In this tutorial we used a ``minimalistic'' processor architecture as our
starting point. The machine had only one transport bus and 5 registers so it
could not fully exploit the parallel capabilities of TTA. As mentioned 
earlier, if you are interested you can open the custom.adf in ProDe and 
add more transport buses, register files and function units and see how 
they affect the performance.

\section{From C to VHDL as Quickly as Possible}
\label{section:fromCtoVHDL}
% NOTE: when updating this update also c2vhdl systemtest if needed.
% TODO: Also add compilation and simulation of VHDL code using GHDL
 
This tutorial introduces a fast way to generate a processor VHDL model
from C source code using the Design Space Explorer.
 
If you haven't already downloaded the tutorial file package, you can get it 
from:\\
\url{http://tce.cs.tut.fi/tutorial_files/tce_tutorials.tar.gz} \\
Unpack it to a working directory and then cd to tce\_tutorials/c2vhdl/

\subsection{Compile C code}

% compile C code.
Compile the application C code. It is important that the name of the compiled
code is ``program.bc''.
 
\textit{tcecc application1/complex\_multiply.c -O2 -o application1/program.bc}
 
\subsection{Use explorer}

% use minimal.adf as a initial machine
Use minimal.adf (data/mach/minimal.adf) as a starting point machine and store
it to a design space exploration database to be used by the Design Space
Explorer. At the same time data directory can be assigned for the database
from where, for example, the compiled program is found and used by the Explorer.

\textit{explore -a minimal.adf -d application1 ExpRes.dsdb}

% use the simple ic optimizer
Next you can try to optimize the initial configuration with the
SimpleICOptimizer explorer plugin. The plugin looks which connections are
needed by the program and removes all others and with default parameters
removes also components that are left unconnected with the
RemoveUnconnectedComponents plugin (plugins sometimes call other plugins).
 
\textit{explore -e SimpleICOptimizer -s 1 ExpRes.dsdb}
 
Where \textit{-s 1} is the id of the configuration to be optimized.
 
If the plugin found something to optimize, it adds a new configuration(s) into
the database and reports the final configuration(s) as ``Best result
configurations ids'' in the database.
 
\subsection{Generate required files}
 
The created configurations can be written as file with command:
 
\textit{explore -w <configuration\_ID> ExpRes.dsdb}
 
If the optimization was successful you should export the optimized version,
otherwise use the original version.
 
Now the ADF and IDF files are written into the current directory and are named
as ``1.adf'' and ``1.idf'' or ``2.adf'' and ``2.idf'' depending on which
version you wrote out.

If implementation definition file ``idf'' is not generated automatically
by the plugin you can generate it with the ImplementationSelector plugin.
Remember to write out the new configuration as explained above.

\textit{explore -e ImplementationSelector -s <configuration\_ID> ExpRes.dsdb}
 
% Schedule
Schedule each program against the new architecture:

(replace <id> with the proper number)
 
\textit{tcecc -a <id>.adf -o application1/complex\_multiply.scheduled.tpef
application1/program.bc}
 
This step creates file ``ApplicationDir/complex\_multiply.scheduled.tpef''.
 
% create bem
Now create an encoding for the instructions in our new architecture:
 
\textit{createbem <id>.adf}
 
This step creates file <id>.bem.
Now you should have all the needed parts for the processor VHDL.
 
\subsection{Generate processor VHDL implementation}
 
% start ProGe
Use ProGe to generate the processor VHDL:
 
\textit{generateprocessor -b <id>.bem -i <id>.idf <id>.adf}
 
Now directory called ``proge-output'' should include the VHDL implementation
of the processor.
 
% start PIG
Generate instruction memory bit image using PIG. First cd to application1
directory and then generate the bit image:

\textit{generatebits -b ../<id>.bem -d -p complex\_multiply.scheduled.tpef
../<id>.adf}

This generates files ``Application.scheduled.img'' into the application
directory.

Tutorial is now finished and you can simulate the generated VHDL
implementation of the processor with a VHDL simulator and synthesize it.


%%%%%%%%%%%%%%%%%%%%%%%%%%%%%%%%%%%%%%%%%%%%%%%%%%%%
% THESE ARE TODO's
%%%%%%%%%%%%%%%%%%%%%%%%%%%%%%%%%%%%%%%%%%%%%%%%%%%%
%
%\section{Manual Exploration}
%\label{section:manualExpl}
% Modify the arch manually until it meets the requirements

%\section{Automated Design Space Exploration}
%\label{section:automatedExploration}
% Add requirements and sequential programs, 
% Click, click, *wait a couple of weeks*, done! ;-)


\section{Hello TTA World!}

What would a tutorial be without the traditional ``Hello World!''
example? Interestingly enough, printing out ``Hello World'' in a
standalone (operating system free) platform like the TTA of TCE is not 
totally straightforward. That is the reason this tutorial is not the first 
one in this tutorial chapter.

The first question is: where should I print the string? Naturally 
it is easy to answer that question while working with the simulator: to
the simulator console, of course. However, when implementing the final
hardware of the TTA, the output is platform dependent. It can be a
debugging interface, serial port output, etc.

In order to make printing data easier from TTAs designed
with TCE, we have included an operation called STDOUT in the ``base operation 
set'' that is installed with the TCE. The simulation behavior of the 
operation reads its input, expects it to be a 8bit char and writes the char
verbatim to the simulator host's standard output. Of course, in case the 
designer wants to use the operation in his final system he must provide the 
platform specific implementation for the function unit (FU) implementing the 
operation, or just remove the FU after the design is fully debugged and
verified.

The default implementation of \textit{printf()} in TCE uses the STDOUT
operation to print out data. Therefore, implementing a ``Hello World''
application with TCE is as simple as adding an FU that implements
the STDOUT to the processor design and then calling \textit{printf()} in
the simulated program. The simulator should print out the greeting
as expected.

Here is a simple C code (hello.c) that prints the magic string:

\begin{verbatim}
#include <stdio.h>

int main() {
    printf("Hello TTA World!");
    return 0;
}
\end{verbatim}

Next, compile it to an architecture that implements the
STDOUT. In this case we add an FU with STDOUT (see the previous
tutorial for instructions on adding FUs to your designs) to the minimal.adf, 
after copying it to minimalWithStdout.adf:

\begin{verbatim}
cp $(tce-config --prefix)/share/tce/data/mach/minimal.adf minimalWithStdout.adf
prode minimalWithStdout.adf &
... add the IO function unit with the STDOUT operation to the ADF ...
tcecc -O0 hello.c -a minimalWithStdout.adf -o hello.tpef
\end{verbatim}

It should compile without errors. Beware: the compilation can take
a couple of minutes on a slower machine! This is because \textit{printf()} 
is actually quite a large function and the compiler is not yet optimized for 
speed.

Finally, simulate the program to get the greeting:

\begin{verbatim}
ttasim -a minimalWithStdout.adf -p hello.tpef --no-debugmode
Hello TTA World!
\end{verbatim}

That's it. Happy debugging!


\section{Streaming I/O}
\label{section:streamIO}

Because TTA/TCE is an environment without operating system, there is also
no file system available for implementing file-based I/O. Therefore, one
popular way to get input and output to/from the TTA is using shared memory
for communicating the data. For stream processing type of applications,
one can also use an I/O function unit that implements the required 
operations for streaming.

TCE ships with example operations for implementing stream type input/output. 
These operations can be used to read and write samples from streams in the
designed TTA processor. The basic interface of the operations allows reading and 
writing samples from the streams and querying the status of an input or output stream 
(buffer full/empty, etc.). The status operations are provided to
allow the software running in the TTA to do something useful while the
buffer is empty or full, for example switch to another thread. Otherwise,
in case one tries to read/write a sample from/to a stream whose buffer is 
empty/full, the TTA is locked and the cycles until the situation resolves
are wasted.

The example streaming operations in the base operation set are 
called STREAM\_IN, STREAM\_OUT, STREAM\_IN\_STATUS, and STREAM\_OUT\_STATUS.
These operations have a simulation behavior definition which simulates the
stream I/O by reading/writing from/to files stored in the file system of 
the simulator host. The files are called \textit{ttasim\_stream.in} for 
the input stream and \textit{ttasim\_stream.out} for the output stream and
should reside in the directory where the simulator is started. 
The file names can be customized using environment variables 
TTASIM\_STREAM\_OUT\_FILE and TTASIM\_STREAM\_IN\_FILE, respectively.

Here is an example C code that implements streaming I/O with the operations:

\begin{verbatim}
#include "tceops.h"

int main()
{
    char byte;
    int status;

    while (1)
    {
        _TCE_STREAM_IN_STATUS(0, status);

        if (status == 0)
            break;

        _TCE_STREAM_IN(0, byte);
        _TCE_STREAM_OUT(byte);
    }

    return 0;
}
\end{verbatim}

This code uses the TCE operation invocation macros from \textit{tceops.h}
to read bytes from the input stream and write the same bytes to the
output stream until there is no more data left. This situation is indicated
with the status code 0 queried with the STREAM\_IN\_STATUS operation. The
value means the stream buffer is empty, which means the file simulating the
input buffer has reached the end of file.

You can test the code by creating a file \textit{ttasim\_stream.in}
with some test data. The code should create a copy of that file to the
stream output file \textit{ttasim\_stream.out}.

\section{Implementing Programs in Parallel Assembly Code}
\label{section:parallelAssemblyCoding}
% TODO: add example of implementing a sort RISC assembly code as
% a parallel TTA program. How to compile it (refer to assembler section) and
% how to simulate and estimate it, etc.

This tutorial will introduce you to TTA assembly programming. It is
recommended that you go through this tutorial because it will certainly
familiarize you with TTA architecture and how TTA works.

\subsection{Preparations}

For the tutorial you need to download file package from
\url{http://tce.cs.tut.fi/tutorial_files/tce_tutorials.tar.gz} and unpack it
to a working directory. Then cd to parallel\_assembly-directory.

The first thing to do is to compile the custom operation set called cos16 shipped
within the parallel\_assembly-directory. The easiest way to do this is:

\shellcmd{buildopset cos16}

This should create a file named \file{cos16.opb} in the directory.

\subsection{Introduction to DCT}

Now you will be introduced to TCE assembler language and assembler usage.
Your task is to write TCE assembly code for 2-Dimensional 8 times 8 point 
Discrete Cosine Transform (DCT\_8x8).
First take a look at the C code of DCT\_8x8
\file{dct\_8x8\_16\_bit\_with\_sfus.c}.  
The code is written to support fixed point datatype with sign plus 15 fragment
bits, which means coverage from $-1\ to\ 1-2^{15}$. The fixed point
multiplier, function \textit{mul\_16\_fix}, and fixed point adder, function
\textit{add\_16\_fix},  used in the code scale inputs automatically to prevent
overflow. Function \textit{cos16} takes $x(2i+1)$ as input and returns
the corresponding cosine value $\frac{cos(x(2i+1)\pi)}{16}$. The code calculates
following equations:

\begin{equation}
F(x) = \frac{C(x)}{2}\sum_{i = 0}^7 \left[f(i)cos\left(\frac{x
\left(2i+1\right)\pi}{16}\right)\right] \nonumber
\end{equation}
\begin{equation}
F(y) = \frac{C(y)}{2}\sum_{i = 0}^7 \left[f(i)cos\left(\frac{y
\left(2i+1\right)\pi}{16}\right)\right] \nonumber
\end{equation}

\begin{equation}
C(i) = 
\left\{
\begin{array}{c c}
\frac{2}{\sqrt2} &,  i = 0 \\
1 &,  else 
\end{array} \right . . \nonumber
\end{equation}

\begin{equation}
F(x,y) = F(x)F(y) \nonumber
\end{equation}

\subsection{Introduction to TCE assembly}

First take a look at assembly example in file \file{example.tceasm}
to get familiar with syntax. More help can be found from
section \ref{section:TCEAsm}

Compilation of the example code is done by command:

\shellcmd{tceasm \ -o \ example.tpef \ dct\_8x8\_16\_bit\_with\_sfus.adf \
example.tceasm}

The assembler will give some warnings saying that ``Source is
wider than destination.'' but these can be ignored.

The compiled tceasm code can be simulated with TCE
simulator, ttasim or proxim(GUI).

\shellcmd{ttasim -a \ dct\_8x8\_16\_bit\_with\_sfus.adf -p example.tpef} , or

\shellcmd{proxim \ dct\_8x8\_16\_bit\_with\_sfus.adf example.tpef}

It is recommended to use proxim because it is more illustrating to track the
execution with it. Especially if you open the Machine Window (View -> Machine
Window) and step through the program.

Check the result of example code with command (you can also write this in
proxim's command line at the bottom of the main window):

\shellcmd{x /a IODATA /n 1 /u b 2}. 

the output of x should be 0x40.

\subsection{Implementing DCT on TCE assembly}

Next try to write assembly code which does the same functionality as the
C code. The assembly code must be functional with the given machine
\file{dct\_8x8\_16\_bit\_with\_sfus.adf}. Take a look at the processor by
using prode:

\shellcmd{prode dct\_8x8\_16\_bit\_with\_sfus.adf \&}

The processor's specifications are the following:

\textbf{Supported operations} \\
Operations supported by the machine are: \textit{mul}, \textit{mul\_16\_fix},
\textit{add}, \textit{add\_16\_fix}, \textit{ldq}, \textit{ldw}, \textit{stq},
\textit{stw}, \textit{shr}, \textit{shl}, \textit{eq}, \textit{gt},
\textit{gtu}, \textit{jump} and \textit{immediate transport}.

When you program using TTA assembly you need to take into account 
operation latencies. The \textit{jump} latency is four clock cycles and load
latencies (\textit{ldq} and \textit{ldw}) are three cycles. Latency for
multiplications (\textit{mul} and \textit{mul\_16\_fix}) are two clock cycles.

\textbf{Address spaces} \\
The machine has two separate address spaces, one for data and another for
instructions. The data memory is 16-bit containing 128 memory slots and the
MAU of data memory is 16-bits. The intruction memory has 1024 memory slots
which means that the maximum number of instructions of 1024.

\textbf{Register files} \\
The machine contains 4 register files, each of which have 4 16-bit registers,
leading to total of 16 16-bit registers.  The first register file
has 2 read ports.

\textbf{Transport buses} \\
The machine has 3 16-bit buses, which means maximum of 3 concurrent
transports. Each bus can contain a 8-bit short immediate.

\textbf{Immediates} \\
Because the transport buses can only contain 8-bit short immediates you must
use the immediate unit if you want to use longer immediates. The immediate unit
can hold a 16-bit immediate. There is an example of immediate unit usage in
file \file{immediate\_example.tceasm}. Basically you need to transfer the value to the
immediate register. The value of immediate register can be read on the next
cycle.

The initial input data is written to memory locations 0-63 in the file
\file{assembler\_tutorial.tceasm}. Write your assembly code in that file.

\subsubsection{Verifying the assembly program}

The reference output is given in \file{reference\_output}. You need to
compare your assembly program's simulation result to the reference
output. Comparision can be done by first dumping the memory contents in the
TCE simulator with following command:

\shellcmd{x /a IODATA /n 64 /u b 0}

The command assumes that output data is stored to memory locations 0-63.

The easiest way to dump the memory into a text file is to execute ttasim with
the following command:

\shellcmd{ttasim -a dct\_8x8\_16\_bit\_with\_sfus.adf -p
assembler\_tutorial.tpef < input\_command.txt > dump.txt}

After this you should use sed to divide the memory dump into separete lines
to help comparison between your output and the reference output. Use the
following command to do this (there is an empty space between the
first two slashes of the sed expression):

\shellcmd{cat dump.txt | sed 's/ /$\backslash$n/g' > output.txt}

And then compare the result with reference:

\shellcmd{diff -u output.txt reference\_output}

When the TCE simulator memory dump is the same as the reference output your
assembly code works and you have completed this tutorial. Of
cource you might wish to improve your assembly code to minimize cycle
count or/and instruction count.

If it is too hard to visualize the whole program in parallel assembly
you can start by writing sequential code and then write it to
parallel assembly.

You should also compile the C program and run it because it gives more
detailed information which can be used as reference data if you need to debug
your assembly code.

To compile the C code, enter:

\shellcmd{gcc -o c\_version dct\_8x8\_16\_bit\_with\_sfus.c}

If you want the program to print its output to a text file, you can use the
following command:

\shellcmd{./c-version > output.txt}


To get some idea of the performance possibilities of the machine,
one assembly code has 52 instructions and it runs the DCT8x8 in
3298 cycles.



\chapter{PROCESSOR DESIGN TOOLS}
\label{chapter:procgen}

\section{TTA Processor Designer (ProDe)}
\label{sec:prode}

Processor Designer (\textbf{ProDe}) is a graphical application mainly for
viewing, editing and printing processor architecture definition files. 
It also allows selecting implementation for each component of the processor,
and generating the HDL implementation of the processor. The application is
very easy to use and intuitive, thus this section provides help only for the
most common problematic situations encountered while using the toolset.

\textbf{Input}: ADF

\textbf{Output}: ADF, VHDL

The main difficulty in using the tool is to understand what is being
designed, that is, the limitations placed by the processor template. 
Details of the processor template are described in \cite{ADF-specs}.

\subsubsection{Usage}

Processor Desinger can simply be executed from command line with:

\shellcmd{prode}


\section{Operation Set Abstraction Layer (OSAL) Tools}
\label{sec:osalTools}

 
\textbf{Input}: OSAL definitions 

\textbf{Output}: OSAL definitions

\subsection{Operation Set Editor (OSEd)}
\label{sec:osed}

Operation Set Editor (\textbf{OSEd}) is a graphical application for managing
the OSAL (Section~\ref{section:osal}) operation database. OSEd makes it possible
to add, simulate, edit and delete operation definitions. 

\subsubsection{Capabilities of the OSEd}

OSEd is capable of the following operations:

\begin{enumerate}
\item%
  All operations found in pre-defined search paths (see Section
  ~\ref{sec:osalpaths}) are organised in a tree-like structure which can be
  browsed.
\item%
  Operation properties can be examined and edited.
\item%
  Operations with a valid behavior model can be tested (simulated).
\item%
  New operation modules can be added to search paths.
\item%
  Operation definitions can be added to a module.
\item%
  Modules containing operation behaviors can be compiled, either all at
  once, or separately.
\item%
  Modules can be removed.
\item%
  Contents of the memory can be viewed and edited.
\end{enumerate}

\subsubsection{Usage}

This chapter introduces the reader to the usage of OSEd. Instructions to
accomplish the common tasks are given in detail.

Operation Set Editor can simply be executed from command line with:

\shellcmd{osed}

\begin{figure}[tb]
\centerline{\psfig{figure=eps/osed/MainWindow.eps,width=0.75\textwidth}}
\caption{OSEd Main window.}
\label{fig:osed_main_window}
\end{figure}

The main window is split in two areas. The left area always displays a
tree-like structure consisting of search paths for operation definition
modules, operation modules, and operations.  The right area view depends on the
type of the item that is currently selected in the left area. Three cases are
possible.
%
\begin{enumerate}
\item %
  If the selected item is a search path, the right area shows all operation
modules in that path.
\item %
  If the item is a module, the right area shows all the operations defined
  in the module.
\item %
  If the item is an operation, the right area displays all the properties of
  the operation.
\end{enumerate}

Figure~\ref{fig:osed_main_window} shows an example of the second situation, in
which the item currently selected is a module. The right area of the window
shows all the operations in that module. If an operation name is shown in
bold text, it means that the operation definition is ``effective'', that is,
it will actually be used if clients of OSAL request an operation with that
name. An operation with a given name is effective when it is the first
operation with that name to be found in the search paths. Other operations
with the same name may be found in paths with lower search priority. Those
operations are not effective.

Figure~\ref{fig:osed_operation_property_view} shows an example of an operation 
property view, that is shown in the right side when an operation is selected 
on the left side.

\begin{figure}[tb]
\centerline{\psfig{figure=eps/osed/StaticOperationProperties.eps,
width=0.50\textwidth}}
\caption{Operation property view.}
\label{fig:osed_operation_property_view}
\end{figure}

\paragraph{Editing Static Operation Properties}

Figure~\ref{fig:osed_operation_window} shows the dialog for editing the
static properties of an operation. 
 
\begin{figure}[tb]
\centerline{\psfig{figure=eps/osed/OperationProperties.eps,
width=0.70\textwidth}}
\caption{Operation property window}
\label{fig:osed_operation_window}
\end{figure}

Operation inputs and outputs (henceforth, ``terminal'' is used to denote both)
can be deleted by selecting an item from the list and clicking the \emph{Delete}
button. New terminals can be added
by clicking the \emph{Add} button, and can be modified by clicking the \emph{Modify}
button. The order of the terminals can be changed by selecting a terminal
and pushing on of the arrow buttons. By pushing the downward arrow button, the
terminal is moved one step downwards in the list; by pushing the upward arrow
button, it is moved one step up on the list.

Operand properties can be modified by clicking on the check boxes. The set
of operands that can be swapped with the operand being edited are shown as a
list of references to operands (input indentification numbers).  A reference
to an operand can be removed by selecting the corresponding item in the `can
swap' list and clicking the \emph{Delete} button.  A new reference to another
operand can be added by selecting an item from the choice list and clicking
the \emph{Add} button.

\paragraph{Operation Behaviour Model}

Behaviour models of operations are stored in separate source files.  If the
operation definition includes a behaviour model, the behaviour source file
can be opened in a text editor of choice by clicking on the \emph{Open}
button. If the operation does not have behavior source file, clicking
\emph{Open} will open an empty file in an editor. The text editor to use can
be defined in the options dialog. All changes to operation properties are 
committed by clicking the \emph{OK} button and canceled by clicking the \emph{Cancel} 
button.

\paragraph{Operation Directed Acyclic Graph}

By treating each operation as a node and each input-output pair as an directed
arc, it is possible to construct operation's Directed Acyclic Graph (DAG)
presentation. For primitive operations which do not call any other operations,
this graph is trivial; one node (operation itself) with input arcs from root
nodes and output arcs to leafs. With OSAL DAG language, it is possible to
define operation behavior model by composing it from multiple operations'
respective models.

Operation's OSAL DAG code sections can be edited by pressing the \emph{Open DAG}
button, which opens the DAG editor window. Code section shows the currently
selected DAG code from the list box below. A new DAG section can be created
either by selecting \emph{New DAG} list box item or pressing the \emph{New} button.
By pressing the \emph{Undo} button, it is possible to revert changes to current
code from the last saved position. DAG can be saved to operation's definition
file by pressing the \emph{Save} button. Unneccessary DAG sections can be deleted
by pressing the \emph{Delete} button.

If code section contains valid OSAL DAG code, then the editor window shows
a DAG presentation of that code.  In order to view the graph, a
program called \file{dot} must be installed. This is included in Graphviz graph
visualization software package and can be obtained from www.graphviz.org. 

\paragraph{Operation Modules}

Figure~\ref{fig:osed_add_module_dialog} shows the dialog for adding a new
operation module to a search path. The name of the module can be entered
into the text input field.  

Operation modules may consist also of a behaviour source file. Before
operation behaviour modules can be simulated, it is necessary to compile the
source file.  Figure~\ref{fig:osed_build_result_dialog} shows a result dialog of
module compilation. 

%
\begin{figure}[tb]
\centerline{\psfig{figure=eps/osed/AddModule.eps,width=0.32\textwidth}}
\caption{Dialog for adding new operation module.}
\label{fig:osed_add_module_dialog}
\end{figure}
%
\begin{figure}[tb]
\centerline{\psfig{figure=eps/osed/BuildResult.eps,width=0.55\textwidth}}
\caption{Result of module compilation}
\label{fig:osed_build_result_dialog}
\end{figure}

\paragraph{Data Memory Simulation Model}

The contents of the data memory simulation model used to simulate
memory accessing operations can be viewed and edited.
Figure~\ref{fig:osed_memory_dialog}
shows the memory window. Memory can be viewed as 1, 2, 4, or 8 MAUs.
The format of the data is either in binary, hexadecimal, signed integer,
unsigned integer, float, or double format.
The contents of the memory can be changed by double clicking a memory cell. 

\begin{figure}[tb]
\centerline{\psfig{figure=eps/osed/Memory.eps,width=0.60\textwidth}}
\caption{Memory window}
\label{fig:osed_memory_dialog}
\end{figure}

\paragraph{Simulating Operation Behavior}

The behavior of operation can be simulated using the dialog in 
Figure~\ref{fig:osed_simulate_dialog}. Input values can be edited by selecting
a input from the input list and typing the new value in a text field below
the input list. Change can be committed by pushing \emph{Update} button.
Trigger command and advance clock command are executed 
by pushing \emph{Trigger} and \emph{Advance clock} buttons.
The format of the inputs and outputs can be modified by selecting a new format
from the choice list above the \emph{Trigger} button.

\begin{figure}[tb]
\centerline{\psfig{figure=eps/osed/Simulate.eps,width=0.85\textwidth}}
\caption{Operation Simulation}
\label{fig:osed_simulate_dialog}
\end{figure}

\subsection{Operation Behavior Module Builder (buildopset)}
\label{sec:buildopset}

The OSAL Builder is an external application that simplifies the process of
compiling and installing new (user-defined) operations into the OSAL system.

The OSAL Builder is invoked with the following command line:

\shellcmd{buildopset <options> operation\_module} 

where \emph{operation\_module} is the name of the operation module. 
Operation module is the base name of a definition file, e.g., the module
name of \emph{base.opp} is `base'. The \emph{operation\_module} can also be
a full path, e.g., `/home/jack/.tce/opset/custom/mpeg'.

The behavior definition source file is searched in the directory of the
\emph{operation\_module}. The directory of the \emph{operation\_module} is
by 
default the current working directory. User may also enter the directory of
the source file explicitly with switch `-s'. The suffix of the
behavior definition source file is `.cc'.
%
If no options are given, the output file is a dynamic module and is stored
in the directory of the \emph{operation\_module}.  

The OSAL Builder accepts the following command line options:

\begin{center}
\begin{longtable}[htb]{@{}p{.10\textwidth}@{}p{.20\textwidth}@{}p{.15\textwidth}%
                     @{}p{.65\textwidth}}

\textbf{Short Name} &\textbf{Long Name} &\textbf{Description} \\
\hline

k & \verb|install|        & \emph{keyword} &

Installs the data file and the built dynamic module into one of the allowed
paths. The paths are identified by the following keywords: 
\emph{base}, \emph{custom}, \emph{user}.\\

b & \verb|ignore|         & \emph{boolean} &

Ignores the case whereby the source file containing operation behavior model
code are not found.  By default, the OSAL Builder aborts if it cannot build
the dynamic module.  This option may be used in combination with
\emph{install} option to install XML data files before operation behavior
definitions are available.\\

s & \verb|source-dir|         & \emph{directory} &

Enter explicit directory where the behavior definition source file to
be used is found.

\end{longtable}
\end{center}

\subsection{OSAL Tester (testosal)}
\label{sec:testosal}

The OSAL Tester is a small external application meant for debugging 
operation behavior models.  The Tester lets user to specify lists of
operations and constant input values and the outputs the corresponding sequence
of result values.

The OSAL Tester can be run in interactive mode or in batch mode (like a
script interpreter).  The batch mode is especially useful to create input
data sets of regression tests.

For all operations having state, a single operation state instance is
constructed the first time the operation is simulated. This state instance
is then used through the testing session by all operations that share the
same operation state.

When started, the OSAL Tester enters interactive mode and prompts the user
for operations. Any nonempty line entered by the user is interpreted as one
operation, unless the character `!' is given at the beginning of the string.
This character (which is not allowed in operation names) introduces
\emph{meta-commands} that are recognized by the interpreter and are not
treated as potential operations in the architecture repertoire. For example, the
OSAL Tester can be quit with the meta-command \textit{!quit}.

Single operands of operations may only be constants and are separated by
blanks.

For any line entered by the user, the Tester responds with a text line
containing the results computed by the operation.

\section{OSAL search paths}
\label{sec:osalpaths}

Default paths, where OSAL operations are seached, are the following: \\
(in descending search order)

\begin{enumerate}
\item
   \textit{\$PWD/data/} \\ where \$PWD is your current working directory

\item
   \textit{TCE\_SRC\_ROOT/opset/base/} \\ where TCE\_SRC\_ROOT/ is the
   TCE source code directory.

\item
   Default predefined and standard operations: \\
   \textit{TCE\_INSTALLATION\_DIR/opset/base/} \\ where
   TCE\_INSTALLATION\_DIR is the path where TCE accessories is installed (for
   example /usr/local/share/tce).

\item
   Users local custom operations: \\
   \textit{\$HOME/.tce/opset/custom/} \\ where \$HOME is users home directory.

\item
   System-wide shared custom operations:
   \textit{TCE\_INSTALLATION\_DIR/opset/base/} \\ where
   TCE\_INSTALLATION\_DIR is the path where TCE accessories is installed (for
   example /usr/local/share/tce).

\item
   Operations in current working directory:\\
   \textit{\$PWD/}

\end{enumerate}

\textbf{NOTE!} Search paths 1 and 2 are not used in Distributed versions!

Search paths are in descending search order meaning that operations are first
searched from first defined paths. If an operation exists in multiple search
paths, the last found version is used.


\section{Processor Generator (ProGe)}
\label{sec:proge}

Processor Generator (\textbf{ProGe}) produces a synthesizable hardware
description of a TTA target processor specified by an architecture
definition file (Section~\ref{sec:adf}) and implementation definition file
(Section~\ref{sec:idf}).

\textbf{Input}: HDB, ADF, IDF

\textbf{Output}: VHDL implementation of the processor

There is a command line client for executing this functionality, but the
functionality can also be used in the Processor Designer
(Section~\ref{sec:prode}). This section is a manual for the command line
client.

Processor generation can be customized with plugin modules. The customizable
parts are the generation of control unit and the interconnection network.

The CLI-based version of the processor generator is invoked by means of a
command line with the following syntax:

  \textit{generateprocessor <options> target}

The sole, mandatory argument \emph{target} gives the name of the file that
contains the input data necessary to generate the target processor. The file
can be either a processor configuration file or an architecture definition
file. If the file specified is a PCF, then the names of ADF, BEM and IDF are
defined in it.

The given PCF may be incomplete; its only mandatory entry is the reference
to the ADF that defines the architecture of the target processor.  If the
file specified is an ADF, or if PCF does not contain BEM and IDF references,
then a BEM is generated automatically and the default implementation of every
building block that has multiple implementations is used.

The processor architecture must be synthesizable, otherwise an error message
is given. \\

\begin{center}
\begin{longtable}[htb]{@{}p{.10\textwidth}@{}p{.20\textwidth}%
                     @{}p{.65\textwidth}}

\textbf{Short Name} &\textbf{Long Name} &\textbf{Description} \\
\hline

l & \verb|hdl| &
Specifies the HDL of the top-level file which wires together the blocks
taken from HDB with IC and GCU.\footnote{In the initial version, the only
keyword accepted is `vhdl'.} \\

g & \verb|gen| & %
Specifies the IC/Decoder generator. It is name of an executable module
linked at run time (plug-in) that generates the implementation of the
processor interconnection network, the instruction decoder of the control
unit.
%
The plug-in file is searched in the paths described in
Section~\ref{sec:plugin-generators}.
%
This option overrides the plug-in specified in IDF, if present. If neither
IDF nor the command line give the plug-in, the default plug-in defined in
the ProGe configuration file is used.
\\
d & \verb|decomp| &%
Specifies the file that contains the decompressor module of the
GCU. Normally, the file is generated by the compressor plugin of
Program Image Generator (PIG).
\\
o & \verb|output| &%
Name of the output directory. If not given, an output directory called
`proge-output' is created inside the current working directory.
\\
u & \verb|plugin-parameter| &%
Shows the parameters accepted by an IC/Decoder generator plug-in and a short
description of the plug-in. When this options are given, any other type of
option is ignored and ProGe returns immediately without generating any output
file or directory.  Even though other options are ignored they must
be valid.
\\
\end{longtable}
\end{center}


\subsubsection{IC/Decoder Generators}

IC/Decoder generators are implemented as external plug-in
modules. This enables users to provide customizable instruction
decoder and IC implementations without recompiling or modifying the
existing code base. One can easily add different plug-ins to
experiment with different implementation alternatives, and as easily
switch from one to another plug-in. To create a new plug-in, a
software module that implements a certain interface must be created,
compiled to a shared object file and copied to appropriate
directory. Then it can be given as command line parameter to the
generator.

\section{Hardware Database Editor (HDB Editor)}
\label{sec:hdbedit}
% TODO: some text, describe at least the most important use cases [VP]
% describe createhdb
% Hardware Database (HDB) is the main database used by the
% Processor Generator and the Cost Estimator.  The data stored in HDB consist
% of hardware description language definitions (HDL) of TTA components
% (function units, register files, buses and sockets) and ``meta-data'' that
% describe certain parameters used in implementations. In addition, HDB may
% include data of each implementation needed by the cost estimation
% algorithms.
% TCE ships with an example HDB that includes implementations for several
% function units and register files and cost data for the default
% interpolating cost estimation plugin.
 

HDB Editor (hdbeditor) is a graphical frontend for creating and
modifying Hardware Databases i.e. HDB files (see Section~\ref{section:hdb}
for details). By default, all the example HDB files are stored in the directory
hdb/ of the TCE installation directory.

\subsection{Usage}

This section is intended to familiarize the reader to basic usage of the HDB
Editor.

HDB editor can be launched from command line by entering:

\shellcmd{hdbeditor}

You can also give a .hdb-file as parameter for the hdbeditor:

\shellcmd{hdbeditor customHardware.hdb}

\subsubsection{Creating a new HDB file}

Choose ``File'' | ``Create HDB...''. 
From there, type a name for your .hdb file and save it in the default HDB path
(tce/hdb).

After that, you can start adding new TTA components such as function units,
register files, buses and sockets from ``Edit'' | ``Add''.

\subsubsection{Adding new components}
A new function unit's architecture can only be added through an existing ADF
file unlike register files, which can only be added by hand. The ADF files
can be done in the ProDe tool. After adding a new architecture, one can add an
implementation for it by right-clicking on it and choosing ``Add
implementation''

The architecture implementation can be given either by hand or by a VHDL file.

After setting up the architecture, one can add new entries (function units,
register files, buses, sockets) for the architectures.

\subsubsection{Adding FU/RF HDL source files}
HDL files of Function Unit and Register File implementations must be added in
right compilation order i.e. the source file which needs to be compiled first
is first in the list and so on.

\begin{figure}[tb]
\centerline{\psfig{figure=eps/hdbeditor/MainWindow.eps,width=0.66\textwidth}}
\caption{HDB Editor Main window.}
\label{fig:hdbeditor_main_window}
\end{figure}


\section{Function Unit Interface}

Function unit interfaces follow a certain de facto standard in TCE. Here is
an example of a such interface:

\begin{verbatim}
entity fu_add_sub_eq_gt_always_1 is
  generic (
    dataw : integer := 32;
    busw  : integer := 32);
  port (
    -- trigger port / operand2 input port
    t1data   : in std_logic_vector (dataw-1 downto 0);
    t1opcode : in std_logic_vector (1 downto 0);
    t1load   : in std_logic;
    -- operand1 input port
    o1data : in std_logic_vector (dataw-1 downto 0);
    o1load : in std_logic;
    -- result output port
    r1data : out std_logic_vector (busw-1 downto 0);
    -- control signals
    glock  : in  std_logic;
    rstx   : in  std_logic;
    clk    : in  std_logic);
end fu_add_sub_eq_gt_always_1;
\end{verbatim}

As you can see the actual implementation interface has more inputs/outputs
than the architecture model of a FU. Architecture input ports are
called o1data and t1data and the output port is called r1data. In addition
there are load ports, operation code port and some control signals.

It is good practice to use (integer) generics for the port widths in the data
I/O ports. This makes the managing of Hardware Database (HDB) easier because
you can use the name of the generic to define port width formulas instead of
using magic numbers. Whether the value of a generic is constant or
parametrizable depends on how you define it in the HDB. In the example
entity declaration we have used two different generics, data width (dataw)
and bus width (busw). The input data ports depend on the dataw and the
output port is defined using busw. You can also use only one generic but
with 2 generics you can define different widths for input and output data if
needed.

The names you set for the ports in VHDL code is up to you. When you define
an FU implementation in HDB you'll have to do port mapping where you map the
VHDL implementation names to the HDB counterparts. I strongly recommend to
use a consistent naming convention to avoid confusion. In the future,
HDBEditor might include prefilled dialog fields with the defacto standard
names. If you use your own naming style for the ports I still recommend to
use the standard names for control signals which are used in the example
above (glock = Global Lock, rstx = reset, x stands for active low, clk =
clock).

The input ports have also load ports. The load signal is active high when the
input value of an input port is valid and should be read (most probably to a
register inside the FU). If the input port is also a trigger port the
operation should be initiated.

If the FU has multiple operations there is also an operation code port in the
interface. Opcode port is usually bound to the trigger port because operations
have at least one input. And if you wish to initiate an operation with one
input operand, the operand has to be written to the trigger port. Operation
codes also have to be declared in HDB.

FUs can also have ports connecting to the external of the TTA processor
core. They are declared like any other ports in the entity declaration but
these ports must be defined as external in HDB.

See the TCE Tour-tutorial (section \ref{sec:tcetour}) for an example how to
add an FU implementation to hdb.

\subsection{Summary of interface ports}
\subsubsection{Input/Output operand ports}
\verb|VHDL-type: std_logic_vector| \\
TCE naming convention is to use \verb|t1data| for the triggering input port
and \verb|o1data, o2data ...| etc. for the rest of input operands. Output port
is usually called \verb|r1data|.

Use generics when you define widths for these ports.

\subsubsection{Input load ports}
\verb|VHDL-type: std_logic| \\
The load ports are named after the input operand ports. For example load port
of t1data is \verb|t1load| and o1data is \verb|o1load|.

\subsubsection{Operation code port}
\verb|VHDL-type: std_logic_vector| \\
Operation code port is usually called \verb|t1opcode| and it is bound to the
trigger port.

\subsubsection{Control signals}
\verb|VHDL-type: std_logic| \\
There are four control signals available in FUs:

\verb|clk| is the most obvious, it is the clock signal

\verb|rstx| is active low reset (x stands for active low)

\verb|glock| is a Global Lock signal. For example if the global control unit
(GCU) issues global lock the ongoing operation should freeze and wait for
the global lock signal to become inactive before resuming execution.

\verb|glock_r| is global lock request. Rarely used in normal function units.
It can be used to request global lock state, for example, in case the
result does not arrive in the time set by the static latency of the
operation. 

A common example using this signal is a load-store unit of which
memory latency is sometimes larger than the static latency informed to the
compiler due to dynamic behavior of the memory system. In that case, a
global lock is requested after the static number of cycles has passed and
the data has not arrived to the load-store-unit from the memory system.

\subsection{Reserved keywords in generics}

Currently there is only one reserved keyword in generic definitions and it is
\verb|addrw|. This keyword is used in load-store units to define trigger
port's width according to the data address space width. Processor Generator
identifies this keyword and defines the port width from the data address space
assigned to the LSU in the ADF.


\chapter{CODE GENERATION TOOLS}

\section{TCE Compiler}
\label{sec:frontend}

\textbf{TCE compiler} compiles high level language (such as 
C/C++) source files provided by the toolset user and produces a bitcode
program or a parallel TTA program. Bitcode program is a non-architecture
specific sequential program. Parallel TTA program, however, is 
a architecture specific program that is optimized for the target architecture.

The main idea between these two program formats is that you can compile your
source code into bitcode and then compile the bitcode into architecture
dependent parallel code. This way you do not have to compile the source code
again every time you make changes in the architecture. Instead you only need
to compile the bitcode into parallel code.

The frontend compiler uses the LLVM C compiler which again is built on GCC
version 4.

\textbf{Input}: program source file(s) in high-level language

\textbf{Output}: a fully linked TTA program

\subsection{Usage of TCE compiler}

The usage of the \emph{tcecc} application is as follows:

\begin{verbatim}
tcecc <options> source-or-bc-file1 source-or-bc-file2 ...
\end{verbatim}

The possible options of the application are the following:\\
\begin{longtable}[htb]{@{}p{.10\textwidth}@{}p{.20\textwidth}%
                     @{}p{.65\textwidth}}
%\begin{longtable}[htb]{@{}p{0.10\textwidth}@{}p{0.20\textwidth}%
%                     @{}p{0.75\textwidth}}
\textbf{Short Name} &\textbf{Long Name} &\textbf{Description} \\
\hline
a & adf-file & Architectures for which the program is scheduled after the 
                        compilation. This switch can be used once for each 
                        target architecture. Note: there must be 'schedule'
                        installed.\\
s & scheduler-config & Configure file for scheduling command. \\
O & optimization-level & Optimization level. 0=no optimizations, 1=preserve
                       program API, 2=do not respect original API, 3 = same
                       that 2 \\
k & keep-symbols & List of symbols whose optimization away is prevented.
                   If you are using this, remember to define at least the
                   'main' symbol. \\
o & output-name & File name of the output binary. \\
d & leave-dirty & Does not delete files from each compilation phase. \\
c & compile-only & Compiles only. Does not link or optimize. \\
v & verbose & Prints out commands and outputs for each phase. \\
h & help & Prints out help info about program usage and parameters.\\
D & preprocessor-define & Preprocessor definition to be passed to gcc. \\
I & include-directory & Include directory to be passed to gcc. \\
L & library-directory & Passed to gcc. \\
l & library-link & Passed to gcc. \\
W & warning & Ignored. \\
- & scheduler-binary & Scheduler binary to use instead of 'schedule' in path. \\
- & extra-llc-flags & Options passed to llc. \\
- & plugin-cache-dir & Directory for cached llvm target plugins. \\
- & no-plugin-cache & Do not cache generated llvm target plugins. \\
- & rebuild-plugin & Rebuild plugin in the cache \\
- & clear-plugin-cache & Clear plugin cache completely. \\
\end{longtable}


\subsubsection{Examples of usage}
% introduce most common use cases

Usage of tcecc quite alike to gcc, excluding that warning options are
ignored.

If you wish to compile your source code into optimized bitcode the
usage is:

\shellcmd{tcecc -O2 -o myProg myProg.c} \\

On the other hand if you already have an architecture definition file of the
target processor you can compile the source code directly to parallel program:

\shellcmd{tcecc -O2 -a myProcessor.adf -o myProg.tpef myProg.c} \\

To compile the bitcode program into parallel program use:

\shellcmd{tcecc -a myProcessor.adf -o myProg.tpef myProg.bc}

Or if you want to a different scheduling configuration than the default:

\shellcmd{tcecc -s /path/to/mySchedulerConfiguration.conf -a myProcessor.adf
-o myProg.tpef myProg.bc} \\

Tcecc also has a ``leave dirty'' flag -d which preserves the intermediate
files created by the compiler. After compilation is complete tcecc will tell
you where to find these files (usually it is /tmp/tcecc-xxxxxx/). For example
if you try to compile your C-code straight into a scheduled program and
something goes wrong in scheduling you can find the bitcode program from the
temp directory.

\shellcmd{tcecc -d -O2 -a myProcessor.adf -o myProg.tpef myProg.c}

After compilation you should see this kind of message:

\textit{Intermediate files left in build dir /tmp/tcecc-xxxxxx}

where \textit{xxxxxx} is a random pattern of characters. \\

If you only want to compile the source code without linking
(and optimization) use -c flag. Output file is named after the source file
with .o appendix if you do not define an output name with -o.

\shellcmd{tcecc -c myProg.c}

\shellcmd{tcecc -c -o ~/another/path/myProg.o myProg.c} \\

With tcecc you can explicitly define symbols you wish to preserve in the
binary. This can be useful in debugging and profiling if the compiler removes
needed function labels.
Symbols are given in a comma seperated list.

\shellcmd{tcecc -O2 -a myMach.adf -k main,foo,bar -o myProg.tpef myProg.c}

\paragraph{Plugins}
% TODO: explain about tcecc plugins

\subsection{Custom operations}
\label{sec:tceccCustOp}

Tcecc compiler automatically defines macros for operations found from
operation definition files in OSAL search paths (see section 
~\ref{sec:osalpaths} for more details). You can use these macros in C code
by defining:
\begin{verbatim}
#include "tceops.h"
\end{verbatim}

Macros use the following format:
\begin{verbatim}
_TCE_<name>(input1, ... , inputN, output1, ... , outputN);
\end{verbatim}
where <name> is the operation name defined in OSAL. Number of input and
output operands depends on the operation.


\subsection{Known issues}
\label{sec:tceccIssues}
\begin{enumerate}

\item
Currently it is not possible to simulate a bitcode format program. But the
advantage of bitcode simulation is quite non-existent because the bitcode
does not even contain the final basic blocks that the architecture dependent
program has.
\end{enumerate}

% TODO: Add an explanation of llvm target plugins?


\section{Binary Encoding Map Generator (BEMGenerator)}
\label{section:bemgen}
% TODO: from Lasse's thesis, specs
Binary Encoding Map Generator (\textbf{BEMGenerator}) creates a file that
describes how to encode TTA instructions for a given target processor into
bit patterns that make up the executable bit image of the program (before
compression, if used).

\textbf{Input}: ADF

\textbf{Output}: BEM

\subsection{Usage}

The usage of BEMgenerator is the following:

\shellcmd{createbem -o outName.bem myProcessor.adf}

\section{Parallel Assembler and Disassembler}
\label{section:TCEAsm}

\textbf{TCE Assembler} compiles parallel TTA assembly programs to a TPEF
binary. The \textbf{Disassembler} provides a textual disassembly of
parallel TTA programs. Both tools can be executed only from the command
line.

\paragraph{Assembler}

\textbf{Input}: program source file in the TTA parallel assembler language
and an ADF

\textbf{Output}: parallel TPEF

\paragraph{Disassembler}

\textbf{Input}: parallel TPEF

\textbf{Output}: textual disassembly of the program

Rest of this section describes the textual appearance of TTA programs, that
is, how a TTA program should be disassembled.  The same textual format is
accepted and assembled into a TTA program.

\subsection{Usage of Disassembler}

The usage of the \emph{tcedisasm} application is as follows:

\begin{verbatim}
tcedisasm <options> adffile tpeffile
\end{verbatim}

The \emph{adffile} is the ADF file. \\
The \emph{tpeffile} is the parallel TPEF file.

The possible options of the application are as follows:\\

\begin{tabular}{p{0.10\textwidth}p{0.20\textwidth}
                p{0.75\textwidth}}
\textbf{Short Name} &\textbf{Long Name} &\textbf{Description} \\
\hline
o & outputfile  & The name of the output file.\\
h & help        & Prints out help info about program usage and parameters.\\
\end{tabular}\\

The application disassembles given parallel TPEF file according to given ADF
file. Program output is directed to standard output stream if specific output
file is not specified. The output is TTA parallel assembler language.

The program output can then be used as an input for the assembler program
tceasm.

The options can be given either using the short name or long name. If
short name is used, a hyphen (-) prefix must be used. For example -o
followed by the name of the output file. If the long name is used, a
double hyphen (- -) prefix must be used, respectively.

\subsubsection{An example of the usage}

The following example generates a disassemble of a parallel TPEF in the file
\emph{add4\_schedule.tpef} and writes the output to a file named
\emph{output\_dis.asm}.
 
\begin{verbatim}
tcedisasm -o output_dis.asm add4_supported.adf add4_schedule.tpef 
\end{verbatim}

\subsection{Usage of Assembler}

The usage of the \emph{tceasm} application is as follows:

\begin{verbatim}
tceasm <options> adffile assemblerfile
\end{verbatim}

The \emph{adffile} is the ADF file.\\
The \emph{assemblerfile} is the program source file in TTA parallel assembler
language.\\

The possible options of the application are as follows:\\

\begin{tabular}{p{0.10\textwidth}p{0.20\textwidth}
                p{0.75\textwidth}}
\textbf{Short Name} &\textbf{Long Name} &\textbf{Description} \\
\hline
o & outputfile  & The name of the output file.\\
q & quiet       & Do Not print warnings.\\
h & help        & Help info about program usage and parameters.\\
\end{tabular}\\

The application creates a TPEF binary file from given assembler file. 
Program output is written to a file specified by outputfile parameter. If
parameter is not given, the name of the output file will be the base name of
the given assembler file concatenated with \emph{.tpef}.

The options can be given either using the short name or long name. If
short name is used, a hyphen (-) prefix must be used. For example -o
followed by the name of the ouput file. If the long name is used, a
double hyphen (- -) prefix must be used, respectively.

\subsubsection{An example of the usage}

The following example generates a TPEF binary file named \emph{program.tpef}.
 
\begin{verbatim}
tceasm add4_schedule.adf program.asm
\end{verbatim}

\subsection{Memory Areas}

A TTA assembly file consists of several memory areas. Each area specifies
the contents (instructions or data) of part of an independently addressed
memory (or address space). There are two kinds of memory areas: \emph{code}
areas and \emph{data} areas. Code areas begin a section of the file that
defines TTA instructions.  Data areas begin a section that define groups of
memory locations (each group is collectively termed ``memory chunk'' in this
context) and reserve them to variables. By declaring data labels (see
Section~\ref{ssec:labels}), variables can be referred to using a name
instead of their address.

Memory areas are introduced by a header, which defines the type of area and
its properties. The header is followed by several logical lines (described
in Section~\ref{ssec:lines}), each declaring a TTA instruction or a memory
chunk.  The end of an area in the assembly file is not marked. Simply, a
memory area terminates when the header of another memory area or the end of
the assembly file is encountered.

The memory area header has one of the following formats:
\begin{verbatim}\tt
  CODE [\parm{start}] ;\\
  DATA \parm{name} [\parm{start}] ;
\end{verbatim}

A code area begins the declaration of TTA instructions that occupy a segment
of the instruction memory. A data area begins the declaration of memory
chunks reserved to data structures and variables.

A TTA program can work with several independently addressed data memories.
The locations of different memories belong to different address spaces. The
\emph{name} parameter defines the address space a memory area belongs
to. The code area declaration does not have a name parameter, because TTA
programs support only one address space for instruction memory, so its name
is redundant.

The \emph{start} parameter defines the starting address of the area being
declared within its address space. The start address can and usually is
omitted. When omitted, the assembler will compute the start address and
arrange different memory area declarations that refer to the same address
space. The way the start address is computed is left to the assembler, which
must follow only two rules:

\begin{enumerate}
\item %
  If a memory area declaration for a given address space appears before
  another declaration for the same address space, it is assigned a lower
  address.
\item %
  The start address of a memory area declaration is \emph{at least} equal to
  the size of the previous area declared in the same address space plus its
  start address.
\end{enumerate}

The second rule guarantees that the assembler reserves enough memory for an
area to contain all the (chunk or instruction) declarations in it.

\subsection{General Line Format}
\label{ssec:lines}

The body of memory areas consists of \emph{logical lines}.  Each line can
span one or more physical lines of the text. Conversely, multiple logical
lines can appear in a single physical lines.  All logical lines are
terminated by a semicolon `;'.

The format of logical lines is free. Any number of whitespace characters
(tabs, blanks and newlines) can appear between any two tokens in a line.
Whitespace is ignored and is only useful to improve readability. See
Section~\ref{ssec:style} for suggestions about formatting style and use of
whitespaces.

Comments start with a hash character (`\#') and end at the end of the
physical line.  Comments are ignored by the syntax.  A line that contains
only a comment (and possibly whitespaces before the hash character) is
completely removed before interpreting the program.

\subsection{Allowed characters}
\label{sec:names}

Names (labels, procedures, function units etc.) used in assembly code must
obey the following format:

\begin{verbatim}
 [a-zA-Z_][a-zA-z0-9_]*
\end{verbatim}

Basically this means is that a name must begin with a letter from range
a-z or A-Z or with an underscore. After the first character numbers can also
be used.

Upper case and lower case letters are treated as different characters.
For example labels \textbf{main:} and \textbf{Main:} are both unique.

\subsection{Literals}
\label{ssec:literals}

Literals are expressions that represent constant values. There are two
classes of literals: numeric literals and strings.

\paragraph{Numeric literals.}

A numeric literal is a numeral in a positional system. The base of the
system (or radix) can be decimal, hexadecimal or binary. Hexadecimal numbers
are prefixed with `0x', binary numbers are prefixed with `0b'. Numbers in
base 10 do not have a prefix. Floating-point numbers can only have decimal
base.

\emph{Example: Numeric literals.}
%
\begin{verbatim}
  0x56F05A
  7116083
  0b11011001001010100110011
  17.759
  308e+55
\end{verbatim}
The first three literals are interpreted as integer numbers expressed in
base, respectively, 16, 10 and 2. An all-digit literal string starting with
`0' digit is interpreted as a decimal number, not as an octal number, as is
customary in many high level languages.\footnote{
%
  This notation for octal literals has been deprecated.}
%
The last two literals are interpreted as floating point numbers. Unlike
integer literals, floating-point literals can appear only in initialisation
sequences of data declarations (see Section~\ref{ssec:data-line} for
details).

\paragraph{String literals.}

A string literal consists of a string of characters. The the numeric values
stored in the memory chunk initialised by a string literal depend on the
character encoding of the host machine.
%
\note{EXTENSION: charset directive}
%
The use of string literals makes the assembly program less portable.

Literals are defined as sequences of characters enclosed in double
(\verb|"|) or single (\verb|'|) quotes. A literal can be split into multiple
quoted strings of characters. All strings are concatenated to form a single
sequence of characters.

Double quotes can be used to escape single quotes and vice versa. To escape
a string that contains both, the declaration must be split into multiple
strings.

\emph{Example: String literals.}
The following literals all declare the same string 
\verb|Can't open file "%1"|.
%
\begin{verbatim}
  "Can't open file" '"%1"'
  'Can' "'" 't open file "%1"'
  "Can't open" ' file "%1"'
\end{verbatim}

String literals can appear only in initialisation sequences of data
declarations (see Section~\ref{ssec:data-line} for details).

\paragraph{Size, encoding and layout of string literals.}
By default, the size (number of MAU's) of the value defined by a string
literal is equal to the number of characters. If one MAU is wider than the
character encoding, then the value stored in the MAU is padded with zeroes.
The position of the padding bits depends on the byte-order of the target
architecture: most significant if ``big endian'', least significant if
``little endian''.

If one character does not fit in a singe MAU, then each character is encoded
in $\lceil m/n \rceil$ MAU's, where \emph{n} is the MAU's bit width and
\emph{m} is the number of bits taken by a character.

When necessary (for example, to avoid wasting bits), it is possible to
specify how many characters are packed in one MAU or, vice versa, how many
MAU's are taken to encode one character.
%
The size specifier for characters is prefixed to a quoted string and
consists of a number followed by a semicolon.

If $n > m$, the prefixed number specifies the number of characters packed in
a single MAU. For example, if one MAU is 32 bits long and a character takes
8 bits, then the size specifier in
\begin{verbatim}
  4:"My string"
\end{verbatim}
means: pack 4 characters in one MAU. The size specifier cannot be greater
than $\lceil n/m \rceil$. The size `1' is equivalent to the default.

If $m > n$, the prefixed number specifies the number of adjacent MAU's used
to encode one character. For example, if MAU's are 8-bit long and one
character takes 16 bits, then the same size specifier means: reserve 4 MAU's
to encode a single character. In this case, a 16-bit character is encoded in
32 bits, and padding occurs as described above. The size of the specifier in
this case cannot be smaller than $\lceil n/m \rceil$, which is the default
value when the size is not specified explicitly.

\subsection{Labels}
\label{ssec:labels}

A label is a name that can be used in lieu of a memory address.  Labels
``decorate'' data or instruction addresses and can be used to refer to,
respectively, the address of a data structure or an instruction.
%
The address space of a label does not need to be specified explicitly,
because it is implied by the memory area declaration block the label belongs
to.

A label declaration consists of a name string followed by a colon:

\begin{verbatim}\tt
  \parm{label-name}:
\end{verbatim}

Only a restricted set of characters can appear in label names. See
Section~\ref{sec:names} for details.

A label must always appear at the beginning of a logical line and must be
followed by a normal line declaration (see Sections~\ref{ssec:data-line},
\ref{ssec:code-line} for details). Only whitespace or another label can
precede a label. Label declarations always refer to the address of the
following memory location, which is the start location of the element (data
chunk or a TTA instruction) specified by the line.

Labels can be used instead of the address literal they represent in data
definitions and instruction definitions. They are referred to simply by
their name (without the colon), as in the following examples:
\begin{verbatim}
  # label reference inside a code area (as immediate)
  aLabel -> r5 ;

  # label reference inside a data area (as initialisation value)
  DA 4 aLabel ;
\end{verbatim}

\subsection{Data Line}
\label{ssec:data-line}

A data line consists of a directive that reserves a chunk of memory
(expressed as an integer number of minimum addressable units) for a data
structure used by the TTA program:

\begin{verbatim}\tt
  DA \parm{size} [\parm{init-chunk-1} \parm{init-chunk-2} \ldots] ;
\end{verbatim}

The keyword `DA' (Data Area) introduces the declaration of a memory chunk.
The parameter \emph{size} gives the size of the memory chunk in MAU's of the
address space of the memory area.

Memory chunks, by default, are initialised with zeroes. The memory chunk can
also be initialised explicitly. In this case, \emph{size} is followed by a
number of literals (described in Section~\ref{ssec:literals}) or labels
(Section~\ref{ssec:labels}) that represent initialisation values.
%
An initialisation value represents a constant integer number and takes
always an integer number of MAU's.

\paragraph{Size of the initialisation values.}
The size of an initialisation value can be given by prepending the size (in
MAU's) followed by a semicolon to the initialisation value.
%
If not defined explicitly, the size of the initialisation values is computed
by means of a number of rules. If the declaration contains only one
initialisation value, then the numeric value is extended to \emph{size},
otherwise, the rules are more complex and depend on the type of
initialisation value.

\begin{enumerate}
\item %
  If the initialisation value is a numeric literal expressed in base 10,
  then it is extended to \emph{size} MAU's.
\item %
  If the initialisation value is a numeric literal expressed in base 2 or
  16, then its size is extended to the minimum number of MAU's necessary to
  represents all its digits, even if the most significant digits are zeroes.
\item %
  If the initialisation value is a label, then it is extended to \emph{size}
  MAU's.
\end{enumerate}

\paragraph{Extension sign.}
Decimal literals are sign-extended. Binary, hexadecimal and string literal
values are zero-extended. Also the initialisation values represented by
labels are always zero-extended.

\paragraph{Partial Initialisation.}
If the combined size of the initialisation values (computed or specified
explicitly, it does not matter) is smaller than the size declared by the `DA'
directive, then the remaining MAU's are initialised with zeroes.

\emph{Example: Padding of single initialisation elements.}
%
Given an 8-bit MAU, the following declarations:
\begin{verbatim}
  DA 2 0xBB ; # equivalent to 2:0xBB
  DA 2 0b110001 ; # 0x31 (padded with 2 zero bits)
  DA 2 -13 ;
\end{verbatim}
define 2-MAU initialisation values: 0x00BB, 0x0031, and 0xFFF3,
respectively.

\emph{Example: Padding of of multi-element initialisation lists.}
%
The following declarations:
\begin{verbatim}
  DA 4 0x00A8 0x11;
  DA 4 0b0000000010100100 0x11 ;
\end{verbatim}
are equivalent and force the size of the first initialisation value in each
list to 16 bits (2 MAU's) even if the integer expressed by the declarations
take less bits. The 4-MAU memory chunk is initialised, in both declarations,
with the number 0x00A81100.
%
Another way to force the number of MAU's taken by each initialisation value
is to specify it explicitly.
%
The following declarations are equivalent to the declarations above:
\begin{verbatim}
  DA 4 2:0xA8  0x11;
  DA 4 2:0b10100100  0x11;
\end{verbatim}
%
Finally, the following declarations:
\begin{verbatim}
  DA 2 1:0xA8  0x11;
  DA 2 1:0b10100100  0x11;
\end{verbatim}
define a memory chunk initialised with 0xA8110000. The initialisation value
(in case of the binary literal, after padding to MAU bit width) defines
only the first MAU.

When labels appear in initialisation sequences consisting of multiple
elements, the size of the label address stored must be specified explicitly.

\emph{Example. Initialisation with Labels.}
%
The following declaration initialises a 6-MAU data structure where the first
2 MAU's contain characters `A' and `C', respectively, and the following 4
MAU's contain two addresses. The addresses, in this target architecture,
take 2 MAU's.
\begin{verbatim}
  DA 6 0x41 0x43 2:nextPointer 2:prevPointer ;
\end{verbatim}

\subsection{Code Line}
\label{ssec:code-line}

A code line defines a TTA instruction and consists of a comma-separated,
fixed sequence of bus slots.  A bus slot in any given cycle can either
program a data transport or encode part of a long immediate and program the
action of writing it to a destination (immediate) register for later
use.\footnote{
%
  The action of actually writing the long immediate to a destination
  register is encoded in a dedicated instruction field, and is not repeated
  in each move slot that encodes part of the long immediate.  This detail is
  irrelevant from the point of view of program specification.  Although the
  syntax is slightly redundant, because it repeats the destination register
  in every slot that encodes a piece of a long immediate, it is chosen
  because it is simple and avoids any chance of ambiguity.}

A special case of code line that defines an empty TTA instruction. This line
contains only three dots separated by one or more white spaces:
\begin{verbatim}
  . . . ; # completely empty TTA instruction
\end{verbatim}

A special case of move slot is the empty move slot. An empty move slot does
not program any data transport nor encodes bits of a long immediate. A
special token, consisting of three dots represents an empty move slot.
Thus, for a three-bus TTA processor, the following code line represents an
empty instruction:
\begin{verbatim}
  ... , ... , ... ; # completely empty TTA instruction
\end{verbatim}

\subsection{Long Immediate Chunk}

When a move slot encodes part of a long immediate, its declaration is
surrounded by square brackets and has the following format:

\begin{verbatim}\tt
  \parm{destination}=\parm{value}
\end{verbatim}

where \emph{destination} is a valid register specifier and \emph{value} is a
literal or a label that gives the value of the immediate.  The only valid
register specifiers are those that represent a register that belongs to an
immediate unit.  See section~\ref{ssec:move-terminal} for details on
register specifiers.

When the bits of a long immediate occupy more than one move slot, the format
of the immediate declaration is slightly more complex.  In this case, the
value of the immediate (whether literal or label) is declared in one and
only one of the slots (it does not matter which one). The other slots
contain only the destination register specifier.

\subsection{Data Transport}

A data transport consists of one optional part (a guard expression) and two
mandatory parts (a source and a destination). All three can contain an port
or register specifier, described in Section~\ref{ssec:move-terminal}.

The guard expression consists of a single-character that represents the
invert flag followed by a source register specifier. The invert flag is
expressed as follows:
\begin{enumerate}
\item %
  Single-character token `!': the result of the guard expression evaluates
  to zero if the source value is nonzero, and evaluates to one if the source
  value is equal to zero.
\item %
  Single-character token `?': the result of the guard expression evaluates
  to zero if the source value is zero, and evaluates to one if the source
  value is not zero.
\end{enumerate}

The move source specifier can be either a register and port specifier or an
in-line immediate. Register and port specifiers can be GPR's, FU output
ports, long immediate registers, bridge registers. The format of all these
is specified in Section~\ref{ssec:move-terminal}.
%
The in-line immediate represents an integer constant and can be defined as a
literal or as a label. In the latter case, the in-line immediate can be
followed by an equal sign and a literal corresponding to the value of the
label.  The value of the labels is more likely to be shown as a result of
disassembling an existing program than in user input code, since users can
demand data allocation and address resolution to the assembler.

\emph{Example: Label with value.}
%
The following move copies the label `LAB', which represents the address
0x051F0, to a GPR:
\begin{verbatim}
  LAB=0x051F0 -> r.4
\end{verbatim}

The move destination consists of a register and port specifier of two types:
either GPR's or FU input ports.

\subsection{Register Port Specifier}
\label{ssec:move-terminal}

Any register or port of a TTA processor that can appear as a move or guard
source, or as a move destination is identified and referred to by means of a
string.
%
There are different types of register port specifiers:
\begin{enumerate}
\item %
  General-purpose register.
\item %
  Function unit port.
\item %
  Immediate register.
\item %
  Bridge register.
\end{enumerate}

GPR's are specified with a string of the following format:

\begin{verbatim}\tt
  \parm{reg-file}[.\parm{port}].\parm{index}
\end{verbatim}

where \emph{reg-file} is the name of the register file,
%
\note{DISCUSS: pending \ref{ch:pending:rf-iu-names}}
%
\emph{port}, which can be omitted, is the name of the port through which the
register is accessed, and \emph{index} is the address of the register within
its register file.

Function unit input and output ports are specified with a string of the
following format:

\begin{verbatim}\tt
  \parm{function-unit}.\parm{port}.[\parm{operation}]
\end{verbatim}

where \emph{function-unit} is the name of the function unit, \emph{port} is
the name of the port through which the register is accessed, and
\emph{operation}, which is required only for opcode-setting ports,
identifies the operation performed as a side effect of the transport.
%
It is not an error to specify \emph{operation} also for ports that do not
set the opcode.  Although it does not represent any real information encoded
in the TTA program, this could improve the readability of the program.

Immediate registers are specified with a string if the following format:
\begin{verbatim}\tt
  \parm{imm-unit}[.\parm{port}].\parm{index}
\end{verbatim}
where \emph{imm-unit} is the name of the immediate unit,
%
\note{DISCUSS: pending \ref{ch:pending:rf-iu-names}}
%
\emph{port}, which can be omitted, is the name of the port through which the
register is accessed, and \emph{index} is the address of the register within
its unit.

Since any bus can be connected to at most two busses through bridges, it is
not necessary to specify bridge registers explicitly.  Instead, the string
that identifies a bridge register can only take one of two values:
`\{prev\}' or `\{next\}'.  These strings identify the bus whose value in
previous cycle is stored in the register itself.  A bus is identified by
`\{prev\}' if it is programmed by a bus slot that precedes the bus slot that
reads the bridge register.  Conversely, if the bus is identified by
`\{next\}', then it is programmed by a bus slot that follows the bus slots
that reads the bridge register.  In either case, the source bus slot must be
adjacent to the bus slot that contains the moves that reads the bridge
register.

\emph{Example: possible register and port specifiers.}
\begin{verbatim}
\begin{tabular}{lp{0.75\textwidth}}
\texttt{IA.0}    & immediate unit `IA', register with index 0\\
\texttt{RFA.5}   & register file `RFA', register with index 5\\
\texttt{U.s.add} & port `s' of function unit `U', opcode for operation
`add'\\
\verb|{|\texttt{prev}\verb|}|  & bridge register that contains the value on
the
                   bus programmed by the previous bus slot in previous
cycle\\
\end{tabular}
\end{verbatim}

\paragraph{Alternative syntax of function unit sources and destinations.}
Most clients, especially user interfaces, may find direct references to
function unit ports inconvenient. For this reason, an alternative syntax is
supported for input and output ports of function units:
\begin{verbatim}\tt
  \parm{function-unit}.\parm{operation}.\parm{index}
\end{verbatim}
where \emph{function-unit} is the name of the function unit,
\emph{operation} identifies the operation performed as a side effect of the
transport and \emph{index} is a number in the range [1,\emph{n}], where
\emph{n} is the total number of inputs and outputs of the operation.
%
The operation input and output, indirectly, identifies also the FU input or
output and the port accessed.  Contrary to the base syntax, which is
requires the operation name only for opcode-setting ports, this alternative
syntax makes the operation name not optional.
%
The main advantage of this syntax is that is makes the code easier to read,
because it removes the need to know what is the operation input or output
bound to a port, because.  The main drawback is an amount of (harmless)
``fuzziness'' and inconsistency, because it forces the user to define an
operation for ports that do not set the opcode, even in cases where the
operand is shared between two different operations.  For example, suppose
that the operand `1' of operations `add' and `mul' is bound to a port that
does not set the opcode and its value is shared between an `add' and a
`mul':
\begin{verbatim}
  r1 -> U1.add.1, r2 -> U1.add.2;
  U1.add.3 -> r3, r4 -> U1.mul.2;
  U1.mul.3 -> r5
\end{verbatim}
it looks as if the shared move belonged only to `add'. One could have also
written, correctly but less clearly:
\begin{verbatim}
  r1 -> U1.mul.1, r2 -> U1.add.2;
  # same code follows
\end{verbatim}
or even, assuming that operation `sub' is also supported by the same unit
and its operand `1' is bound to the same port:
\begin{verbatim}
  r1 -> U1.sub.1, r2 -> U1.add.2;
  # same code follows
\end{verbatim}

This alternative syntax is the only one permitted for TTA moves where
operations are not assigned to a function unit of the target machine.

When operations are not assigned to a function unit of the target machine,
they are formally assigned to the Universal Function Unit of the Universal
Machine. See Section~\ref{sec:um-conventions} for details on the Universal
Machine and other conventions that apply to unscheduled TTA code. The name
of the unit, in this case, may be omitted from the string.
%
\note{how to distinguish a RF name from an operation name then?}

\subsection{Assembler Command Directives}
\label{ssec:directives}

Command directives do not specify any code or data, but change the way the
assembler treats (part of) the code or data declared in the assembly
program.
%
A command directive is introduced by a colon followed by the name string
that identifies it, and must appear at the beginning of a new logical line
(possibly with whitespace before).

The assembler recognises the following directives.

\paragraph{procedure}
The `\verb|:procedure|' directive defines the starting point of a new
procedure. This directive is followed by one mandatory parameter: the name
of the procedure. Procedure directives should appear only in code areas. The
procedure directive defines also, implicitly, the end of procedure declared
by the previous `:procedure' directive. If the first code section of the
assembly program contains any code before a procedure directive, the code
is assumed to be part of a nameless procedure. Code in following code areas
that precede any procedure directive is considered part of the last
procedure declared in one of the previous code areas.

\emph{Example: declaration of a procedure.}
\begin{verbatim}
  CODE ;
  :procedure Foo ;
  Foo:
      r5 -> r6 , ... ;
      . . . ;
      ... , r7 -> add.1 ;
\end{verbatim}
%
In this example, a procedure called `Foo' is declared and a code label with
the same name is declared at the procedure start point. The code label could
be given any name, or could be placed elsewhere in the same procedure. In
this case, the code label `Foo' marks the first instruction of procedure
`Foo'.

\paragraph{global}
The `\verb|:global| directive declares that a given label is globally
visible, that is, it could be linked and resolved with external code. This
directive is followed by one mandatory parameter: the name of the label. The
label must be defined in the same assembly file. The label may belong to the
data or the code section, indifferently.

\paragraph{extern}
The `\verb|:extern| directive declares that a given label is globally
visible and must be resolved an external definition. This directive is
followed by one mandatory parameter: the name of the label. The label must
not be defined in the assembly file.

There can be only one label with any given name that is declared global or
external.

\emph{Example: declaration of undefined and defined global labels.}
\begin{verbatim}
  DATA dmem  0x540;
  aVar:
      DA 4 ;
  :global aVar ;
  :extern budVar ;
\end{verbatim}
%
In this example, `aVar' is declared to have global linkage scope (that is,
it may be used to resolve references from other object files, once the
assembly is assembled). Also `budVar' is declared to have global linkage,
but in this case the program does not define a data or code label with that
name anywhere, and the symbol must be resolved with a definition in an
external file.

\subsection{Assembly Format Style}
\label{ssec:style}

This section describes a number of nonbinding guidelines that add to the
assembly syntax specification and are meant to improve programs'
readability.

\paragraph{Whitespaces.}
Although the format of the assembly is completely free-form, tabs,
whitespaces and new lines can be used to improve the assembly layout and the
readability. The following rules are suggested:
\begin{enumerate}
\item %
  Separate the following tokens with at least one whitespace character:
  \begin{enumerate}
  \item %
    Label declaration `\emph{name}:' and first move or `DA' directive.
  \item %
    Moves of an instruction and commas.
  \item %
    Move source and destination and the `->' or `<-' token.
  \end{enumerate}
\item %
  Do not separate the following tokens with whitespaces:
  \begin{enumerate}
  \item %
    Long immediate chunk declaration and the surrounding brackets.
  \item %
    Label and, literal and the `=' token in between.
  \item %
    Any part of a register specifier (unit, port, operation, index) and the
    `.' separator token.
  \item %
    Register specifier, label or literal and the `=' in between.
  \item %
    Invert flag mark (`!' or `?') and the register specifier of a guard
    expression.
  \item %
    Initialisation chunk, the number of MAU's it takes and the `:' token in
    between.
  \item %
    Colon `:' and the nearby label or directive name.
  \end{enumerate}
\end{enumerate}

\paragraph{End of Line.}
The length of physical lines accepted is only limited by client
implementation.  Lines up to 1024 characters must be supported by any
implementation that complies with these specifications.
%
However, it is a good rule, to improve readability, that physical line
should not exceed the usual line length of 80 or 120 characters. If a TTA
instruction or a data declaration does not fit in the standard length, the
logical line should be split into multiple physical lines. The physical
lines following the first piece of a logical line can be indented at the
same column or more to the right.
%
In case of data declarations, the line is split between two literals that
form an initialisation data sequence.  In case of TTA instructions, logical
lines should never be split after a token of type `X' if it is recommended
that no whitespace should follow `X' tokens.
%
To improve readability, TTA instructions should be split only past the comma
that separates two move slots:
\begin{verbatim}
    # good line breaking
    r2 -> U.sub.1 , [i1=var] , r3 -> U.sub.2 , i1 -> L.ld.1 , [i1] ,
    0 -> U.eq.2 ;

    # bad line breaking
    r2 -> U.sub.1 , [i1=var] , r3 -> U.sub.2 , i1 -> L.ld.1 , [i1] , 0 ->
    U.eq.2 ;

    # really bad line breaking
    r2 -> U.sub.1 , [i1=var] , r3 -> U.sub.2 , i1 -> L.ld.1 , [i1] , 0 -> U.
    eq.2 ;
\end{verbatim}

\paragraph{Tabulation.}
The following rules can be taken as starting point for a rather orderly
layout of the assembly text, which resembles the layout of traditional
assembly languages:
\begin{enumerate}
\item %
  The first \emph{n} characters of the assembly lines are reserved to
  labels.  Instruction or data declarations are always indented by \emph{n}
  characters.
\item %
  Labels appear in the same physical line of the instruction or data
  declaration they refer to. Labels are no more than $n-2$ characters long.
\end{enumerate}
This layout is particularly clean when the TTA instructions contain few bus
slots and when multiple labels for the same data chunk or instruction do not
occur.

\emph{Example: Assembly layout style best suited target architectures with
few busses.}
\begin{verbatim}
DATA DMEM
var:      DA 4;

CODE
lab_A:    spr -> U.add.1 , 55 -> U.add.2 , spr -> r12 ;
          [i0=0x7F] , U.add.3 -> spr , i0 -> r2 ;
loop_1:   r2 -> U.sub.1 , r3 -> U.sub.2 , var -> L.ld.1 ;
          r2 -> U.eq.1 , U.sub.3 -> r2 , 0 -> U.eq.2 ;
          ?U.eq.3 loop_1 -> C.jump.1 , L.ld.2 -> U.and.2 ;
          0x1F -> U.and.1 , ... , ... ;
          ... , U.and.3 -> r8 , ... ;
\end{verbatim}

An alternative layout of the assembly text is the following:
\begin{enumerate}
\item %
  Instruction and data declarations are always indented by \emph{n}
  characters.
\item %
  Each label declaration appears in a separate physical line of the
  instruction or data declaration they refer to, and starts from column 0.
\end{enumerate}
This layout could be preferable when the TTA instructions contain so many
bus slots that the logical line is usually split into multiple physical
lines,
because it separates more clearly the code before and after a label (which
usually marks also a basic block entry point).  In addition, this layout
looks better when an instruction or data declaration has multiple labels and
when the label name is long.

\emph{Example: Assembly layout style best suited targets with many busses.}
\begin{verbatim}
DATA DMEM
var:
    DA 4;

CODE
a_long_label_name:
    spr -> U.add.1 , 55 -> U.add.2 , spr -> r12 , [i0=0x7F], i0 -> r2,
    ... ;
    ... ,  U.add.3 -> spr , ... , ... , ... , ... ;
loop_1:
    r2 -> U.sub.1 , [i1=var] , r3 -> U.sub.2 , i1 -> L.ld.1 , [i1] ,
    0 -> U.eq.2 ;
    r2 -> U.eq.1 , U.sub.3 -> r2 , ... , ?U.eq.3 loop_1 -> C.jump.1 ,
    0x1F -> U.and.1 , ... ;
    L.ld.2 -> U.and.2 , ... , ... , ... , ... , ... ;
    ... , ... , ... , U.and.3 -> r8 , ... , ... ;
\end{verbatim}
This example of assembly code is exactly equivalent to the code of previous
example, except that the address of `var' data chunk (a 4-MAU word) is
encoded in a long immediate and takes 2 move slots.

\paragraph{Layout of Memory Area Declarations.}
It is preferable to subdivide the contents of memories into several memory
area declarations and to group near each other area declarations of
different address spaces that are related to each other.  This underlines
the relation between data and code.
%
The alternative, a single area for each address space, mixes together all
data and all procedures of a program.

\paragraph{Mixing Alternative Syntaxes.}
It is preferable to not mix alternative styles or even syntaxes, although
any client that works with the assembly language is expected to deal with
syntax variants.

\subsection{Error Conditions}
\label{ssec:errors}

This section describes all the possible logical errors that can occur while
assembling a TTA program.

\paragraph{Address Width Overflow in Initialisation.}
A label is used as initialisation value of a data declaration, and the label
address computed by the assembler exceeds the number of MAU's in the data
declaration that must be initialised.

\paragraph{Address Width Overflow in Long Immediate.}
A label is used as initialisation value of a long immediate declaration, and
the label address computed by the assembler exceeds total width of the move
slots that encode the immediate, when concatenated together.

\paragraph{Address Width Overflow in In-line Immediate.}
A label is used as initialisation value of an in-line immediate, and the
label address computed by the assembler exceeds width of source field that
encodes the immediate.

\paragraph{Unspecified Long Immediate Value.}
A long immediate is defined, but none of the move slots that encode its bits
defines the immediate value.

\paragraph{Multiply Defined Long Immediate Value.}
More than one of the move slots that contain the bits of a long immediate
defines the immediate value.\footnote{
%
  If the values are identical in every move slot, then the client could
  issue a warning rather than a critical error.}

\paragraph{Overlapping Memory Areas.}
The start address specified in the header of two memory area declarations is
such that, once computed the sizes of each memory area, there is an
overlapping.

\paragraph{Multiple Global Symbols with Same Name.}
A `:global' directive declares a symbol with given name as globally visible,
but multiple labels with given name are declared in the program.

\paragraph{Unknown Command Directive.}
A command directive has been found that is not one of the directives
supported by the assembler.

\paragraph{Misplaced Procedure Directive.}
A `:procedure' directive appears inside a data area declaration block.

\paragraph{Procedure Directive Past End of Code.}
A `:procedure' directive appears after the last code line in the program.

\paragraph{Label Past End of Area.}
A label has been declared immediately before an area header or at the end of
the assembly program. Labels must be followed by a code line or a data line.

\paragraph{Character Size Specifier too Big.}
A size specified for the characters of a string literal is greater than the
maximum number of characters that can fit in one MAU.

\paragraph{Character Size Specifier too Small.}
A size specified for the characters of a string literal is smaller than the
minimum number of MAU's necessary to encode one character.

\paragraph{Illegal Characters in Quoted String.}
A quoted string cannot contain non-printable characters (that is, characters
that cannot be printed in the host encoding) and end-of-line characters.

\subsection{Warning Conditions}
\label{ssec:warnings}

This section describes all conditions of target architecture or assembly
syntax for which the client should issue an optional warning to prepare
users for potential errors or problematic conditions.

\paragraph{Equally Named Register File and Function Unit.}
A register file and a function unit of the target architecture have the same
name.  This is one of the conditions for the activation of the
disambiguation rule.

\paragraph{Port with the Name of an Operation.}
A register file or a function unit port are identified by a string name that
is also a name of a valid operation supported by the target architecture.
The first condition (port of register file) is more serious, because it may
require triggering a disambiguation rule.  The second condition (FU port) is
not ambiguous, but is confusing and ugly.  The second condition may be more
or less severe depending, respectively, whether the operation with the same
name is supported by the same FU or by another FU.

\paragraph{Code without Procedure.}
The first code area of the program begins with code lines before the first
`:procedure' directive. A nameless procedure with local visibility is
automatically created by the assembler.

\paragraph{Procedure Spanning Multiple Code Areas.}
A code area contains code line before the first `:procedure' directive, but
it is not the first code area declared in the code. The code at the
beginning of the area is attached to the procedure declared by the last
`:procedure' directive.

\paragraph{Empty Quoted String.}
Empty quoted strings are ignored.

\subsection{Disambiguation Rules}

Certain syntactic structures may be assigned different and (in principle)
equally valid interpretations. In these cases, a disambiguation rule assigns
priority to one of the two interpretation.
%
Grammatically ambiguous assembly code should be avoided.  Clients that
operate on TTA assembly syntax should issue a warning whenever a
disambiguation rule is employed.

\paragraph{Disambiguation of GPR and FU terms.}
When a GPR term includes also the RF port specifier, it can be interpreted
also as a function unit input or output.

Normally, the names of units, ports and operation rule out one of the two
interpretations.  Ambiguity can only occur only if:
\begin{enumerate}
\item%
  The target architecture contains a RF and a FU with identical name.
\item%
  One of the RF ports has a name identical to one of the operations
  supported by the FU that has the same name of the RF.
%
  \note{PENDING: disambiguation of unscheduled TTA
    code~\ref{ch:pending:disambiguation}}
\end{enumerate}

Ambiguity is resolved in favour of the GPR interpretation. No condition
applies to the indices (register index or operation input or output index).
The first interpretation is chosen even when it results in a semantic error
(an index out of range) whereas the other interpretation would be valid.

\emph{Example. Disambiguation rule.} The following move is interpreted as a
move that writes the constant 55 to the register register with index 2 of
register file `xx' through port `yy'.  If there exists an FU called `xx'
that supports an operation `yy' which has an input with index 2, this
interpretation of the move is never possible.
\begin{verbatim}
  55 -> xx.yy.2
\end{verbatim}

Even if the disambiguation rule is not triggered, clients should warn when
the target architecture satisfies one of the conditions above (or a
similar condition).  See Section~\ref{ssec:warnings} for a description of
this and other conditions for which a warning should be issued.

\paragraph{Disambiguation of variables and operation terms.}
In unscheduled code, operation terms cannot be confused with variables. The
special RF names `r', `f' and `b' are reserved, respectively, to integer,
floating-point and Boolean register files of the universal machine. The
assembler does not allow any operation to have one of these names.
%
\note{PENDING: disambiguation of mixed TTA
  code~\ref{ch:pending:disambiguation}}

\emph{Example. Unambiguous move term accessing a variable.} The following
move is interpreted as ``copy constant 55 to variable with index 2 of
variable pool `r'''. There cannot exist an operation `r', so the
interpretation of the move destination as operation term is impossible.
\begin{verbatim}
  55 -> r.2
\end{verbatim}

\paragraph{Disambiguation of Register File and Immediate Unit names}

Assembler syntax does not differentiate unit names of immediate units from
unit names of register files.  The same register specifier of a move source
\begin{verbatim}
   x.2 -> alu.add.1
\end{verbatim}
can represents a GPR or an immediate register depending on whether `x' is an
RF or a IU.

In this case the GPR interpretation is always preferred over the IU 
interpretation. However using the same naming for IUs and GPRs
restricts severely the programmability of target machine and is not
encouraged.




\section{Program Image Generator (PIG)}

Program Image Generator (\textbf{PIG}) generates the bit image which can be
uploaded to the target machine's memory for execution. Compression can be
applied to the instruction memory image by means of instruction compression
algorithm plugins.

% TODO: from Lasse's thesis, specs
\textbf{Input}: TPEF, BEM, ADF

\textbf{Output}: program bit image in alternative formats

\subsection{Usage}

The usage of the \emph{generatebits} application is as follows:

\begin{verbatim}
generatebits <options> ADF
\end{verbatim}

The possible options of the application are as follows:\\

\begin{tabular}{p{0.10\textwidth}p{0.20\textwidth}
                p{0.75\textwidth}}
\textbf{Short Name} &\textbf{Long Name} &\textbf{Description} \\
\hline
b & bem         & The binary encoding map.\\
c & compressor  & Name of the code compressor plugin file.\\
u & compressorparam & Parameter to the code compressor in form 'name=value'. \\
d & dataimages  & Creates data images.\\
g & decompressor & Generates a decompressor block. \\
o & diformat & The output format of data image(s) ('ascii', 'array'
or 'binary'). Default is 'ascii'. \\
w & dmemwidthinmaus & Width of data memory in MAUs. Default is 1. \\
i & imemwidthinmaus & Width of instruction memory in MAUs. Affects the ASCII
output formats. If not given, each instruction is printed on different line. \\
f & piformat & Determines the output format of the program image and data
images. Value may be 'ascii' or 'binary'.\\
s & showcompressors & Shows the compressor plugin descriptions. \\
p & program & The TPEF program file(s).\\
\end{tabular}\\

The application prints the program image to the standard output
stream. It can be easily forwarded to a file, if wanted. The data
images are generated to separate files, one for each address
space. Names of the files are determined by the name of the address
space and the suffix is \emph{.img}. The files are generated to 
the directory in which the application is executed.

The binary encoding map input parameter may be omitted. If so, the BEM
to be used is generated automatically. The BEM is used in the
intermediate format of the program image, before it is compressed by
the code compressor plugin. However, if no code compression is
applied, the program image output matches the format defined in the
BEM.

The options can be given either using the short name or long name. If
short name is used, a hyphen (-) prefix must be used. For example -a
followed by the name of the ADF file. If the long name is used, a
double hyphen (- -) prefix must be used, respectively.

\subsubsection{An example of the usage}

The following example generates a program image and data images of the address spaces in ASCII format without code compression.

\begin{verbatim}
generatebits -b encodings.bem -t program.tpef -f ascii -d machine.adf
\end{verbatim}

\subsection{Dictionary Compressor}
% The default plugin shipped with TCE.
% TODO: dictionary_tool.
% TODO: describe usage

\subsubsection{Defining New Code Compressors}

By default, PIG does not apply any code compression to the program
image. However, user can create a dynamic code compressor module which
is loaded and used by PIG. To define a code compressor, a C++ class
derived from \emph{CodeCompressorPlugin} class must be created and
compiled to a shared object file. The class is exported as a plugin
using a macro defined in \emph{CodeCompressor.hh} file. The 
\emph{buildcompressor} script can be used to compile the plugin
module.

\subsubsection{Creating the Code Compressor Module}

As mentioned, the code compressor class must be derived from
\emph{CodeCompressorPlugin} base class. The code compressor class
must implement the virtual compress() method of the
\emph{CodeCompressorPlugin} class. An example of a simple dictionary
compressor is defined in \emph{compressors/simple\_dictionary.cc} file in the source code distribution.

\paragraph{Compress Method}

The compress method is the heart of the compressor. The compressor
method returns the complete program image as a bit vector. The task of
the method is to call the \emph{addInstruction} method of the base class
sequentially to add the instructions to the program image. The base
class does the rest of the job. You just have to provide the bits of
each instruction in the parameter of addInstruction calls. Finally,
when all the instructions are added, the program image can be returned
by calling the \emph{programBits} method of the base class.

The instruction bits are provided as \emph{InstructionBitVector}
instances. That class provides capability to mark which bits of the
instruction refer to address of another instruction and thus are
likely to be changed when the real address of the referred instruction
is known. It is the responsibility of code compressor to mark that
kind of bits to the instruction bit vectors given in \emph{addInstruction}
calls. The base class will change the bits when the referred
instruction is added (when its address is known).

The code compressor should tell the base class which instructions must be
placed in the beginning of a MAU block. For example, the jump targets
should be in the beginning of MAU. Otherwise instruction fetching will
get difficult. This can be done by calling the
\emph{setInstructionToStartAtBeginningOfMAU} method of the base class. If all
the instructions are meant to start at the beginning of MAU,
\emph{setAllInstructionsToStartAtBeginningOfMAU} method is handy. By default,
all the instructions are concatenated without any pad bits in between
them, whatever the MAU of the instruction memory is. Note
that \emph{setInstructionToStartAtBeginningOfMAU} /
\emph{setAllInstructionsToStartAtBeginningOfMAU} method(s) must be called
before starting to add instructions with \emph{addInstruction} method.

\paragraph{Helper Methods Provided by the Base Class}

The \emph{CodeCompressorPlugin} base class provides some handy methods
that might be useful for the code compressor implementation. The
following lists the most important ones: 

\begin{itemize}
\item \emph{program()}: Returns the POM of the program.
\item \emph{tpefProgram()}: Returns the TPEF of the program.
\item \emph{binaryEncoding()}: Returns the BEM used to encode the instructions.
\item \emph{machine()}: Returns the machine.
\item \emph{bemBits()}: Returns the program image encoded with the rules of BEM.
\item \emph{bemInstructionBits()}: Returns the bits of the given instruction
encoded with the rules of BEM.
\end{itemize}

% TODO: describe the generation of the decompressor.vhdl

\subsubsection{Building the Shared Object}

When the code compressor module is coded, it must be compiled to a
shared object. It can be done with the \emph{buildcompressor}
script. The script takes the name of the object file to be created and
the source file(s) as command line parameters. The output of the
script, assuming that everything goes fine, is the dynamic module that
can be given to the \emph{generatebits} application.

\section{TPEF Dumper (dumptpef)}
\label{section:dumptpef}

\textbf{TPEF Dumper} is a program used for displaying information from the
given TPEF.

\textbf{Input}: TPEF

\textbf{Output}: dumped data from TPEF (printed to the standard output)

\subsection{Usage}

The usage of the \emph{dumptpef} application is as follows:

\begin{verbatim}
dumptpef <options>
\end{verbatim}

The possible options of the application are as follows:\\

\begin{tabular}{p{0.10\textwidth}p{0.15\textwidth}
                p{0.75\textwidth}}
\textbf{Short Name} &\textbf{Long Name} &\textbf{Description} \\
\hline
f & file-headers & Prints the file headers. \\
l & logical & Prints only logical information. Can be used for
checking if two files contain the same program and data and connections even if
it is in different order. \\
m & mem & Print information about memory usage of reserved sections. \\
r & reloc & Prints the relocation tables. \\
j & section & Prints the elements of section by section index. \\
s & section-headers & Prints the section headers. \\
t & syms & Prints the symbol tables. \\
\end{tabular}\\

% TODO: perhaps a simple example of the resulting dump in here?


\chapter{CO-DESIGN TOOLS}
\label{section:codesign}
% Tools that deal with both the program and architecture: not completely
% clear which one is more important object.

% \section{TCE Project Manager}
% \label{section:projectManager}
% The "TCE IDE." 
% TODO: V-P specifies it here.

\section{Architecture Simulation and Debugging}

TTA Processor \textbf{Simulator} simulates the process of running a TTA
program on its target TTA processor. Provides profiling, utilization, and 
tracing data for Explorer, Estimator and Compiler Backend.  Additionally, 
it offers debugging capabilities. 

\textbf{Input}: TPEF, [ADF]

\textbf{Output}: TraceDB

There are two user interfaces to the simulating and debugging
functionalities. One for the command line more suitable for scripting, and
another with more user-friendly graphical interface more suitable for
program debugging sessions. Both interfaces provide a console which
supports the Tcl scripting language.

\subsection{Processor Simulator CLI (ttasim)}
\label{section:ttasim}

The command line user interface of TTA Simulator is called 'ttasim'. The command
line user interface is also visible in the graphical user interface in form of a
console window. This manual covers the simulator control language used to
control the command line simulator and gives examples of its usage.

\subsubsection{Usage}

The usage of the command line user interface of the simulator is as follows:

\begin{verbatim}
ttasim <options>
\end{verbatim}

In case of a parallel simulation, a machine description file can be given before
giving the simulated program file. Neither machine file or the program file are mandatory;
they can also be given by means of the simulator control language.

The possible options for the application are as follows:\\

\begin{tabular}{p{0.10\textwidth}p{0.15\textwidth}
                p{0.75\textwidth}}
\textbf{Short Name} &\textbf{Long Name} &\textbf{Description} \\
\hline
a & adf & Sets the architecture definition file (ADF). \\
d & debugmode & Start simulator in interactive "debugging mode". This is enabled by default. Use --no-debugmode to disable.\\
e & execute-script & Executes the given string as a simulator control language script. For an examples of usage, see later in this section. \\
p & program & Sets the program to be simulated. Program must be given as a TTA program exchange format file (.TPEF) \\
q & quick & Simulates the program as fast as possible using the compiled simulation engine. \\
\end{tabular}\\



\paragraph{Example: Simulating a Parallel Program Without Entering Interactive
Mode}

The following command simulates a parallel program until the program ends,
without entering the debugging mode after simulation.

\begin{verbatim}
ttasim --no-debugmode -a machine.adf -p program.tpef
\end{verbatim}

\paragraph{Example: Simulating a Program Until Main Function Without
Entering Interactive Mode}

The following command simulates a sequential program until its main function and
prints consumed clock cycles so far. This is achieved by utilizing the simulator
control language and the '-e' option, which allows entering scripts 
from the command line.

\begin{verbatim}
ttasim --no-debugmode -e "until main; puts [info proc cycles];" -a machine.adf -p program.tpef
\end{verbatim}

\paragraph{Using the Interactive Debugging Mode}

Simulator is started in debugging mode by default. In interactive mode,
simulator prints a prompt "(ttasim)" and waits for simulator control language
commands. This example uses simulator control language to load a machine and a
program, run the simulation, print the consumed clock cycles, and quit
simulation.

\begin{verbatim}
ttasim
(ttasim) mach machine.adf
(ttasim) prog program.tpf
(ttasim) run
(ttasim) info proc cycles
54454
(ttasim) quit
\end{verbatim}

\subsection{Fast Compiled Simulation Engine}

The command line version of the Simulator, 'ttasim', supports two
different simulation engines. The default simulation engine interprets each
instruction and then simulates the processor behavior accordingly. While this is
good for many cases, it can be relatively slow when compared to the computer it
is being simulated on. Therefore, the Simulator also has a highly optimized 
mode that uses compiled simulation techniques for achieving faster simulation
execution. In this simulation, the TTA program and machine are compiled
into a single binary plug-in file which contains functions for simulating basic
blocks directly in native machine code, allowing as fast execution as possible.

\subsubsection{Usage}

\paragraph{Example: Simulating a Parallel Program Using The Compiled Simulation
Engine}

The following command simulates a parallel program using the compiled
simulation engine. (``-q'')

\begin{verbatim}
ttasim -a machine.adf -p program.tpef -q
\end{verbatim}

Currently, the behaviour of the compiled simulation can only be controlled with
a limited set of Simulator commands (such as 'stepi', 'run', 'until', 'kill').
Also, the simulation runs only at an accuracy of basic blocks so there 
is no way to investigate processor components between single cycles.

The following environment variables can be used to control the compiled
simulation behavior: \\

\begin{tabular}{p{0.35\textwidth}p{0.40\textwidth}
                p{0.15\textwidth}}
\textbf{Environment variable} &\textbf{Description} &\textbf{Default value} \\
\hline
TTASIM\_COMPILER & Specifies the used compiler. & ``gcc'' \\
TTASIM\_COMPILER\_FLAGS & Compile flags given to the compiler. & ``-O0'' \\
TTASIM\_COMPILER\_THREADS & Number of threads used to compile. & ``3'' \\

\end{tabular}\\


\subsubsection{ccache}
http://ccache.samba.org/

The compiled simulator can benefit quite a bit from different third party
software. The first one we describe here is a compiler cache software called
ccache. Ccache works by saving compiled binary files into a cache. When ccache
notices that a file about to be compiled is the same as file found in the
cache, it simply reloads file from the cache, thus eliminating recompilation
of unmodified files and saving time. This can be very useful when running the
same simulation program again, due to drastically reduced compilation times.

\subsubsection{distcc}
http://distcc.samba.org/

Another useful tool to use together with the compiled simulator is a
distributed compiler called distcc. Distcc works by distributing compilation to
multiple computers and compiling the generated source files in parallel.

After installing distcc, you can set ttasim to use the distcc compiler using the
following environment variable:
\begin{verbatim}
export TTASIM_COMPILER="distcc"
\end{verbatim} or if ccache is also installed, use:
\begin{verbatim}
export TTASIM_COMPILER="ccache distcc"\end{verbatim}

Also, remember to set the amount of used threads high enough. A good number
of threads to use would be approximately the amount of CPU cores
available. For example, setting 6 compiler threads can be done like following:
\begin{verbatim}
export TTASIM_COMPILER_THREADS=6
\end{verbatim}


\subsection{Simulator Control Language}

This section describes all the Simulator commands that can be entered when
the Simulator runs in debug mode.  The Simulator displays a new line with
the prompt string only when it is ready to accept new commands (the simulation
is not running).  The running simulation can be interrupted at any time by
the key combination CTRL-c.  The simulator stops simulation and prompts the
user for new commands as if it had been stopped by a breakpoint.

The Simulator control language is based on the Toolset Control
Language.  It extends the predefined set of Tcl commands
with a set of commands that allow to perform the functions listed above.  In
addition to predefined commands, all basic properties of Tcl (expression
evaluation, parameter substitution rules, operators, loop constructs,
functions, and so on) are supported.

\subsubsection{Initialization}
When the Simulator is run in debug mode, it automatically reads and executes
the initialization command file `.ttasim-init' if found in the user home
directory. The `.ttasim-init' file allows user to define specific simulator
settings (described in section ~\ref{ssec:debug-set}) which are enabled
everytime ttasim is executed.

After the initialization command sequence is completed, the
Simulator processes the command line options, and then reads the
initialization command file with the same name in current working directory.

After it has processed the initialization files and the command line options,
the Simulator is ready to accept new commands, and prompts the user for
input. The prompt line contains the string `(ttasim)'.

\subsubsection{Simulation Settings}
\label{ssec:debug-set}

Simulation settings are inspected and modified with the following commands.

\begin{description}
\item[setting \emph{variable} \emph{value}] %
  Sets a new value of environment variable \emph{variable}.
\item[setting \emph{variable}] %
  Prints the current value contained by environment variable
  \emph{variable}.
\item[setting] %
  Prints all settings and their current values. 
\end{description}

Currently, the following settings are supported.

\begin{description}
\item[bus\_trace {\emph{boolean}}] %
  Enables writing of the bus trace. Bus trace stores values written to each bus
in each simulated clock cycle.

\item[execution\_trace  {\emph{boolean}}] %
  Enables writing of the basic execution trace. Basic execution trace stores the
address of the executed instruction in each simulated clock cycle. 

\item[history\_filename {\emph{string}}]%
  The name of the file to store the command history, if command history saving
is enabled.

\item[history\_save {\emph{boolean}}] %
  Enables saving command history to a file.

\item[history\_size {\emph{integer}}]%
 Maximum count of last commands stored in memory. This does not affect writing
of the command history log, all commands are written to the log if logging is
enabled.

\item[next\_instruction\_printing {\emph{boolean}}]%
Print the next executed instruction when simulation stops, for example,
after single-stepping or at a breakpoint. 

\item[procedure\_transfer\_tracking {\emph{boolean}}]%
 Enables procedure transfer tracking. This trace can be used to easily
observe which procedures were 
 called and in which order. The trace is saved in 'procedure\_transfer'
table of Trace DB. This information
 could be derived from 'execution\_trace', but simulation is very slow when
it is enabled, this type of
 tracking should be faster.

\item[profile\_data\_saving {\emph{boolean}}]%
Save program profile data to trace database after simulation.

\item[rf\_tracking {\emph{boolean}}]%
 Enables concurrent register file access tracking. This type of tracking
makes the simulation
 speed much worse, so it is not enabled by default. The produced statistics can
be browsed after simulation by using the command 'info proc stats'.

\item[simulation\_time\_statistics {\emph{boolean}}]%
Prints time statistics for the last command ran (run, until, nexti, stepi).

\item[simulation\_timeout {\emph{integer}}]%
Stops the simulation after specified timeout. Value of zero means no timeout.

\item[static\_compilation {\emph{boolean}}]%
Switch between static and dynamic compilation when running compiled simulation. 


\item[utilization\_data\_saving {\emph{boolean}}]%
Save processor utilization data to trace database after simulation.

\end{description}

\subsubsection{Control of How the Simulation Runs}
\label{ssec:debug-running}

The commands described in this section allow to control the simulation process.

Before simulation can start, a program must be loaded into the Simulator.
If no program is loaded, the command \emph{run} causes the following
message:
\begin{verbatim}
  Simulation not initialized.
\end{verbatim}

\begin{description}
\item[run] %
  Starts simulation of the program currently loaded into the Simulator.  The
  program can be loaded by \emph{prog} command (see
  Section~\ref{ssec:debug-files}) or may be given directly as argument, on
  the command line.  Simulation runs until either a breakpoint is
  encountered or the program terminates.

\item[resume {[\emph{count}]}] %
  Resume simulation of the program until the simulation is finished or a
  breakpoint is reached.  The \emph{count} argument gives the
  number of times the \emph{continue} command is repeated, that is, the
  number of times breakpoints should be ignored.

\item[stepi {[\emph{count}]}] %
  Advances simulation to the next machine instructions, stepping into the
  first instruction a new procedure if a function call is simulated.  The
  \emph{count} argument gives the number of machine instruction to simulate.

\item[nexti {[\emph{count}]}] %
  Advances simulation to the next machine instructions in current procedure.
  If the instruction contains a function call, simulation proceeds until
  control returns from it, to the instruction past the function call.  The
  \emph{count} argument gives the number of machine instruction to simulate.

\item[until {[\emph{arg}]}] %
  Continue running until the program location specified by \emph{arg} is
  reached.  Any valid argument that applies to command \emph{break} (see
  Section~\ref{ssec:debug-breakpoints}) is also a valid argument for
  \emph{until}.  If the argument is omitted, the implied program location
  is the next instruction.  In practice, this command is useful when
  simulation control is inside a loop and the given location is outside it:
  simulation will continue for as many iterations as required in order to
  exit the loop (and reach the designated program location).

\item[kill] %
  Terminate the simulation.  The program being simulated remains loaded and
  the simulation can be restarted from the beginning by means of command
  \emph{run}.  The Simulator will prompt the user for confirmation
  before terminating the simulation.

\item[quit] %
  This command is used to terminate simulation and exit the Simulator.
\end{description}

\subsubsection{Examining Program Code and Data}
\label{ssec:debug-data}

The Simulator allows to examine the program being simulated and the data it
uses

\begin{description}
\item[x {[/\emph{nfu}][\emph{addr}]}] %
  This low-level command prints the data in memory starting at specified
  addresses \emph{addr}.  The optional parameters \emph{n}, \emph{f} and
  \emph{u} specify how much memory to display and how to format it.
  \begin{description}
  \item[\emph{n}] %
    Repeat count: how many data words (counting by units \emph{u}) to
    display.  If omitted, it defaults to 1.
  \item[\emph{f}] %
    Display format is one of the output (specified below), `s'
    (null-terminated string) or `i' (machine instruction).  If omitted, it
    defaults to `x' (hexadecimal).
  \item[\emph{u}] %
    Unit size: `b' (MAU, a byte in byte-addressed memories), `h' (double
    MAU), `w' (quadruple word, a `word' in byte-addressed 32-bit
    architectures), `g' (giant words, 8 MAU's).  The unit size is ignored
    for formats `s' and `i'.
  \end{description}

  If \emph{addr} is omitted, then the first address past the last address
  displayed by the previous \emph{x} command is implied.  If the value of
  \emph{n}, \emph{f}, or \emph{u} is not specified, the value given in the
  most recent \emph{x} command is maintained.

  The values printed by command \emph{x} are not entered in the value
  history (see Section~\ref{ssec:debug-history}).

\item[symbol\_address \emph{datasym} ] %

  Returns the address of the given data symbol (usually
  a global variable). 

\item[disassemble {[\emph{addr1} [\emph{addr2}]]}] %
  Prints a range of memory addresses as machine instructions.  When two
  arguments \emph{addr1}, \emph{addr2} are given, \emph{addr1} specifies the
  first address of the range to display, and \emph{addr2} specifies the last
  address (not displayed).  If only one argument, \emph{addr1}, is given,
  then the function that contains \emph{addr1} is disassembled.  If no
  argument is given, the default memory range is the function surrounding
  the program counter of the selected frame.

\end{description}

\subsubsection{Control Where and When to Stop Program Simulation}
\label{ssec:debug-breakpoints}

A breakpoint stops the simulation whenever the Simulator reaches a certain
point in the program. It is possible to add a condition to a breakpoint, to
control when the Simulator must stop with increased precision. There are two
kinds of breakpoints: \emph{breakpoints} (proper) and \emph{watchpoints}. A
watchpoint is a special breakpoint that stops simulation as soon as the
value of an expression changes.

%
where \emph{num} is a unique number that identifies the breakpoint or
watchpoint and \emph{description} describes the properties of the
breakpoint. The properties include: whether the breakpoint must be deleted
or disabled after it is reached; whether the breakpoint is currently
disabled; the program address of the breakpoint, in case of a program
breakpoint; the expression that, when modified by the program, causes the
Simulator to stop, in case of a watchpoint.

\begin{description}
\item[bp \emph{address}] %
  Sets a breakpoint at address \emph{address}. Argument can also be a code
  label such as global procedure name (e.g. 'main').
\item[bp \emph{args} if] %
  Sets a conditional breakpoint.  The arguments \emph{args} are the same as
  for unconditional breakpoints.  After entering this command, Simulator
  prompts for the condition expression. Condition is evaluated each time 
  the breakpoint is reached, and the simulation only when the condition 
  evaluates as true.
\item[tbp \emph{args}] %
  Sets a temporary breakpoint, which is automatically deleted after the
  first time it stops the simulation.  The arguments \emph{args} are the
  same as for the \emph{bp} command.  Conditional temporary breakpoints
  are also possible (see command \emph{condition} below).
\item[watch] %
  Sets a watchpoint for the expression \emph{expr}.  The Simulator will stop
  when the value of given expression is modified by the program.
  Conditional watchpoints are also possible (see command \emph{condition}
  below). 
\item[condition {[\emph{num}] [\emph{expr}]}] %
  Specifies a condition under which breakpoint \emph{num} stops simulation.
  The Simulator evaluates the expression \emph{expr} whenever the breakpoint
  is reached, and stops simulation only if the expression evaluates as true
  (nonzero).  The Simulator checks \emph{expr} for syntactic correctness as
  the expression is entered.

  When \emph{condition} is given without expression argument, it removes any
  condition attached to the breakpoint, which becomes an ordinary
  unconditional breakpoint.
\item[ignore {[\emph{num}] [\emph{count}]}] %
  Sets the number of times the breakpoint \emph{num} must be ignored when
  reached.  A \emph{count} value zero means that the breakpoint will stop
  simulation next time it is reached.

\item[enablebp {[delete|once] [\emph{num} \ldots]}] %
  Enables the breakpoint specified by \emph{num}.  If \emph{once} flag is
  specified, the breakpoint will be automatically disabled after it is
  reached once.  If \emph{delete} flag is specified, the breakpoint will be
  automatically deleted after it is reached once.
\item[disablebp {[\emph{num} \ldots]}] %
  Disables the breakpoint specified by \emph{num}.  A disabled breakpoint
  has no effect, but all its options (ignore-counts, conditions and
  commands) are remembered in case the breakpoint is enabled again.
\item[deletebp {[\emph{num} \ldots]}] %
  Deletes the breakpoint specified by \emph{num}.  If no arguments are
  given, deletes all breakpoints currently set, asking first for
  confirmation.
\item[info breakpoints {[\emph{num}]}] %
  Prints a table of all breakpoints and watchpoints.  Each breakpoint is
  printed in a separate line.  The two commands are synonymous.
\end{description}

\subsubsection{Specifying Files and Directories}
\label{ssec:debug-files}

The Simulator needs to know the file name of the program to
simulate/debug and, usually, the Architecture Definition File (ADF)
that describes the architecture of the target processor on which the
program is going to run.  

\begin{description}
\item[prog {[\emph{filename}]}]%
  Load the program to be simulated from file \emph{filename}.  If no
  directory is specified with \emph{set directory}, the Simulator will
  search in the current directory.

  If no argument is specified, the Simulator discards any information it has
  on the program.

\item[mach {[\emph{filename}]}]%
  Load the machine to be simulated from file \emph{filename}.  If no
  directory is specified with \emph{set directory}, the Simulator will
  search in the current directory.

  In case a parallel program is tried to be simulated without machine,
  an error message is printed and simulation is terminated immediately.
  In some cases the machine file can be stored in the TPEF file.

\item[conf {[\emph{filename}]}]%
  Load the processor configuration to be simulated from file \emph{filename}.
  If no directory is specified with \emph{set directory}, the Simulator will
  search in the current directory.

  Simulator expects to find the simulated machine from the processor 
  configuration file. Other settings are ignored. This can be used as
  replacement for the \emph{mach} command.

\end{description}

\subsubsection{Examining State of Target Processor and Simulation}
\label{ssec:debug-state}

The current contents of any programmer visible state, which includes any
programmable register, bus, or the last data word read from or written to a
port, can be displayed. The value is displayed in base 10 to allow using it
easily in Tcl expressions or conditions. This makes it possible, for
example, to set a conditional breakpoint which stops simulation only if the
value of some register is greater than some constant.

\begin{description}
\item[info proc cycles] %
  Displays the total execution cycle count and the total stall cycles count.

\item[info proc mapping] %
  Displays the address spaces and the address ranges occupied by the
  program: address space, start and end address occupied, size.

\item[info proc stats] %
  In case of parallel simulation, displays current processor utilization
  statistics. In case 'rf\_tracking' setting is enabled and running parallel
  simulation, also lists the detailed register file access information. 

\item[info regfiles] %
  Prints the name of all the register files of the target processor.

\item[info registers \emph{regfile} {[\emph{regname}]}] %
  Prints the value of register \emph{regname} in register file
  \emph{regfile}, where \emph{regfile}
  is the name of a register file of the target processor, and \emph{regname}
  is the name of a register that belongs to the specified register file.

  If \emph{regname} is omitted, the value of all registers of the specified
  register file is displayed.

\item[info funits] %
  Prints the name of all function units of the target processor.

\item[info iunits] %
  Prints the name of all immediate units of the target processor.

\item[info immediates \emph{iunit} {[\emph{regname}]}] %
  Prints the value of immediate register \emph{regname} in immediate unit
  \emph{iunit}, where \emph{iunit} is the name of an immediate unit of 
  the target processor, and \emph{regname} is the name of a register 
  that belongs to the specified unit.

  If \emph{regname} is omitted, the value of all registers of the specified
  immediate unit is displayed.

\item[info ports \emph{unit} {[\emph{portname}]}] %
  Prints the last data word read from or written to port \emph{portname} of
  unit \emph{unit}, where \emph{unit} may be any function unit, register
  file or immediate unit of the target processor.  The value of the data
  word is relative to the selected stack frame.

  If \emph{portname} is omitted, the last value on every port of the
  specified unit is displayed.

\item[info busses {[\emph{busname}]}] %
  Displays the name of all bus segments of transport bus \emph{busname}.  If
  the argument is omitted, displays the name of the segments of all busses
  of the target processor.%

\item[info segments \emph{bus} {[\emph{segmentname}]}]] %
  Prints the value currently transported by bus segment \emph{segmentname}
  of the transport bus \emph{busname}.

  If no segment name is given, the Simulator displays the contents of all
  segments of transport bus \emph{bus}.

\item[info program] %
  Displays information about the status of the program: whether it is
  loaded or running, why it stopped.

\item[info program is\_instruction\_reference \emph{ins\_addr} \emph{move\_index}]
  Returns 1 if the source of the given move refers to an instruction
  address, 0 otherwise.

\item[info stats executed\_operations] %
  Prints the total count of executed operations.

\item[info stats register\_reads] %
  Prints the total count of register reads.

\item[info stats register\_writes] %
  Prints the total count of register writes.

\end{description}

\subsubsection{Miscellaneous Support Commands and Features}
\label{ssec:debug-misc}

The following commands are facilities for finer control on the behaviour of
the simulation control language.

\begin{description}
\item[help [\emph{command}]] %
  Prints a help message briefly describing command \emph{command}.
 If no argument is given, prints a general help message and a listing of
 supported commands.
\end{description}

\subsubsection{Command and Value History Logs}
\label{ssec:debug-history}

All commands given during a simulation/debugging session are saved in a
\emph{command history log}.  This forms a complete log of the session, and
can be stored or reloaded at any moment.  By loading and running a complete
session log, it is possible to resume the same state in which the session
was saved.

It is possible to run a sequence of commands stored in a command file at any
time during simulation in debug mode using the \emph{source} command.  The
lines in a command file are executed sequentially and are not printed as
they are executed.  An error in any command terminates execution of the
command file.

\begin{description}
\item[commands [\emph{num}]]%
  Displays the last \emph{num} commands in the command history log. If
  the argument is omitted, the \emph{num} value defaults to 10.

\item[source \emph{filename}]%
  Executes the command file \emph{filename}.

\end{description}

\subsection{Traces}
\label{sec:traces}
% TODO: Add examples of usage!

All simulation traces are stored in a SQLite 3 binary file. The file is
named after the program file by appending '.trace' to its end. This file can be
browsed with the sqlite client. The sqlite client allows browsing contents of
the tables using standard SQL queries.

\subsection{Example Queries}
\label{ssec:tracedb_tables}

This section provides several useful example SQL queries to retrieve data from
the trace database. To query data in a generated trace database file, one should
use the 'sqlite3' client to open the database after which any SQL query can
be entered. The most useful queries are provides as scripts of which
names start with 'dump\_*'.

\paragraph{Instruction Execution Statistics}

Following query returns the instruction execution counts starting from the
most frequently executed instruction. In addition, the procedure name to which
each instruction belongs is printed.

\begin{verbatim}

SELECT instruction_execution_count.address AS address, 
       instruction_execution_count.count AS count, 
       procedure_address_range.procedure_name AS procedure 
FROM instruction_execution_count, procedure_address_range 
WHERE address >= procedure_address_range.first_address AND 
      address <= procedure_address_range.last_address
ORDER BY count DESC;

\end{verbatim}

\paragraph{Procedure Execution Profile}

Following query returns the count of clock cycles spent in each procedure of
the program. Listing is ordered such that the procedure in which most time was
spent is printed first.

\begin{verbatim}
SELECT procedure_address_range.procedure_name AS procedure, 
      SUM(instruction_execution_count.count) AS cycles 
FROM instruction_execution_count, procedure_address_range 
WHERE address >= procedure_address_range.first_address AND
      address <= procedure_address_range.last_address
GROUP BY procedure
ORDER BY cycles DESC;
\end{verbatim}

This query is provided in script \emph{dump\_function\_profile}.

\paragraph{Procedure Transfer Trace}

This query lists the order in which each procedure was executed in the program.

\begin{verbatim}
SELECT procedure_transfer.cycle AS cycle,
       procedure_transfer.address AS address,
       procedure_transfer.type AS is_exit,
       procedure_address_range.procedure_name AS procedure
FROM procedure_transfer,procedure_address_range
WHERE address >= procedure_address_range.first_address AND
      address <= procedure_address_range.last_address
ORDER BY cycle;
\end{verbatim}

This query is provided in script \emph{dump\_call\_trace}.

\subsection{Processor Simulator GUI (Proxim)}
\label{section:proxim}

Processor Simulator GUI (Proxim) is a graphical frontend for 
the TTA Processor Simulator.

\subsubsection{Usage}

This section is intended to familiarize the reader to basic usage of 
Proxim. This chapter includes instructions to accomplish only the most
common tasks to get the user started in using the Simulator GUI.

The following windows are available:
\begin{itemize}
\item \emph{Machine State window} Displays the state of the simulated
processor. 
\item \emph{Disassembly window} for displaying machine level source
code of the simulated application. 
\item \emph{Simulator console} for controlling the simulator using
the simulator control language. 

\item \emph{Simulation Control Window}: Floating tool window with shortcut
buttons for items in the \emph{Program} menu.
\end{itemize}

\paragraph{Console Window}

Textual output from the simulator and all commands sent to the simulator
engine are displayed in the \emph{Simulator Console} window, as well as the
input and output from the simulated program. Using the window,
the simulator can be controlled directly with Simulator Control Language
accepted also by the command line interface of the simulator. For list of
available commands, enter \emph{'help'} in the console.

Most of the commands can be executed using graphical dialogs and
menus, but the console allows faster access to simulator functionality for
users familiar with the Simulator Control Language. Additionally, all commands
performed using the GUI are echoed to the console, and appended to the
console command history.

The console keeps track of performed commands in command history. Commands in
the command history can be previewed and reused either by selecting the
command using up and down arrow keys in the console window, or by selecting
the command from the \textbf{Command History}.

The \emph{Command} menu in the main window menubar contains all GUI
functionality related to the console window.


\paragraph{Simulation Control Window}

Running simulation can be controlled using the \textbf{Simulation Control}
window.

Consequences of the window buttons are as follows:
\begin{itemize}
\item \textbf{Run/Stop:} If simulation is not running, the button is
labeled 'Run', and it starts simulation of the program loaded in the
simulator. If simulation is running, the button is labled 'Stop', and
it will stop the simulation.

\item \textbf{Stepi:} Advances simulation to the next machine instructions.

\item \textbf{Nexti:} Advances simulation to the next machine instructions
in current procedure.

\item \textbf{Continue:} Resumes simulation of the program until the
simulation is finished or a breakpoint is reached.
\item \textbf{Kill:} Terminates the simulation. The program being simulated
remains loaded and the simulation can be restarted from the begining.
\end{itemize}


\paragraph{Disassembly Window}
\label{sec:mcode_win}


The disassembly window displays the
machine code of the simulated program. The machine code is displayed
one instruction per line. Instruction address and instruction
moves are displayed for each line.
Clicking right mouse button on an instruction displays a context menu with the
following items:
\begin{itemize}
\item \textbf{Toggle breakpoint:} Sets a breakpoint or deletes existing 
breakpoint at the selected instruction.
\item \textbf{Edit breakpoint...:} Opens selected breakpoint in
\textbf{Breakpoint Properties} dialog.

\end{itemize}


\paragraph{Machine State Window}

The \emph{Machine State Window}
displays the state of the processor running the simulated program.
The window is split horizontally to two subwindows. The window on the
left is called \emph{Status Window}, and it displays general information
about the state of the processor, simulation and the selected processor
block. The subwindow on the right, called \emph{Machine Window},
displays the machine running the simulation.

The blocks used by the current instruction are drawn in red color. The block
utilization is updated every time the simulation stops.

Blocks can be selected by clicking them with LMB.
When a block is selected, the bottom of the \emph{status window}
will show the status of the selected block.

\subsubsection{Profiling with Proxim}
\label{sec:ProfileProxim}

Proxim offers simple methods for profiling your program. After you have
executed your program you can select ``Source'' -> ``Profile data'' ->
``Highlight top execution count'' from the top menu. This opens a dialog which
shows execution counts of various instruction address ranges. The list is
arranged in descending order by the execution count.

If you click a line on the list the disassembly window will focus on the
address range specified on that line. You can trace in which function the
specific address range belongs to by scrolling the disassembly window up
until you find a label which identifies the function. You must understand at
least a little about assembly coding to find the actual spot in C code that
produces the assembly code.

\section{Processor Cost/Performance Estimator (estimate)}
\label{section:estimate}

Processor Cost/Performance \textbf{Estimator} provides estimates of energy
consumption, die area, and maximum clock rate of TTA designs (program and
processor combination).

\textbf{Input}: ADF, IDF, [TPEF and TraceDB]

\textbf{Output}: Estimate (printed to the standard output)

\subsection{Command Line Options}

The usage of the \emph{estimate} application is as follows:

\begin{verbatim}
estimate {-p [TPEF] -t [TraceDB]} ADF IDF
\end{verbatim}

The possible options of the application are as follows:\\

\begin{tabular}{p{0.10\textwidth}p{0.15\textwidth}
                p{0.75\textwidth}}
\textbf{Short Name} &\textbf{Long Name} &\textbf{Description} \\
\hline
p & program & Sets the TTA program exchange format file (TPEF) from which to
load the estimated program (required for energy estimation only). \\
t & trace & sets the simulation trace database (TraceDB) from which to load
the simulation data of the estimated program (required for energy estimation
only). \\
a & total-area & Runs total area estimation.\\
l & longest-path & Runs longest path estimation.\\
e & total-energy & Runs total energy consumption estimation.\\
\end{tabular}\\

If \verb|tpef| and \verb|tracedb| are not given, the energy estimation
will NOT be performed since the Estimator requires utilization
information about the resources. However, the area and timing
estimation will be done. If only one of \verb|tracedb| and \verb|tpef|
is given, it is ignored.

\section{Automatic Design Space Explorer (explore)}
\label{section:explore}

Automatic Design Space \textbf{Explorer} automates the
process of searching for target processor configurations with favourable
cost/performance characteristics for a given set of applications by
evaluating hundreds of processor configurations.

\textbf{Input}: ADF (a starting point architecture), TPEF, HDB

\textbf{Output}: ExpResDB (Section~\ref{sec:expresdb})

\subsection{Command Line Options}

The exploration result database \emph{<output\_dsdb>} is required always. The
database can be queried applications can be added into and removed from the 
database and the explored configurations in the database can be written as
files for further examination.

The possible options of the application are as follows:\\

\begin{tabular}{p{0.08\textwidth}p{0.17\textwidth}p{0.75\textwidth}}
\textbf{Short Name} &\textbf{Long Name} &\textbf{Description} \\
\hline
d & add\_app\_dir & Path(s) of the test application(s) to be added into the DSDB.\\
a & adf & ADF to add into the DSDB.\\
n & conf\_count & Print the number of machine configurations in the DSDB.\\
c & conf\_summary & Print the summary of machine configurations in the DSDB
ordered by: Ordering may be one of the following: \emph{I} ordering by 
configuration Id, \emph{P} ordering by application path, \emph{C} ordering by
cycle count, \emph{E} ordering by energy estimate.\\
e & explorer\_plugin & Design Space Explorer plugin to be used.\\
b & hdb & HDB to use with exploration.\\
i & idf & IDF to add into the DSDB, needs also ADF.\\
l & list\_apps & List the applications in the DSDB.\\
u & plugin\_param & Parameter to the explorer plugin in form 'name=value'.
If parameter value is boolean, use 'true', 'false', 1 or 0.\\
r & rm\_app & ID(s) of the test program path(s) to be removed from the DSDB.\\
s & start & Starting point configuration ID in the DSDB.\\
w & write\_conf & export the ADF and IDF files from the DSDB with given
configuration id. Does not remove the configuration from the DSDB.\\
\end{tabular}\\

Depending on the exploration plugin, the exploring results machine
configurations in to the exploration result database dsdb. The best results
from the previous exploration run are given at the end of the exploration:

\shellcmd{explore -e RemoveUnconnectedComponents -a data/FFTTest --hdb=data/initial.hdb 
data/test.dsdb}

\begin{verbatim}
Best result configurations:
 1
\end{verbatim}
Exploration plugins may also estimate the costs of configurations with the
available applications. If there are estimation results for the configuratios
those can be queried with option \textbf{--conf\_summary} by giving the
ordering of the results.

\subsection{Explorer Plugin: SimpleICOptimizer}
SimpleICOptimizer is an explorer plugin that optimizes the interconnection
network of the given configuration by removing the connections that are not
used in the parallel program. Parameters that can be passed to the
SimpleICOptimizer are:

\begin{tabular}{p{0.20\textwidth}p{0.20\textwidth}
                p{0.60\textwidth}}
\textbf{Param Name} &\textbf{Default Value} &\textbf{Description} \\
\hline
tpef & no default value &  name of the scheduled program file \\
add\_only & false &  Boolean value. If set true the connections of the given \\
evaluate & true &  Boolean value. True evaluates the result config. \\
configuration won't be emptied, only new ones may be added\\
\end{tabular}\\

If you pass a scheduled tpef to the plugin, it tries to optimize the configuration
for running the given program. If multiple tpefs are given, the first one will be used
and others discarded.
Plugin tries to schedule sequential program(s) from the application path(s) defined in
the dsdb and use them in optimization if tpef is not given.

Using the plugin requires user to define the configuration he wishes optimize.
This is done by giving \textit{-s <configuration\_ID>} option to the explorer.

Let there be 2 configurations in database.dsdb and application directory path app/.
You can optimize the first configuration with:

\shellcmd{explore -e SimpleICOptimizer -s 1 database.dsdb}

If the optimization was succesfull, explorer should output:
\begin{verbatim}
Best result configuration:
 3
\end{verbatim}
Add\_only option can be used for example if you have an application which isn't
included in application paths defined in database.dsdb but you still want to run
it with the same processor configuration.
First export the optimized configuration (which is id 3 in this case):

\shellcmd{explore -w 3 database.dsdb}

Next schedule the program:

\shellcmd{schedule -t 3.adf -o app\_dir2/app2.scheduled.tpef app\_dir2/app2.seq}

And then run explorer:

\shellcmd{explore -e SimpleICOptimizer -s 3 -u add\_only=true -u tpef=app\_dir2/app2.scheduled.tpef database.dsdb}

The plugin now uses the optimized configuration created earlier and adds
connections needed to run the other program. If the plugin finds a new 
configuration it will be added to the database, otherwise the existing
configuration was already optimal. Because the plugin won't
remove existing connections the new machine configuration is able to run
both programs.

\subsection{Explorer Plugin: RemoveUnconnectedComponents}
Explorer plugin that removes unconnected ports from units or creates
connections to these ports if they are FUs, but removes FUs that have no 
connections. Also removes unconnected buses. If all ports from a unit 
are removed, also the unit is removed.

You can pass a parameter to the plugin:

\begin{tabular}{p{0.20\textwidth}p{0.20\textwidth}
                p{0.60\textwidth}}
\textbf{Param Name} &\textbf{Default Value} &\textbf{Description} \\
\hline
allow\_remove & false & Allows the removal of unconnected ports and FUs \\
\end{tabular}\\

When using the plugin you must define the configuration you wish the plugin
to remove unconnected components. This is done by passing 
\textit{-s <configuration\_ID>} to explorer.

If you do not allow removal the plugin will connect unconnected ports to some
sockets. It can be done with: \\
\shellcmd{explore -e RemoveUnconnectedComponents -s 3 database.dsdb}\\
or \\
\shellcmd{explore -e RemoveUnconnectedComponents -s 3 -u allow\_remove=false
database.dsdb}\\
if you wish to emphasise you do not want to remove components. This will
reconnect the unconnected ports from the configuration 3 in database.dsdb.

And if you want to remove the unconnected components: \\
\shellcmd{explore -e RemoveUnconnectedComponents -s 3 -u allow\_remove=true
database.dsdb}\\

\subsection{Explorer Plugin: GrowMachine}
GrowMachine is an Explorer plugin that adds resources to the machine until
cycle count doesn't go down anymore.

Parameters that can be passed to the GrowMachine are:

\begin{tabular}{p{0.20\textwidth}p{0.20\textwidth}
                p{0.60\textwidth}}
\textbf{Param Name} &\textbf{Default Value} &\textbf{Description} \\
\hline
superiority & 2 & Percentage value of how much faster schedules are wanted
    until cycle count optimization is stopped \\
\end{tabular}\\

Using the plugin requires user to define the configuration he wishes optimize.
This is done by giving \textit{-s <configuration\_ID>} option to the explorer.

Example of usage:

\shellcmd{explore -e GrowMachine -s 1 database.dsdb}

\subsection{Explorer Plugin: ImmediateGenerator}
ImmediateGenerator is an Explorer plugin that creates or modifies machine
instruction templates. Typical usage is to split an instruction template slot
among buses.

Parameters that can be passed to the ImmediateGenerator are:

\begin{tabular}{p{0.20\textwidth}p{0.20\textwidth}
                p{0.60\textwidth}}
\textbf{Param Name} &\textbf{Default Value} &\textbf{Description} \\
\hline
print & false & Print information about machines instruction templates.\\
remove\_it\_name & no default value & Remove instruction template with a
given name \\
add\_it\_name & no default value & Add empty instruction template with a
given name. \\
modify\_it\_name & no default value & Modify instruction template with a
given name. \\
width & 32 & Instruction template supported width.\\
width\_part & 8 & Minimum size of width per slot.\\
split & false & Split immediate among slots.\\
dst\_imm\_unit & no default value & Destination immediate unit.\\
\end{tabular}\\

Example of adding a new 32 width immediate template named newTemplate that is
splitted among busses:

\shellcmd{explore -e ImmediateGenerator -s 1 -u add\_it\_name="newTemplate" -u
width=32 -u split=true database.dsdb}

\subsection{Explorer Plugin: ImplementationSelector}
ImplementationSelector is an Explorer plugin that selects implementations for
units in a given configuration ADF. It creates a new configuration with a IDF.

Parameters that can be passed to the ImmediateGenerator are:

\begin{tabular}{p{0.20\textwidth}p{0.20\textwidth}
                p{0.60\textwidth}}
\textbf{Param Name} &\textbf{Default Value} &\textbf{Description} \\
\hline
ic\_dec & DefaultICDecoder & Name of the ic decoder plugin. \\
ic\_hdb & asic\_130nm\_1.5V.hdb & name of the HDB where the implementations are \\
selected. \\
adf & no default value & An ADF for the implementations are selected if no \\
database is used. \\
\end{tabular}\\

Example of creating implementation for configuration ID 1, in the database:

\shellcmd{explore -e ImplementationSelector -s 1 database.dsdb}

\subsection{Explorer Plugin: MinimizeMachine}
MinimizeMachine is an Explorer plugin that removes resources from a machine
until the real time requirements of applications are not reached anymore.

Parameters that can be passed to the ImmediateGenerator are:

\begin{tabular}{p{0.20\textwidth}p{0.20\textwidth}
                p{0.60\textwidth}}
\textbf{Param Name} &\textbf{Default Value} &\textbf{Description} \\
\hline
min\_bus & true & Minimize buses. \\
min\_fu & true & Minimize function units. \\
min\_rf & true & Minimize register files. \\
frequency & no default value & Running frequency for applications. \\
\end{tabular}\\

Example of minimizing configuration ID 1, in the database, with frequency
50 MHz:

\shellcmd{explore -e MinimizeMachine -s 1 -u frequency=50 database.dsdb}

\chapter{FREQUENTLY ASKED QUESTIONS}
\label{chapter:faq}

\section{Memory Related}

Questions related to memory accessing.

\subsection{What is the endianness of the TTA processors designed with TCE?}

Big endian. At the moment, endianness cannot be customized.

\subsection{What is the alignment of words when reading/writing memory?}

The memory acessing operations in the base operation set (ldq, ldh, ldw,
ldd, stq, sth, stw, and ldd) are aligned with their size. Operations stq/ldq
are for accessing single minimum addressable units (MAU, usually bytes),
thus their alignment is 1 MAU, for sth/ldh it is 2 MAUs, and for stw/ldw 
it is 4 MAUs. Thus, one cannot access, for example, a 4 MAU word at address
3. 

Double precision floating point word operations std/ldd which access
64-bit words are aligned at 8-byte addresses. Thus, if your memory is
addressed in 16-bit units, double words can be stored at addresses
divisible by 4, if memory is byte-addressed, then addresses must be
divisible by 8, and so on.

\subsection{Load Store Unit}
In the LSUs shipped with TCE the two LSB bits are used by the LSU to control
a so called write mask that handles writing of bytes. This means that the
memory address outside the processor is 2 bits narrower than inside the
processor. When you set the data address space width in ProDe, the width
is the address width inside the processor.

\subsection{Instruction memory}
The default GCU assumes that instruction memory is intruction addressable. In
other words the instruction word must fit in the MAU of the instruction
memory. This way the next instruction can be referenced by incrementing the
current memory address by one.

How to interface the instruction memory with an actual memory chip is out of
scope in TCE because there are too many different platforms and chips and
possibilities. But as an advantage this gives the user free hands to
implement almost any kind of memory hierarchy the user wants. Most probably
you must implement a memory adapter to bind the memory chip and the TTA
processor interfaces together.


\chapter{TROUBLESHOOTING}
\label{chapter:troubleshooting}

This chapter gives solutions to common problems encountered while using TCE.

\section{Simulation}

Problems with simulation, both with command line (ttasim) and graphical
user interfaces (proxim) are listed here.

\subsection{Failing to Load Operation Behavior Definitions}

It might be possible that you have defined some custom operations to
\verb|~/.tce/opset/custom| which conflict with your new definitions, or the
simulation behaviors are compiled with an older compiler and not compatible
with your new compiler. Workaround for this is to either delete the old
definitions or rebuild them.

\section{Limitations of the Current Toolset Version}

This section lists the most user-visible limitations placed by the current
toolset version.

\subsection{Integer Width}

The simulator supports only integer computations with maximum word width of
32 bits. Floating point computations can use the single (32 bits) or double
(64 bits) precision floating point types.

\subsection{Instruction Addressing During Simulation}

The details of encoding and compression of the instruction memory are not
taken into account before the actual generation of the bit image of the
instruction memory. This decision was taken to allow simplification in the
other parts of the toolset, and to allow easy "exploration" with different
encodings and compression algorithms in the bit generation phase.

This implies that every time you see an instruction address in
architectural simulation, you are actually seeing an instruction index.
That is, instruction addressing (one instruction per instruction memory
address) is assumed.

We might change this in the future toolset versions to allow seeing exact
instruction memory addresses during simulation, if that is seen as a
necessity. Currently it does not seem to be a very important feature.

\subsection{Data Memory Addressing}

There is no tool to map data memory accesses in the source code to the
actual target's memories. Therefore, you need to have a data memory which
provides byte-addressing with 32-bit words. The data will be accessed using
operations LDQ, LDH, LDW, STQ, STH, STW, which access the memory in 1 (q),
2 (h), and 4 (w) byte chunks. This should not be a problem, as it is rather
easy to implement byte-addressing in case the actual memory is of width of
2's exponent multiple of the byte. The parallel assembler allows any kind
of minimum addressable units (MAU) in the load/store units. In that case,
LDQ/STQ just access a single MAU, etc. One should keep in mind the 32-bit
integer word limitation of simulation. Thus, if the MAU is 32-bits, one
cannot use LDH or LDW because they would require 64 and 128 bits,
respectively.

\subsection{Ideal Memory Model in Simulation}

The simulator assumes ideal memory model which generates no stalls and
returns data for the next instruction. This so called 'Ideal SRAM' model
allows modeling all types of memories in the point of view of the
programmer. It is up to the load/store unit implementation to generate the
lock signals in case the memory model does not match the ideal model.

There are hooks for adding more memory models which generate lock signals
in the simulation, but for the v1.0 the simulator does not provide other
memory models, and thus does not support lock cycle simulation.

\subsection{Guards}

The guard support as specified in the ADF specification~\cite{ADF-specs} is
only partially supported in TCE. `Operators other than logical negation are
not supported. That is, supported guards always ``watch'' a single register
(FU output or a GPR). In addition, the shipped default scheduling algorithm
in compiler backend requires a register guard. Thus, if more exotic guarded
execution is required, one has to write the programs in parallel assembly
(Section~\ref{section:TCEAsm}).

\subsection{Operation Pipeline Description Limitations}

Even though supported by the ADF and ProDe, writing of operands after
triggering an operation is not supported neither by the compiler nor
the simulator. However, setting different latencies for outputs of 
multi-result operations is supported. For example, using this
feature one can have an iterative operation pipeline which computes 
several results which are ready after the count of stages in an iteration.

\subsection{Encoding of XML Files}

TCE uses XML to store data of the architectures and implementation 
locations (see Section~\ref{sec:adf} and Section~\ref{sec:idf}). The
encoding of the XML files must be in 7-bit ascii. If other encodings are
used, the result is undefined.

\chapter{Copyright notices}

Here are the copyright notices of the libraries used in the software.

\section{Xerces}

Xerces-C++ is released under the terms of the Apache License, version
2.0.  The complete text is presented here.


                                 Apache License
                           Version 2.0, January 2004
                        http://www.apache.org/licenses/

   TERMS AND CONDITIONS FOR USE, REPRODUCTION, AND DISTRIBUTION

   1. Definitions.

      "License" shall mean the terms and conditions for use, reproduction,
      and distribution as defined by Sections 1 through 9 of this document.

      "Licensor" shall mean the copyright owner or entity authorized by
      the copyright owner that is granting the License.

      "Legal Entity" shall mean the union of the acting entity and all
      other entities that control, are controlled by, or are under common
      control with that entity. For the purposes of this definition,
      "control" means (i) the power, direct or indirect, to cause the
      direction or management of such entity, whether by contract or
      otherwise, or (ii) ownership of fifty percent (50%) or more of the
      outstanding shares, or (iii) beneficial ownership of such entity.

      "You" (or "Your") shall mean an individual or Legal Entity
      exercising permissions granted by this License.

      "Source" form shall mean the preferred form for making modifications,
      including but not limited to software source code, documentation
      source, and configuration files.

      "Object" form shall mean any form resulting from mechanical
      transformation or translation of a Source form, including but
      not limited to compiled object code, generated documentation,
      and conversions to other media types.

      "Work" shall mean the work of authorship, whether in Source or
      Object form, made available under the License, as indicated by a
      copyright notice that is included in or attached to the work
      (an example is provided in the Appendix below).

      "Derivative Works" shall mean any work, whether in Source or Object
      form, that is based on (or derived from) the Work and for which the
      editorial revisions, annotations, elaborations, or other modifications
      represent, as a whole, an original work of authorship. For the purposes
      of this License, Derivative Works shall not include works that remain
      separable from, or merely link (or bind by name) to the interfaces of,
      the Work and Derivative Works thereof.

      "Contribution" shall mean any work of authorship, including
      the original version of the Work and any modifications or additions
      to that Work or Derivative Works thereof, that is intentionally
      submitted to Licensor for inclusion in the Work by the copyright owner
      or by an individual or Legal Entity authorized to submit on behalf of
      the copyright owner. For the purposes of this definition, "submitted"
      means any form of electronic, verbal, or written communication sent
      to the Licensor or its representatives, including but not limited to
      communication on electronic mailing lists, source code control systems,
      and issue tracking systems that are managed by, or on behalf of, the
      Licensor for the purpose of discussing and improving the Work, but
      excluding communication that is conspicuously marked or otherwise
      designated in writing by the copyright owner as "Not a Contribution."

      "Contributor" shall mean Licensor and any individual or Legal Entity
      on behalf of whom a Contribution has been received by Licensor and
      subsequently incorporated within the Work.

   2. Grant of Copyright License. Subject to the terms and conditions of
      this License, each Contributor hereby grants to You a perpetual,
      worldwide, non-exclusive, no-charge, royalty-free, irrevocable
      copyright license to reproduce, prepare Derivative Works of,
      publicly display, publicly perform, sublicense, and distribute the
      Work and such Derivative Works in Source or Object form.

   3. Grant of Patent License. Subject to the terms and conditions of
      this License, each Contributor hereby grants to You a perpetual,
      worldwide, non-exclusive, no-charge, royalty-free, irrevocable
      (except as stated in this section) patent license to make, have made,
      use, offer to sell, sell, import, and otherwise transfer the Work,
      where such license applies only to those patent claims licensable
      by such Contributor that are necessarily infringed by their
      Contribution(s) alone or by combination of their Contribution(s)
      with the Work to which such Contribution(s) was submitted. If You
      institute patent litigation against any entity (including a
      cross-claim or counterclaim in a lawsuit) alleging that the Work
      or a Contribution incorporated within the Work constitutes direct
      or contributory patent infringement, then any patent licenses
      granted to You under this License for that Work shall terminate
      as of the date such litigation is filed.

   4. Redistribution. You may reproduce and distribute copies of the
      Work or Derivative Works thereof in any medium, with or without
      modifications, and in Source or Object form, provided that You
      meet the following conditions:

      (a) You must give any other recipients of the Work or
          Derivative Works a copy of this License; and

      (b) You must cause any modified files to carry prominent notices
          stating that You changed the files; and

      (c) You must retain, in the Source form of any Derivative Works
          that You distribute, all copyright, patent, trademark, and
          attribution notices from the Source form of the Work,
          excluding those notices that do not pertain to any part of
          the Derivative Works; and

      (d) If the Work includes a "NOTICE" text file as part of its
          distribution, then any Derivative Works that You distribute must
          include a readable copy of the attribution notices contained
          within such NOTICE file, excluding those notices that do not
          pertain to any part of the Derivative Works, in at least one
          of the following places: within a NOTICE text file distributed
          as part of the Derivative Works; within the Source form or
          documentation, if provided along with the Derivative Works; or,
          within a display generated by the Derivative Works, if and
          wherever such third-party notices normally appear. The contents
          of the NOTICE file are for informational purposes only and
          do not modify the License. You may add Your own attribution
          notices within Derivative Works that You distribute, alongside
          or as an addendum to the NOTICE text from the Work, provided
          that such additional attribution notices cannot be construed
          as modifying the License.

      You may add Your own copyright statement to Your modifications and
      may provide additional or different license terms and conditions
      for use, reproduction, or distribution of Your modifications, or
      for any such Derivative Works as a whole, provided Your use,
      reproduction, and distribution of the Work otherwise complies with
      the conditions stated in this License.

   5. Submission of Contributions. Unless You explicitly state otherwise,
      any Contribution intentionally submitted for inclusion in the Work
      by You to the Licensor shall be under the terms and conditions of
      this License, without any additional terms or conditions.
      Notwithstanding the above, nothing herein shall supersede or modify
      the terms of any separate license agreement you may have executed
      with Licensor regarding such Contributions.

   6. Trademarks. This License does not grant permission to use the trade
      names, trademarks, service marks, or product names of the Licensor,
      except as required for reasonable and customary use in describing the
      origin of the Work and reproducing the content of the NOTICE file.

   7. Disclaimer of Warranty. Unless required by applicable law or
      agreed to in writing, Licensor provides the Work (and each
      Contributor provides its Contributions) on an "AS IS" BASIS,
      WITHOUT WARRANTIES OR CONDITIONS OF ANY KIND, either express or
      implied, including, without limitation, any warranties or conditions
      of TITLE, NON-INFRINGEMENT, MERCHANTABILITY, or FITNESS FOR A
      PARTICULAR PURPOSE. You are solely responsible for determining the
      appropriateness of using or redistributing the Work and assume any
      risks associated with Your exercise of permissions under this License.

   8. Limitation of Liability. In no event and under no legal theory,
      whether in tort (including negligence), contract, or otherwise,
      unless required by applicable law (such as deliberate and grossly
      negligent acts) or agreed to in writing, shall any Contributor be
      liable to You for damages, including any direct, indirect, special,
      incidental, or consequential damages of any character arising as a
      result of this License or out of the use or inability to use the
      Work (including but not limited to damages for loss of goodwill,
      work stoppage, computer failure or malfunction, or any and all
      other commercial damages or losses), even if such Contributor
      has been advised of the possibility of such damages.

   9. Accepting Warranty or Additional Liability. While redistributing
      the Work or Derivative Works thereof, You may choose to offer,
      and charge a fee for, acceptance of support, warranty, indemnity,
      or other liability obligations and/or rights consistent with this
      License. However, in accepting such obligations, You may act only
      on Your own behalf and on Your sole responsibility, not on behalf
      of any other Contributor, and only if You agree to indemnify,
      defend, and hold each Contributor harmless for any liability
      incurred by, or claims asserted against, such Contributor by reason
      of your accepting any such warranty or additional liability.

   END OF TERMS AND CONDITIONS

   APPENDIX: How to apply the Apache License to your work.

      To apply the Apache License to your work, attach the following
      boilerplate notice, with the fields enclosed by brackets "[]"
      replaced with your own identifying information. (Do Not include
      the brackets!)  The text should be enclosed in the appropriate
      comment syntax for the file format. We also recommend that a
      file or class name and description of purpose be included on the
      same "printed page" as the copyright notice for easier
      identification within third-party archives.

   Copyright [yyyy] [name of copyright owner]

   Licensed under the Apache License, Version 2.0 (the "License");
   you may not use this file except in compliance with the License.
   You may obtain a copy of the License at

       http://www.apache.org/licenses/LICENSE-2.0

   Unless required by applicable law or agreed to in writing, software
   distributed under the License is distributed on an "AS IS" BASIS,
   WITHOUT WARRANTIES OR CONDITIONS OF ANY KIND, either express or implied.
   See the License for the specific language governing permissions and
   limitations under the License.

\section{wxWidgets}

Copyright (c) 1998 Julian Smart, Robert Roebling [, ...]

Everyone is permitted to copy and distribute verbatim copies
of this licence document, but changing it is not allowed.

WXWINDOWS LIBRARY LICENCE

TERMS AND CONDITIONS FOR COPYING, DISTRIBUTION AND MODIFICATION

This library is free software; you can redistribute it and/or modify it
under the terms of the GNU Library General Public Licence as published by
the Free Software Foundation; either version 2 of the Licence, or (at
your option) any later version.

This library is distributed in the hope that it will be useful, but
WITHOUT ANY WARRANTY; without even the implied warranty of
MERCHANTABILITY or FITNESS FOR A PARTICULAR PURPOSE. See the GNU Library
General Public Licence for more details.

You should have received a copy of the GNU Library General Public Licence
along with this software, usually in a file named COPYING.LIB. If not,
write to the Free Software Foundation, Inc., 59 Temple Place, Suite 330,
Boston, MA 02111-1307 USA.

EXCEPTION NOTICE

1. As a special exception, the copyright holders of this library give
permission for additional uses of the text contained in this release of
the library as licenced under the wxWindows Library Licence, applying
either version 3 of the Licence, or (at your option) any later version of
the Licence as published by the copyright holders of version 3 of the
Licence document.

2. The exception is that you may use, copy, link, modify and distribute
under the user's own terms, binary object code versions of works based
on the Library.

3. If you copy code from files distributed under the terms of the GNU
General Public Licence or the GNU Library General Public Licence into a
copy of this library, as this licence permits, the exception does not
apply to the code that you add in this way. To avoid misleading anyone as
to the status of such modified files, you must delete this exception
notice from such code and/or adjust the licensing conditions notice
accordingly.

4. If you write modifications of your own for this library, it is your
choice whether to permit this exception to apply to your modifications.
If you do not wish that, you must delete the exception notice from such
code and/or adjust the licensing conditions notice accordingly.

\section{L-GPL}

\begin{center}
	  GNU LIBRARY GENERAL PUBLIC LICENSE

	  ==================================

                Version 2, June 1991

 Copyright (C) 1991 Free Software Foundation, Inc.
                    675 Mass Ave, Cambridge, MA 02139, USA
 Everyone is permitted to copy and distribute verbatim copies
 of this license document, but changing it is not allowed.

[This is the first released version of the library GPL.  It is
 numbered 2 because it goes with version 2 of the ordinary GPL.]

Preamble
\end{center}

The licenses for most software are designed to take away your
freedom to share and change it.  By contrast, the GNU General
Public Licenses are intended to guarantee your freedom to share
and change free software--to make sure the software is free for
all its users.

This license, the Library General Public License, applies to
some specially designated Free Software Foundation software, and
to any other libraries whose authors decide to use it.  You can
use it for your libraries, too.

When we speak of free software, we are referring to freedom, not
price.  Our General Public Licenses are designed to make sure
that you have the freedom to distribute copies of free software
(and charge for this service if you wish), that you receive
source code or can get it if you want it, that you can change
the software or use pieces of it in new free programs; and that
you know you can do these things.

To protect your rights, we need to make restrictions that forbid
anyone to deny you these rights or to ask you to surrender the
rights. These restrictions translate to certain responsibilities
for you if you distribute copies of the library, or if you
modify it.

For example, if you distribute copies of the library, whether
gratis or for a fee, you must give the recipients all the rights
that we gave you.  You must make sure that they, too, receive or
can get the source code.  If you link a program with the
library, you must provide complete object files to the
recipients so that they can relink them with the library, after
making changes to the library and recompiling it.  And you must
show them these terms so they know their rights.

Our method of protecting your rights has two steps: (1)
copyright the library, and (2) offer you this license which
gives you legal permission to copy, distribute and/or modify the
library.

Also, for each distributor's protection, we want to make certain
that everyone understands that there is no warranty for this
free library.  If the library is modified by someone else and
passed on, we want its recipients to know that what they have is
not the original version, so that any problems introduced by
others will not reflect on the original authors' reputations.
 
Finally, any free program is threatened constantly by software
patents.  We wish to avoid the danger that companies
distributing free software will individually obtain patent
licenses, thus in effect transforming the program into
proprietary software.  To prevent this, we have made it clear
that any patent must be licensed for everyone's free use or not
licensed at all.

Most GNU software, including some libraries, is covered by the
ordinary GNU General Public License, which was designed for
utility programs.  This license, the GNU Library General Public
License, applies to certain designated libraries.  This license
is quite different from the ordinary one; be sure to read it in
full, and do not assume that anything in it is the same as in the
ordinary license.

The reason we have a separate public license for some libraries
is that they blur the distinction we usually make between
modifying or adding to a program and simply using it.  Linking a
program with a library, without changing the library, is in some
sense simply using the library, and is analogous to running a
utility program or application program.  However, in a textual
and legal sense, the linked executable is a combined work, a
derivative of the original library, and the ordinary General
Public License treats it as such.

Because of this blurred distinction, using the ordinary General
Public License for libraries did not effectively promote
software sharing, because most developers did not use the
libraries.  We concluded that weaker conditions might promote
sharing better.

However, unrestricted linking of non-free programs would deprive
the users of those programs of all benefit from the free status
of the libraries themselves.  This Library General Public
License is intended to permit developers of non-free programs to
use free libraries, while preserving your freedom as a user of
such programs to change the free libraries that are incorporated
in them.  (We have not seen how to achieve this as regards
changes in header files, but we have achieved it as regards
changes in the actual functions of the Library.)  The hope is
that this will lead to faster development of free libraries.

The precise terms and conditions for copying, distribution and
modification follow.  Pay close attention to the difference
between a "work based on the library" and a "work that uses the
library".  The former contains code derived from the library,
while the latter only works together with the library.

Note that it is possible for a library to be covered by the
ordinary General Public License rather than by this special one.

\begin{center}
                GNU LIBRARY GENERAL PUBLIC LICENSE

 TERMS AND CONDITIONS FOR COPYING, DISTRIBUTION AND MODIFICATION
\end{center}

0. This License Agreement applies to any software library which
contains a notice placed by the copyright holder or other
authorized party saying it may be distributed under the terms of
this Library General Public License (also called "this
License").  Each licensee is addressed as "you".

A "library" means a collection of software functions and/or data
prepared so as to be conveniently linked with application
programs (which use some of those functions and data) to form
executables.

The "Library", below, refers to any such software library or
work which has been distributed under these terms.  A "work
based on the Library" means either the Library or any derivative
work under copyright law: that is to say, a work containing the
Library or a portion of it, either verbatim or with
modifications and/or translated straightforwardly into another
language.  (Hereinafter, translation is included without
limitation in the term "modification".)

"Source code" for a work means the preferred form of the work
for making modifications to it.  For a library, complete source
code means all the source code for all modules it contains, plus
any associated interface definition files, plus the scripts used
to control compilation and installation of the library.

Activities other than copying, distribution and modification are
not covered by this License; they are outside its scope.  The
act of running a program using the Library is not restricted,
and output from such a program is covered only if its contents
constitute a work based on the Library (independent of the use
of the Library in a tool for writing it).  Whether that is true
depends on what the Library does and what the program that uses
the Library does.
  
1. You may copy and distribute verbatim copies of the Library's
complete source code as you receive it, in any medium, provided
that you conspicuously and appropriately publish on each copy an
appropriate copyright notice and disclaimer of warranty; keep
intact all the notices that refer to this License and to the
absence of any warranty; and distribute a copy of this License
along with the Library.

You may charge a fee for the physical act of transferring a
copy, and you may at your option offer warranty protection in
exchange for a fee.
 
2. You may modify your copy or copies of the Library or any
portion of it, thus forming a work based on the Library, and
copy and distribute such modifications or work under the terms
of Section 1 above, provided that you also meet all of these
conditions:

    a) The modified work must itself be a software library.

    b) You must cause the files modified to carry prominent notices
    stating that you changed the files and the date of any change.

    c) You must cause the whole of the work to be licensed at no
    charge to all third parties under the terms of this License.

    d) If a facility in the modified Library refers to a function or a
    table of data to be supplied by an application program that uses
    the facility, other than as an argument passed when the facility
    is invoked, then you must make a good faith effort to ensure that,
    in the event an application does not supply such function or
    table, the facility still operates, and performs whatever part of
    its purpose remains meaningful.

    (For example, a function in a library to compute square roots has
    a purpose that is entirely well-defined independent of the
    application.  Therefore, Subsection 2d requires that any
    application-supplied function or table used by this function must
    be optional: if the application does not supply it, the square
    root function must still compute square roots.)

These requirements apply to the modified work as a whole.  If
identifiable sections of that work are not derived from the
Library, and can be reasonably considered independent and
separate works in themselves, then this License, and its terms,
do not apply to those sections when you distribute them as
separate works.  But when you distribute the same sections as
part of a whole which is a work based on the Library, the
distribution of the whole must be on the terms of this License,
whose permissions for other licensees extend to the entire
whole, and thus to each and every part regardless of who wrote
it.

Thus, it is not the intent of this section to claim rights or
contest your rights to work written entirely by you; rather, the
intent is to exercise the right to control the distribution of
derivative or collective works based on the Library.

In addition, mere aggregation of another work not based on the
Library with the Library (or with a work based on the Library)
on a volume of a storage or distribution medium does not bring
the other work under the scope of this License.

3. You may opt to apply the terms of the ordinary GNU General
Public License instead of this License to a given copy of the
Library.  To do this, you must alter all the notices that refer
to this License, so that they refer to the ordinary GNU General
Public License, version 2, instead of to this License.  (If a
newer version than version 2 of the ordinary GNU General Public
License has appeared, then you can specify that version instead
if you wish.)  Do not make any other change in these notices.
 
Once this change is made in a given copy, it is irreversible for
that copy, so the ordinary GNU General Public License applies to
all subsequent copies and derivative works made from that copy.

This option is useful when you wish to copy part of the code of
the Library into a program that is not a library.

4. You may copy and distribute the Library (or a portion or
derivative of it, under Section 2) in object code or executable
form under the terms of Sections 1 and 2 above provided that you
accompany it with the complete corresponding machine-readable
source code, which must be distributed under the terms of
Sections 1 and 2 above on a medium customarily used for software
interchange.

If distribution of object code is made by offering access to
copy from a designated place, then offering equivalent access to
copy the source code from the same place satisfies the
requirement to distribute the source code, even though third
parties are not compelled to copy the source along with the
object code.

5. A program that contains no derivative of any portion of the
Library, but is designed to work with the Library by being
compiled or linked with it, is called a "work that uses the
Library".  Such a work, in isolation, is not a derivative work
of the Library, and therefore falls outside the scope of this
License.

However, linking a "work that uses the Library" with the Library
creates an executable that is a derivative of the Library
(because it contains portions of the Library), rather than a
"work that uses the library".  The executable is therefore
covered by this License. Section 6 states terms for distribution
of such executables.

When a "work that uses the Library" uses material from a header
file that is part of the Library, the object code for the work
may be a derivative work of the Library even though the source
code is not. Whether this is true is especially significant if
the work can be linked without the Library, or if the work is
itself a library.  The threshold for this to be true is not
precisely defined by law.

If such an object file uses only numerical parameters, data
structure layouts and accessors, and small macros and small
inline functions (ten lines or less in length), then the use of
the object file is unrestricted, regardless of whether it is
legally a derivative work.  (Executables containing this object
code plus portions of the Library will still fall under Section
6.)

Otherwise, if the work is a derivative of the Library, you may
distribute the object code for the work under the terms of
Section 6. Any executables containing that work also fall under
Section 6, whether or not they are linked directly with the
Library itself.
 
6. As an exception to the Sections above, you may also compile
or link a "work that uses the Library" with the Library to
produce a work containing portions of the Library, and
distribute that work under terms of your choice, provided that
the terms permit modification of the work for the customer's own
use and reverse engineering for debugging such modifications.

You must give prominent notice with each copy of the work that
the Library is used in it and that the Library and its use are
covered by this License.  You must supply a copy of this
License.  If the work during execution displays copyright
notices, you must include the copyright notice for the Library
among them, as well as a reference directing the user to the
copy of this License.  Also, you must do one of these things:

    a) Accompany the work with the complete corresponding
    machine-readable source code for the Library including whatever
    changes were used in the work (which must be distributed under
    Sections 1 and 2 above); and, if the work is an executable linked
    with the Library, with the complete machine-readable "work that
    uses the Library", as object code and/or source code, so that the
    user can modify the Library and then relink to produce a modified
    executable containing the modified Library.  (It is understood
    that the user who changes the contents of definitions files in the
    Library will not necessarily be able to recompile the application
    to use the modified definitions.)

    b) Accompany the work with a written offer, valid for at
    least three years, to give the same user the materials
    specified in Subsection 6a, above, for a charge no more
    than the cost of performing this distribution.

    c) If distribution of the work is made by offering access to copy
    from a designated place, offer equivalent access to copy the above
    specified materials from the same place.

    d) Verify that the user has already received a copy of these
    materials or that you have already sent this user a copy.

For an executable, the required form of the "work that uses the
Library" must include any data and utility programs needed for
reproducing the executable from it.  However, as a special
exception, the source code distributed need not include anything
that is normally distributed (in either source or binary form)
with the major components (compiler, kernel, and so on) of the
operating system on which the executable runs, unless that
component itself accompanies the executable.

It may happen that this requirement contradicts the license
restrictions of other proprietary libraries that do not normally
accompany the operating system.  Such a contradiction means you
cannot use both them and the Library together in an executable
that you distribute.
 
7. You may place library facilities that are a work based on the
Library side-by-side in a single library together with other
library facilities not covered by this License, and distribute
such a combined library, provided that the separate distribution
of the work based on the Library and of the other library
facilities is otherwise permitted, and provided that you do
these two things:

    a) Accompany the combined library with a copy of the same work
    based on the Library, uncombined with any other library
    facilities.  This must be distributed under the terms of the
    Sections above.

    b) Give prominent notice with the combined library of the fact
    that part of it is a work based on the Library, and explaining
    where to find the accompanying uncombined form of the same work.

8. You may not copy, modify, sublicense, link with, or
distribute the Library except as expressly provided under this
License.  Any attempt otherwise to copy, modify, sublicense,
link with, or distribute the Library is void, and will
automatically terminate your rights under this License.
However, parties who have received copies, or rights, from you
under this License will not have their licenses terminated so
long as such parties remain in full compliance.

9. You are not required to accept this License, since you have
not signed it.  However, nothing else grants you permission to
modify or distribute the Library or its derivative works.  These
actions are prohibited by law if you do not accept this
License.  Therefore, by modifying or distributing the Library
(or any work based on the Library), you indicate your acceptance
of this License to do so, and all its terms and conditions for
copying, distributing or modifying the Library or works based on
it.

10. Each time you redistribute the Library (or any work based on
the Library), the recipient automatically receives a license
from the original licensor to copy, distribute, link with or
modify the Library subject to these terms and conditions.  You
may not impose any further restrictions on the recipients'
exercise of the rights granted herein. You are not responsible
for enforcing compliance by third parties to this License.
 
11. If, as a consequence of a court judgment or allegation of
patent infringement or for any other reason (not limited to
patent issues), conditions are imposed on you (whether by court
order, agreement or otherwise) that contradict the conditions of
this License, they do not excuse you from the conditions of this
License.  If you cannot distribute so as to satisfy
simultaneously your obligations under this License and any other
pertinent obligations, then as a consequence you may not
distribute the Library at all.  For example, if a patent license
would not permit royalty-free redistribution of the Library by
all those who receive copies directly or indirectly through you,
then the only way you could satisfy both it and this License
would be to refrain entirely from distribution of the Library.

If any portion of this section is held invalid or unenforceable
under any particular circumstance, the balance of the section is
intended to apply, and the section as a whole is intended to
apply in other circumstances.

It is not the purpose of this section to induce you to infringe
any patents or other property right claims or to contest
validity of any such claims; this section has the sole purpose
of protecting the integrity of the free software distribution
system which is implemented by public license practices.  Many
people have made generous contributions to the wide range of
software distributed through that system in reliance on
consistent application of that system; it is up to the
author/donor to decide if he or she is willing to distribute
software through any other system and a licensee cannot impose
that choice.

This section is intended to make thoroughly clear what is
believed to be a consequence of the rest of this License.

12. If the distribution and/or use of the Library is restricted
in certain countries either by patents or by copyrighted
interfaces, the original copyright holder who places the Library
under this License may add an explicit geographical distribution
limitation excluding those countries, so that distribution is
permitted only in or among countries not thus excluded.  In such
case, this License incorporates the limitation as if written in
the body of this License.

13. The Free Software Foundation may publish revised and/or new
versions of the Library General Public License from time to
time. Such new versions will be similar in spirit to the present
version, but may differ in detail to address new problems or
concerns.

Each version is given a distinguishing version number.  If the
Library specifies a version number of this License which applies
to it and "any later version", you have the option of following
the terms and conditions either of that version or of any later
version published by the Free Software Foundation.  If the
Library does not specify a license version number, you may
choose any version ever published by the Free Software
Foundation.

14. If you wish to incorporate parts of the Library into other
free programs whose distribution conditions are incompatible
with these, write to the author to ask for permission.  For
software which is copyrighted by the Free Software Foundation,
write to the Free Software Foundation; we sometimes make
exceptions for this.  Our decision will be guided by the two
goals of preserving the free status of all derivatives of our
free software and of promoting the sharing and reuse of software
generally.

\begin{center}
NO WARRANTY
\end{center}

  15. BECAUSE THE LIBRARY IS LICENSED FREE OF CHARGE, THERE IS NO
WARRANTY FOR THE LIBRARY, TO THE EXTENT PERMITTED BY APPLICABLE LAW.
EXCEPT WHEN OTHERWISE STATED IN WRITING THE COPYRIGHT HOLDERS AND/OR
OTHER PARTIES PROVIDE THE LIBRARY "AS IS" WITHOUT WARRANTY OF ANY KIND,
EITHER EXPRESSED OR IMPLIED, INCLUDING, BUT NOT LIMITED TO, THE
IMPLIED WARRANTIES OF MERCHANTABILITY AND FITNESS FOR A PARTICULAR
PURPOSE.  THE ENTIRE RISK AS TO THE QUALITY AND PERFORMANCE OF THE
LIBRARY IS WITH YOU.  SHOULD THE LIBRARY PROVE DEFECTIVE, YOU ASSUME
THE COST OF ALL NECESSARY SERVICING, REPAIR OR CORRECTION.

  16. IN NO EVENT UNLESS REQUIRED BY APPLICABLE LAW OR AGREED TO IN
WRITING WILL ANY COPYRIGHT HOLDER, OR ANY OTHER PARTY WHO MAY MODIFY
AND/OR REDISTRIBUTE THE LIBRARY AS PERMITTED ABOVE, BE LIABLE TO YOU
FOR DAMAGES, INCLUDING ANY GENERAL, SPECIAL, INCIDENTAL OR CONSEQUENTIAL
DAMAGES ARISING OUT OF THE USE OR INABILITY TO USE THE
LIBRARY (INCLUDING BUT NOT LIMITED TO LOSS OF DATA OR DATA BEING
RENDERED INACCURATE OR LOSSES SUSTAINED BY YOU OR THIRD PARTIES OR A
FAILURE OF THE LIBRARY TO OPERATE WITH ANY OTHER SOFTWARE), EVEN IF
SUCH HOLDER OR OTHER PARTY HAS BEEN ADVISED OF THE POSSIBILITY OF SUCH DAMAGES.

\begin{center}
END OF TERMS AND CONDITIONS
\end{center}

\begin{center}
Appendix: How to Apply These Terms to Your New Libraries
\end{center}

If you develop a new library, and you want it to be of the
greatest possible use to the public, we recommend making it free
software that everyone can redistribute and change.  You can do
so by permitting redistribution under these terms (or,
alternatively, under the terms of the ordinary General Public
License).

To apply these terms, attach the following notices to the
library.  It is safest to attach them to the start of each
source file to most effectively convey the exclusion of
warranty; and each file should have at least the "copyright"
line and a pointer to where the full notice is found.

    <one line to give the library's name and a brief idea of what it does.>
    Copyright (C) <year>  <name of author>

    This library is free software; you can redistribute it and/or
    modify it under the terms of the GNU Library General Public
    License as published by the Free Software Foundation; either
    version 2 of the License, or (at your option) any later version.

    This library is distributed in the hope that it will be useful,
    but WITHOUT ANY WARRANTY; without even the implied warranty of
    MERCHANTABILITY or FITNESS FOR A PARTICULAR PURPOSE.  See the GNU
    Library General Public License for more details.

    You should have received a copy of the GNU Library General Public
    License along with this library; if not, write to the Free
    Software Foundation, Inc., 675 Mass Ave, Cambridge, MA 02139, USA.

Also add information on how to contact you by electronic and paper mail.

You should also get your employer (if you work as a programmer) or your
school, if any, to sign a "copyright disclaimer" for the library, if
necessary.  Here is a sample; alter the names:

  Yoyodyne, Inc., hereby disclaims all copyright interest in the
  library `Frob' (a library for tweaking knobs) written by James Random Hacker.

  <signature of Ty Coon>, 1 April 1990
  Ty Coon, President of Vice

That's all there is to it!

\section{TCL}

Tcl/Tk License Terms

This software is copyrighted by the Regents of the University of California, Sun Microsystems, Inc., Scriptics Corporation, and other parties. The following terms apply to all files associated with the software unless explicitly disclaimed in individual files.

The authors hereby grant permission to use, copy, modify, distribute, and license this software and its documentation for any purpose, provided that existing copyright notices are retained in all copies and that this notice is included verbatim in any distributions. No written agreement, license, or royalty fee is required for any of the authorized uses. Modifications to this software may be copyrighted by their authors and need not follow the licensing terms described here, provided that the new terms are clearly indicated on the first page of each file where they apply.

IN NO EVENT SHALL THE AUTHORS OR DISTRIBUTORS BE LIABLE TO ANY PARTY FOR DIRECT, INDIRECT, SPECIAL, INCIDENTAL, OR CONSEQUENTIAL DAMAGES ARISING OUT OF THE USE OF THIS SOFTWARE, ITS DOCUMENTATION, OR ANY DERIVATIVES THEREOF, EVEN IF THE AUTHORS HAVE BEEN ADVISED OF THE POSSIBILITY OF SUCH DAMAGE.

THE AUTHORS AND DISTRIBUTORS SPECIFICALLY DISCLAIM ANY WARRANTIES, INCLUDING, BUT NOT LIMITED TO, THE IMPLIED WARRANTIES OF MERCHANTABILITY, FITNESS FOR A PARTICULAR PURPOSE, AND NON-INFRINGEMENT. THIS SOFTWARE IS PROVIDED ON AN "AS IS" BASIS, AND THE AUTHORS AND DISTRIBUTORS HAVE NO OBLIGATION TO PROVIDE MAINTENANCE, SUPPORT, UPDATES, ENHANCEMENTS, OR MODIFICATIONS.

GOVERNMENT USE: If you are acquiring this software on behalf of the U.S. government, the Government shall have only "Restricted Rights" in the software and related documentation as defined in the Federal Acquisition Regulations (FARs) in Clause 52.227.19 (c) (2). If you are acquiring the software on behalf of the Department of Defense, the software shall be classified as "Commercial Computer Software" and the Government shall have only "Restricted Rights" as defined in Clause 252.227-7013 (c) (1) of DFARs. Notwithstanding the foregoing, the authors grant the U.S. Government and others acting in its behalf permission to use and distribute the software in accordance with the terms specified in this license.

\section{SQLite}

SQLite is public domain. For more information, see http://www.sqlite.org/copyright.html

\section{Editline}

   -

   Copyright (c) 1997 The NetBSD Foundation, Inc.

   All rights reserved.
  
   This code is derived from software contributed to The NetBSD Foundation
   by Jaromir Dolecek.
  
   Redistribution and use in source and binary forms, with or without
   modification, are permitted provided that the following conditions
   are met:

   1. Redistributions of source code must retain the above copyright
      notice, this list of conditions and the following disclaimer.

   2. Redistributions in binary form must reproduce the above copyright
      notice, this list of conditions and the following disclaimer in the
      documentation and/or other materials provided with the distribution.

   3. All advertising materials mentioning features or use of this software
      must display the following acknowledgement:
        This product includes software developed by the NetBSD
        Foundation, Inc. and its contributors.

   4. Neither the name of The NetBSD Foundation nor the names of its
      contributors may be used to endorse or promote products derived
      from this software without specific prior written permission.
  
   THIS SOFTWARE IS PROVIDED BY THE NETBSD FOUNDATION, INC. AND CONTRIBUTORS
   ``AS IS'' AND ANY EXPRESS OR IMPLIED WARRANTIES, INCLUDING, BUT NOT LIMITED
   TO, THE IMPLIED WARRANTIES OF MERCHANTABILITY AND FITNESS FOR A PARTICULAR
   PURPOSE ARE DISCLAIMED.  IN NO EVENT SHALL THE FOUNDATION OR CONTRIBUTORS
   BE LIABLE FOR ANY DIRECT, INDIRECT, INCIDENTAL, SPECIAL, EXEMPLARY, OR
   CONSEQUENTIAL DAMAGES (INCLUDING, BUT NOT LIMITED TO, PROCUREMENT OF
   SUBSTITUTE GOODS OR SERVICES; LOSS OF USE, DATA, OR PROFITS; OR BUSINESS
   INTERRUPTION) HOWEVER CAUSED AND ON ANY THEORY OF LIABILITY, WHETHER IN
   CONTRACT, STRICT LIABILITY, OR TORT (INCLUDING NEGLIGENCE OR OTHERWISE)
   ARISING IN ANY WAY OUT OF THE USE OF THIS SOFTWARE, EVEN IF ADVISED OF THE
   POSSIBILITY OF SUCH DAMAGE.

\bibliographystyle{alpha}
\cleardoublepage
% Equivalent to a chapter in the table of contents
\addcontentsline{toc}{chapter}{BIBLIOGRAPHY}
\bibliography{Bibliography}

\end{document}
