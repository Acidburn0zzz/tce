\documentclass[twoside]{tce}

\begin{document}
\author{Lasse Laasonen}
\title{Hardware Database}
\ver{0.5.3}
\firstday{08.09.2005}
\lastday{19.01.2005}
% id number in S- sequence
\docnum{(undefined)}
% draft/complete/committed
\state{draft}
\maketitle

% Version history
\chapter*{Version History}

\begin{table}[htb]
\begin{center}
\begin{tabular}{|p{0.10\textwidth}|p{0.15\textwidth}
		|p{0.15\textwidth}|p{0.46\textwidth}|}
\hline
\textbf{Version} &\textbf{Date} &\textbf{Author} &
\textbf{Description}\\
\hline
0.1  & 09.08.2005 & L. Laasonen     & 
Contents of the database and most of the tables defined.\\
0.2  & 09.09.2005 & L. Laasonen     &
Changes in table names and in data contents concerning FU architecture.\\
0.2.1 & 20.09.2005 & L. Laasonen &
Added guard support attribute to RF architecture.\\
0.3 & 22.09.2005 & P. J��skel�inen &
Added an explicit id to 'format' table and an explanation of what the table is (was a bit unclear).
Added tables needed by Estimator.\\
0.3.1 & 22.09.2005 & P. J��skel�inen &
Fixes to minor flaws noted by Lasse. \\
0.3.2 & 26.09.2005 & P. J��skel�inen &
Added fu/rf/plugin references to the cost\_data table. \\
0.4 & 26.09.2005 & L. Laasonen &
Added HDBManager module.\\
0.4.1 & 27.09.2005 & L. Laasonen &
Cosmetic changes.\\
0.4.2 & 28.09.2005 & L. Laasonen &
Changed direction of \emph{fu}->\emph{fu\_implementation} and \emph{rf}->\emph{rf\_implementation} references. Fixed an error in the data content diagram.\\
0.4.3 & 28.09.2005 & L. Laasonen &
A couple of cosmetic changes in field names yet.\\
0.4.4 & 29.09.2005 & L. Laasonen &
Added \emph{action} field to \emph{io\_usage} table.\\
0.4.5 & 05.10.2005 & L. Laasonen &
Renamed a field in \emph{block\_source\_file} table.\\
0.4.6 & 05.10.2005 & L. Laasonen &
Combined path and file name in the \emph{block\_source\_file} table.
Architecture is now optional in FU or RF entries.\\
0.4.7 & 06.10.2005 & P. J��skel�inen &
Added name to \emph{cost\_function\_plugin} table.\\
0.4.8 & 11.10.2005 & P. J��skel�inen &
Added \emph{\{bus,socket\}\_architecture}.\\
0.4.9 & 12.10.2005 & P. J��skel�inen &
Renamed \emph{\{bus,socket\}\_architecture} to \emph{bus and socket}\\
0.5   & 18.10.2005 & L. Laasonen &
Added \emph{fu\_implementation\_parameter} table. Replaced the 
\emph{width\_param} fields from \emph{fu\_port\_map} and 
\emph{fu\_external\_port} tables with the \emph{width\_formula} field. Added
\emph{opcode\_port\_width\_formula} field to \emph{rf\_data\_port} table.
Removed the \emph{direction} field from \emph{fu\_data\_port} table.\\
0.5.1 & 14.11.2005 & L. Laasonen &
Added \emph{guard\_support} field to \emph{fu\_data\_port} table and 
\emph{guard\_port} field to \emph{fu\_port\_map} table.\\
0.5.2 & 23.11.2005 & L. Laasonen &
Moved \emph{opcode\_port} field from \emph{fu\_port\_map} table to 
\emph{fu\_implementation} table.\\
0.5.3 & 19.01.2005 & L. Laasonen &
Added fu\_ext\_port\_parameter\_dependency table.\\
\hline
\end{tabular}
\end{center}
\end{table}

% Table of contents
\tableofcontents


% Document text
\chapter{INTRODUCTION}
\label{cha:intro}

\section{Purpose and Scope}

This is the design document of Hardware Database (HDB). The purpose of
this document is to describe the detailed design of the tables in the
database and to give an overview how the data is accessed in different
use cases.

\section{Definitions}

\section{Acronyms and Abbreviations}

\begin{table}[htb]
\begin{center}
\begin{tabular}{p{0.15\textwidth}p{0.85\textwidth}}
HDB & Hardware Database. \\
SQL & Structured Query Language \\
TCE & TTA Codesign Environment. \\
TTA & Transport Triggered Architecture.\\
UML & Unified Modeling Language. \\
\end{tabular}
\end{center}
\end{table}

\chapter{Contents of the Database}

This chapter describes the contents of the database in detail. In the
beginning, an overview of the contents is given and the latter
sections describes the relation tables in detail.

\section{Overview}

HDB is a database that contains all the necessary information of the
provided and user defined processor building blocks, such as function
units and register files that is needed by the tools of TCE. It
consists of architural, cost and HDL implementation-specific
information to make them usable in processor generation. The data in
HDB lie in relation tables but HDB provides also an interface to
clients that is in higher abstraction level. HDB is always accessed
through this interface which hides the real implementation. Therefore
the implementation of the database can be changed without any changes
in the clients of the database. That is, clients do not access the
database directly using SQL queries.

\section{Data Content}

Figure \ref{fig:data_content} depicts the data content of the
database. It is an UML class diagram where each class represents a
relation table in the database.

\begin{figure}[tb]
\centerline{\psfig{figure=eps/hdb.eps,width=1.2\textwidth}}
\caption{Data content diagram of the HDB.}
\label{fig:data_content}
\end{figure}

\section{Relation Tables}

This section defines the relation tables in the database. For each
table, all the attributes are defined as well as the primary key.

\begin{longtable}{|l||l|l|l|}
\caption{Tables, fields and field types of the hardware database}
\label{tab:database_contents}
\cr\hline
\bf{Table} & \bf{Field} & \bf{Field type} \\
\hline
              & @id            & id \\\cline{2-4}
fu            & architecture   & id->fu\_architecture \\\cline{2-4}
              & cost\_funtion  & cost\_function\_plugin->id \\\cline{2-4}
\hline
\hline
fu\_architecture & @id         & id \\\cline{2-4}
\hline

\hline
              & @id            & id \\\cline{2-4}
\raisebox{1.5ex}[0pt]{pipeline\_resource}
              & fu\_arch           & fu\_architecture->id \\\cline{2-4}
\hline
\hline
              & @id            & id \\\cline{2-4}
operation\_pipeline & fu\_arch      & fu\_architecture->id \\\cline{2-4}
              & operation      & operation->id \\\cline{2-4}
\hline
\hline
              & @id            & id \\\cline{2-4}
              & cycle & integer number \\\cline{2-4}
\raisebox{1.5ex}[0pt]{pipeline\_resource\_usage}
              & resource       & pipeline\_resource->id \\\cline{2-4}
              & pipeline       & operation\_pipeline->id \\\cline{2-4}
\hline
\hline
              & @id            & id \\\cline{2-4}
              & cycle          & integer number \\\cline{2-4}
io\_usage     & io\_number     & integer number \\\cline{2-4}
              & action         & boolean \\\cline{2-4}
              & pipeline       & operation\_pipeline->id \\\cline{2-4}
\hline
\hline
              & @id            & id \\\cline{2-4}
\raisebox{1.5ex}[0pt]{operation}
              & name           & string \\\cline{2-4}
\hline
\hline
              & @id            & id \\\cline{2-4}
              & triggers       & boolean \\\cline{2-4}
              & sets\_opcode  & boolean \\\cline{2-4}
\raisebox{1.5ex}[0pt]{fu\_data\_port}
              & guard\_support  & boolean \\\cline{2-4}
              & width          & integer number \\\cline{2-4}
              & fu\_arch       & fu\_architecture->id \\\cline{2-4}
\hline
\hline
              & @id            & id \\\cline{2-4}
              & io\_number     & integer number \\\cline{2-4}
\raisebox{1.5ex}[0pt]{io\_binding}
              & port           & fu\_data\_port->id \\\cline{2-4}
              & operation      & operation->id \\\cline{2-4}
\hline
\hline
              & @id            & id \\\cline{2-4}
              & name           & string \\\cline{2-4}
              & opcode\_port   & string \\\cline{2-4}
              & clk\_port      & string \\\cline{2-4}
\raisebox{1.5ex}[0pt]{fu\_implementation}
              & rst\_port & string \\\cline{2-4}
              & glock\_port    & string \\\cline{2-4}
              & glock\_req\_port & string \\\cline{2-4}
              & fu             & fu->id \\\cline{2-4}
\hline
\hline
              & @id            & id \\\cline{2-4}
              & opcode         & integer number \\\cline{2-4}
\raisebox{1.5ex}[0pt]{opcode\_map}   
              & operation      & operation->id \\\cline{2-4}
              & fu\_impl       & fu\_implementation->id \\\cline{2-4}
\hline
\hline
              & @id            & id \\\cline{2-4}
              & name           & string \\\cline{2-4}
              & width\_formula & string \\\cline{2-4}
fu\_port\_map & load\_port     & string \\\cline{2-4}
              & guard\_port    & string \\\cline{2-4}
              & fu\_impl       & fu\_implementation->id \\\cline{2-4}
              & arch\_port     & fu\_data\_port->id \\\cline{2-4}
\hline
\hline
              & @id            & id \\\cline{2-4}
              & name           & string \\\cline{2-4}
              & direction      & enumeration \\\cline{2-4}
\raisebox{1.5ex}[0pt]{fu\_external\_port}   
              & width\_formula & string \\\cline{2-4}
              & description    & string \\\cline{2-4}
              & fu\_impl       & fu\_implementation->id \\\cline{2-4}
\hline
\hline
              & @id            & id \\\cline{2-4}
              & name & string \\\cline{2-4}
\raisebox{1.5ex}[0pt]{fu\_implementation\_parameter}   
              & default\_value  & integer number \\\cline{2-4}
              & fu\_impl       & fu\_implementation->id \\\cline{2-4}
\hline
\hline
              & @id            & id \\\cline{2-4}
\raisebox{1.5ex}[0pt]{fu\_ext\_port\_parameter\_dependency} & port & fu\_external\_port->id \\\cline{2-4}
              & parameter      & fu\_implementation\_parameter->id \\\cline{2-4}
\hline
\hline
              & @id            & id \\\cline{2-4}
rf            & architecture   & rf\_architecture->id \\\cline{2-4}
              & cost\_function & cost\_function\_plugin->id \\\cline{2-4}
\hline
\hline
              & @id            & id \\\cline{2-4}
              & size           & integer number \\\cline{2-4}
              & width          & integer number \\\cline{2-4}
              & read\_ports    & integer number \\\cline{2-4}
              & write\_ports   & integer number \\\cline{2-4}
\raisebox{1.5ex}[0pt]{rf\_architecture}
              & bidir\_ports   & integer number \\\cline{2-4}
              & latency        & integer number \\\cline{2-4}
              & max\_reads     & integer number \\\cline{2-4}
              & max\_rw        & integer number \\\cline{2-4}
              & guard\_support & boolean \\\cline{2-4} 
\hline
\hline
              & @id            & id \\\cline{2-4}
              & name           & string \\\cline{2-4}
              & size\_param    & string \\\cline{2-4}
              & width\_param   & string \\\cline{2-4}
rf\_implementation & clk\_port  & string \\\cline{2-4}
              & rst\_port      & string \\\cline{2-4}
              & glock\_port    & string \\\cline{2-4}
              & guard\_port    & string \\\cline{2-4}
              & rf             & rf->id \\\cline{2-4}
\hline
\hline
              & @id            & id \\\cline{2-4}
              & name           & string \\\cline{2-4}
              & direction      & enumeration \\\cline{2-4}
\raisebox{1.5ex}[0pt]{rf\_data\_port}
              & load\_port     & string \\\cline{2-4}
              & opcode\_port   & string \\\cline{2-4}
              & rf\_impl        & rf\_implementation->id \\\cline{2-4}
\hline
\hline
              & @id            & id \\\cline{2-4}
block\_source\_file & file     & string \\\cline{2-4}
              & format         & format->id \\\cline{2-4}
\hline
\hline
              & @id            & id \\\cline{2-4}
rf\_source\_file & rf\_impl    & rf\_implementation->id \\\cline{2-4}
              & file           & block\_source\_file->id \\\cline{2-4}
\hline
\hline
              & @id            & id \\\cline{2-4}
fu\_source\_file & fu\_impl     & fu\_implementation->id \\\cline{2-4}
              & file           & block\_source\_file->id \\\cline{2-4}
\hline
\hline
bus           & @id         & id \\\cline{2-4}
\hline
\hline
socket        & @id         & id \\\cline{2-4}
\hline
\hline
              & @id            & id \\\cline{2-3}
format        & format\_name   & string \\\cline{1-3}
\hline
\hline
              & @id            & id \\\cline{2-4}
cost\_function\_plugin & description & string \\\cline{2-4}
              & name           & string \\\cline{2-4}
              & plugin\_file\_path           & string \\\cline{2-4}
	      & type & string \{'fu' | 'rf' | 'decomp' | 'icdec'\} \\\cline{2-4}
\hline
\hline
              & @id            & id \\\cline{2-4}
cost\_estimation\_data   & plugin\_reference  & cost\_function\_plugin->id
\\\cline{2-4}
              & bus\_reference & bus->id \\\cline{2-4}
              & socket\_reference & socket->id \\\cline{2-4}
              & rf\_reference & rf->id \\\cline{2-4}
              & fu\_reference & fu->id \\\cline{2-4}
              & name & string \\\cline{2-4}
              & value          & string \\\cline{2-4}
\hline
\hline
              & @id            & id \\\cline{2-4}
cost\_estimation\_metadata   & parent & cost\_estimation\_metadata->id
\\\cline{2-4}
              & name & string \\\cline{2-4}
              & value          & string \\\cline{2-4}
\hline


\end{longtable}

\subsection{format}

Lists all possible formats the block implementations can be stored in (e.g.,
'VHDL', 'Verilog', 'VHDL/Compiled'...).

\subsection{cost\_function\_plugin}

All cost estimation plugins must be registered in HDB. This includes, in
addition to estimation plugins for FUs and RFs, also estimation plugins of
decompressor and IC/decoder costs. \textbf{cost\_function\_plugin} table 
lists the plugins that contain functions used to calculate costs for different
types of elements. 

Cost functions are stored in external plugin modules of which
path is given in \textit{plugin\_file\_path} field. The path can be
relative or absolute path. In case of a relative path, the default search
locations are used to find the plugin. The class name of the plugin
is stored in \textit{name}.

Field \textit{type} tells the type of the cost estimation plugin. Type
\textit{fu} is for function unit cost estimation function, \textit{rf} for
register file cost estimation function, \textit{decomp} for decompressor
generator plugin (which includes also estimation functions used to estimate
costs of the particular decompressor), and \textit{icdec} for IC/decoder
generator plugin (which includes also its cost estimation functions).

\subsection{cost\_estimation\_data}

Each cost estimation algorithm may store and retrieve arbitrary data in HDB.
Such data can be pure measurement results of certain implementations, or
anything that is needed to estimate the costs, for example, by interpolating
between different cost data. The contents of data is purely decided by each
estimation algorithm, therefore the data is stored as simple name/value pairs,
in table \textbf{cost\_estimation\_data}. The table includes a list of names 
(\textit{name}) and their values (\textit{value}). Each name/value pair can be
connected to a cost estimation plugin (\textit{plugin\_reference}), to an FU
entry (\textit{fu\_reference}), and  a RF entry (\textit{rf\_reference}). None
of these connections are mandatory, thus can be NULL.  In addition, each
name/value pair is identified with an unique id, thus it's possible to have
multiple rows with identical fields -- only the row id must differ. This allows
defining lists.

\chapter{Access paths}

This chapter illustrates how the information is obtained from the
database in different use cases.

\section{Use Case: Finding HW implementation of an FU}

Input: ID of the function unit.\\ 
Output: Implementation info of the FU.

\emph{fu\_implementation} entry is obtained directly from the attribute
\emph{fu->implementation}.

\section{Use Case: Finding FU that has minimum cost}

Input: Architecture definition of the function unit.\\
Output: ID of the function unit.

Architecture matching must be done at first. It can be done by
selecting an \emph{fu\_architecture} entry one by one and checking its
structure. Operation set is obtained by selecting the pipelines that
refer to the particular \emph{fu\_architecture} entry. Data ports are
obtained in the same way: by selecting the \emph{fu\_data\_port}
entries that refer to that particular \emph{fu\_architecture}. When
the architecture match is done, the \emph{fu} entries that refer to
that architecture are selected. Then we have a set of \emph{fu}
entries and we select the one that has the minimal cost.

\chapter{HDBManager Module}

\section{Overview}

\emph{HDBManager} is a class that provides the interface used to access HDB. It contains methods to search, add and remove blocks from/to HDB. In 
addition to the \emph{HDBManager} class, the module contains also
other classes which are used to represent the contents of the
database. The class diagram of the module is depicted in Figure
\ref{fig:hdb_classdiagram}.

\begin{figure}[tb]
\centerline{\psfig{figure=eps/hdb_classdiagram.eps,width=1.2\textwidth}}
\caption{Class diagram of HDB module.}
\label{fig:hdb_classdiagram}
\end{figure}

\section{HDBManager class}

The HDBManager class provides an API to search, add and remove
hardware blocks to/from HDB. This is the class that is used to access
HDB. \emph{HDBManager} retrieves data from HDB by using SQL-queries and
constructs C++ objects that represent the data. The objects can be
handled by the clients then.

\subsection{Interface}

The HDBManager class has the following interface:

\begin{description}
\item[HDBManager(hdbFile : string)]%
  Constructor. The HDB file to be managed by the HDBManager
  instance is given as parameter.
\item[addFunctionUnit(fu : FUEntry\&)]%
  Add a new function unit to the HDB.
\item[addRegisterFile(rf : RFEntry\&)]%
  Add a new register file to the HBL.
\item[removeFunctionUnit(id : int)]%
  Remove the function unit that has the given id.
\item[removeRegisterFile(id : int)]%
  Remove the register file that has the given id.
\item[fuByLooseArchitecture(architecture : FUArchitecture\&) : set<FUEntry*>]%
  Returns all the FUs that has the given architecture. The returned
  FUs may have extra operations that are not defined in the given
  architecture.
\item[rfByArchitecture(architecture : RFArchitecture\&) : set<RFEntry*>]%
  Returns all the RFs that has the given architecture.
\item[fuByID(id : integer) : FUEntry*]
  Returns the FU that has the given ID.
\item[rfByID(id : integer) : RFEntry*]
  Returns the RF that has the given ID.
\end{description}

\section{FUEntry}

\emph{FUEntry} class represents an FU in the HDB. It consists of 
architecture and implementation parts (and cost part in the
future). The implementation part may be missing, if there is only cost
information of the FU in the HDB.

\subsection{Interface}

The \emph{FUEntry} class has the following interface:

\begin{description}
\item[FUEntry(id : integer, architecture : FUArchitecture*)]
  The constructor. ID of the entry and the architecture are given as
  parameters.
\item[hasImplementation() : bool]
  Tells whether the FU entry has an implementation.
\item[implementation() : FUImplementation\&]
  Returns the implementation of the FU.
\item[architecture() : FUArchitecture\&]
  Returns the architecture of the FU.
\item[id() : integer]
  Returns the ID of the entry.
\item[hdbFile() : string]
  Returns the name of the HDB file that contains the entry.
\end{description}

\section{FUArchitecture}

FUArchitecture represents the visible architecture of a function
unit. The architecture is mainly represented by the contained
\emph{TTAMachine::FunctionUnit} instance but since the architecture
may be parameterized, \emph{FUArchitecture} instance tells what is
parameterized in the architecture.

\subsection{Interface}

\emph{FUArchitecture} class has the following interface:

\begin{description}
\item[FUArchitecture(fu : TTAMachine::FunctionUnit*)]
  The constructor.
\item[hasParameterizedWidth(port : string) : bool]
  Tells whether the port of the given name has parameterized width.
\item[setParameterizedWidth(port : string)]
  Sets parameterized width for port of the given name.
\item[architecture() : TTAMachine::FunctionUnit\&]
  Returns the architecture.
\end{description}

\section{HWBlockImplementation}

\emph{HWBlockImplementation} is a base class for \emph{FUImplementation} and \emph{RFImplementation}. It bundles up the common properties of those classes. It contains \emph{BlockImplementationFile} object(s) each of which represent a file that implements (a part of) the hardware block.

\subsection{Interface}

The \emph{HWBlockImplementation} class has the following interface:

\begin{description}
\item[HWBlockImplementation(moduleName : string, clkPort : string, rstPort : string, glockPort : string)]
  The constructor. Name of the module and general ports are given as
  parameters.
\item[moduleName() : string]
  Returns the name of the module.
\item[clkPort() : string]
  Returns the name of the clock port.
\item[rstPort() : string]
  Returns the name of the reset port.
\item[glockPort() : string]
  Returns the name of the global lock port.
\item[addImplementationFile(file : BlockImplementationFile*)]
  Adds the given implementation file for the implementation.
\end{description}

\section{FUImplementation}

\emph{FUImplementation} class represents one particular hardware 
implementation of a function unit. It tells the names of the ports in
the hardware implementation and how the ports are mapped to the
corresponding ports in the architecture.

\subsection{Inteface}

The \emph{FUImplementation} class has the following interface:

\begin{description}
\item[FUImplementation(name : string, clkPort : string, rstPort : string, glockPort : string, glockReqPort : string, parent : FUEntry\&)]
  The constructor. Module name and port names are given as
  parameters. Empty string is given if some port is missing. Also the
  parent \emph{FUEntry} is given and the instance is automatically
  registered to the parent.
\item[glockRequestPort() : string]
  Returns the name of the global lock request port.
\item[setOperationCode(operation : string, opcode : int)]
  Sets opcode for the given operation.
\item[operationCode(operation : string)]
  Returns the opcode of the given operation.
\end{description}

\section{FUExternalPort}

This class represents an external port in the FU. The port is not
visible in the architecture of the FU since it is external to the
whole processor.

\subsection{Interface}

The \emph{FUExternalPort} class has the following interface:

\begin{description}
\item[FUExternalPort(name : string, direction : enum, width : int, description : string, parent : FUImplementation\&)]
  The constructor. A non-parameterized port is created by this constructor.
\item[FUExternalPort(name : string, direction : enum, widthParameter : string, description : string, parent : FUImplementation\&)]
  The constuctor. Parameterized port is created by this constructor.
\item[name() : string]
  Returns the name of the port.
\item[direction() : enum]
  Returns the direction of the port.
\item[hasParameterizedWidth() : bool]
  Tells whether the port has parameterized width.
\item[widthParameter() : string]
  Returns the name of the width parameter.
\item[width() : integer]
  Returns the bit width of the port.
\item[description() : string]
  Returns the description of the port.
\end{description}

\section{FUPortImplementation}

This class represents a port in the hardware implementation. It tells
also which is the corresponding port in the architecture.

\subsection{Interface}

The \emph{FUPortImplementation} class has the following interface:

\begin{description}
\item[FUPortImplementation(name : string, architecturePort : string, widthParameter : string, loadPort : string, opcodePort : string, parent : FUImplementation\&)]
  The constructor. Name of the port, the corresponding architectural
  port, the width parameter, the load port and the opcode port are
  given as parameters. Empty string is given if some doesn't exist.
\item[name() : string]
  Returns the name of the port.
\item[architecturePort() : string]
  Returns the name of the architecture port.
\item[widthParameter() : string]
  Returns the name of the width parameter.
\item[loadPort() : string]
  Returns the name of the load enable port.
\item[opcodePort() : string]
  Returns the name of the opcode port.
\end{description}

\section{RFEntry}

The \emph{RFEntry} class is corresponding class to \emph{FUEntry} but
it represents a register file. It consists of architecture part and
possibly implementation part. It the future also cost part is added.

\subsection{Interface}

The \emph{RFEntry} class has the following interface:

\begin{description}
\item[RFEntry(id : integer, architecture : RFArchitecture*)]
  The constructor. ID and architecture are given as parameters.
\item[architecture() : RFArchitecture\&]
  Returns the architecture of the register file.
\item[hasImplementation() : bool]
  Tells whether the RF has a hardware implementation also.
\item[implementation() : RFImplementation\&]
  Returns the implementation of the register file.
\item[id() : int]
  Returns the id of the register file.
\item[hdbFile() : string]
  Returns the name of the HDB file that contains the entry.
\end{description}

\section{RFArchitecture}

The \emph{RFArchitecture} class represents the architectural
properties of a register file.

\subsection{Interface}

The \emph{RFArchitecture} class has the following interface.

\begin{description}
\item[RFArchitecture(maxReads : int, maxWrites : int, readPorts : int, writePorts : int, bidirPorts : int, latency : int)]
  The constructor. Creates an architecture with parameterized width and size.
\item[hasParameterizedWidth() : bool]
  Tells whether the architecture has parameterized width.
\item[hasParameterizedSize() : bool]
  Tells whether the architecture has parameterezed size.
\item[setSize(size : int)]
  Sets a fixed size.
\item[setWidth(width : int)]
  Sets a fixed width.
\item[size() : int]
  Returns the size of the register file.
\item[width() : int]
  Returns the width of the register file.
\item[maxReads() : int]
  Returns the maximum simultaneous reads of the register file.
\item[maxWrites() : int]
  Returns the maximum simultaneous writes of the register file.
\item[readPortCount() : int]
  Returns the number of read ports.
\item[writePortCount() : int]
  Returns the number of write ports.
\item[bidirPortCount() : int]
  Returns the number of bidirectional ports.
\item[latency() : int]
  Returns the latency of the register file.
\end{description}

\section{RFImplementation}

\emph{RFImplementation} class represents one particular implementation of a register file. It is corresponding to the \emph{FUImplementation} class. It is always consisted by \emph{RFEntry} object.

\subsection{Interface}

The \emph{RFImplementation} class has the following interface:

\begin{description}
\item[RFImplementation(sizeParam : string, widthParam : string, guardPort : string, parent : RFEntry\&)]
  The constructor. Registers the instance automatically to the given parent.
\item[sizeParameter() : string]
  Returns the name of the size parameter.
\item[widthParameter() : string]
  Returns the name of the width parameter.
\item[guardPort() : string]
  Returns the name of the guard port.
\end{description}

\section{RFPortImplementation}

\emph{RFPortImplementation} class represents a read, write or bidirectional port of a register file implementation.

\subsection{Interface}

The \emph{RFPortImplementation} class has the following interface:

\begin{description}
\item[RFPortImplementation(name : string, direction : enum, loadPort : string, opcodePort : string, parent : RFImplementation\&)]
  The constructor. Name of the port, direction and names of the load
  enable port and opcode port are given as parameters. The port is
  registered automatically to the given parent.
\item[name() : string]
  Returns the name of the port.
\item[direction() : enum]
  Returns the direction of the port.
\item[loadPort() : string]
  Returns the name of the load enable port.
\item[opcodePort() : string]
  Returns the name of the load enable port.
\end{description}

\section{BlockImplementationFile}

This class represents a file that contains hardware implementation of
a module.

\subsection{Interface}

The \emph{BlockImplementationFile} class has the following interface:

\begin{description}
\item[BlockImplementationFile(pathToFile : string, format : enum)]
  The constructor.
\item[pathToFile() : string]
  Returns the name of the file.
\item[format() : string]
  Returns the format of the file.
\end{description}

\chapter{Open Issues}

None at the moment.

\end{document}
