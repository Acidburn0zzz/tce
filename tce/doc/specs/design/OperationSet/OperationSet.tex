\documentclass[a4paper,twoside]{tce}

\usepackage{pslatex}

\begin{document}
\author{Andrea Cilio, Pekka J\"a\"askel\"ainen, Mikael Lepist\"o}
\title{Operation Set Abstraction Layer}
\ver{0.11.1}
\firstday{24.02.2004}
\lastday{25.04.2006}
% id number in S- sequence
\docnum{028}
% draft/complete/committed
\state{draft}

\maketitle


\chapter*{Version History}

\begin{HistoryTable}

 0.1    & 24.02.2004 & A. Cilio   &
 First fragments. \\

 0.1.1  & 11.03.2004 & P. J\"a\"askel\"ainen &   
 Added information of the proxy mechanism used in behavior loading
 function of \emph{Operation}. \\

 0.2    & 22.03.2004 & A. Cilio   &
 Added chapter 4, Operation Model. \\

 0.3    & 26.03.2004 & A. Cilio   &
 Added chapter 5, Language.  Added pending issues: Proxy design, state
 lookup, 8. Minor changes. \\

 0.4    & 29.03.2004 & A. Cilio   &
 Minor, partial changes after P. J\"a\"askel\"ainen's review. \\

 0.4.1    & 17.05.2004 & P. J\"a\"askel\"ainen &

 Text revision to match current design status.  Still some classes missing.
 Clarified issues: operation behavior proxy and macro USES\_STATE.\\

 0.4.2      & 17.05.2004 & P. J\"a\"askel\"ainen &
 Cleanups, prepare to add data from the new class diagram.\\

 0.5.0      & 18.05.2004 & P. J\"a\"askel\"ainen &

 Added classes: \emph{OperationSerializer}, \emph{OperationBehavior}.
 Extended classes: \emph{OperationPool}, \emph{Operation},
 \emph{Operand}.\\

 0.5.1      & 19.05.2004 & P. J\"a\"askel\"ainen &
 Added or revised descriptions of remaining classes.\\

 0.6        & 21.05.2004 & A. Cilio &
 Major revision of the design.  Introduced classes \emph{OperationIndex} and
 \emph{OperationModule}.  \\

 0.6.1      & 24.05.2004 & P. J\"a\"askel\"ainen &
 Added algorithm for loading \emph{Operations}.  Added state-management API
 to \emph{OperationBehavior} and \emph{OperationContext}.  Minor
 corrections.  \\
 
 0.6.2      & 25.05.2004 & P. J\"a\"askel\"ainen &
 Updated class diagrams.  Added simplified diagram: OSAL relations with its
 client.  Renamed \emph{OperationContext} methods for state management.
 Minor updates and fixes. \\

 0.6.3      & 29.05.2004 & P. J\"a\"askel\"ainen &
 Not an error if a search path of \emph{OperationIndex} does not exist.
 Added state-management API to \emph{Operation} class. \\

 0.7        & 31.05.2004 & A. Cilio &
 Revised state-management API.  Extensive text revision. \\

 0.8        & 05.06.2004 & A. Cilio &
 Added \emph{OperationPropertyLoader}.  \emph{OperationState} name.  Special
 state via \emph{SimValue} objects.  Replaced nonmember assignment operators
 (not C++) with conversion operators.  Updated some of the behaviour
 language macros. \\

 0.8.1      & 08.06.2004 & P. J\"a\"askel\"ainen &
 Added \emph{OperationPropertyLoader} to the class diagram. No method in
 \emph{OperationBehavior} is pure virtual anymore; all create a ``null''
 OperationBehavior object. \\

 0.8.2      & 10.06.2004 & P. J\"a\"askel\"ainen &
 Reduced ``Behavior Definition Language'' chapter to a reference to comments
 in OSAL.hh. \\

 0.8.3      & 10.06.2004 & J. Nyk\"anen &
 Added two new methods to \emph{OperationPropertyLoader}.\\

 0.8.4      & 10.06.2004 & P. J\"a\"askel\"ainen &
 Added \emph{Operation} argument to the behavior factory functions.\\

 0.8.5      & 11.06.2004 & J. Nyk\"anen &
 Removed methods from \emph{OperationPropertyLoader}. Added information
 about \emph{OperationIndex} (that it uses \emph{OperationSerializer}). \\

 0.8.6      & 14.06.2004 & P. J\"a\"askel\"ainen &
 Updated information on \emph{SimValue}; added overloaded operators. \\

 0.8.7      & 17.06.2004 & P. J\"a\"askel\"ainen &
 Added parent \emph{Operation} argument to method lookupBehavior of
 \emph{OperationBehaviorLoader}. \\

 0.8.8      & 22.07.2004 & J. Nyk\"anen &
 Added two methods to \emph{OperationIndex} and four methods to
 \emph{Operation}. \\

 0.8.9      & 19.08.2004 & P. J\"a\"askel\"ainen &
 Added information about MEM macro and its implementation. Added the memory
 natural word width to \emph{OperationContext}.\\

 0.9        & 26.08.2004 & P. J\"a\"askel\"ainen &
 Major modifications to behavior simulation API due to changes to Memory
 Model. Added definitions of load and store operations as examples. Added
 indentification code to \emph{OperationContext}.  Added syscall registers
 to \emph{OperationContext}. \\

 0.10       & 31.08.2004 & A. Cilio &
 Removed \emph{SimValue} overloaded operators. Several corrections
 throughout the document. Fixed document history table.\\

 0.10.1     & 31.08.2004 & P. J\"a\"askel\"ainen &
 Method \emph{lateResult} back to \emph{OperationBehavior}. Moved examples
 of load and store implementations to requirement document. Updated class
 diagrams.\\

 0.10.2     & 01.09.2004 & P. J\"a\"askel\"ainen &
 Added method \emph{lateResult} to \emph{OperationState}. Changed access of
 \emph{SimValue} contents (now direct).\\

 0.10.3     & 06.09.2004 & P. J\"a\"askel\"ainen &
 Reorganisation of methods between classes \emph{OperationBehavior},
 \emph{OperationState}, and \emph{OperationContext}.\\

 0.10.4      & 03.08.2005 & A. Cilio &
 Assigned document number.\\

 0.11        & 08.03.2006 & A. Cilio &
 Partial update of operation behaviour and state to ``as is''. \\

 0.11.1      & 25.04.2006 & A. Cilio &
 Minor update.\\

 0.12.0      & 18.12.2007 & M. Lepist\"o &
 Added section about Operation DAG.\\

 0.13.0      & 14.03.2008 & P. J. &
 Removed lateResult() and the syscall methods due to the OSAL and Memory
 model cleanup.

\end{HistoryTable}


\tableofcontents



\chapter{INTRODUCTION}

\section{Purpose and Scope}

This document describes the software architecture and the design of the
Operation Set Abstraction Layer (OSAL).


\section{Definitions}

\begin{description}
\item[Application]%
  A stand-alone executable program that provides services for users of the
  TCE toolset. It may be controlled by a command-line interface or by a
  graphical user interface, or via a driver application that integrates a
  group of related applications.
\item[Client]%
  Any part of a module or application that interacts with an object or
  module by means of the interface provided by it.
\item[Module]%
  Part of an application or library which can be accessed and manipulated by
  clients solely by means of the interface it provides.
\item[Operation] %
  The minimum amount of activity that can be performed by a function unit.
  In a sequential processor architecture, operations are equivalent to
  machine instructions.
\item[Repertoire] (of operations) %
  The set of all operations that may be implemented in a processor that
  instantiates the TTA template.
\end{description}



\section{Acronyms and Abbreviations}

\begin{table}[htb]
\begin{center}
\begin{tabular}{p{0.15\textwidth}p{0.85\textwidth}}
OSAL  & Operation Set Abstraction Layer.  \\
TCE   & TTA Co-design Environment. \\
TTA   & Transport Triggered Architectures. \\
\end{tabular}
\end{center}
\end{table}


\chapter{MODULE OVERVIEW}

This chapter describes the high-level architecture of the OSAL module.

\section{Philosophy of Design}

% Describe the principles behind this design and give an overview of
% it. Motivate briefly the design if necessary.

%% No description of the module purpose and motivation, that belongs to the
%% functional specification document.

From the logical point of view, OSAL provides an abstract model of the
target TTA operation repertoire and of every operation contained in it.  The
repertoire is represented by a single class object, \emph{OperationPool},
whereas the properties of operations are represented by as many objects as
are the operations that clients need to work with.

From the deployment point of view, the OSAL module can be divided into two
very different parts:
\begin{enumerate}
\item %
  A fixed submodule of the domain library.
\item %
  A variable number of files that contain operation definitions.
\end{enumerate}

The domain library part provides the interface to access operation
repertoires and lookup the definition of single operations.

The operation definition data is accessed solely \emph{at run time} and is
organized into a variable number of dynamic modules and data files.  
Any modification in the dynamic modules and data files affects
the operation repertoire data set automatically, without any need to
recompile or statically re-link the domain library or the applications that
are client of OSAL module.

\paragraph{Open-Ended Repertoire.}
The operation repertoire is open-ended. Its design does not rely on
hard-coded, reserved ``operation codes'' nor it reserves ``free operation
slots'' for future user-defined operations.

\paragraph{Transparent dynamic loading.}
One of the useful services provided by OSAL is encapsulated interaction
with dynamic modules. Because of this, user of OSAL doesn't need to know 
that behavior is loaded from dynamic modules and static properties from 
XML files.  The system- and implementation-dependent aspects
of opening and linking symbols at run time (for example, by means of the
Solaris and Linux \verb#dlopen()# interface) is hidden in toolkit module
\emph{PluginTools}.

\section{External Modules/ Application Environment}

% Overview of module from an external perspective:
% - dependencies with external modules-subsystems
% - IFs of external modules that are used
% - external (file) data formats used

This section describes the modules of TCE system on which OSAL depends.

\subsection{MemoryModel}

Operations that access memory are clients of the Memory Model. An instance
of Memory Model is accessed via OSAL macros and is stored in the operation
context.

\subsection{PluginTools}

Handling of dynamic modules, also known as ``plugin modules'' is delegated
to a toolkit module called \emph{PluginTools}.

\subsection{XMLSerializer}

Toolkit module \emph{XMLSerializer} works as a layer between the actual XML
library and TCE.  It is used to load the static properties of operations.

\section{Architecture of Databases}

% Describes the data types defined and used by the module and their
% relations.  The description is at type-accurate level. For example, the
% description may consist of UML class diagrams with the attributes and
% interfaces of classes of the target program. Another example, the database
% tables are defined precisely, at level of single fields.

The operation repertoire source data is stored in two types of files: XML
data files and source files of the implementation language (C++).  The
source files contain code that must be imported into client applications in
order to simulate operation behavior.  The source files are compiled to
independent plugin modules with the aid of a helper application ``OSAL
builder''.

The XML files contain specifications of operation properties, including a
specification based on C++ of the operation behavior that can be integrated
into the operation set simulator.


%% \section{? Main Restrictions of Implementation}

%% This section is in design document of SW document course
%% I need to know precisely what it should contain to decide if it's useful
%% enough for us.


\section{Module Communication}

The diagram depicted in Figure~\ref{fig:OSAL-communication} summarizes the
dependencies and communication links between OSAL main classes and clients.
Diagram is simplified for clarity.  For example, the Machine State Model
(MSM) doesn't access OperationState objects directly, but holds them via
\emph{OperationContext} objects, one in each class that models the Function
Unit state.

The Operation Set Abstraction Layer communicates and directly uses the
interface of the Memory Model (described in~\cite{MemoryModuleDesign}
and~\cite{SimulatorSpecs}).

Except for memory and the toolkit modules, OSAL is completely independent
from other modules. Communication with clients occurs only through operation
input and output values implemented by type \emph{SimValue} and the
operation context, modeled by class \emph{OperationContext}.

The role of \emph{SimValue} and \emph{OperationContext} is crucial, because
these classes provide the level of indirection necessary to insulate OSAL
from direct dependencies of other domain modules.  In particular, the
operation behavior model depends on data held in the Machine State Model and
Machine Object Model.  For this reason, \emph{SimValue} and
\emph{OperationContext} must provide a copy of such data.  For example, the
bit width of a data word is included in the representation of the word given
by a \emph{SimValue}.


\begin{figure}[tb]
  \centerline{\psfig{figure=eps/OSALComm.eps,width=0.60\textwidth}}
  \caption{Communication between parts of OSAL module and between OSAL and
    client modules.}
  \label{fig:OSAL-communication}
\end{figure}

\chapter{CLASSES}

A detailed class diagram of Operation Set Abstraction Layer can be found
from osal doxygen documentation.

Figure~\ref{fig:OSAL_class_diagram_simple} shows a simplified class diagram 
with relations of the OSAL classes to the client (``ClientUnit'').

\begin{figure}[tb]
  \centerline{\psfig{figure=eps/OSALClassesSimplified.eps,width=1.0\textwidth}}
  \caption{Simplified class diagram of OSAL classes and their relations with
    client.}
  \label{fig:OSAL_class_diagram_simple}
\end{figure}

Details of the classes of OSAL are described in following sections.

\section{OperationPool}
\label{sec:operationPool}

\subsection{Overview}

Clients access the models of operations of TTA processors through
\emph{OperationPool}.  It provides interface for getting references to
operations accessed with operation name strings.

\emph{OperationPool} is responsible for loading operation models into its
internal structures and hold them.  Most of its work is delegated to helper
classes: \emph{OperationBehaviorLoader}, \emph{OperationIndex},
\emph{OperationSerializer}, \emph{XMLSerializer}.  Helper classes are used,
for example, to speed up the operation lookup by identifying and keeping a
record of all external files that contain operation definitions.

\emph{OperationPool} is also responsible for defining the path list used to
search operation definitions to lookup.  Such path list is hard-coded to the
default paths defined in \cite{OSAL-specs}. Currently it's not possible to
edit the list of lookup paths dynamically.

\subsection{Interface}

\begin{description}

\item[OperationPool()]%

  The constructor creates and prepares the internal helper objects that are
  necessary to perform its job: \emph{OperationBehaviorLoader}
  (Section~\ref{sec:opBehaviorLoader}), \emph{OperationIndex}
  (Section~\ref{sec:opIndex}) and \emph{OperationSerializer}
  (Section~\ref{sec:operationSerializer}).

\item[operation(name : const string\&) : Operation\&] %

  Look up an operation identified by \emph{name} and returns a reference to
  it.

  Multiple calls to the lookup methods with the same operation name will
  return always the same instance.  Such instance is unique and shared among
  all clients of the given instance of operation repertoire (see
  Section~\ref{ssec:operation-construction}).

  When a client requests an operation lookup, the operation repertoire first
  inspects its set of cached operations.  Operations are privately cached by
  the \emph{OperationPool} instance.  If the requested operation is found in
  this set, a reference to it is simply returned.

  Clients are not allowed to make copies or delete operation objects.  See
  Section~\ref{ssec:operation-construction} for details.

  If no operation with given name is found in the cache,
  \emph{OperationPool} scans all available operation data files in search of
  the requested operation definition, and creates a new operation object
  with the data from the definition.  Thus, operations are created only on
  demand.  If the requested operation is not defined, a special null
  operation (see \ref{ssec:nullOperation}) is returned.

\end{description}

\subsection{Null Operation}
\label{ssec:nullOperation}

The \emph{NullOperation} is a special singleton class that represents an
operation that is undefined or not available.  The \emph{NullOperation}
instance should be returned by all functions that return \emph{Operation}
instances whenever the actual operation could not be returned due to some
reason.  For example, \emph{OperationPool} returns it when operation lookup
fails.  By means of the null operation, it is possible to postpone error
detection to the time when it is not absolutely possible to ignore the
error.  The null operation is an application of the \emph{Null Object}
design pattern and can be seen as an alternative to throwing an exception
immediately.

All methods of the \emph{NullOperation} instance inherited from
\emph{Operation} abort the program with an error message.

In addition to the operation API, \emph{NullOperation} provides a static
method to access its unique instance:
\begin{description}
\item[Operation\& it()]
\end{description}

Clients can test for null operation simply by comparing the operation with
the \emph{NullOperation} singleton, given by \verb|NullOperation::instance()|.

\subsection{Implementation}

\subsubsection{Construction of OperationPool}

Operation static data is obtained from \emph{OperationSerializer}
(Section~\ref{sec:operationSerializer}), which uses \emph{XMLSerializer}
\cite{ToolkitDesign} to read the data from XML file.  These
\emph{Serializer} implementations allow also to write back the operation
definition.  Writing capability is needed for client applications such an
OSAL editor GUI.

Operation data is not loaded during construction of an \emph{OperationPool}
instance.  Instead, it is loaded only when a new operation (not yet cached)
is requested.

The operation pool cache is an associative map:
\begin{quote}
\verb|operationCache : map<string name, Operation>|
\end{quote}
Caching rationale is efficiency.  The \emph{OperationPool} instance avoids
to repeatedly load the same static operation data from an external file when
the data is already available.

Operation uses a variant of the proxy design pattern~\cite{DesignPatterns}
to avoid unnecessary loading of behavior in cases where it is not
needed. The mechanism behind this pattern is explained in
Section~\ref{sec:opBehaviorProxy}.

\subsubsection{Loading an Operation}

An algorithm for loading and constructing a single operation in
\verb|operation()| is presented below.

\begin{enumerate}
\item %
  Obtain the name of the operation module (an \emph{OperationModule}
  instance) where the operation with the given name is defined from the
  \emph{OperationIndex} instance of \emph{OperationPool}.
\item %
  Obtain from \emph{OperationSerializer} the object state tree of the
  requested operation.  The data is loaded from the XML file indicated by
  the \emph{OperationModule}.
\item %
  Create a new \emph{OperationBehaviorProxy} instance for the
  \emph{Operation}, with a reference to the \emph{OperationBehaviorLoader}
  held in \emph{OperationPool}.  Store the created proxy instance so it can
  be later retrieved and deleted.
\item %
  Create a new \emph{Operation} with given name and created proxy.
\item %
  Let the newly created \emph{Operation} load its state from the object
  state tree returned by \emph{OperationSerializer}. Or if there is no 
  plugin behavior for an operation, then create behavior model out of 
  a DAG presentation of the operation.
\item % 
  Store the constructed \emph{Operation} in the cache and return a reference
  to it.
\end{enumerate}

\subsection{Error Handling}

Error conditions occurring while reading the XML or dynamic module data are
propagated to \emph{OperationPool} clients by throwing an exception.  This
lets the client respond to error conditions in any way wanted.

\subsubsection{Multiple Definitions of the Same Operation}

All operations in a single data file must have a unique string.  If two
operations have the same name, an error is generated at the time the
operation module is scanned for the first time.

Different databases and operation modules are independent and may
contain operations with the same name.  In these cases, the first module
that is scanned will return its operation definition, and the other
definitions will be ignored.


\section{OperationSerializer}
\label{sec:operationSerializer}

\subsection{Overview}

\emph{OperationSerializer} is an implementation of the \emph{Serializer}
interface (see \cite{ToolkitDesign}).  Its purpose is to construct an
\emph{ObjectState} tree that represents the information contents of an
operation from a given external file.  The bulk of the work is delegated to
a more primitive class, called \emph{XMLSerializer}.

\emph{XMLSerializer} transforms the information contents of an XML file into
an \emph{ObjectState} tree.  Each XML element in the file has a
corresponding object in the representation and each attribute in a node has
a matching attribute in the object. This \emph{ObjectState} tree is
converted by \emph{OperationSerializer} to an \emph{ObjectState} tree that
matches the structure of the \emph{Operation} class.

Clients can pass the tree constructed by \emph{OperationSerializer} to
method \verb|loadState()| of \emph{Operation}, which redefines itself from
the tree (and delegates \emph{Operand} specific data to \emph{Operand}
objects).

Both \emph{Operation} and \emph{Operand} implement the \emph{Serializable}
interface, which means that they are able to load their data from an
\emph{ObjectState} object. Implementing the interface guarantees that the
class is able to save its data to an \emph{ObjectState} object.

The conversion step between the object state tree returned by
\emph{XMLSerializer} and the tree returned by \emph{OperationSerializer} is
needed because the XML representation doesn't match the internal structure
of the operation object model.  For example, in the XML file there might be
XML elements for properties which are stored in operation \emph{ObjectState}
as attributes, not as tree nodes.

\emph{XMLSerializer} receives the absolute path name to the file from which
operation definitions must be loaded (or to which must be stored).  Clients
may use the services of \emph{OperationIndex} (Section~\ref{sec:opIndex}) to
obtain the name of the file from which a given operation definition data
must be loaded.


\section{Operation}
\label{sec:operation}

\subsection{Overview}

\emph{Operation} is the main class that models the static properties and the
behavior of operations of the target processor.  Only properties that do not
depend on the hardware implementation are modeled.  The state of operations
is encapsulated in a separate class object.  The operation behavior is
modeled by methods that simulate execution of the operation.  These methods
are implemented by a concrete \emph{OperationBehavior} subclass.

Operation state and properties that depend on the hardware implementation
(such as the bit width of operation inputs and outputs) are passed by the
clients of \emph{Operation}.

\subsection{Interface}

See doxygen documetation for updated information.

\subsubsection{Redefining Operation Behavior}

The behavior model of an operation can be redefined dynamically.  The
following method is used to replace the \emph{OperationBehavior} instance
assigned to the \emph{Operation}.  This method is currently used only by
\emph{OperationBehaviorProxy} (Section~\ref{sec:opBehaviorProxy}) to replace
itself with the concrete behavior object.

\begin{description}
\item[setBehavior(behavior : OperationBehavior\&)]%

  Replaces the current operation behavior object model with the given
  \emph{OperationBehavior} instance.

  It is responsibility of the client of this method to dispose of the
  operation object being replaced.

\item[behavior() : OperationBehavior\&]%

  Return the \emph{OperationBehavior} object assigned to the operation.
  This method should be only used when the behavior needs to be redefined,
  in order to dispose of the replaced behavior object.

\end{description}

Simulation methods are delegated to an \emph{OperationBehavior} instance.
Methods that simulate operation behavior are listed in
Section~\ref{sec:opBehavior}.

An operation is never responsible for the deallocation of the
\emph{OperationBehavior} instance assigned to it.  Every operation behavior
proxy is responsible (owns) the concrete behavior instance it created, and
the proxies themselves are owned by \emph{OperationPool}, which deallocates
them.  As a result, \emph{OperationPool} must deallocate operations before
their operations behavior instances, otherwise a dangling reference
(albeit, temporary) is created.

\subsubsection{Operation State Allocation}

For operations that have state, \emph{Operation} provides methods for
creating an operation state instance.  Such state is accessed by behavior
simulation methods.

\begin{description}
\item[createState(OperationContext\&)]%
\item[deleteState(OperationContext\&)]%
\end{description}

Above methods delegate construction and destruction of operation state to
the only class that has specific type knowledge about the operation state
and can construct and destruct it: the concrete \emph{OperationBehavior}
subclass.  See Section~\ref{sec:opBehavior} for details.

\subsubsection{Construction}
\label{ssec:operation-construction}

Within OSAL, operation objects are created and managed solely by the
operation pool, described in Section~\ref{sec:operationPool}.

Thanks to the fact that \emph{Operation} class does not contain any built-in
dependency to hardware implementation properties or state, the operation
model implemented by an \emph{Operation} instance can be shared by all the
hardware implementations of the operation.  Clients, therefore, cannot copy
or delete operation objects: for any given operation in a repertoire
instance there can be one and only one \emph{Operation} instance.

The operation pool assigns an \emph{OperationBehaviorProxy} to the operation
to construct instead of an actual operation behavior object.  See
Section~\ref{sec:opBehaviorProxy} for details on the behavior proxy.

\section{OperationDAG}
\label{sec:operationdag}

\subsection{Overview}

\emph{OperationDAG} is a graph description of an operation. Each Operation
may contain multiple DAG presentations and those can be used to simulate 
behavior of \emph{Operation} if there is not behavior plugin available for the 
\emph{Operation}.

Operation DAG is consists of two types of nodes \emph{OperationNode} and 
\emph{TerminalNode} and \emph{OperationDAGEdge}. \emph{TerminalNode}s
represents input and output operands of the DAG and \emph{OperationNode}s the 
operations that are done for the operands.

\emph{OperationDAG} is created from operation DAG language which is described
in TCE User Manual. Construction of DAG is done with \emph{OperationDAGConverter}
class.

\subsection{Interface}

Complete interface can be found from doxygen documantation.

\section{OperationDAGConverter}
\label{sec:operationdagconverter}

\subsection{Overview}

\emph{OperationDAGConverter} is a class for doing DAG Language -> DAG -> DAG Language
conversions and other DAG related transformations like expanding DAG format etc.

For building DAG class uses \emph{OperationDAGBuilder} class.

\subsection{Interface}

Complete interface can be found from doxygen documantation.

\section{Operand}
\label{sec:operand}

\subsection{Overview}

\subsection{Interface}

The static properties of operation inputs and outputs, described in the
specification document~\cite{OSAL-specs}. Latest class interface can be found
from doxygen documentation.

\section{OperationBehavior}
\label{sec:opBehavior}

\subsection{Overview}

\emph{OperationBehavior} is an abstract class.  A class that implements it
provides the behavior model of one operation.  Custom, user-defined
operations require users of TCE to define new behavior classes.  Operation
behavior classes are defined in plugin modules and are dynamically
imported. There is also \emph{OperationDAGBehavior} class which is used for
simulating \emph{OperationDAG}s.

The operation behavior model is not needed by every client of OSAL. For this
reason, concrete, operation-specific behavior classes are individually
imported and instantiated on demand, using a proxy mechanism (detailed in
Section~\ref{sec:opBehaviorProxy}).

\subsection{Interface}

All methods  have empty default implementation.  The implementation code for 
these methods is generated with the macro definition language, automatically, 
in \emph{OperationBehavior} subclasses. Complete interface is found from 
doxygen documentation.

\begin{description}
\item[simulateTrigger(inouts : SimValue**, context : OperationContext\&) :
  bool] %
  Simulate the process of starting the execution of an operation. The first
  argument is a vector of input and/or output values, accessible with macros
  INT, UINT, FLT, and DBL.

  The operation context is an object that models the state of the function
  unit on which the given operation is simulated. Through this object, the
  method can access and change the state of its operation, as described in
  Section~\ref{sec:context-if}.

  Result a flag that tells whether all results of the operation could be
  computed and are stored in the output vector (value \emph{true}) or not
  (return value \emph{false}).

  A version of this method without operation context argument is available
  for operations that do not have state.

\item[void createState(OperationContext\&)]%
  Create a new instance of operation state and register it into given
  operation context.

  This method is overridden only by concrete operation behavior subclass
  that need state.  Its empty, default implementation is enough for behavior
  classes that need no state.

  If an instance of the operation state class is already found in the
  \emph{OperationContext}, this method does nothing.  This guarantees that
  in any given context only one operation state instance for a given
  operation group exists.  This method can be safely called multiple times
  on the same operation context.  In particular, it can be called from
  several different concrete \emph{OperationBehavior} objects that share the
  same state.

\item[void deleteState(OperationContext\&)]%
  Remove an instance of operation state from the operation context and
  deletes it.

  This method is overridden only by concrete operation behavior subclass
  that need state.  Its empty, default implementation is enough for behavior
  classes that need no state.

  If no instance of the operation state class exists in the given
  \emph{OperationContext}, this method does nothing.  Therefore, this method
  can be safely called multiple times from the same concrete
  \emph{OperationBehavior} object or by several different object that share
  the same state.
\end{description}

The input (output) values are passed (returned) through vectors of input or
output values, described in Section~\ref{sec:simValue}.  The size of the
vectors is not known to the \emph{Operation} object, but it is guaranteed
that there are at least as many elements as the operation inputs plus
operation outputs.

Although the design would allow to build in protection of input and output
values from (most of) erroneous uses, an explicit design decision is made to
avoid protections of input and output values.  The choice is motivated by
the high performance impact that it could have, given that input and output
values are accessed in every simulation cycle, for every operation.

\section{OperationState}
\label{sec:operationState}

\subsection{Overview}

\emph{OperationState} is the base class of all concrete operation state
classes.  One such class stores state information for one or more operations
that share the same state.  A group of operations sharing the same state is
termed \emph{operation family}.

\subsection{Interface}

\begin{description}
\item[name() : string] %

  Return the name that identifies the operation state.

  This name consists of a unique string and identifies the concrete
  operation state class of an operation family. Since it identifies a static
  entity (the class), and not an instance, the string is hard-coded.

\item[isAvailable(context : OperationContext\&) : bool] %

  Return true if operation state is in ``available'' state, that is, it can
  accept new operation invocations. In case operation is ``not available'',
  the client should not go on with new operation invocations.

\item[advanceClock(context : OperationContext\&)] %

  This method must be invoked in every clock cycle until all results are
  computed. Failing to do so may result in loss of some of the pending
  results.

  The argument gives the operation context, described in
  Section~\ref{sec:context-if}.


\end{description}


\subsubsection{Construction of operation state.}
The clients of operation behavior delegate construction of operation state
objects (see Section~\ref{sec:operationState}) to the only entity that has
specific knowledge of such state: the operation behavior subclass.  To this
purpose, every user defined \emph{OperationBehavior} class that needs state
data must implement factory method \verb|createState()| and destructor
method \verb|deleteState()|.

Typically, a client will invoke the factory method once for each operation
object it is client of.  Multiple invocations do no harm.  The factory
method checks whether an operation-specific state instance is already
present in the operation context passed by the client.  If it is not, then
it creates and adds one instance, otherwise it does nothing.

\emph{Example.} Given 3 operations that share the same state and are
implemented in the same unit, the first call to the \verb#createState()#
method of any of the operations will construct a new operation state object.
Later calls (even to the same operation method) will have no effect on
\emph{OperationContext} object.

The mechanism for deleting an operation state instance is symmetrical with
respect to the factory method.  The client typically invokes the
\verb|deleteState()| method for each operation it is client for; no harm is
done if two or more of its operations share the same operation state.  Only
the first invoked will delete the instance.


\section{OperationBehaviorProxy}
\label{sec:opBehaviorProxy}

\subsection{Overview}

\emph{OperationBehaviorProxy} is a special implementation of the
\emph{OperationBehavior} interface (Section~\ref{sec:opBehavior}).  The
\emph{OperationPool} initializes the instances of \emph{Operation} it
creates with an instance of this class.  An operation behavior proxy object
waits for the first access to any of the \emph{OperationBehavior} methods.
Instead of simulating the operation behavior, the methods of the proxy find
the appropriate operation behavior class for the operation, and then
replace the reference to itself in the containing \emph{Operation} instance
with the created object.  The \emph{OperationBehavior} subclass is imported
from a dynamic module.  Future calls to behavior simulation methods of the
operation are thus delegated to the behavior object constructed by the
proxy, which completely disappears from the operation instance.

Because \emph{OperationBehaviorProxy} effectively obsoletes itself when
operation behavior is needed the first time, the pattern does not fully
match the model of ``Gang of Four'' Proxy pattern described
in~\cite{DesignPatterns}.  We decided to name our pattern ``Ghost Proxy'',
because the proxy ``disappears'' from the subject after carrying out its
instantiation task.

The behavior proxy delegates actual importing of the appropriate operation
behavior subclass to the helper class \emph{OperationBehaviorLoader}
(Section~\ref{sec:opBehaviorLoader}).  A reference to an instance of this
class is passed to the proxy constructor.

Figure~\ref{fig:proxySequence} shows a sequence diagram of the proxy
mechanism.

\begin{figure}[tb]
  \centerline{\psfig{figure=eps/proxySequence.eps,width=1.0\textwidth}}
  \caption{The Proxy Mechanism to Load Behavior Functions On-demand.}
  \label{fig:proxySequence}
\end{figure}

\subsection{Interface}

\begin{description}
\item[OperationBehaviorProxy(Operation\& targetOperation, OperationBehaviorLoader\& loader)]%
  A behavior proxy is assigned to one operation only.  The constructor
  records the operation it is assigned to and the object (behavior loader)
  to which delegate the actual work of finding and importing the behavior
  methods.

\item[$\sim$OperationBehaviorProxy()]%
  The destructor makes sure that the operation behavior instance imported by
  the proxy is deleted with the correct deletion function from the dynamic
  module.

  Deletion is delegated to the behavior loader, described in
  Section~\ref{sec:opBehaviorLoader}.
\end{description}


\subsection{Behavior simulation function implementations}

The behavior simulation methods of the \emph{OperationBehavior} interface
are implemented by the proxy with the following code:

\begin{verbatim}
bool simulateTrigger(...) {
   // next call results in loading the real behavior to the Operation
   OperationBehavior& ob_ = loader_.lookupBehavior(operation_.name());
   operation_.setBehavior(ob_);
   // operation has an instance of the actual behavior object now
   // let's call operation API and put proxy's job to test:
   return operation_.simulateTrigger(...);
}
\end{verbatim}

\section{OperationPropertyLoader}
\label{sec:opPropertyLoader}

\subsection{Overview}

The purpose of this class is to import the static information contents of an
operation from a data file.  This task is delegated to
\emph{OperationSerializer}, which hides the details of accessing the data
file, building the ObjectState tree structures, and selecting the sub-tree
corresponding to the requested operation.

\subsection{Interface}

\begin{description}
\item[void loadOperationProperties(Operation\&, const OperationModule\&)]%
  Redefines a given operation with the data loaded from the external file
  given by \emph{OperationModule}. This method should be called only on a
  newly created operation.

  If no data is found for the given operation (identified by its name), this
  method throws an ObjectNotFound exception.
  %

\end{description}

\subsection{Implementation}

The operation property loader minimises the input from external files by
keeping an internal data structure corresponding to each file that has been
accessed at least once.  This structure consists of a list of
\emph{ObjectState} trees, one for each operation definition found in the
file.  For a given operation module, the loader creates one such list the
first time the module is accessed.

Notice that the loader does not keep an index mapping operation names to
operation modules, it does not need it, because clients request an operation
specifying directly the sole module where such operation is expected to be
defined.


\section{OperationBehaviorLoader}
\label{sec:opBehaviorLoader}

\subsection{Overview}

The purpose of this class is to import operation behavior definitions from
dynamic modules.  Before loading a definition, the loader must locate the
operation module which contains it.  This task is delegated to
\emph{OperationIndex}.  The loader keeps a reference to its operation index
instance, which is directly managed by \emph{OperationPool}.  The lookup of
operation modules occurs in the order given by the operation index.

\subsection{Interface}

\begin{description}

\item[OperationBehaviorLoader(const OperationIndex\&)] %
  The constructor records a reference to an \emph{OperationIndex}, which
  will be used throughout loader extent to search for the operation module
  that contains the operation behavior definition.

\item[OperationBehavior\& lookupBehavior(name : const string\&, 
parent : const Operation\&)] %

  Look up the behavior definition for the operation identified by
  \emph{name}.

  First, the operation module that contains the operation is determined by
  means of the operation index.  Then, the appropriate \emph{PluginTools}
  instance is queried for the factory function that creates the behavior
  class instance.

  Method \verb|importSymbol()| of \emph{PluginTools} is invoked to lookup
  the factory function.  The method version with an explicit module name
  argument is used; the module name is obtained from \emph{OperationIndex}.

  The factory function to search for is:
  \begin{quote}\tt
    OperationBehavior* createOpBehavior\verb|_|\textsl{name}(const Operation\&)
  \end{quote}
  where \emph{name} is the given operation name.

  The module name argument of \verb|importSymbol()| is necessary to make
  sure that the definition is imported from the correct file.  The name of
  the module is retrieved from the \emph{OperationModule} instance
  (Section~\ref{sec:opModule}) found by \emph{OperationIndex}.

  The \emph{Operation} instance given as parameter to \verb|createOpBehaviorX|
  call is the operation the behavior belongs to, it's the parent operation
  given as argument for \verb|lookupBehavior()| call.

  The loaded factory function is then invoked in order to create an instance
  of the operation behavior subclass defined for the given operation.

  In case behavior module or factory function for the given operation cannot
  be found, a \emph{DynamicLibraryException} is thrown.

\item[freeBehavior(name  : const string\&)]

  Delete the behavior object model of the operation identified by
  \emph{name}.

  The \emph{OperationBehavior} instance of \emph{Operation} cannot be
  deleted safely with \verb|delete|, because delete is an operator that can
  be overloaded in the dynamic module or in the client application.  In
  either case, there would be a mismatch between the \verb|new| and
  \verb|delete| operators.  Therefore, a destructor function for the
  specific class is looked up from the same plugin module that contains the
  factory function of the operation behavior subclass.

  Method \verb|importSymbol()|  of \emph{PluginTools} is invoked to lookup
  the destructor function.  The method version with an explicit module name
  argument is used; the module name is obtained from \emph{OperationIndex}.

  The destructor function to search for is:
  \begin{quote}\tt
    void deleteOpBehavior\verb|_|\textsl{name}(OperationBehavior*)
  \end{quote}
  where \emph{name} is the given operation name.

  The module name argument of \verb|importSymbol()| is necessary to make
  sure that the definition is imported from the correct file.  The name of
  the module is retrieved from the \emph{OperationModule} instance
  (Section~\ref{sec:opModule}) found by \emph{OperationIndex}.

  The imported function is invoked in order to delete the behavior object.

  In case behavior module or destructor function for the given operation 
  cannot be found, a \emph{DynamicLibraryException} is thrown.

\end{description}

The operation behavior loader relies on
\emph{PluginTools}~\cite{ToolkitDesign} services for looking up and
importing operation definitions from dynamic modules.  The loader contains
one instance of \emph{PluginTools} for each lookup path that is accessed.
The instances are constructed lazily.  Construction is postponed until the
first time an operation from a module at the given path is required.  The
lookup paths are stored in \emph{OperationIndex}
(Section~\ref{sec:opIndex}).

\section{OperationContext}
\label{sec:context-if}

\subsection{Overview}

Class \emph{OperationContext} holds any implementation- and
context-dependent data that may affect the result of an operation. In
typical client applications, operation implementations are related to a
function unit.

\subsection{Interface}

\subsubsection{Operation State Lookup}

The most important operation implemented by \emph{OperationContext} is the
lookup of operation state:

\begin{description}
\item[state(name : const string\&) : OperationState\&] %
  Look up the (concrete) operation state identified by the string
  \emph{name}.

  Throw \emph{KeyNotFound} exception if no state with given name is found.
\end{description}

\subsubsection{Managing Instances of Operation State}

The following API allows to register and unregister concrete operation state
instances into the operation context.  These methods are public, but not
visible to the client.  They are only used by \emph{OperationBehavior}
objects when creating or deleting an operation state object (see
Section~\ref{sec:opBehavior}).

\begin{description}
\item[registerState(state : OperationState*)] %
  Register an operation state instance. Called in method \emph{createState}
  of concrete \emph{OperationBehavior} subclasses.

  Abort if an instance of the same type is already registered.

\item[unregisterState(state : OperationState*)]%
  Unregister an operation state instance.  Called in method
  \verb|deleteState()| of concrete \emph{OperationBehavior} subclasses.

  Abort if given operation state instance is not registered in the context.
\end{description}

\subsubsection{Advancing the Clock of the State Instances}

The following method is used to advance the internal clock of each state
instance registered to the \emph{OperationContext} instance.

\begin{description}
\item[advanceClock()] %
  Advance the clock of each registered \emph{OperationState} instance.
\end{description}

\subsubsection{Special State}
In addition to operation-specific state, the operation context includes
information with a larger scope, which affects certain kinds of operations.
Unlike operation state, which affects only a group of related operations,
this information can affect unrelated operations in the same function unit
or even in different function units. Special state is directly handled by
the function unit state model and not by operations.

The following methods let the clients access special state:
\begin{description}
\item[memory() : Memory\&]%

  Return a pointer to the Memory Module that can be accessed by the
  function unit and behavior simulation methods. 
  
\item[setMemory(memory : Memory\&)]%

  Set the Memory Module pointer. 

\item[programCounter() : SimValue\&]%

  Return a reference to the current value of the program counter register.
  The value of the program counter can be changed through this reference.
  This is used to implement control transfer operations like jumps and
  calls to subroutines. 

\item[returnAddress() : SimValue\&] %

  Return a reference to the current value of the return address
  register.  The value of the return address can be changed through
  this reference.  This is used in implementing calls to
  subroutines. 

\item[contextId() : int] %
  
  Return the unique number that identifies the context. This id is used,
  for example, to make memory access requests from different operation
  implementations unique.  No two instances of \emph{OperationContext} have
  the same value in execution time. 

\end{description}


\section{SimValue}
\label{sec:simValue}

\subsection{Overview}

The input and output values accessed by operations through simulation
functions of \emph{OperationBehavior} class (see Section~\ref{sec:opBehavior}) 
are modeled by the class \emph{SimValue}.
This class represents any data type that can be manipulated by operations of
the target architecture template, and provides the interface to access the
data in three predefined types.

\emph{SimValue} contains a union of built-in types IntWord, FloatWord and
DoubleWord, whose properties are described in~\cite{OSAL-specs}.

\subsection{Interface}

When a SimValue is constructed, its bit width is defined:
\begin{description}
\item[SimValue(width : const int)]
\end{description}

The data contained by \emph{SimValue} is accessed directly. Data is stored
in union with name \emph{value\_}. Complete interface is described in doxygen 
documentation.

The width of a \emph{SimValue} instance is set once and for all at
construction time.

\section{OperationModule}
\label{sec:opModule}

\subsection{Overview}

\emph{OperationModule} is a simple concrete type that represents the names
and the path of external files where operation definitions are stored.  A
collection of these files related to the same group of operations is termed
operation module.  Currently, an operation module consists of two files: a
dynamic module containing the behavior object model of operations and a
data file containing the properties of the same operations.  The dynamic
module is optional and may be completely omitted.

\subsection{Interface}

\begin{description}
\item[OperationModule(const string\& name, const string\& path)]
\item[string name()]%
  Name of the operation module.
\item[string path()]%
  Path of the operation module.
\item[bool definesBehavior()]%
  Return true if the operation module includes a file containing behavior
  object model definitions.
\item[string behaviorModule()]%
  Name of the dynamic module where definitions of behavior object models
  for the operation module are stored.  Throw exception \emph{FileNotFound}
  if no behavior module is defined for this operation module.
\item[string propertiesModule()]%
  Name of the data file where definitions of operation properties for the
  operation module are stored.
\end{description}

\section{OperationIndex}
\label{sec:opIndex}

\subsection{Overview}

The class \emph{OperationIndex} has a crucial role in making lookup of
operation data (properties and behavior) efficient.  Operation indexes are
owned by individual \emph{OperationPool} objects; their services are used by
the operation behavior loader (Section~\ref{sec:opBehaviorLoader}) and
certain type of clients of \emph{OperationPool}.


\subsection{Interface}

The methods of \emph{OpertionIndex} may be divided in three groups:
\begin{enumerate}
\item Methods to manage the search paths.
\item Methods to manage the operation modules.
\item Methods to manage the operation index.
\end{enumerate}

\subsubsection{Construction}
\emph{OperationIndex} objects are created with an empty constructor.  Before
an object thus created can be used for anything useful, it must be
initialized with one or more search paths.

\subsubsection{Path List Management}

An \emph{OperationIndex} keeps a list of absolute search paths.  These paths
are searched for operation modules.  An operation module consists of two
files: an XML data file and a dynamic module.  \emph{OperationSerializer}
and \emph{OperationBehaviorLoader} use the operation module to lookup the
required operation definitions.

Path lookup order of operations is important because the first found
operation takes precedence over the possibly later found operations. See
Chapter~4 of Functional Specification Document~\cite{OSAL-specs} for a list
of searched paths.

The API for managing search path is the following:
\begin{description}
\item[addPath(const string\& path)]%
  Append \emph{path} to the search path list.  The added path is the last
  that will be searched for operation definitions.

 Whenever a path is added to an operation index, it is immediately scanned,
 and a list of all operation modules found in it is built and integrated with
 the list of modules already found. 

 If the given path is not found, function returns immediately. It is
 also not an error if path is found but doesn't provide any operation 
 definitions.

\item[string path(int i)]%
  Return the path in \emph{i}-th position in the search list.
\item[int pathCount()]%
  Return the number of paths registered into the operation index.
\end{description}


\subsubsection{Operation Modules Management}


The single operation modules found by an operation index can be accessed
using the following methods:
\begin{description}
\item[OperationModule module(int i)]%
  Return the operation module in \emph{i}-th position in the overall search
  list.
\item[int moduleCount()]%
  Return the total number of operation modules registered into the
  operation index.
\item[OperationModule module(int i, const string\& path)]%
  Return the operation module in \emph{i}-th position in the list of
  modules found in \emph{path}.
\item[int moduleCount(const string\& path)]%
  Return the number of operation modules registered into the operation
  index and found in \emph{path}.
\item[void addModule(OperationModule* module, string path)]%
  Add a module in a given path.
\item[void removeModule(const std::string\& path, const std::string\& modName)]%

  Remove the module with given search path and name from OperationIndex.
  Throw \emph{PathNotFound} in case path is not found, and 
  \emph{InstanceNotFound} if module with given name is not found.

\item[void refreshModule(const std::string\& path, const std::string\& modName)]%

  Refresh (force reload) of module with given search path and name.
  Throw \emph{PathNotFound} in case path is not found, and 
  \emph{InstanceNotFound} if module with given name is not found.

\end{description}

\subsubsection{Operation Name List Management}

The main purpose of \emph{OperationIndex}, as the name suggests, is to keep
an index of operations.  The most important service offered by
\emph{OperationIndex} is to give the name of the operation module where a
given operation is defined.  The following methods let clients access the
operation names:
\begin{description}
\item[OperationModule moduleOf(const string\& name)]%
  Return the operation module where the operation with name \emph{name} is
  defined.
\item[string operationName(int i, const OperationModule\& om)]%
  Return the name of operation found at \emph{i}-th position in operation
  module \emph{om}.
\item[int operationCount(const OperationModule\& om)]%
  Return the number of operations defined in the operation module
  \emph{om}.
\end{description}

\subsection{Implementation}

\subsubsection{Data Structures}
The main data structures that implement \emph{OperationIndex} are:
\begin{description}
\item[list<string>]%
  An ordered list of paths that are searched for operation modules.
\item[map<string, OperationModule*>]%
  An associative map that relates operation with operation modules.
  Operations are identified by their name.
\end{description}

Note: the actual \emph{OperationModule} objects are not stored in the
associative map, but in a separate private container.  This ensures
efficient handling of the map.  No problem can occur with pointers, because
the APIs that relate to \emph{OperationModule} return copies of the stored
objects.

\subsubsection{Lazy Design}
All activity in an \emph{OperationIndex} is postponed until it is made
necessary by a request through its public API.  This design choice is
motivated by efficiency.

When a search path is added to the list, the list of operation modules is
updated with all the modules found in the new path.  The single modules,
however, are not scanned for operation definitions.

When an operation-related method is invoked, \emph{OperationIndex} scans,
one, by one, the operation modules until it finds a module where an
operation with the given name is found. \emph{OperationIndex} uses
\emph{OperationSerializer} class for reading the XML definitions
of operations.  The operation container is
updated with all the operations defined in each scanned module.  Thus, is
may well occur that a module is never accessed simply because all operations
requested by clients are defined in modules that are searched before it.

Clients can force scanning of a operation module by invoking method
\verb|operationCount()| with the module as argument.


\chapter{BEHAVIOR DEFINITION LANGUAGE}
\label{cha:language}

Behavior definition language is implemented with a set of C preprocessor
\#define macros. These macro definitions are located in OSAL.hh which is
thoroughly commented.

Operation DAG Language is described in TCE User Manual.

\chapter{REJECTED ALTERNATIVES}

% Rejected alternatives for (parts of) the design should be listed here with
% the reasoning and date the alternative was dumped. For future reference.
%
%18.11.2003 . . .
\begin{description}
\item[24.05.2004 --- Public API to load operation behaviour.] %
  Instead of using a Proxy mechanism, the operation behaviour could be
  loaded explicitly by clients before they start to use the behaviour
  interface.  This alternative would make the implementation much simpler,
  but has been rejected because it is less convenient for clients, it
  requires discipline and is error prone.  Efficiency is not a concern,
  because a Proxy machanism without permanent overhead can be implemented
  (Ghost Proxy).
\end{description}


\chapter{IDEAS FOR FURTHER DEVELOPMENT}

% Ideas that are not part of the design yet but that might be added in the
% future are listed here. This is to have ideas written down somewhere.
% This list should contain the date of addition, the idea described briefly
% and the inventor of the idea.

\begin{description}
\item[23.02.2004]%
  Additional \emph{OperationPool} interface to manage paths in which the
  data files of operation definitions are stored.  Currently, the path list
  is hidden inside \emph{OperationIndex} and is hard-coded by
  \emph{OperationPool}.  A possible API would be the following:
  \begin{description}
  \item[pathList()]%
    Return a colon-separated list of search paths.
  \item[addToPathList()]%
    Add one or more colon-separated paths to the list.
  \item[removeFromPathList()]%
    Remove one or more colon-separate paths from the list.
  \end{description}
  --A. Cilio
\item[25.04.2006] ---

  Documentation filed in operation properties.

  Add support for an (optional?) additional field that contains the textual
  description of the operation.
\end{description}


\chapter{PENDING ISSUES}

% Pending issues concerning this design, a sort of TODO list.
% This chapter should be empty when implementation of this module/subsystem
% is completed.

\chapter{MAINTENANCE}

% This chapter treats the problem of extending the design with new
% capabilities that fit in a predefined framework.

% For example, in case of the TPEF module, this chapter could describe the
% steps that must be taken in order to add a new Section subclass and/or a
% SectionReader subclass.

% Remove this chapter no obvious and standardized way to extend the design
% with new capabilities exists.


% ------------------------------------------------------------------------

%% Remove this part if there are no references.  Usually there will be at
%% least a reference to the functional specifications of the same module.

%  References are generated with BibTeX from a bibtex file.
\bibliographystyle{alpha}
\cleardoublepage
%% Equivalent to a chapter in the table of contents
\addcontentsline{toc}{chapter}{BIBLIOGRAPHY}
\bibliography{Bibliography}



\end{document}

%%% Local Variables: 
%%% mode: latex
%%% TeX-master: t
%%% End: 
