\documentclass[a4paper,twoside]{tce}
\usepackage{pslatex}

\begin{document}
\author{Andrea Cilio, Ari Mets\"ahalme}
\title{Program Object Model}
\ver{0.13}
\firstday{08.08.2004}
\lastday{25.09.2006}
% id number in S- sequence
\docnum{017}
% draft/complete/committed
\state{draft}

\maketitle


\chapter*{Document History}

\begin{HistoryTable}
 0.1    & 08.08.2004 & A. Cilio        &
 Initial, incomplete revision.\\

 0.1.1  & 13.08.2004 &  P. J\"a\"askel\"ainen &
 Moved \emph{MemoryContents} to memory module design doc.\\

 0.2    & 30.12.2004 & A. Cilio        &
 Major rewrite. Added \emph{CodeSnippet}, \emph{DataDef}, symbolic
 expressions, \emph{Terminal} classes, and much more.\\

 0.2.1 & 07.01.2005 & P. J\"a\"askel\"ainen &
 Fixed a couple of typos and formatting issues. \\

 0.3   & 23.01.2005 & A. Cilio &
 Corrections following review by A.\ Mets\"ahalme and  P. J\"a\"askel\"ainen.
 Clarified merging of unit and global scopes.\\

 0.4   & 29.01.2005 & A. Cilio &
 Added methods to retrieve the address of instructions.\\

 0.5   & 07.02.2005 & A. Cilio &
 Corrected interface of move and terminal classes (error found by P.\
 J\"a\"askel\"ainen).\\

 0.6 & 06.10.2005 & A. Mets\"ahalme &
 Major update to current implementation status. Added classes for keeping
 jump addresses up-to-date.\\

 0.6.1 & 08.10.2005 & A. Cilio &
 Minor updates. Name change to a copy method.\\

 0.7    & 11.10.2005 & A. Cilio &
 Partial update to planned implementation.\\

 0.8    & 12.10.2005 & A. Cilio &
 Revised FU terminal ports, clarified ``pseudo-operation''.\\

 0.8.1  & 13.10.2005 & A. Cilio &
 Renamed hint operation method of FU port terminal.\\

 0.8.2  & 14.10.2005 & A. Cilio &
 Changed behaviour of operation methods of terminals.\\

 0.8.3  & 14.10.2005 & A. Cilio &
 Added alternative design (pending) of move guards.\\

 0.9    & 21.12.2005 & A. Cilio &
 Corrected wrong behaviour of \emph{Program} constructor.\\

 0.10   & 29.03.2006 & A. Cilio &
 Added terminal method for accessing operation in/out indices.\\

 0.11   & 11.04.2006 & A. Cilio &
 Revised \emph{Terminal} interface, changed access to address. \\

 0.12   & 13.04.2006 & A. Cilio &
 \emph{ImmediateValue} replaces \emph{SimValue} in \emph{Immediate}.\\

0.13   & 05.06.2006 & M. Lepist\"o &
 Added design notes of data memory support.\\

0.14  & 25.09.2006 & P. J��skel�inen &
 Added the annotation API. \\

\end{HistoryTable}

\tableofcontents


\chapter{INTRODUCTION}

\section{Purpose and Scope}

This document describes the Program Object Model (POM) of TTA Codesign
Environment (TCE). It describes how the Program Object Model is designed and
the reasons behind the design.

This document is a reference manual destined to developers and maintainers
who need to know how Program Object Model is designed.

This document is not the primary source of information for developers of
client applications that use POM. This document provides a userful overview
of the object model from this document, but the bulk of the interface
documentation is found in the API reference hypertext manual.


\section{Definitions}

% Uncomment this to enable the ``term explanation box''
\begin{description}
\item[Inline Immediate]%
  A constant number encoded in the source field of a TTA instruction slot
  that is used to program a move.
\item[Long Immediate]%
  A constant number encoded in one or more slots of a TTA instruction.
\item[Program]%
  A TTA program destined to run on a TTA target processor.
\item[Machine]%
  A TTA target processor in the TCE environment.  A machine can be
  represented as an XML file or as an object hierarchy in client
  applications.
\item[Move]%
  A programmed data transport on a TTA processor.  Data transports are the
  minimum amount of independent, programmable activity.
\end{description}


\section{Acronyms and Abbreviations}

\begin{tabular}{p{0.10\textwidth}p{0.80\textwidth}}
ADF   & Architecture Definition File\\
GPR   & General-Purpose Register\\
MAU   & Minimum Addressable Unit (of a memory address space)\\
MOM   & Machine Object Model \\
OSAL  & Operation Set Abstraction Layer \\
PC    & Program Counter \\
POM   & Program Object Model \\
TCE   & TTA Codesign Environment \\
TTA   & Transport Triggered Architectures \\
XML   & eXtensible Markup Language \\
\end{tabular}


\chapter{MODULE ARCHITECTURE}

This chapter describes the high-level architecture of the POM module.

\section{``Philosophy of Design''}

Programs for a TTA target processor are written in high-level language by
TCE users and then compiled into TTA code.  Since the main goal of the TCE
toolset is an efficient target application (or a set thereof) that runs on a
target TTA processor, TTA programs are fundamental entities of the TCE
toolset.

The Program Object Model provides a model of target TTA programs for TCE
applications that need to access and manipulate the TTA code of a target
program.

\subsection{High-Level Symbolic Information}

The Program Model tries to preserve the symbolic and source-level
information available; in particular:
\begin{enumerate}
\item Variable, label and procedure names.
\item Binding between symbols and addresses.
\item Scope of symbols.
\end{enumerate}

This information, usually lost and not available in low-level program
representations, is extremely useful for effective code analysis and
optimisations even in later, machine-dependent phases of code generation.

\subsection{Minimal, Client-Neutral Design}

The guiding principle of the POM is simplicity of design.  Another important
attribute is client-neutrality, which can be stated as ``Only services of
general use are provided''.  Client-specific services, no matter how simple
or essential, are not provided in POM.

A motivation for not supporting client-specific information is that the
contents of information of POM should be always up to date.
%
\note{related: pending issue \ref{ch:pending:aux-sections}}
%
Any information that is only applicable or relevant to a specialised client
(or the lack of such information) would have to be managed throught any POM
instance extent, no matter if created by a client that does not need it.

Another problem that arises when supporting client-specific information is
that such support is open ended: it cannot be determined in advance what
additional support is convenient for a new specialised client.  Whenever
additional support is implemented, the interface needs extensions.

\subsection{Dependencies among Model Parts}

For efficiency reasons and to keep complexity of the design under control,
the parts of the model (for example, instructions, moves, symbols) have
minimal cross dependencies and require limited bookkeeping. For example: an
instruction does not know its relative position within a sequence of
instructions that forms a basic block or a procedure. Instead, instructions
have a back pointer to their parent procedure.

Model parts can always exist in isolation (that is, not registered within a
higher-level part). The responsibility of efficiently figuring out relative
positions of model parts (for example, previous and next instruction) lies
on the parent objects.


\section{External Modules}

Figure~\ref{fig:ext-mod} depicts the dependencies of POM module with other
domain modules and with external files. POM lies at the fundations of the
domain library and depends only on two domain modules:
OSAL~\cite{OSAL-specs} and MOM~\cite{MOM-design}. Like most domain modules,
POM depends also on toolkit library services.

\begin{figure}[tb]
  \centerline{\psfig{figure=eps/ExternalModules.eps,width=0.60\textwidth}}
  \caption{Dependencies between POM and other modules and external files.}
  \label{fig:ext-mod}
\end{figure}

This section describes the domain modules POM depends on.

\subsection{TPEF Files}

The primary data source of TTA programs is TPEF files. Typically, a
\emph{Program} instance is built by a client module out of a TPEF object
model. However, POM has no dependency with TPEF files and with their object
model. The client (for example, a \emph{Builder} class) completely insulates
POM from the TPEF interface.

TPEF is by no means the only data source, and a builder is not the only way
in which a program can be created. Client applications can implement other
services that let one create TTA programs.

\subsection{Machine Object Model}

A scheduled TTA program necessarily contains information about the target
architecture it is scheduled for. The Program Object Model contains
references to objects of the Machine Object Model, and depends on the
interfaces implemented by MOM classes.
%
\note{PENDING: \ref{ch:pending:MOM-api-dep}}

Uncheduled TTA programs contain references to a special, \emph{Singleton}
instance of the target architecture model, called Universal
Machine~\cite{TCE-progTemplate}.

\subsection{Operation Set Abstraction Layer}

Although TTA programs are ultimately composed by data transports, or moves,
POM retains a link to the actual operation that is performed as side effect
of executing certain moves. POM relies on the Operation Set Abstraction
Layer to represent any properties of the operations it contains.

POM depends on the static properties of the operations in order to organise
its internal representation of the TTA program. Thus, POM is a client of
OSAL interface for accessing operation properties.

\section{Architecture of Databases}

The Program Object Model does not use any database proper. Instead, all
information contents of a TTA program is encoded in a binary format (TPEF)
and stored in an external file.

Like traditional object file formats, TPEF is organised in file sections.
TPEF is an extensible format: it is possible to define new special-purpose
sections for specific TCE applications. Thus, the format of additional base
of data that is logically attached to the program must ultimately conform to
the TPEF file section framework. An example of such base of data is the
execution profile data.

The TPEF specification document~\cite{TPEF-specs} and the Binary Handling
Module Design Notes~\cite{TPEF-design} describe TPEF and how the design of
the corresponding used in TCE.


\section{Module Communication}

The Program Object Model is a monolithic software module.  Internal
dependencies are at level of single class objects.


\chapter{Module Design}

%% This chapter is repeated for each module in case this design document
%% describes a multi-module larger module (or application).


\section{Overview}

% Overview of the module design in concrete terms.

The Program Object Model implements a model of target TTA applications at a
symbolic level of abstraction. The main class of POM is \emph{Program}. The
program model has a simple structure:
\begin{enumerate}
\item %
  The program is seen as an ordered sequence of nonoverlapping procedures:
  class \emph{Procedure}.
\item %
  Each procedure is a flat sequence of instructions: class
  \emph{Instruction}.
\item %
  Each instruction is a nonordered set of moves: class \emph{Move}.
\item %
  An instruction can also hold slots that encode long immediate bits and
  specify a destination for them: class \emph{Immediate}.
\end{enumerate}

This class hierarchy does not model control flow, data flow or intermediate
data structures such as basic blocks. Client applications that need more
sophisticated levels of abstractions must use the graph
framework~\cite{graph-design} or define a specialised object model.

\begin{figure}[tb]
  \centerline{\psfig{figure=eps/Overview.eps,width=0.35\textwidth}}
  \caption{Fundamental classes of the Program Object Model and their main
    relations.}
  \label{fig:overview}
\end{figure}

Figure~\ref{fig:overview} shows the classes that represent the main entities
of the Program Object Model (program, procedure and scopes) and their most
important relations.
%
In addition to classes that model the code, POM comprises a number of helper
classes that model important aspects of TTA programs, such as:
\begin{enumerate}
\item %
  Symbolic expressions that represent memory addresses: classes
  \emph{SymbolicExpr} and \emph{AddressTerm}.
\item %
  Program symbols: classes \emph{CodeLabel}, \emph{DataLabel}.
\item %
  Scope of program symbols: class \emph{Scope}.
\end{enumerate}

In POM, the contents of the \emph{data} memory used by variables is modeled
with bit accuracy, whereas the \emph{instruction} memory is only modeled at
a symbolic level, with the objects listed above.

Contents of the \emph{data} memory are maintained in a \emph{MemoryContents}
object \cite{MemoryModuleDesign}.

When program and data are stored in the same memory, the word ``memory
segment'' should be intended where ``memory'' is found. Throughout this
document the word ``memory'' is sometimes used to mean either a memory
address space or a segment thereof. For example ``a program symbol in a data
memory'' should be intended ``a program symbol in a data segment of memory''
when instruction and data are in the same memory.

\section{Interfaces}

\subsection{Program}
\label{ssec:prog-if}

A \emph{Program} instance represents a TTA program in a given area of the
instruction memory.  \emph{Program} does not define the memory image
corresponding to the program that should be downloaded into the target
memory system.  For this, additional information from an instruction binary
encoding map~\cite{ProGeSpecs} is required.

The addresses of instructions are stored in the \emph{Program}
representation. Addresses may have to be adjusted (relocated) as a side
effect of, for example, adding a procedure or removing a number of
instructions from the program. Relocation of instruction and data addresses
requires a certain amount of automatic, internal bookkeeping.
%
Relocations need not occur only as side effects. \emph{Program} provides
also an interface to explicitly relocate an entire data area in one of the
target address spaces.
%
In some target TTA processors it may not even be possible to relocate
parallel code without having to completely reschedule the program. This is
because of the way long immediates are encoded. A program signals such cases
by throwing an exception.

\begin{description}
\item[Program(space : AddressSpace, start: Address, entry: Address)]%
  Constructor. Create an empty program in given address space. Automatically
  create an empty global scope to store all program symbols with external
  linkage. If the start address is not specified, the default value zero is
  used instead. If neither start or entry address is specified, both default
  to zero. If only the start address is specified, then the entry address
  defaults to the start address value.

\item[globalScope() : GlobalScope\&]%
  Return the global scope of this program.

\item[targetProcessor() : Machine\&]%
  Return the target TTA processor (\emph{Machine} instance) of the program.

\item[startAddress() : Address]%
  Return the start (lowest) address of the program. The start address of the
  program is not necessarily the lowest address of the instruction address
  space.

\item[entryAddress() : Address]%
  Return the address of the program entry point. The entry point is not
  necessarily the instruction  of the program with the lowest address.

\item[setEntryAddress(address: Address)]
  Set the entry address of the Program.

\item[addProcedure(proc: Procedure*)]
  Insert the given procedure as the last procedure of the program.
  The ownership of the procedure will be passed to the program.

\item[addInstruction(ins: Instruction*)]
  Insert the given instruction as the last instruction of the last
  procedure in the program.
  The ownership of the instruction will be passed to that procedure.

\item[relocate(proc: Procedure\&, howMuch: IntWord)]
  Relocate all the procedures after the given procedure by \emph{howMuch}.

\item[firstProcedure() : Procedure\&]
  Return the first procedure of the program.

  Throw \emph{InstanceNotFound} if there are no procedures.

\item[lastProcedure() : Procedure\&]
  Return the last procedure of the program.

  Throw \emph{InstanceNotFound} if there are no procedures.

\item[nextProcedure(proc: Procedure\&) : Procedure\&]
  Return the procedure following the given procedure or a null
  procedure, if the given procedure is the last procedure in the program.

  Throw \emph{IllegalRegistration} if the given procedure does not
  belong to the program.

\item[procedureCount() : int]
  Return the number of procedures in the program.

\item[procedure(index: int) : Procedure\&]
  Return the procedure at the given index.

\item[procedure(name : const string\&) : Procedure\&]%
  Return the procedure with given name. In case several procedures exist
  with same name (from different compilation units), it returns one of the
  procedures (it is undefined which one).

  Throw \emph{KeyNotFound} exception if no procedure with given name is
  defined in the entire program.

\item[procedure(name : const string\&, index : const int) : Procedure\&]%
  Return the procedure at position \emph{index} in the list of all
  procedures of given name.

  Throw \emph{OutOfRange} exception if given index exceeds the number of
  procedures with given name.

\item[firstInstruction() : Instruction\&]
  Return the first instruction in the program.

  Throw \emph{InstanceNotFound} if there are no instructions in the
  program.

\item[instructionAt(address : IntWord) : Instruction\&]%
  Return the instruction at a given address. The address space of the
  address is implied, since there is only one address space for
  instructions.

  Throw exception \emph{KeyNotFound} if given address is illegal.

\item[nextInstruction(const Instruction\&) : Instruction\&] %
  Return the next instruction in the instruction stream.

  Return a special \emph{null instruction} if the given instruction is the
  last instruction of the program.

  Throw \emph{IllegalRegistration} exception if given instruction does
  not belong to the program.

\item[lastInstruction() : Instruction\&]
  Return the last instruction in the program.

  Throw \emph{InstanceNotFound} if there are no instructions in the
  program.

\item[instructionReferenceManager() : InstructionReferenceManager\&]
  Return the instruction reference manager of the program.

\end{description}

\note{Following Methods not yet implemented}

\begin{description}

\item[procedure(name : const string\&, scope : const UnitScope) : Procedure\&]%
  Return the procedure with given name in the compilation unit identified by
  \emph{scope}.

  Throw\note{
%
    DISCUSS: exception type}
%
  exception if no procedure with given name is defined in the unit scope.

\item[procedureCount(name : const string\&) : int]%
  Return the number of procedures (defined in different compilation units)
  with the requested name.

\item[previousInstruction(const Instruction\&) : Instruction\&] %
  Return the previous instruction in the instruction stream.

  Return a special \emph{null instruction} if the given instruction is the
  first instruction of the program.

  Throw \emph{IllegalRegistration} exception if given instruction does not
  belong to the program.

\item[relocate(space : AddressSpace, offset : int)]%
  Relocate the memory address space \emph{space} by the given relative
  offset. All references to addresses are updated by adding the
  corresponding offset.

\item[relocateProgram(offset : int)]%
  Relocate the entire area of memory that contains the program code. If the
  program data is stored in the same memory, also the data area is
  relocated. All instruction addresses are shifted by given relative offset,
  and all references to addresses (stored as immediates or as data
  definitions) are updated correspondingly.

  This method is just a shortcut for \verb|relocate()| with the address
  space that contains the program given as parameter.

\item[insertBefore(ins : const CodeSnippet\&, pos : const Instruction\&]%
  Insert the sequence of instructions \emph{ins} before instruction
  \emph{pos}. All instructions following \emph{pos} are relocated to make
  room for the new instructions, and all references to relocated
  instructions are updated.

  Throw exception \emph{IllegalRegistration} if \emph{pos} does not belong
  to the program, or if the instructions in the sequence \emph{ins} already
  belong to a procedure.

\item[insertAfter(ins : const CodeSnippet\&, pos : const Instruction\&)]%
  Insert the sequence of instructions \emph{ins} after instruction
  \emph{pos}. All instructions from \emph{pos} on are relocated to make
  room for the new instructions, and all references to relocated
  instructions are updated.

  Throw exception \emph{IllegalRegistration} if \emph{pos} does not belong
  to the program, or if the instructions in the sequence \emph{ins} already
  belong to a procedure.

\item[remove(inStart : const Instruction\&, inEnd : const Instruction\&) :
  CodeSnippet]%
  Remove from the program the instructions from \emph{inStart} to
  \emph{inEnd}, both included. Return a code snippet with the instructions
  removed. The end instruction can be omitted; in that case, only
  \emph{inStart} is removed. The instructions become independent.  All
  instructions that follow \emph{inEnd} are relocated so that the address
  that was occupied by \emph{inStart} is occupied by the instruction that
  immediately follows \emph{inEnd}. All references to relocated instructions
  are updated.
%
  \note{DISCUSS: leak-prone IF. Return \textsl{CodeSnippet} to simplify
    deallocation and prevent leaks?}
%
  Throw exception \emph{IllegalRegistration} if either of the given
  instructions does not belong to the program.

\item[insertProcedureBefore(proc : Procedure\&, pos : const
  Procedure\&, scope : const UnitScope\&)]%
  Insert procedure \emph{proc} before procedure \emph{pos}. All instructions
  following \emph{proc} are relocated to make room for the new instructions.

  Throw exception \emph{IllegalRegistration} if \emph{pos} does not belong
  to the program, or if \emph{proc} belongs to another program.
%
  Throw [[exception]] \note{DISCUSS: exception type} if the unit scope is
  not specified and the program owns more than one unit scope.
%
  Throw [[exception]] if any of the labels in the global or the unit scope
  of \emph{proc} clashes with a label in the corresponding scope of the
  program. See Section~\ref{ssec:proc-if} for details).

\item[insertProcedureAfter(proc : Procedure\&, pos : const Procedure\&,
  scope : const UnitScope\&)]%
  Insert procedure \emph{proc} after procedure \emph{pos}. All instructions
  following \emph{proc} are relocated to make room for the new instructions,
  and all references to relocated instructions are updated. The symbols in
  the local copy of the unit scope are copied into the given units scope,
  \emph{scope}.

  Throw \emph{IllegalRegistration} exception if \emph{pos} does not belong
  to the program, or if \emph{proc} belongs to another program.
%
  Throw [[exception]] \note{DISCUSS: exception type} if the unit scope is
  not specified and the program owns more than one unit scope.
%
  Throw [[exception]] if any of the labels in the global or the unit scope
  of \emph{proc} clashes with a label in the corresponding scope of the
  program. See Section~\ref{ssec:proc-if} for details).

\item[removeProcedure(proc : Procedure\&)]%
  Remove procedure \emph{proc} from the program. The procedure becomes
  independent (see Section~\ref{ssec:proc-if} for details about independent
  procedures). All instructions following \emph{proc} are relocated
  appropriately.

  Throw\note{
%
    DISCUSS: exception type}
%
  exception if \emph{proc} does not belong to the program.

\item[addUnitScope(name : const string\&, prefix : const string\&)]%
  Add a new unit scope to the program. A unit scope is uniquely identified
  by the name of the compilation unit (typically, an object file) and a
  prefix string (typically, the absolute path of the object file).

  Throw exception \emph{ObjectAlreadyExists} if the program already contains
  a unit scope with same identification strings.

\end{description}

\subsection{Procedure}
\label{ssec:proc-if}

When a procedure is created, it is \emph{independent}, that is, not
registered to any program. However, such a procedure may contain code that
refers to a global symbol, and global (or unit-scope) symbols belong to
\emph{GlobalScope} and \emph{UnitScope}, which are managed by
\emph{Program}. How to solve this conflict?
%
Direct references between a procedure and the objects that belong to a
program must be avoided. An independent procedure, therefore, maintains its
own copy of the global and the scope corresponding to the compilation unit
it belongs to.

Creating a new independent procedure is trivial, since the scopes are
initially all empty. There are two cases, however, in which a completely
nonempty, independent procedure is obtained:
\begin{enumerate}
\item %
  When a procedure is removed from its program.
\item %
  When a procedure is copied.
\end{enumerate}

A procedure created by copying another procedure is always independent, even
if the original procedure is registered in a program.  The scope tree rooted
at the procedure scope is simply duplicated and passed to the copy. The
outer scopes (unit and global), however, are treated differently. The
\verb|copy()| method scans the original procedure and collects the global
symbols referenced by the code. These symbols are all duplicated, and the
duplicate is stored in the unit and global scopes that are managed by the
copy.

\begin{figure}[tb]
  \begin{tabular*}{\textwidth}[h]{c@{\hspace{0.15\textwidth}}c}
    \psfig{figure=eps/ProcedureCloningA.eps,width=0.25\textwidth} &
    \psfig{figure=eps/ProcedureCloningB.eps,width=0.50\textwidth} \\
    (a) & (b) \\
  \end{tabular*}
  \caption{Copying a registered procedure: (a) before copy; (b) after copy.}
  \label{fig:proc-copy}
\end{figure}

Figure~\ref{fig:proc-copy} depicts the object diagram corresponding to a
registered procedure before and after it is copied.  Unit scope $F'$ and
global scope $G'$ are subsets of the unit and global scopes of the program
that contains the copied procedure.  $F'$ and $G'$ contain only those
symbols that are referenced within the copied procedure.

The addresses of the procedure copy are unchanged, that is, they remain the
same as the original procedure.  The procedure still belongs to the original
address space.

The procedure-owned global and unit scopes undergo an opposite process of
merging when an independent procedure is inserted into a program. All labels
contained in the procedure-owned global and unit scope are added into the
corresponding scope of the program. If a label with the same name already
exists in the corresponding scope of the program, two things can occur:
\begin{enumerate}
\item %
  If the address of the equally-named labels is identical, the label of the
  procedure-owned scope is simply discarded: all references to it will
  correctly point to the label in the scopes owned by program.
\item %
  If the address or the address space of the equally named labels is
  different, then the labels are clashing, and a [[exception]].
  \note{DISCUSS: exception type} is thrown.
\end{enumerate}

POM makes no attempt to determine when two labels with the same name and
different address refer to the same symbol that just happened to be
relocated in the program.

New procedures are created with an empty global scope and an empty, nameless
unit scope. Once attached to a program, however, a procedure must be treated
as coming from an actual compilation unit. This is necessary for the clients
that construct the program object model from some external data source, like
TPEF, in order to preserve information about the original compilation units.
In other words, the symbols referenced inside the procedure code belong to a
given unit scope. For this reason, the methods that insert a procedure into
a \emph{Program} accept the destination unit scope of the procedure, as
described in Section~\ref{ssec:prog-if}.
%
Conversely, when a procedure is removed from a program, its copy of the unit
scope is turned into a nameless scope. In other words, procedures do not own
any specific information about the actual compilation unit they come from.

\begin{description}
\item[Procedure(name: const string\&, space: AddressSpace, start: IntWord)]%
  Constructor. Create an empty procedure in given address space. If the
  start address is not specified, the default value zero is used instead.
  Automatically create an empty procedure scope to store all program symbols
  that are local to this procedure.

\item[parent() : Program\&]%
  Return the parent program of the procedure.  If the procedure is not
  registered in a program,
  %
  \note{more specific exception, e.g., MissingRegistration?}
  %
  thow exception \emph{IllegalRegistration}.

\item[setParent(prog : Program\&)]
  Set a new parent for the procedure.

\item[isInProgram() : bool] %
  Return true if the procedure belongs to a program (it is registered).

\item[name() : string]
  Return the name of the procedure.

\item[alignment() : int]
  Return the alignment.

\item[address(ins: Instruction\&) : Address]
  Return the address of the given instruction.

  Throw \emph{IllegalRegistration} if the instruction does not belong
  to this procedure.

\item[startAddress() : Address]%
  Return the start (lowest) address of the procedure.

\item[setStartAddress(start : Address)]
  Set the start address of the procedure.

\item[endAddress() : Address]
  Return the end address of the procedure, that is, the next address
  that doesn't belong to this procedure.

\item[instructionCount() : int]
  Return the number of instructions in the procedure.

\item[instructionAt(address : IntWord) : Instruction\&]%
  Return the instruction at a given address. The address space of the
  address is implied, since there is only one address space for
  instructions.

  Throw \emph{KeyNotFound} if given address is illegal.

\item[hasNextInstruction(ins: Instruction\&) : boolean]
  Tell whether the given instruction is the last instruction of the procedure.

\item[entryPointAddress(index : int) : Address]%
  Return the address of the entry point at position \emph{index}. A
  procedure can have several entry points. Entry points are stored in
  ascending order by their address.

\item[entryPointCount() : int]%
  Return the number of known\footnote{
  %
    Due to limitations of program analysis, a procedure may have more entry
    points than what the program model knows.}
  %
  entry points of the procedure.

\item[nextInstruction(const Instruction\&) : Instruction\&]%
  Return the next instruction in the instruction stream.
%
  Return a special \emph{null instruction} if the given instruction is the
  last instruction of the procedure.  Compare this behaviour with that of
  the method by the same name in \emph{Program}
  (Section~\ref{ssec:prog-if}).

  Throw \emph{IllegalRegistration} exception if given instruction does not
  belong to the procedure.

\item[lastInstruction() : Instruction\&]
  Return the last instruction in the procedure.

  Throw \emph{IllegalRegistration} if there are no instructions in the
  procedure.

\item[addInstruction(ins: Instruction*)]
  Add an instruction as the last instruction in the procedure.
  Remaining procedures in the parent program will be relocated as
  needed. The ownership of the instruction will be passed to the procedure.

\item[insertInstructionAfter(pos: Instruction\&, ins: Instruction*)]
  Inserts \emph{ins} after \emph{pos}. Instructions from \emph{pos}
  are relocated to make ``room'' for the new instruction.

  Throw \emph{IllegalRegistration} if pos does not belong to the
  procedure or ins already belongs to a procedure.

\end{description}

\note{Following methods not implemented yet}

\begin{description}

\item[previousInstruction(const Instruction\&) : Instruction\&] %
  Return the previous instruction in the instruction stream.
%
  Return a special \emph{null instruction} if the given instruction is the
  start instruction of the procedure.  Compare this behaviour with that of
  the method by the same name in \emph{Program}
  (Section~\ref{ssec:prog-if}).

  Throw \emph{IllegalRegistration} exception if given instruction does not
  belong to the procedure.

\item[label(name : const string\&) : Label\&]%
  Return the label with given name.  If no label with given name exists in
  the scope of the procedure, returns the special \emph{null label}.

\item[label(name : const string\&, scope : const Scope\&) : Label\&]%
  Return the label with given name in the given scope.  The scope must be a
  local scope of the procedure, or the procedure scope itself.  If the scope
  is the scope of the procedure, this method is equivalent to the
  one-argument variant.  If no label with given name exists in the scope of
  the procedure, returns the special \emph{null label}.

  Throw \emph{IllegalRegistration} if the given scope is not a scope that
  belongs to this procedure.

\item[insertBefore(ins : const CodeSnippet\&, pos : const Instruction\&)]%
  Insert the sequence of instructions \emph{ins} before instruction
  \emph{pos}. All instructions from \emph{pos} on are relocated to make room
  for the new instructions.

  Throw exception if \emph{pos} does not belong to the procedure, or if
  \emph{ins} belongs to another procedure.

\item[insertAfter(ins : const CodeSnippet\&, pos : const Instruction\&)]%
  Insert the sequence of instructions \emph{ins} after instruction
  \emph{pos}.  All instructions following \emph{pos} are relocated to make
  room for the new instruction.

  Throw \emph{IllegalRegistration} exception if \emph{pos} does not belong
  to the procedure, or if \emph{ins} belongs to another procedure.

\item[remove(inStart : const Instruction\&, inEnd : const Instruction\&) :
  CodeSnippet]%
  Remove from the procedure instructions from \emph{inStart} to
  \emph{inEnd}, both included. Return a code snippet with the instructions
  removed. The end instruction can be omitted; in that case, only
  \emph{inStart} is removed. The instructions become independent.  All
  instructions that follow \emph{inEnd} are relocated so that the address
  that was occupied by \emph{inStart} is occupied by the instruction that
  immediately followed \emph{inEnd}.

  Throw \emph{IllegalRegistration} exception if \emph{ins} does not belong
  to the procedure.

\item[nearestLabel(addr : const Address\&) : CodeLabel\&]%
  Return the label closest to address \emph{addr} with an address that is
  not greater than \emph{addr}.  If several labels have the same instruction
  address, return one of the labels.  The label is chosen arbitrarily, but
  it is guaranteed that always the same label of a group of equivalent
  labels is returned.

\item[copy() : Procedure\&]%
  Make a complete copy of the procedure. The instructions are copied one by
  one (see section~\ref{ssec:ins-if}). The copy is identical, except that it
  is not registered to the program of the original procedure. Therefore, any
  address it refers to is not meaningful. Scopes are created and attached to
  the copy, as explained above and depicted in Figure~\ref{fig:proc-copy}.
\end{description}

\subsection{Instruction}
\label{ssec:ins-if}

Class \emph{Instruction} represents a TTA instruction.  TTA instructions
consist of two main parts: a set of data transports (represented by class
\emph{Move}, described in Section~\ref{ssec:move-if}) and a set of long
immediates (represented by class \emph{Immediate}, described in
Section~\ref{ssec:longimm-if}).

\emph{Instruction} class is not responsible for knowing:
\begin{enumerate}
\item %
  The instruction address.
\item %
  The neighbouring instructions within a program (when applicable).
\end{enumerate}

The only way to obtain this information is through the parent procedure of
the instruction.

Instructions can be marked with code labels (section~\ref{ssec:clabel-if}).
These labes are stored in the innermost scope that contains the instruction
(it may be a procedure scope or a local scope).

\begin{description}
\item[parent() : Procedure\&] %
  Return the parent procedure that contains the instruction.  If the
  instruction is not registered anywhere,
  %
  \note{more specific exception, e.g., MissingRegistration?}
  %
  throw exception \emph{IllegalRegistration}.

\item[setParent(proc: Procedure\&)]
  Set a new parent for the instruction.

\item[isInProcedure() : bool]%
  Return true if the instruction belongs to a procedure (it is registered).

\item[addMove(move: Move*)]
  Add a move to the instruction.

  Throw \emph{ObjectAlreadyExists} if the move has already been added
  to this instruction.

\item[moveCount() : int] %
  Return the number of moves contained in this instruction.

\item[move(i : const int) : Move\&] %
  Return the move at position index \emph{i} in this instruction.  The order
  of moves is arbitrary, no assumption should be made by clients.

\item[immediateCount() : int] %
  Return the number of immediate registers written by the instruction
  template of this instruction.

\item[immediate(i : const int) : Immediate\&]%
  Return the immediate write action (\emph{Immediate} class, see
  section~\ref{ssec:longimm-if}) at position index \emph{i} in this
  instruction.  The order of immediates is arbitrary, no assumption should
  be made by clients.

\item[address() : Address]%
  Return the address of the instruction.

  Throw \emph{IllegalRegistration} if the instruction does not belong to a
  procedure that is registered to a program.

\item[copy() : Instruction\&]%
  Make a complete copy of the instruction.  The moves and the immediates
  containted in the instructions are copied one by one (see
  sections~\ref{ssec:move-if} and~\ref{ssec:longimm-if}).  The copy is
  identical, except that it is not registered to the procedure of the
  original instruction (and therefore, any address it refers to is not
  meaningful).

\end{description}

\note{Following methods not yet implemented}

\begin{description}

\item[moveCount(mb : Bus) : int]%
  Return the number of moves programmed onto bus \emph{mb}. In architectures
  without segmented busses, the move count can only be one or zero.

\item[move(mb : const Bus\&) : Move\&] %
  Return the move that programs bus \emph{mb} in this instruction.  Return
  the special \emph{null move} object if the instruction does not program
  \emph{mb}.\note{
%
    future extension: multiple moves on one segmented bus}

\item[immediate(mb : const Bus\&) : Immediate\&] %
  Return the immediate that is programmed on bus \emph{mb} in this
  instruction. Return the special \emph{null immediate} object if the
  instruction does not use the slot of \emph{mb} to store immediate
  bits. Note that several bus slots may be used to store immediate
  bits for the same immediate.

\end{description}

\subsection{Move}
\label{ssec:move-if}

Class \emph{Move} represents a data transport through the TTA programmable
interconnection network, also called simply ``move''. A move has a source
and a destination, and, optionally, a Boolean predicate.

A move destination may represent an operand of an operation; at least one of
the operands of any operation triggers the operation; thus, a move may start
an operation as ``side effect'' of moving one of its operands. The
destination of a move may be a GPR; in that case, the move simply copies the
source value into a temporary storage for later, possibly repeated, use.

\emph{Example: Moves.}
The following moves: (1) write a constant into one of the operands of a
multiply operation; (2) copy one of the results of a multiply operation into
a GPR; (3) copy a GPR into another register of the same register file.
\begin{verbatim}
1:  0x78900 -> mul.1
2:  mul.4 -> r9
3:  r6 -> r7
\end{verbatim}

The source of certain moves represents the memory address of other TTA
instructions.  Some of such moves (jumps and calls) perform a control
transfer. Control transfer moves contain an indirect reference to the
target instruction.

When a move source represents an address expression (pointing to another
instruction or a data location), it is possible to retrieve the instruction,
respectively the data symbol, from the parent program or procedure.

\begin{description}
\item[Move(src : const Terminal\&, dst : const Terminal\&, bus : Bus\&,
  guard : Guard\&)]%
  Constructor. Create a move with \emph{src} and \emph{dst}, respectively,
  as source and destination. The move is predicated by the Boolean
  expression \emph{guard}. The transport is carried on the transport bus
  \emph{bus}.

\item[Move(src : const Terminal\&, dst : const Terminal\&, bus : Bus\&)]%
  Constructor. Create an unguarded (unconditional) move with \emph{src} and
  \emph{dst}, respectively, as source and destination. The transport is
  carried on the transport bus \emph{bus}.

\item[parent() : Instruction\&]%
  Return the parent instruction of the move.  If the move is independent,
  %
  \note{more specific exception, e.g., MissingRegistration?}
  %
  throw exception \emph{IllegalRegistration}.

\item[setParent(ins: Instruction\&)]
  Set a new parent instruction for the move.

\item[isInInstruction() : bool] %
  Return true if this move belongs to an \emph{Instruction} instance.

\item[isUnconditional() : bool] %
  Return true if this move is not predicated.

\item[source() : Terminal]%
  Return the source of this move.

\item[setSource(src: Terminal*)]
  Set a new source for the move.

\item[destination() : Terminal]%
  Return the destination of this move.

\item[setDestination(src: Terminal*)]
  Set a new destination for the move.

\item[guard() : Guard]%
  Return the Boolean expression that guards this move.

\item[setGuard(guard: MoveGuard*)]
  Set a new guard for the move.

\item[bus() : Bus]%
  Return the transport bus programmed by this move.

\item[setBus(bus: Bus\&)]
  Set a new bus for the move.

\item[copy() : Move\&]%
  Make a copy of the move.  The copy is identical, except that it is not
  registered to the instruction of the original move (and therefore, any
  address it refers to is not meaningful).

\end{description}

\note{Following methods not yet implemented}

\begin{description}

\item[target() : Instruction\&] %
  Return the target instruction of the move.  Return the special \emph{null
  instruction} if the move represents and indirect control transfer (for
  example, a computed jump or an indirect function call).
%
  Return a special \emph{null instruction} when the target instruction of
  the move cannot be determined by static analysis. This is the case, for
  example, of indirect jumps.

  Throw \emph{WrongObject}
  %
  \note{DISCUSS: type of exception}
  %
  exception if the method is called on a move that does not perform a
  control transfer.

  A move can look up its target instruction automatically only if
  registered. Throw \emph{IllegalRegistration} if the move is not registered
  in a procedure, or if the move is not registered in a program and the
  target instruction does not belong to the same procedure.

\item[addressExpression() : SymbolicExpr]%
  Return the symbolic address expression referenced by this move in symbolic
  form (see Section~\ref{ssec:symexpr-if}).  Return a special \emph{null
    expression} object if given move does not refer to an address.  Return a
  special \emph{undefined expression} object if given move refers to an
  address whose symbolic expression cannot be computed.

\item[isSourceMemoryAddress() : bool]%
  Return true if the source of this move is used as a memory address.

\item[isDestinationMemoryAddress() : bool]%
  Return true if the destination of this move can be used as a memory
  address by some of the moves that read it.

\end{description}

\subsection{Immediate}
\label{ssec:longimm-if}

In addition to data transports programmed by moves
(Section~\ref{ssec:move-if}), a TTA instruction (Section~\ref{ssec:ins-if})
can specify also one or more special data transports. These data transports
copy a constant value encoded in a TTA instruction (called long immediate)
into a register that belongs to an immediate unit.

Constant values are represented by instances of \emph{ImmediateValue}
(Section~\ref{ssec:ImmediateValue}).

The actual bit width of the instruction field(s) where the immediate bits
are encoded is defined by the hardware resources assigned to the immediate;
the overall width of the immediate value is implied by the slot widths.

Class \emph{Immediate} represents the long immediate value and its transport
into a dedicated register, and implements the following methods:
\begin{description}
\item[Immediate(value : ImmediateValue, dst : Terminal, template : const
  InstructionTemplate\&)]%
  Constructor. Create a long immediate with the specified integer value. The
  destination register is given by \emph{dst} and the set of instruction
  slots where the immediate bits are encoded is specified by the
  \emph{template} argument.

  Throw \emph{IllegalParameters} exception if the destination immediate unit
  of \emph{dst} is not one of the destinations of the given template.

\item[destination() : Terminal]%
  Return the destination register of this immediate.

\item[value() : ImmediateValue]%
  Return the value of this immediate.

\item[instructionTemplate() : InstructionTemplate\&]%
  Return the instruction template used to encode the bits of this immediate.

\item[copy() : Immediate\&]
  Makes a copy of the immediate.

  The copy is identical, except that it is not registered to the instruction
  of the original immediate (and therefore, any address it refers to is not
  meaningful).

\end{description}

Note that the value of an \emph{Immediate} object cannot be modified once
created. Clients must simply create a new instance and replace the old
instance if a change in immediate value is required.

\subsection{ImmediateValue}
\label{ssec:ImmediateValue-if}

\emph{ImmediateValue} represents the actual value of an immediate, whether
it is a long immediate encoded in instruction slots or an in-line immediate
stored in the source field of a move.

\emph{ImmediateValue} is a plain typedef to a 32-bit signed integer.
%
\note{could be a class in the future, with shortcut methods}
%
Typical operations performed on instances of this class are MathTools'
\emph{RequiredBits} and \emph{RequiredBitsSigned}.
%
Furthermore, to ensure that the value actually stored in an
\emph{ImmediateValue} fits into the destination (slot, source field),
functions \emph{SignExtendTo} and \emph{ZeroExtendTo} are used.

\subsection{MoveGuard}
\label{ssec:guard-if}

Guards are one-term Boolean expressions that represent execution
predicates. Class \emph{MoveGuard} wraps in the \emph{Guard} class of
the Machine Object Model, and implements the following methods:

%
\begin{description}
\item[MoveGuard(guard : Guard\&)]%
  Constructor.

\item[isUnconditional() : bool]%
  Return whether this guard is a constant true or false value. Return always
  false.

\item[isInverted() : bool]%
  Return true if the Boolean value computed out of the contents of the
  source register is inverted.

\item[guard() : Guard\&]%
  Return the guard object of the Machine Object Model.

\end{description}

\paragraph{Unconditional guards.}
A specialised guard class, \emph{UnconditionalGuard}, models Boolean guard
expressions with a constant value (either always true or always false).
Unconditional guards do not read any register or port. If the invert flag is
true, then the guard is always `false', and the guarded move is never
executed. This case corresponds to a special type of ``no-operation''.
Conversely, if the invert flag is false, the guard is always 'true'. This
case applies to unconditional moves, including moves carried on busses that
have no guard at all.

Only methods \verb|isUnconditional()| and \verb|source()| of
\emph{UnconditionalGuard} are redefined. The first returns `true', the
second throws an exception.
%
\note{DISCUSS: exception of type \emph{IllegalOperation}?}


\subsection{Terminal}
\label{ssec:terminal-if}

Class \emph{Terminal} models any possible source or destination of a
move. A terminal can be one of the following:
\begin{enumerate}
\item%
  An inline immediate: a constant number, represented by an instance of
  \emph{ImmediateValue}. The immediate can represent an address, or even
  more precisely, an instruction address.
\item%
  A GPR: a register of a register file.
\item%
  A long immediate: a register of an immediate unit.
\item%
  A function unit result: a port of a function unit.
\item%
  A function unit input: a port of a function unit.
\end{enumerate}

Class \emph{Terminal} provides the following interface:
\begin{description}

\item[isImmediate() : bool]%
  Tell whether the terminal is an inline immediate value.

\item[isAddress() : bool]
  Tell whether the terminal is an inline immediate value that refers
  to an address.

\item[isInstructionAddress() : bool]
  Tell whether the terminal is an inline immediate value that refers
  to an instruction address.

\item[isImmediateRegister() : bool]%
  Tell whether the terminal is a long immediate register.

\item[isGPR() : bool]%
  Tell whether the terminal is a general-purpose register.

\item[isFUPort() : bool]%
  Tell whether the terminal is a function unit port (possibly, including an
  operation code.

\item[isOpcodeSetting() : bool]
  Tell whether terminal transports an opcode to a function unit port.

  Throw \emph{WrongSubclass} if the terminal is of wrong class type.

\item[value() : ImmediateValue]%
  Return the value of the inline immediate.

  Throw \emph{WrongSubclass} if the terminal is of wrong class type.

\item[address() : Address]%
  Return the address of an immediate terminal which represents a memory
  address.

  Throw \emph{WrongSubclass} if the terminal is of wrong class type.

\item[instructionReference() : InstructionReference\&]
  Return a reference to the instruction at which the immediate points.

  Throw \emph{WrongSubclass} if the terminal is of wrong class type.

\item[registerFile() : RegisterFile\&]%
  Return the register file of the general-purpose register.

  Throw \emph{WrongSubclass} if the terminal is of wrong class type.

\item[immediateUnit() : ImmediateUnit\&]%
  Return the immediate unit of the long immediate register.

  Throw \emph{WrongSubclass} if the terminal is of wrong class type.

\item[functionUnit() : FunctionUnit\&]%
  Return the function unit of the port.

  Throw \emph{WrongSubclass} if the terminal is of wrong class type.

\item[index() : IntWord]%
  Return the index of the register (either a long immediate or a
  general-purpose register).

  Throw \emph{WrongSubclass} if the terminal is of wrong class type.

\item[operation() : Operation\&]%
  Return the operation code transported into the function unit port.

  Throw \emph{WrongSubclass} if the terminal is of wrong class type.
  Throw \emph{InvalidObject} if the terminal is of the the right class type,
  but the FU port of the terminal is not opcode-setting.

\item[port() : Port\&]%
  Return the port. The port can belong to a register file, an immediate unit
  or a function unit.

  Throw \emph{WrongSubclass} if the terminal is of wrong class type.

\item[setIndex(index : IntWord)]%
  Change the register of the register file to the given index.

  Throw \emph{OutOfRange} exception if \emph{index} is negative or is not
  smaller than the size of the register file or immediate unit it belongs
  to. Throw \emph{WrongSubclass} if the terminal is of wrong class type.

\item[setOperation(operation : Operation\&)]%
  Change the operation code transported into the function unit port to the
  given operation.

  Throw \emph{InstanceNotFound}
%
  \note{DISCUSS: more specific exception type?}
%
  exception if the function unit does not support the operation.
  Throw \emph{WrongSubclass} if the terminal is of wrong class type.

\item[operationIndex() : int]%
  Return the index that identifies the operation input or output that is
  represented by this terminal.

  Throw \emph{WrongSubclass} if the terminal is of wrong class type. Throw
  \emph{InvalidObject} if the terminal is of the the right class type, but
  the FU port of the terminal is not opcode-setting and bears no auxiliary
  ``hint operation'' information.

\item[copy() : Terminal\&]%
  Create an exact copy of the terminal and return it. Every concrete class
  implements a specific version of this method.
\end{description}

\paragraph{Restrictions on the implementation of the Terminal API.}
Obviously, not all methods of the \emph{Terminal} interface can be used with
all types of terminal. The following table shows which method applies (cell
marked with `x') to which type of terminal. Terminal types are: \emph{ii}
(inline immediate), \emph{gpr}, \emph{ir} (long immediate register),
\emph{ifu} function unit input port, \emph{ofu} function unit output port.

\begin{quote}
\begin{tabular}{|l|c|c|c|c|c|}
\hline
 method & \emph{ii} & \emph{gpr} & \emph{ir} & \emph{ifu} & \emph{ofu} \\
\hline
 \verb|value|          & x & - & - & - & - \\
\hline
 \verb|index|          & - & x & x & - & - \\
\hline
 \verb|operation|      & - & - & - & x & x \\
\hline
 \verb|operationIndex| & - & - & - & x & x \\
\hline
 \verb|port|           & - & x & x & x & x \\
\hline
 \verb|registerFile|   & - & x & - & - & - \\
\hline
 \verb|immediateUnit|  & - & - & x & - & - \\
\hline
 \verb|functionUnit|   & - & - & - & x & x \\
\hline
\end{tabular}
\end{quote}

The setting methods for index and operation have the same restrictions of
the corresponding inspecting method.

The \emph{operationIndex} method is applicable to FU port terminals only if
the port is opcode-setting, or if it bears auxiliary ``hint operation''
information.

\paragraph{Restriction in position of terminals.}
Not all types of terminals can appear in both source and destination
positions of a move. For example, in normal programs input ports of a
function unit cannot be sources of moves. Inline immediates, on the other
hand, can definitely be only move sources. When a terminal is used as guard
source registers, the same restrictions of move destinations apply, except
that the FU ports are usually output ports.

The following sections describe in detail the concrete specialised
\emph{Terminal} subclasses.

\subsection{TerminalImmediate}

Class \emph{TerminalImmediate} represents an inline immediate. This class
redefines method \emph{isImmediate} and method \emph{value}. The latter
returns the actual constant contained in the inline immediate field of the
move. All other methods throw an exception of type \emph{WrongSubclass}.
%
\note{DISCUSS: more specific exception types?}

The constructor is:
\begin{description}
\item[TerminalImmediate(value : ImmediateValue)]%
  Create an inline immediate containing the constant \emph{value}.
\end{description}

\subsection{TerminalRegister}

Class \emph{TerminalRegister} represents a reference to registers from
general-purpose register files and immediate units. These types of terminals
are characterised by the fact that multiple, functionally identical
registers are accessed through the same port. Registers are identified by an
index. This class redefines the following methods:
\begin{description}
\item[index() : IntWord]%
\item[port() : Port\&]%
\item[registerFile() : RegisterFile\&]%
  Applicable only if the unit of the terminal is an instance of
  \emph{RegisterFile}. Throw exception \emph{WrongSubclass} otherwise.
\item[immediateUnit() : ImmediateUnit\&]%
  Applicable only if the unit of the terminal is an instance of
  \emph{ImmediateUnit}. Throw \emph{WrongSubclass} exception otherwise.
\item[setIndex(index : IntWord)]%
\end{description}

See Section~\ref{ssec:terminal-if} for a description of these methods.

The constructor is:
\begin{description}
\item[TerminalRegister(unit : const Unit\&, port : const Port\&, index :
  IntWord)]%
  Create a move terminal that represents a register in unit \emph{unit},
  accessed through port \emph{port} and identified by \emph{index}.
\end{description}

All other methods throw an exception of type \emph{WrongSubclass}.
%
\note{DISCUSS: more specific exception types?}

\subsection{TerminalFUPort}

Class \emph{TerminalFUPort} represents an input or output port of a function
unit and (when applicable) the operation code written into it. Notice that,
in principle, operation codes can be written into FU output ports.\footnote{
%
  Opcode-setting output ports are a critical requirement for a general
  support of trigger-on-result behaviour.}
%
This class redefines the following methods:
\begin{description}
\item[operation() : Operation\&]%
\item[port() : Port\&]%
\item[functionUnit() : FunctionUnit\&]%
\item[setOperation(operation : Operation\&)]%
\item[isOpcodeSetting() : bool]%
\item[operationIndex() : int]%
\end{description}

See Section~\ref{ssec:terminal-if} for a description of these methods.

There are three types of FU port terminals:
\begin{enumerate}
\item %
  Opcode-setting FU port terminals, which, in addition to the port
  represent also one of its opcodes.
\item %
  Normal FU port terminals, which represent any FU ports except the one that
  sets the opcode. These ports do not carry any operation information, nor
  does the terminal.
\item %
  Annotated FU port terminals, which represent any FU ports except the one
  that sets the opcode. These ports do not carry any operation information,
  but the terminal keeps a ``pseudo-operation'' code. This code, when
  possible, is preserved for several reasons (readability, ease of
  conversion, extra sanity checks).
\end{enumerate}

The constructors are of two kinds:
\begin{description}
\item[TerminalFUPort(op : HWOperation\&, index : const int)]%
  Create a FU port terminal instance representing the FU port bound to the
  input or output of operation \emph{op} that is identified by \emph{index}.

  If the port is opcode-setting, then the terminal represents also the
  opcode carried by the port and that corresponds to \emph{op}. If port does
  not set the opcode, then the operation is just auxiliary information, and
  the terminal represents an annotated FU port.
\item[TerminalFUPort(port : const Port\&)]%
  Create a FU port terminal instance representing the non-opcode-setting
  port \emph{port}. Method \emph{operationIndex} cannot be called on this
  type of terminal.

  Throw \emph{WrongSubclass} if \emph{port} is opcode-setting.
\end{description}

All other methods, when applied to a \emph{TerminalFUPort}, throw an
exception of type \emph{WrongSubclass}.

This class provides a special method that is not part of the terminal
interface to access, when present, pseudo-operation information:
%
\note{this solution forces downcasts, but putting the method in base IF
  would ``pollute'' the real meaning of terminals}
%
\begin{description}
\item[hintOperation() : Operation\&] %
  Return the operation annotated into the terminal. If the FU port of the
  terminal is opcode-setting, return the actual operation. Return the
  \emph{NullOperation} if the port is not opcode-setting and no
  pseudo-operation is associated with the terminal.
\end{description}

\paragraph{``Instability'' of operation indices.}
Different operations can have different input or output indices bound to the
same FU port, so it is possible for a \emph{setOperation} call to change
also the operation index returned by subsequent invocations of
\emph{operationIndex}, even if the FU port of the terminal is unchanged.

\subsection{TerminalAddress}

Class \emph{TerminalAddress} is a specialisation of \emph{TerminalImmediate}
and represents an inline immediate that represents a memory address. It
redefines the following method:
\begin{description}
\item[address() : Address]
\end{description}

The constructor is:
\begin{description}
\item[TerminalAddress(location : ImmediateValue, space : AddressSpace\&)]%
  Using the immediate value given by \emph{location} and the reference to an
  address space, it creates an \emph{Address} instance.
\end{description}

\subsection{TerminalInstructionAddress}

Class \emph{TerminalInstructionAddress} is a specialisation of
\emph{TerminalImmediate} and represents an inline immediate that
refers to an instruction address (usually a jump target). It stores a
reference to the instruction, and the value of the immediate is
determined by the actual address of the referred instruction. This
way, immediates that are jump targets are automatically updated, if
the target instruction is relocated. The class redefines the
following methods:

\begin{description}
\item[value() : ImmediateValue]
\item[instructionReference() : InstructionReference\&]
\end{description}

\subsection{Address}
\label{ssec:addr-if}

\emph{Address} is a simple type that represents an address in the target
memory system of the TTA program. An address consists of a number that
identifies a memory location and a reference to an address space.

An address may be illegal for two reasons. The first reason, which depends
on the context (that is, the way the address is used), is that the address
violates an alignment constraint. The second reason is that the address is
out of the range of its address space. Even the second type of error is not
automatically caught by \emph{Address}.

The methods implemented by \emph{Address} are:
\begin{description}
\item[Address(location : IntWord, space : AddressSpace\&)]%
  Create an address for a given location in given address space.
\item[location() : IntWord]%
  Return the location address.
\item[space() : const AddressSpace\&]%
  Return the address space of the address.
\end{description}

\subsection{Labels}
\label{ssec:labels-if}

Symbolic information is represented by means of \emph{labels}. Labels mark
locations of code or data area of the program.  All labels have three basic
attributes: (1) an address, (2) a name, (3) a scope.  The following API
provides access to these attributes:
\begin{description}
\item[name() : string]%
  Return the name of this label.
\item[address() : Address\&]%
  Return the address of the location corresponding to this label.
\item[scope() : Scope\&]%
  Return the innermost scope that contains this label.
\end{description}

Labels are owned by scopes, described in Section~\ref{ssec:scope-if}.
Scopes do not provide methods to add new labels, because the labels are
automatically registered to a scope by their constructor: a label cannot
exist without its scope.

A label must have a unique name within the owning scope. It does not matter
if there are other labels with the same name in outer scopes, because the
inner scope has precedence. As a result, all labels in global scope must
have a unique name.

\subsection{Code Label}
\label{ssec:clabel-if}

A code label, also simply called ``label'', is a symbolic name that stands
for a location in the program. Class \emph{CodeLabel} models code labels and
implements the following methods:
\begin{description}
\item[CodeLabel(ins : const InstructionReference\&, name : const string\&)]%
  Create a new code label for given instruction, \emph{ins}.  The new label
  is stored in the innermost scope that encloses \emph{ins}.  If the name of
  the label, \emph{name} is not specified, a default name is assigned.  The
  default name is created automatically, and is made unique within its scope
  by appending a number to a base string `\verb|label_|'.

  Throw an exception \emph{IllegalRegistration} if \emph{ins} does not
  belong to a procedure.

\item[instructionReference() : InstructionReference\&]%
  Return a reference to the instruction corresponding to this label.

\item[address() : Address\&]%
  Return the address of the instruction corresponding to this label.

\item[procedure() : Procedure\&]%
  Return the procedure that contains this label.
\end{description}

Multiple labels can represent the same location, see
Section~\ref{ssec:clabel-imp} for a discussion of how to handle multiple
equivalent labels.

\subsection{Data Label}
\label{ssec:dlabel-if}

A data label is a symbol that represents a location in the data area used by
the program.  Usually, a data label is the starting address of a program
variable.

\begin{description}
\item[DataLabel(address : Address, scope : Scope\&, name : constr string\&)]%
  Create a new data label in given scope with name \emph{name}, at given
  address.

\item[address() : Address\&]%
  Return the address of the program variable (or part thereof) corresponding
  to this label.
\end{description}

\subsection{Symbolic Address Expressions}
\label{ssec:symexpr-if}

Class \emph{SymbolicExpr} represents memory addresses as expressions made of
symbols, coefficients and constant offsets.
%
The symbolic expressions modelled by class \emph{SymbolicExpr} have a
tree-like structure whose root node represents the entrire expression, and
each branch node represents a subexpression.  \emph{SymbolicExpr} is an
implementation of the \emph{Composite} design pattern described
in~\cite{DesignPatterns}.

A symbolic expression is made of one or more terms. Multiple terms are
combined two by two, by an operator. Figure~\ref{fig:sym-expr} depicts the
classes that implement simbolic address expressions. A complex symbolic
expression consists of two terms combined by an operator. Each term can be a
leaf of the expression (a label or a constant) or a complex sub-expression.

\begin{figure}[tb]
  \centerline{\psfig{figure=eps/AddressExprClass.eps,width=0.70\textwidth}}
  \caption{Class diagram of address expression and expression term.}
  \label{fig:sym-expr}
\end{figure}

\emph{Exampe.} Expressions associated with a sequence of moves that stores
the constant 500 to symbolic address $4L+52$, where \emph{L} is a data
label.

\begin{tabular}[h]{ll}
  move                                         & symbolic address \\
  \verb|4 -> mul.o|                            &  \\
  \verb|0x8000 -> mul.t|                       & $L$ \\
  \verb|mul.r -> r1|                           & $4L$ \\
  \verb|52 -> add.o|                           &  \\
  \verb|r1 -> add.t|                           & $4L$ \\
  \verb|add.r -> r2|                           & $4L+52$ \\
  \verb|500 -> st.o|                           &  \\
  \verb|r2 -> st.t|                            & $4L+52$ \\
\end{tabular}

\emph{SymbolicExpr} defines the following interface:
\begin{description}
\item[isLeaf() : bool]%
  Return whether the expression is a leaf or contains sub-expressions.
\item[isConstant() : bool]%
  Return whether the expression is a leaf expression representing a constant.
\item[value() : IntWord]%
  Return the numberic value of a constant expression. Undefined.
\item[isLabel() : bool]%
  Return whether the expression is a leaf expression representing a
  reference to a label.
\item[name() : IntWord]
  Return the name of the a label expression. Undefined.
\item[local() : bool]%
  Return whether a label expression refers to a label that is located in the
  innermost scope.\footnote{
  %
    This flag is necessary to support the global-scope operator of C++.  For
    class and namespace scopes, instead, no special support is necessary,
    since the compilers resolve name clashes by mangling the name strings
    with the symbol's scope.}
  %
  Undefined.
\end{description}

\subsection{Constant Symbolic Expression}
\label{ssec:const-expr-if}

Class \emph{ConstantExpr} is a possible leaf of a complex expression
(described in Section~\ref{ssec:complex-expr-if}). It represents a numeric
integer constant.

\emph{ConstantExpr} redefines the following methods:
\begin{description}
\item[isLeaf() : bool]%
  Return always true.
\item[isConstant() : bool]%
  Return always true.
\item[value() : IntWord]
  Return the value of the constant.
\end{description}

In well-formed memory address expressions, a constant is always part (term)
of a larger symbolic expression, and represents either a coefficient or an
offset.

\subsection{Label Symbolic Expression}
\label{ssec:label-expr-if}

Class \emph{LabelExpr} is a possible leaf of a complex expression (described
in Section~\ref{ssec:complex-expr-if}) and represents a reference to a label
(described in Section~\ref{ssec:labels-if} and following).

\emph{LabelExpr} redefines the following methods:
\begin{description}
\item[isLeaf() : bool]%
  Return always true.
\item[isLabel() : bool]%
  Return always true.
\item[name() : IntWord]
  Return the name of the label.
\item[local() : bool]%
  Return true if the symbol is located in the innermost scope, false if any
  symbol in the innermost scope that matches the symbol name must be
  ignored.
\end{description}

A \emph{LabelExpr} refers to a label object solely by name. To identify the
label object, it is necessary to consider the scope of the object that
contains the symbolic expression, and to apply a scoping rule. The default
scoping rule is to give precedence to a label with matching name in the
innermost scope starting from the scope of the object that contains the
symbolic expression. When the label reference is marked as nonlocal, then
the scoping rule is to skip local and procedure scopes and search starting
from the parent unit scope.

\subsection{Complex Symbolic (Sub-)Expression}
\label{ssec:complex-expr-if}

Class \emph{ComplexExpr} represents a symbolic expression consisting of two
sub-expressions combined by an operator. Each sub-expression, in turn, can
be a leaf of the expression tree (a constant or a label, see
Sections~\ref{ssec:const-expr-if} and~\ref{ssec:label-expr-if}) or,
recursively, another complex expression. This makes it possible to define
arbitrary complex expressions in which leaf expressions are combined by a
finite set of operators.

The operators supported by a complex expression are:
\begin{enumerate}
\item %
  Addition (operator EXPR::ADD).
\item %
  Subtraction (operator EXPR::SUB).
\item %
  Product (operator EXPR::MUL).
\end{enumerate}

\emph{ComplexExpr} redefines the following methods:
\begin{description}
\item[isLeaf() : bool]%
  Return always false.
\item[leftTerm() : SymbolicExpr\&]%
  Return the left sub-expression.
\item[rightTerm() : SymbolicExpr\&]%
  Return the right sub-expression.
\item[operator() : SymbolicExprOperator]%
  Return the operator of this expression.
\end{description}

The purpose of \emph{ComplexExpr} is to support the type of symbolic memory
address expressions that are amenable to typical disambiguation analysis.
For this reason, and to make symbolic analysis by POM clients easier,
expressions should satisfy certain constraints with respect to the position
of constant terms (a symbolic expression of class \emph{ConstExpr}).
%
The left term of addition and subtraction operators must not be a constant.
Even the right term of a subtraction subexpression should not be a constant:
the minus sign can be ``moved into'' the constant, and the subtraction
operator can be replaced by an addition. Finally, the product operator is
allowed only if the left term is a constant.

\subsection{Scopes}
\label{ssec:scope-if}

The \emph{Scope} classes represent parts of the program where a given set of
symbols (label names, essentially) is visible. There 4 scope levels, from
the outermost to the innermost:
\begin{enumerate}
\item %
  Global scope: represents the scope of all symbols that have external
  linkage.  Only symbols in this scope can be used across different
  compilation units.
\item %
  Unit scope: represents the scope of a single compilation unit (usually, a
  source file), and usually corresponds to an object file. Only procedures
  defined within the same compilation units can access these symbols.
\item %
  Procedure scope: represents the scope of a procedure.  Symbols in this
  scopes are local to the whole procedure and can be used throught its
  extension.
\item %
  Local scope: represents the scope of a local block within a procedure.
  Symbols in this scope (for example, symbols defined within a loop
  construct) are only visible within the enclosing block.
\end{enumerate}

Local scopes can be nested, all other scopes cannot.\footnote{
%
  Nesting of procedure scope is allowed in some high level languages, but is
  currently not supported by the program model.}

Scopes can exist as independent objects or as parts of the program object
tree (depicted in Figure~\ref{fig:overview}).

All scope classes share the following basic interface:
\begin{description}

\item[isGlobal() : bool] %
  Return true if this is a global scope.

\item[isUnit() : bool] %
  Return true if this is a global scope restricted to a compilation unit.

\item[isProcedure() : bool] %
  Return true if this the local scope of an entire procedure.

\item[isLocal() : bool] %
  Return true if this is the scope of a local block inside a procedure.

\item[containsCodeLabel(name : const string\&) : bool]%
  Return true if this scope contains a code label with given name.

\item[containsDataLabel(name : const string\&) : bool]%
  Return true if this scope contains a data label with given name.

\item[codeLabel(name : const string\&) : CodeLabel\&]%
  Lookup and return the code label with given name.

  Throw \emph{KeyNotFound} exception if no code label with given
  name exists in this scope.

\item[dataLabel(name : const string\&) : DataLabel\&]%
  Lookup and return the data label with given name.

  Throw  \emph{KeyNotFound} exception if no code label with given
  name exists in this scope.

\item[codeLabelCount(address : Address) : int]
  Return the number of code labels in the given address visible in
  this scope.

\item[codeLabel(address : Address, index : int) : CodeLabel\&]
  Return a code label visible in this scope in the given address and index.

  Throw \emph{OutOfRange} if the index is greater than or equal to the
  count of code labels in the given address.

  Throw \emph{KeyNotFound} if there are no code labels in the given
  address.

\item[dataLabelCount(address : Address) : int]
  Return the number of data labels in the given address visible in
  this scope.

\item[dataLabel(address : Address, index : int) : DataLabel\&]
  Return a data label visible in this scope in the given address and index.

  Throw \emph{OutOfRange} if the index is greater than or equal to the
  count of data labels in the given address.

  Throw \emph{KeyNotFound} if there are no data labels in the given
  address.

\item[addCodeLabel(codeLabel : CodeLabel*)]
  Add a code label to this scope.

  Throw \emph{KeyAlreadyExists} if a code label of the same name
  already exists.

\item[addDataLabel(dataLabel : DataLabel*)]
  Add a data label to this scope.

  Throw \emph{KeyAlreadyExists} if a data label of the same name
  already exists.

\item[parent() : Scope\&] %
  Return the parent scope, that is, the scope that contains this scope.

  Throw exception \emph{WrongSubclass}
  %
  \note{DISCUSS: exception type appropriate?}
  %
  if this scope has no parent scope (it is the global scope).

\item[child(i : int) : Scope\&] %
  Return the child scope with index \emph{i}.

  Throw \emph{OutOfRange} if the index exceeds the number of scopes
  contained in this scope.

\item[childCount() : int] %
  Return the number of contained scopes.  Only scopes directly contained in
  this scope are counted, not inner scopes inside one of the child scopes.

\item[dataDefinition(label : const DataLabel\&, offset : IntWord) : DataDef\&]%
  Lookup and return the data definition for the location defined by given
  label and offset. Return a special \emph{null data definition} if no
  symbolic data exists for the initialisation data at the specified
  location.

  Throw \emph{IllegalRegistration} exception if given label does not belong
  to this scope.

\end{description}

Methods \verb|parent()| and \verb|child()| are not public.  These are basic
building blocks for scope subclasses (see sections~\ref{ssec:gscope-if},
\ref{ssec:fscope-if}, \ref{ssec:pscope-if}, \ref{ssec:lscope-if}).

The symbols stored in a scope are of two main types:
\begin{enumerate}
\item %
  Symbols that represent variables, \emph{DataLabel}.
\item %
  Code labels, \emph{CodeLabel}.
\end{enumerate}

\paragraph{Clients of Scope classes in POM.}
When considered as part of a program object hierarchy, different types of
scopes are managed by different parts of the object hierarchy:
\begin{enumerate}
\item %
  Global and unit scopes are managed by \emph{Program} (see
  Section~\ref{ssec:prog-if}) and, in a limited way, by \emph{Procedure}.
\item %
  Procedure and local scopes are managed by \emph{Procedure} (see
  Section~\ref{ssec:proc-if}).
\end{enumerate}

\paragraph{Scopes, name spaces and address spaces.}
Since a label represents a memory address, an address space can be
associated to each label. In this sense, it can be said that ``a label
belongs to an address space''. The address space does not keep labels apart.
Label name does. A scope contains two label name spaces: one for data labels
and one for code labels. This means that, for example, each data label in a
scope is identified by a unique name, but a code label and a data label in
the same scope can have the same name.

Code labels all belong to the same address space, which is the address space
of the instruction memory. (In some target memory systems, this address
space can be shared with data.) Data labels, even if in the same scope, can
have different address spaces.

\subsection{GlobalScope}
\label{ssec:gscope-if}

The \emph{GlobalScope} class represents the scope of all symbols that have
external linkage.
%
\note{PENDING: code label restrictions, \ref{ch:pending:clab-restrict}}

\begin{description}
\item[GlobalScope()]%
  Constructor.  Create an empty global scope.  Called also automatically
  inside a \emph{Program} instance at construction time, to create its
  global scope.

\item[globalCodeLabelCount(address : Address) : int]
  Return the number of code labels at the given address. Applies to
  all child scopes of this global scope.

\item[globalCodeLabel(address : Address, index : int) : CodeLabel\&]
  Return the code label in the given address. Applies to all child
  scopes of this global scope.

  Throw \emph{KeyNotFound} if no code labels in the given address were
  found.

\item[globalCodeLabelCount() : int]
  Return the number of code labels in program. Applies to all child
  scopes of this global scope.

\item[globalCodeLabel(index : int) : CodeLabel\&]
  Return the code label of requested index. Applies to all child
  scopes of this global scope.

  Throw \emph{KeyNotFound} if there are no code labels in the given
  index.

\item[globalDataLabelCount(address : Address) : int]
  Return the number of data labels at the given address. Applies to
  all child scopes of this global scope.

\item[globalDataLabel(address : Address, index : int) : DataLabel\&]
  Return the data label in the given address. Applies to all child
  scopes of this global scope.

  Throw \emph{KeyNotFound} if no data labels in the given address were
  found.

\item[globalDataLabelCount() : int]
  Return the number of data labels in program. Applies to all child
  scopes of this global scope.

\item[globalDataLabel(index : int) : DataLabel\&]
  Return the data label of requested index. Applies to all child
  scopes of this global scope.

  Throw \emph{KeyNotFound} if there are no data labels in the given
  index.

\end{description}

\note{Following methods not yet implemented}

\begin{description}

\item[unitScopeCount() : int]%
  Return the numer of unit scopes contained by this global scope.

\item[unitScope(i : int) : UnitScope\&] %
  Return the unit scope with index \emph{i} contained by this scope.

  Throw \emph{OutOfRange} if the index exceeds the number of scopes
  contained in this scope.

\item[unitScope(name : const string\&) : UnitScope\&] %
  Return the unit scope with given name. The name string may be the full
  name of the scope (prefix and name) or just its name. In the latter case,
  the prefix of the scopes is ignored. If no matching exists when the prefix
  is ignored, and \emph{name} does not include an extension (a trailing
  substring starting with `.'), then a match is searched using the base name
  of the scopes. The base name is obtained by clipping any extension to the
  name.

  Throw \emph{KeyNotFound} exception if no unit scope with given name is
  contained in this scope.
%
  Throw \note{DISCUSS exception type} exception if only the base name is
  given and scopes with the same base name and different prefixes or
  extensions are available.
\end{description}

\subsection{UnitScope}
\label{ssec:fscope-if}

The \emph{UnitScope} class represents the scope of symbols defined by a
compilation unit and visible throughout, but not outside it.
%
\note{PENDING: code label restrictions, \ref{ch:pending:clab-restrict}}

Unit scopes are identified by a pair of name strings. Usually, these strings
identify the compilation unit of the scope. The first string is the ``base''
name of the compilation unit, the second is a ``prefix'' (usually, but not
necessarily, an absolute path) that makes the combination unique.

\begin{description}
\item[UnitScope(parent : const GlobalScope\&)]%
  Constructor. Create an empty, nameless unit scope with parent scope
  \emph{parent}.

\item[UnitScope(parent : const GlobalScope\&, name : string, prefix : string)]%
  Constructor. Create an empty unit scope with given name and prefix. The
  parent scope is \emph{parent}. The prefix can be omitted.

\item[name() : string] %
  Return the name of the scope. Usually, the name of the scope is given by
  the name of the compilation unit (or the resulting object file).

\item[prefix() : string] %
  Return the prefix of the scope name. Usually, the prefix of the scope name
  is the absolute path to the source file that represents the compilation
  unit.

\item[parentScope() : GlobalScope\&] %
  Return the parent (global) scope.

\item[procedureScope(i : int) : ProcedureScope\&] %
  Return the procedure scope with index \emph{i} contained by this scope.

  Throw \emph{OutOfRange} if the index exceeds the number of scopes
  contained in this scope.

\item[procedureScope(name : const string\&) : ProcedureScope\&] %
  Return the procedure scope of the procedure with given name contained in
  this scope.

  Throw \emph{KeyNotFound} if no procedure scope with given name is
  contained in this scope.

\end{description}


\paragraph{Handling multiple compilation units.}

How to support multiple compilation units defining variables or procedures
with identical names? The methods that refer to program symbols by name must
explicitly specify the scope of the symbol as a reference to a
\emph{UnitScope} object.

Only lookup of \emph{UnitScope} objects requires direct, explicit reference
to the name of the unit scope (a file name and possibly a path name), see
Section~\ref{ssec:gscope-if}.

The motivation for this design is that it avoids direct use of file and path
names when selecting a program symbol is not nice. File names and absolute
path are a low-level detail in the context of POM.

\subsection{ProcedureScope}

\label{ssec:pscope-if}
\begin{description}
\item[ProcedureScope(parent : const UnitScope\&, name : string)]%
  Constructor. Create an empty procedure scope with given name.
  The parent scope is \emph{parent}.

\item[name() : string] %
  Return the name of the scope.  Usually, the name of the scope is given by
  the name of the procedure that defines it.

\item[parentScope() : UnitScope\&] %
  Return the parent (unit) scope.

\item[localScope(i : int) : LocalScope\&] %
  Return the local scope with index \emph{i} directly contained in this
  scope.

  Throw \emph{OutOfRange} if the index exceeds the number of scopes
  contained in this scope.
\end{description}

\subsection{LocalScope}
\label{ssec:lscope-if}

\begin{description}
\item[LocalScope(parent : const ProcedureScope\&)] %
  Constructor. Create an empty local scope. The parent scope is the
  procedure scope \emph{parent}.

\item[LocalScope(parent : const LocalScope\&)] %
  Constructor. Create an empty local scope. The parent scope is the local
  scope \emph{parent}.

\item[procedureScope() : ProcedureScope\&] %
  Return the parent procedure scope.  If nested, this method looks up the
  procedure scope from its local parent scopes.

\item[parentLocalScope() : LocalScope\&] %
  If nested, return the parent local scope. Otherwise,
  %
  \note{DISCUSS: which exception? \emph{InstanceNotFound}, seems most
    appropriate, but is derived from \emph{IllegalParameter}, not applicable
    to this case.}
  %
  throw exception [[to define]].

\item[isNested() : bool] %
  Return true if the innermost parent scope is another local scope, false if
  the parent scope is the scope of a procedure.

\item[localScope(i : int) : LocalScope\&] %
  Return the local scope with index \emph{i} directly contained in this
  scope.

  Throw \emph{OutOfRange} if the index exceeds the number of scopes
  contained in this scope.

\item[identifier() : int] %
  Return a number that uniquely identifies this scope among all the scopes
  of the enclosing scope.
\end{description}

\subsection{CodeSnippet}
\label{ssec:code-snippet}

\emph{CodeSnippet} is simple helper class to support efficient insertion of
instructions into a TTA program. A code snippet is an ordered sequence of
adjacent instruction that do not belong to a program (are independent) and
are not assigned an address.

\emph{CodeSnippet} implements the following simple interface:
\begin{description}

\item[CodeSnippet()]%
  Constructor. Create an empty sequence of instructions.

\item[addToBack(ins : const Instruction\&)]%
  Append instruction \emph{ins} to the end of the code snippet.

  Throw \emph{IllegalRegistration} exception if \emph{ins} is owned by
  a procedure.

  Throw \emph{ObjectAlreadyExists} if the given instruction is already
  in the snippet.

\item[removeLastInstruction()]%
  Remove one instruction from the end of the code snippet. Nothing happens
  if the code snippet is empty.

\item[instruction(index : int) : Instruction\&]%
  Return the instruction of the code snippet at position \emph{index}.

  Throw \emph{OutOfRange} exception if \emph{index} is negative or is not
  smaller than the number of instructions in this code snippet.

\item[instructionCount() : int]%
  Return the number of instructions that make up this code snippet.
\end{description}

\subsection{DataMemory}
\label{ssec:data-mem}

Class \emph{DataMemory} models the data of one data address space of a
program. \emph{DataMemory} class contain all the data areas definitions that
are used by the program. If there is an address stored in a data memory, that
information is also stored in this class.

\begin{description}
\item[DataMemory(memory : AddressSpace\&)]%
    Creates \emph{DataMemory} which defines the data of given address space.

\item[addDataDefinition(dataDef : DataDefinition*)]%
    Adds a \emph{DataDefinition} object to the class. \emph{DataDefinition}s
    are automatically sorted to ascending order by start address of a data
    defintion.

\item[dataDefinition(addr : Address) : DataDefinition\&]%
    Returns the \emph{DataDefinition} object, which covers the address in
    parameter. A for example if a \emph{DataDefinition} object defines an area
    starting from address 10 and its size is 30, then the that object is
    returned when data definitions are requested for addresses 10-29.

\item[dataDefinition(index : int) : DataDefinition\&]%
    Returns the \emph{DataDefinition} by an index. \emph{DataDefinition}s are
    sorted by the start address property of a \emph{DataDefinition} object, so
    index of an object might change if \emph{DataDefinition}s are added to
    a memory.

\item[dataDefinitionCount() : int]%
    Retuns the number of \emph{DataDefinitions} stored in the object.

\item[addressSpace() : AddressSpace\&]%
    Returns the address space whose data this class defines.

\end{description}

\subsection{DataDefinition}
\label{ssec:data-def}

Class \emph{DataDefinition} defines one or more MAUs in a data address
space. Stored data can be either initialized or unititialized data. This
class contains all needed interfaces for using different types of
\emph{DataDefinitions}. E.g., \emph{DataAddressDef} or
\emph{DataInstructionAddressDef} can be used through \emph{DataDefintion}
interfaces.

\begin{description}
\item[DataDefinition(start : Address, size : int, initData = NULL :
MinimumAddressableUnit*)]%
    Creates area definition of given size starting from given address. Size
    defines how many MAUs are defined. If initData parameter is given, then
    the area contains initialized data. Otherwise created object defines an
    uninitialized data area.

\item[DataDefinition(start : Address, initData
: vector<MinimumAddressableUnit>\&)]%
    Creates initialized data area starting from given address. Data area size
    is same that the size of the given vector of initialization data.

\item[startAddress() : Address]%
    Returns the address, where from the definition starts defining data.

\item[isInitialized() : bool]%
    Returns if the data definition contains initialization data.

\item[MAU(index : int) : MinimumAddressableUnit]%
    Returns an initialization data value of the index given in parameter.

\item[size() : int]%
    Returns the number of MAUs that are defined by this object.

\item[isAddress() : bool]%
    For this class always false. \emph{See DataAddressDef} and
    \emph{DataInstructionAddressDef} classes.

\item[isInstructionAddress() : bool]%
    For this class always false. See \emph{DataInstructionAddressDef} class. 

\item[destinationAddress(index : int) : DataDefinition\&]%
    For this class always throws an InvalidSubclass exception.

\end{description}

\subsection{DataAddressDef}
\label{ssec:data-addr-def}

Defines an one or multiple MAUs in a data address space, which contains an
address referring data memory. For data area that contain an
address to an instruction see \emph{DataInstructionAddressDef}.

\begin{description}
\item[DataAddressDef(start : Address, size : int, dest : Address)]%
    Creates \emph{DataAddressDef} starting from given address. Size parameter
    defines how many MAUs are used to store the address. Dest parameter
    defines the destination address which is stored in data definition.

\item[startAddress() : Address]%
    Returns the address where in the data memory the definition starts.

\item[isInitialized() : bool]%
    For this class returns always true, since there is always address stored.

\item[MAU(index : int) : MinimumAddressableUnit]%
    Returns one MAU of the initialized area.

\item[size() : int]%
    Returns number of MAUs that are used to store the destination address.

\item[isAddress() : bool]%
    Returns always true, because the class overloads the destinationAddress.

\item[isInstructionAddress() : bool]%
    Returns always false, because the destination address is not in
    the instruction address space.

\item[destinationAddress(index : int) : DataDefinition\&]%
    Returns the destination address of stored data including the destination
    address space.

\end{description}

\subsection{DataInstructionAddressDef}
\label{ssec:data-instr-addr-def}

Defines an one or multiple MAUs in a data address space, which contains an
address referring an instruction.

\begin{description}
\item[DataInstructionAddressDef(start : Address, size : int, dest :
InstructionReference\&*)]%
    Creates an initialized data definition which stores an address of the
    given instruction. Size and start address parameters defines where in
    data memory the address is stored and how many MAUs are used to store
    the value.

\item[startAddress() : Address]%
    Returns the address where in the data memory the definition starts.

\item[isInitialized() : bool]%
    For this class returns always true, since there is always address stored.

\item[MAU(index : int) : MinimumAddressableUnit]%
    Returns one MAU of the initialized area.

\item[size() : int]%
    Returns number of MAUs that are used to store the destination address.

\item[isAddress() : bool]%
    Returns always true, because the class overloads the destinationAddress.

\item[isInstructionAddress() : bool]%
    Returns always true, since the destinationAddress() always refers to
    the instruction address space.

\item[destinationAddress(index : int) : DataDefinition\&]%
    Returns the destination address of stored data including the destination
    address space.

\end{description}

\subsection{InstructionReferenceManager}

Class \emph{InstructionReferenceManager} helps in keeping instruction
addresses referenced in the Program Object Model
up-to-date. Instructions are not referenced directly, but through an
InstructionReference instance. The actual Instruction instance
referred can be changed as needed.

The class provides the following interface:

\begin{description}
\item[createReference(ins: Instruction\&) : InstructionReference\&]
  Create a new reference to the given instruction. If one already
  exists, it will be returned instead.
\item[replace(insA: Instruction\&, insB: Instruction\&)]
  Replace the referred instruction \emph{insA} with the instruction
  \emph{insB}.

  Throw \emph{InstanceNotFound} if the instruction to be replaced is not found.
\end{description}

\subsection{InstructionReference}

Class \emph{InstructionReference} represents a reference to an
Instruction instance. The target of the reference can be changed as
needed.

The class provides the following interface:

\begin{description}
\item[setInstruction(ins: Instruction\&)]
  Set a new referred instruction.

  Throw \emph{IllegalParameters} if the instruction is a null instruction.
\item[instruction() : Instruction\&]
  Return the referred instruction.
\end{description}

\subsection{AnnotatedInstructionElement}

This interface is for classes in POM that can be augmented with
"annotations", which is arbitrary data added to certain elements
of the program. 

This is the API required to support the annotation feature of the TPEF
storage format. Currently two classes of POM inherit this interface,
thus provide annotation services to clients: \emph{Move} and
\emph{TerminalImmediate}. 

The annotations are presented with class \emph{ProgramAnnotation}. 

The details of the API can be seen in the API documentation.

\section{Implementation}

% Describe here important low-level details. Only details about complex
% problems that require ``smart solution'' should be treated.  Do not
% describe every single detail of the implementation when that follows from
% the specification/design in a straightforward manner.

\subsection{Program}
\label{ssec:prog-imp}

\begin{figure}[tb]
  \centerline{\psfig{figure=eps/ProcedureLookup.eps,width=0.55\textwidth}}
  \caption{Efficient procedure lookup by means of procedure table.}
  \label{fig:proc-table}
\end{figure}

\paragraph{Instruction lookup.}
Instruction lookup by \verb|instructionAt()| is implemented by means of a
simple helper structure.  This structure is a simple vector of
address-procedure pairs (see Figure~\ref{fig:proc-table}).  A specialised
binary search algorithm allows to lookup the procedure that \emph{should}
contain the instruction at the requested address. This search algorithm
doesn't look for a perfect match, but for the procedure with the highest
start address that is smaller or equal to the requested address.  The
request is then delegated to the procedure thus found.

\paragraph{Delegations.}

\emph{Program} delegates most of the work at level of single instructions to
the \emph{Procedure} instance that is affected by the operation.  The
program API is merely a convenience shortcut.

\paragraph{Automatic Relocations.}

Insert and remove
%
\note{PENDING: \ref{ch:pending:edit-prog}}
%
operations are very expensive.  They require adjustment of references to
instructions (and, in some cases, symbols) throught the program code and
data.

\paragraph{Inserting and removing procedures.}

When a procedure is removed from a program, it is unregistered from it.
Unregistration is a complex operation:
\begin{enumerate}
\item %
  All instruction following the removed procedures are relocated to ``fill
  the empty area'' left by the removed procedure.  All references to
  relocated instructions, in turn, are adjusted.  This work is not different
  from what described for removing single instructions.

\item %
  All program symbols that are referenced by the unregistered procedure are
  duplicated into a procedure-owned scope hierarchy tree.  Symbol
  duplication involves a rather unusual step: the global and the unit scopes
  (Section~\ref{ssec:scope-if}) are created and filled with a subset of the
  original symbols owned by the program.  Thus, even if an independent
  procedure defines only procedure and local scopes, it actually owns also a
  copy of unit and global scopes.
%
  These duplications ensure that a procedure, once unregistered, is
  completely independent from its original program.
\end{enumerate}

\subsection{Procedure}
\label{ssec:proc-imp}

\paragraph{Relocation workflow.}

\emph{Procedure} is responsible for maintaining the instruction stream.
This means that it must: (1) Relocate instructions as a consequence of
insertion or removal; (2) Adjust addresses of references to relocated
instructions.

However, changes to instructions of a procedure affect the entire program.
A procedure has no knowledge or control over other procedures.  Procedures
are owned by \emph{Program}.  Thus, when a procedure modifies the
instruction it owns, it must notify its program (if any) that other
procedures should update their state according to the modifications.

\paragraph{Instruction lookup.}

Instruction lookup is implemented with a constant-time algorithm using a
helper data structure.

\begin{figure}[tb]
  \centerline{\psfig{figure=eps/InstructionLookup.eps,width=0.55\textwidth}}
  \caption{Constant-time instruction lookup by means of instruction table. The table shown in the example applies to instruction words with  variable-length encoding: 5 or 10 MAU's.}
  \label{fig:instr-table}
\end{figure}

The helper data structure is a table of instruction references (see
Figure~\ref{fig:instr-table}).
%
Unlike for the procedure table used by \emph{Program}, the instruction is
not looked up by a binary search algorithm.  Instead, the requested address
is converted into a table index and the \emph{Instruction} is retrieved by
direct access.

The Address is converted to an index as follows:
\begin{equation}
  \label{eq:addr-to-index}
  \textrm{index} = (\textrm{address} - \textrm{startAddr}) / m.
\end{equation}

Where \emph{m} the alignment of legal instruction addresses, expressed in
MAU's.  In practice, \emph{m} is the minimal number of MAU's taken by a TTA
instruction.

\emph{Example.} Given an instruction memory addressed by 8-bit words (bytes)
and fixed-size instruction words 134-bit long (17 bytes, aligned to 18), the
value of \emph{m} is 18.

This lookup mechanism supports even variable-length TTA instruction
%
\note{FUTURE EXTENSION}
%
encodings, at the price of reserving ``illegal'' table slots for all
potentially legal instruction addresses that fall in the middle of a long
instruction.  This price may be significant, when the maximum common
denominator of the allowed instruction widths (in MAU's) is small.

\emph{Example.} Given the instruction memory of previous example, suppose
that now instructions are variably-encoded, and that three bit width are
possible for instruction words: 40, 80 and 140 bits, that is, 5, 10, and 18
bytes.  The mcd of 5, 10, 18 is 1, so \emph{m} is 18 times smaller than in
the case of a fixed-size instruction 18 bytes long, and the lookup table is
18 times bigger.

In practice, cases of small mcd between different instruction word widths
are unlikely, since variable-length encodings tend to be multiple of a large
word.

\paragraph{Address adjustment.}

Address adjustment is the process of updating addresses that refer to
relocated locations (instructions or data words).

Address adjustment is split in two parts:
\begin{enumerate}
\item Adjustment of references in instructions.
\item Adjustment of addresses stored in data words.
\end{enumerate}

\emph{Procedure} is responsible for adjusting the addresses contained in
instructions.  For adjustment of addresses stored in data words, see
Section~\ref{ssec:prog-imp}.

An address stored in an instruction may point to another instruction or to a
data addresses.

\subsection{Instruction}
\label{ssec:instr-imp}

\emph{Program} and \emph{Procedure} appear to have the same control over the
instruction stream that forms the program.
%
At level of implementation however, instructions are owned and directly
controlled by procedures.  This requires some coordination between
\emph{Program} and \emph{Procedure}, as described in
Section~\ref{ssec:proc-imp}.


\paragraph{Discussion.}

Compared with an implementation whereby all instructions are directly
managed by \emph{Program}, leaving ownership instructions to single
procedures has several advantages: (1) more efficient object hierarchy (less
expensive instruction lookups, for example); (2) possibility to define two
levels of program update: one coarse-grained (entire procedure unaffected or
relocated), one fine-grained (instruction stream locally modified); (3)
simplified procedure cloning; (4) straightforward support for unregistered
procedures; (5) supports ``empty areas'' between procedures gracefully.

The only evident disadvantage of procedure ownership is that it makes
implementation of the ``flat instruction list'' interface that a program
provides to clients more expensive.

\subsection{Move}
\label{ssec:move-imp}

\paragraph{Managing instruction references.}

A crucial problem is managing cross references between moves and
instructions.  When a move is removed from the containing program (for
example, when an instruction is unregistered from its procedure), it is not
anymore under the responsibility of it.  It is important that no reference
between unregistered instruction and other object exist, since there is no
efficient way to maintain these references up to date.

To avoid bookkeeping activity whenever either the move containing the
reference or the referenced instruction are removed from the program, all
references are maintained in symbolic form.

Moves and instructions contain indirect references to other instructions or
data object, called address expressions (see \emph{AddresssExpression},
section~\ref{ssec:symexpr-if}).

When a move that contains a reference to another instruction (target of a
control flow transfer) is removed from its procedure, no reference
bookkeeping is necessary. The address expression that represents the target
instruction contains only a symbolic reference to a code label (and its
scope) by means of a name string and, optionally, an offset.

When the move that contains a reference to a target instruction is
registered, the \emph{Move} class allows to automatically lookup the
instruction object from the symbolic reference.  Moves contain a pointer to
the parent procedure, which in turn contains pointers to the scopes.  Thus,
a move can search the parent scopes for the code label, and the parent
program for the target instruction corresponding to such label.

This design ensures that, if the move is registered back into the procedure,
the instruction reference becomes valid again without any need for
adjustment.

\paragraph{Support for segmented busses.}

The moves internally contain references to the first and the last bus
segment used by the move. This makes it possible to later add full support
for segmented busses without having to change the bulk of the
implementation. The transport bus of a move is either stored in a separate
data member (with invariant that the bus is always the parent of the source
and destination segment) or simply retrieved from the segments. In the
latter case, the API of MOM is used.

\subsection{Labels}
\label{ssec:label-imp}

Label objects do not contain their address. The address of a label (and its
address space) is stored in the owning \emph{Scope}. Labels are thus little
more than strings. See Section~\ref{ssec:scope-imp} for details.

\subsection{Code Labels}
\label{ssec:clabel-imp}

\paragraph{Managing multiple code labels.}
Source program may contain several compiler-generated code labels that
represents the same instruction address. POM handles multiple equivalent
code labels as all valid alternatives for referring to the same instruction
address. Clients can refer to an instruction by one of its labels, it does
not matter which. When an instruction does not have a label, explicit
construction is required.

The main drawback of supporting multiple labels is that looking up all
equivalent code labels is complicated. POM does not provide methods to
lookup every code label pointing at a given instruction. Clients need to
scan through scopes and search for labels. However, such functionality is
needed very rarely, probably only by debugging tools.

\subsection{Symbolic Expressions}
\label{ssec:sym-expr-imp}

For program models built out of TPEF files, the data for symbolic
expressions comes from symbol tables and relocation tables. Symbolic memory
addresses thus created are simple label references (base addresses) attached
to immediate source fields of moves (inline immediates) or \emph{Immediate}
objects (long immediates).

After dataflow analysis, symbolic expressions can be propagated, as shown in
the example in Section~\ref{ssec:symexpr-if}. As a result, the final
symbolic expressions that represent address of memory operations are
obtained, and can be used for memory analysis.

Propagating symbolic expressions implies some knowledge about operations
`ADD', `SUB' and `MUL', and precise assumptions about their behaviour (2's
complement integer arithmetics with bit width sufficient for expressing any
legal address of the target memory being accessed).

The amount of symbolic information stored in POM depends on the needs of the
client application. Clients that do not make use of symbolic expressions
just maintain the simple label references created during POM construction.
Maintenance requires little action, since the API of \emph{Move} and
\emph{Immediate} automatically copies symbolic expressions. Clients that
require extensive symbolic information, on the contrary, must propagate and
then maintain symbolic expressions for intermediate subexpressions that
occur during computation of a memory address.
%
\note{PENDING: TPEF support for complex symbolic expressions,
  \ref{ch:pending:tpef-sym-expr}}

\subsection{Scopes}
\label{ssec:scope-imp}

Scopes are basically specialised containers of labels and data definitions.

\paragraph{Label lookup.}
Labels are stored in containers indexed by the address of the label. This
allows efficient relocation of labels.

\note{if fast lookup of labels by name is also required, a helper map with
  string names as keys and label references as ``payload'' could be used}

A scope can contain data labels for locations in different address
spaces. Since it is redundant and inefficient to store the address
space reference for every label address in the same address space, the label
containers are organised in a hierarchy and indexed by their address space.
%
\begin{figure}[tb]
  \centerline{\psfig{figure=eps/LabelContainers.eps,width=0.60\textwidth}}
  \caption{Hierarchy of label containers in a scope.}
  \label{fig:label-containers}
\end{figure}
%
Figure~\ref{fig:label-containers} illustrates the hierarchy of containers
that stores and maintains the labels of a scope. The address space of the
instruction memory is a special data member of the scope class.



\section{Error Handling}

% How are errors handled? What is considered internal and what is considered
% external source of errors? What is handled by throwing exceptions and what
% by assertions? What are the


\chapter{PROGRAM INPUT OUTPUT MODULE}


\section{Overview}

The program object model is not read directly from the external TPEF file.
Instead, it is constructed out of the TCE object model of a TPEF file
created by the Binary Handling Module~\cite{TPEF-design}.  This is the task
of the program builder.  Similarly, creating the binary file from the
program is not a responsibility of the program model.  A client domain
module \emph{ProgramWriter} scans the program objects and translates them
into a \emph{Binary} object hierarchy.

The input/output module is the only module of the domain library that
depends from (and includes the headers of) both TPEF Binary Handling Module
and Program Object Model.

\section{Interfaces}

\subsection{Program Input Module}

The TPEF program builder \emph{TPEFProgramFactory} is the domain class
responsible for translating a TPEF binary object model into a TTA program
object model.

The program builder is a client of TPEF binary module and scans only the
following sections:
\begin{enumerate}
\item %
  Code, Data and other program sections.
\item %
  Symbol sections and relocation sections (for symbolic information).
\item %
  Address space section.
\item %
  Machine resource section.
\end{enumerate}

All other TPEF sections are ignored.
\note{PENDING: \ref{ch:pending:aux-sections}}

In addition, the program builder depends also on the Machine Object Model
(MOM) of the target processor of the application being scanned.

Just like the program builder is not responsible for loading a TPEF file, it
is not responsible for loading the target processor object model.  The
client of program builder feds complete \emph{Machine} and \emph{Binary}
(the root classes of MOM and Binary Handling Module, respectively) to the
program builder.

The \emph{Program} instance returned by the builder is owned by the client
that requested construction and must be deallocated by it when out of use.

The methods implemented by the program builder are:
\begin{description}
\item[build(machine : const Machine\&, binary : const Binary\&) : Program*]%
  Construct and return the program object model for the TTA program
  contained in given binary. The program is targeted for the TTA processor
  defined by \emph{machine}.
\item[build(binary : const Binary\&) : Program*]%
  Construct and return the program object model for the TTA program
  contained in given binary. The program is unscheduled. The target
  architecture is an implied ``universal machine''.
\end{description}

\subsubsection{Limitations of Input TPEF Files}

A program is uniquely defined by an address space and a nonoverlapping
address range.  A program can be spread over several TPEF code sections.

While TPEF does not force special restriction on the number of program
sections, their address ranges and their address space, the TPEF program
builder can only handle TPEF files that satisfy the following restrictions:
\begin{enumerate}
\item %
  Code sections must all belong to the same address space.
\item %
  The address range of any code section must not overlap with the address
  range of another code section (in fact, with any other program section).
\end{enumerate}

\subsection{Program Output Module}

Output of a TTA program into external TPEF file is implemented by
\emph{ProgramWriter}, a client module of POM and the Binary Handling Module.


The sole method implemented by the program output module is:
\begin{description}
\item[Binary* build(prog : const Program\&)]
\end{description}

\chapter{REJECTED ALTERNATIVES}

\begin{description}
\item[14.08.2004 --- Impossible Program Relocation] %
  When relocation of parallel TTA code in a program fails, the possibility
  to return a failure Boolean flag true was rejected.  Returning a flag is
  dangerous, because clients tend to ignore status flags when the successful
  outcome is by far the most common.  A client may be designed at a time
  when no relocation failure is possible at all.

  The alternative, throwing an exception is rather safe, because a client
  that does not make provision for relocation failures will safely abort.
  On the other hand, relocation failure is not, conceptually speaking, an
  exceptional condition.

\item[14.08.2004 --- Managing Stray Labels] %
  When a procedure is removed from a program, it is not safe to remove any
  label unless a check is performed to verify that no other procedure refers
  to the labels in question.  The choice of having the Program Object Model
  actively maintaining the structures ``clean'' by removing the unused
  labels is rejected because too expensive and even lossy.  POM is not
  responsible to decide which labels should be removed, even if they are
  unused.

\item[06.12.2004 --- Management of Multiple Code Labels] %
  A source program may contain several compiler-generated code labels that
  refer to the same instruction address. Two possible strategies for
  managing multiple labels for the same instruction have been rejected:
  \begin{enumerate}
  \item %
    Program builder normalises label information by destroying all but one
    label referring to the same instruction (and redirecting all references
    to other labels to the ``survivor''. Advantage of this approach is that
    it is guaranteed that there is only one label for each instruction.
    Main disadvantage of this approach is that POM becomes ``lossy'', since
    some source data may be lost forever.
  \item%
    POM handles multiple labels only internally, as debugging information
    for users. Labels are completely banned from POM API: all methods that
    refer to an instruction require an \emph{Instruction} reference, not a
    code label. Advantage of this approach is that is does not require
    explicit or automatic creation of labels. POM could even prohibit
    creation of labels (probably not a good idea). The main disadvantage of
    this approach is that lookup of any code label becomes awkward.
  \end{enumerate}

\item[07.12.2004 --- Automatic code label generation] %
  Force every move that refers to a target instruction (contains an indirect
  symbolic reference), to refer to a code label without offset.  If the
  original program does not contain a label for a target instruction, it is
  necessary to create it.

  Rejected for two reasons: (1) the address expressions of moves provide
  offsets, so support is present and creation of new labels is not strictly
  necessary; (2) automatic creation of labels is expensive because it
  requires to allocate and manage more small objects.
\end{description}



\chapter{IDEAS FOR FURTHER DEVELOPMENT}

These ideas are not part of the design yet, but might be added in the
future are listed here.

\begin{description}

\item[09.06.2004] --- Program builder support for multiple programs

  \emph{TPEFProgramFactory} could be extended so as to support multiple
  programs stored in a single TPEF file.  Currently, it seems unlikely that
  such feature is really needed.

  A possibilty could be to provide a method to build a program given a TPEF
  \emph{CodeSection} instead of an entire \emph{Binary}.

\item[14.08.2004] --- Interface for control-flow information

  Even if POM does not organise the program abstraction at level of basic
  blocks, some information about control flow (and thus basic blocks) is
  readily available at the time the input data source is read.  This
  information could be preserved and made available to clients by means of
  two methods:
  \begin{description}
  \item[bool isLeader(const Instruction\&)]%
    Return true if the given instruction is the entry point of a basic
    block, false if it is inside a basic block.
  \item[const Instruction\& leader(const Instruction\&)]%
    Return the entry point of the basic block (leader) that contains given
    instruction.  If the given instruction is the leader, return it.
  \end{description}

\item[25.11.2004] --- Support for nested procedure scopes

  Certain high-level languages allow for procedure definitions to be nested.
  Since each procedure defines its own scope, support for these languages
  requires that \emph{ProcedureScope} can be nested.

\item[23.12.2004] --- Instruction interface for segmented busses

  A possible interface for supporting segmented busses is:
  \begin{description}
  \item[move(mb : Bus, index : int) : Move\&]%
    Retrieve the move number \emph{index} in the list of moves programmed
    onto the segments of bus \emph{mb}.
  \end{description}

\item[28.12.2004] --- Richer \emph{CodeSnippet} interface

  The interface of \emph{CodeSnippet} is currently very simple. It could be
  extended to let clients add and remove instructions in any position of the
  sequence.

\item[14.10.2005] --- Move guards treated as terminals

  Extend the \emph{MoveGuard} class so that the term of the guard expression
  can be treated as a register or (output) FU port terminal. There are many
  details in this design idea that can vary. See
  Section~\ref{ch:pending:move-guards} for possible variations.
\end{description}

\chapter{PENDING ISSUES}

\section{Maintaining TPEF Sections not Directly Needed for Program}
\label{ch:pending:aux-sections}

Certain sections of TPEF files (for example, profile sections, debugging
data sections, ADF section) are not necessary for Program Object Model.
However, some of these sections may be affected by transformations that
change the program.

This means that there must be some coordinating entity that keeps track of
TPEF auxiliary sections not directly used for building the program object
model (and thus, not recreated by the program output module).  Such
coordinator must either be able to update the information of such auxiliary
sections or mark them as obsolete and prevent their storage on TPEF external
file.

Who is in charge for developing the section maintainers? The same persons
who design the concrete specialisation of TPEF \emph{Section}.

The main stumbling block for the design of a section maintainer framework is
to define how information about the program model changes is conveyed to the
maintainer. Information should make it possible to capture the following
cases (in parenthesis, the cases that affect the control flow of the
program):
\begin{enumerate}
\item%
  Insertion of instructions (insertion of control flow transfers).
\item%
  Modification of instructions (modification of target addresses).
\item%
  Deletion of instructions (deletion of control flow transfers).
\end{enumerate}
Relocation of a piece of code could be treated as a deletion/insertion
pair.

\paragraph{Profile Section.}
Profile section is an example of how much profitable and important section
maintenance can be. Maintenance of a section profile as the program control
flow is modified (due to code transformations or scheduling) is orders or
magnitude faster than redoing a long simulation.

\section{Efficiency of Inserting and Removing Instructions}
\label{ch:pending:edit-prog}

As currently described, operations that modify the instruction stream
(insert or remove instructions) are very expensive.  All TTA instructions of
the entire program are potentially affected.

Instructions (and, if sharing the same address space, data words) that
follow the point of the modification have their addresses relocated.  In
turn, all references to such relocated addresses (both in program code and
program data) need adjustment.

If no elegant solution to this problem exists, then clients should be aware
of efficiency and work consequently.  For example, a client that is building
a new procedure should keep it unregistered from the destination program
until the code modifications are completed.  This would minimise the work
needed to update references throughout the program.

\section{Handling of Addressing Modes}
\label{ch:pending:addr-modes}

Except for the trivial case of absolute addressing, the value of a constant
operand (immediate) may bear no direct relation with the address of the
target instruction.  For example, in the following piece of code containing
a PC-relative jump
\begin{verbatim}
0x100   add.r -> r12;
0x101   b1: -1 -> jumpr.t;  // if b1 then jump to PC - 1
\end{verbatim}
the immediate value $-1$ is not an address, and has only an indirect
relation with the absolute address 0x100 of the target instruction.

A simple and unflexible solution to this problem is to make the Program
Object Model addressing-conscious, as far as that is required for computing
effective addresses.  This means that, if a user devises a new control
transfer operation, he must also update POM implementation, even if the new
operation uses one of the existing addressing modes, simply because POM does
not ``know'' about the new operation.  This solution destroys must of the
flexibility offered by OSAL, and is unacceptable.

Better solutions require to encapsulate addressing modes in special classes.
The Program Object Model has no built-in knowledge about the addressing
modes, all knowledge is stored in addressing mode classes and is accessible
through the class API.  For example, given an object that represents
PC-relative addressing mode,
\begin{verbatim}
  PCRel.effectiveAddress(-1, 0x101)
\end{verbatim}
gives (obviously) the address 0x100.

The problems of this solutions are:
\begin{enumerate}
\item %
  Who creates the addressing mode classes? Only single operations that have
  operands that are addresses can know which specific addressing mode should
  be instantiated. This creates and additional burden for the OSAL operation
  designer.
\item %
  The API of addressing mode classes puts a clear limitation for what new
  addressing modes can actually be defined.  For example, the API must
  certainly require the PC value in order to correctly compute the effective
  address of PC-relative jumps.  What if an addressing mode uses a given
  register as base address?
\end{enumerate}

Notice that this problem is actually serious only for addressing modes that
define static effective addresses, such as, for example, PC or page relative
addressing modes.  Modes that use, for example, the value of a register as a
base address
\begin{verbatim}
  0x500 -> setbase.t; // set base address of a special addressing mode
  ...
  72 -> ldbaserel.t;  // load word at address (base)+72
\end{verbatim}
are generally known only at run time, so the services of an
addressing mode class are not required.

A brutal solution of all these problems is to clearly restrict addressing to
a predefined set including only absolute, PC and page relative modes.

\section{Dependency on MOM API}
\label{ch:pending:MOM-api-dep}

POM should not depend on the API of MOM, but only contains references to MOM
objects. For this POM methods as shortcuts of MOM methods must be ruled out.
However, certain POM services may require, to be implemented, to internally
use MOM objects. For example, \emph{Program} probably uses
\emph{AddressSpace} to perform certain tests on the range of program's
addresses. In that case, the dependency on MOM must be accepted.

\section{Restrictions on the Scope of Code Label}
\label{ch:pending:clab-restrict}

Only a special type of \emph{CodeLabel}, the \emph{ProcedureLabel}, can be
stored in a unit scope. Is this a reasonable restriction? It stems from
typical use of jump target labels in high level languages, but nothing
fundamental restricts a jump operation of a procedure from jumping into
another procedure.

\section{TPEF Support for Complex Symbolic Expressions}
\label{ch:pending:tpef-sym-expr}

TPEF supports simple symbolic expressions (in form of references to symbol
table entries that label base addresses). However, computing intermediate
and final symbolic address expressions for a memory access requires a
nontrivial amount of work (dataflow analysis, expression propagation) and
\emph{ad-hoc} knowledge (operations `ADD', `SUB', `MUL').
%
For this reason, it is profitable to provide TPEF support for storing and
retrieving symbolic information about of intermediate sub-expressions and
final expressions attached to memory addresses.

TPEF annotations to elements of the code section are flexible and powerful
enough to express symbolic expressions of arbitrary complexity. For example,
symbolic expression: `$4\cdot S + 32$' could be stored as a stack of terms
and operators in a 5-element annotation. The stack would consist of the
following elements:

\begin{tabular}[h]{cll}
position & element  & description\\
\hline
 1     & 4        & integer constant, stored ``as is''\\
 2     & \emph{S} & label `S', stored as reference to symbol table entry\\
 3     & MUL      & multiplication operator, stored as reserved code\\
 4     & 32       & integer constant, stored ``as is''\\
 5     & ADD      & sum operator, stored as reserved code\\
\end{tabular}

\section{Move Guards as Terminals}
\label{ch:pending:move-guards}

\emph{MoveGuard} design as an empty wrapper to \emph{Guard} provides only an
objectified guard reference. However, move guards also represent a data use,
just like the move source does. For this reason, it may be convenient for
clients to be able to access guard expression terms as terminals.

\paragraph{Example and motivation.}

Let's assume we have a dataflow graph with methods for adding register use
and definition points to itself, as new nodes. The methods take a
\emph{Terminal} as input parameter. The following code adds any register use
and definition points of a move to the dataflow graph `dflowGraph':
%
\begin{verbatim}
    if (move.source().isRegister()) {
        dflowGraph.addUseNode(move.source());
    }
    if (move.destination().isRegister()) {
        dflowGraph.addDefNode(move.destination());
    }
    Guard& g = move.guard().guard(); // get Guard& from MoveGuard&
    RegisterGuard& rg = dynamic_cast<RegisterGuard*>(&g);
    if (rg != NULL) {
        RegisterFile& gRf = rg.registerFile();
        Port* gRfPort = NULL;
        // find a valid output port
        while (int n = 0; n < gRf.portCount()) {
            if (gRf.port(n).outputSocket() != NULL) {
                gRfPort = gRf.port(n);
            }
        }
        assert(gRfPort != NULL);
        // create a pseudo-terminal
        RegisterTerminal* gTerm;
        gTerm = new RegisterTerminal(gRfPort, gRf.registerIndex());
        dflowGraph.addDefNode(*gTerm);
        delete gTerm; // works assuming dflow graph makes local copies
     }
\end{verbatim}

There are, at least sometimes, valid alternatives to this solution that do
not require changes to move guard interface. For example, if the dataflow
graph can be extended, it could provide another method for adding use
points. The client code is then somewhat simplified. The part that processes
the move guard becomes:
%
\begin{verbatim}
    Guard& g = move.guard().guard(); // get Guard& from MoveGuard&
    RegisterGuard& rg = dynamic_cast<RegisterGuard*>(&g);
    if (rg != NULL) {
        RegisterFile& gRf = rg.registerFile();
        int gRegIndex = rg.registerIndex();
        dflowGraph.addUseNode(gRf, gRegIndex);
    }
\end{verbatim}

In any case, a downcast is necessary and client code is arguably awkward and
less intuitive. Compare this with one client code for the equivalent task
given one of the alternative designs described below:
%
\begin{verbatim}
    if (move.source().isRegister()) {
        dflowGraph.addUseNode(move.source());
    }
    if (move.destination().isRegister()) {
        dflowGraph.addDefNode(move.destination());
    }
    if (move.guard().term().isRegister()) {
        dflowGraph.addUseNode(move.guard().term());
    }
\end{verbatim}

\paragraph{Pure terminal vs. additional member.}

Terminal representation for move guard terms can be implemented in two ways:
\begin{enumerate}
\item %
  The \emph{MoveGuard} is replaced by new types of terminals.
\item %
  Existing \emph{MoveGuard} is augmented with a terminal member. This member
  provides access to all terminal-like properties of a guard expression
  (term): register index or port, register file or function unit. The
  invert bit property is either wrapped in \emph{MoveGuard} (forwarded from
  \emph{Guard}) or simply left out.
\end{enumerate}

The main drawback of the first approach is that we heavily overload the
terminal concept. We have to provide invert bit access and modification,
which do not belong to the concept of terminal.

\paragraph{New guard terminal types vs. reuse of existing terminals.}

Using \emph{Terminal} as representation of guard expression terms in a
program's moves leaves another open problem. Unlike in usual register and FU
port terminals, registers and FU ports of a guard term are not read through
a port and a socket of the programmable transport network. Thus, the port
member of their terminal subclasses is not applicable. There are two main
solutions to this problem:
\begin{enumerate}
\item %
  Reuse and extend the meaning of \emph{TerminalRegister} and
  \emph{TerminalFUPort} classes. When their port is undefined (null object),
  it means that the port is not relevant (unassigned terminal in sequential
  code) or not applicable (guard term).
\item %
  Define two further specialisations of \emph{TerminalRegister} and
  \emph{TerminalFUPort} classes, respectively: \emph{TerminalGuardRegister}
  and \emph{TerminalGuardFUPort}. These can be used as their immediate
  parent classes, but override the port access members.
\end{enumerate}

The main drawback of the first solution is that it muddles the meaning of
register and FU port terminals, and that it requires a new constructor for
the port-less case. Another drawback is that returning a null object can
create inconsistent program state and unexpected errors if client forgets to
test for non-null condition and the port is null. This can lead to
hard-to-track errors, occurring in parts of code seemingly unrelated with
the place where the port of a terminal was accessed.

The main drawback of the second solution is that it requires an
\emph{isGuardTerm} method or it requires a downcast when a client must test
whether a given terminal is a guard term or not.



\chapter{MAINTENANCE}

% This chapter treats the problem of extending the design with new
% capabilities that fit in a predefined framework.

% Remove this chapter no obvious and standardised way to extend the design
% with new capabilities exists.


% ------------------------------------------------------------------------

%  References are generated with BibTeX from a bibtex file.
\bibliographystyle{alpha}
\cleardoublepage
% Equivalent to a chapter in the table of contents
\addcontentsline{toc}{chapter}{BIBLIOGRAPHY}
\bibliography{Bibliography}



\end{document}

%%% Local Variables:
%%% mode: latex
%%% mode: auto-fill
%%% TeX-master: t
%%% End:
