\NeedsTeXFormat{LaTeX2e}
\documentclass[a4paper,twoside]{tce}
\usepackage{pslatex}

\begin{document}
\author{Mikael Lepist�}
\title{TCE v1 Frontend Compiler}
\ver{0.3.3}
\firstday{27.09.2004}
\lastday{17.01.2005}
% id number in S- sequence
\docnum{\#\#\#}
% draft/complete/committed
\state{draft}

\maketitle

\chapter*{Document History}

\begin{HistoryTable}

 0.1    & 04.10.2004 &
 M. Lepist�   & Initial version.\\

 0.1.1  & 04.10.2004 & P. J��skel�inen &
 Small corrections found in review.\\

 0.2    & 05.10.2004 & M. Lepist� &
 Lots of updates to file format specification.\\

 0.3    & 06.10.2004 & M. Lepist� &
 Fixed tables to look nicer. Clarified symbol information and completed
 structure of document. Added text to missing sections.\\

 0.3.1  & 28.10.2004 & M. Lepist� &
 Updated compiling instructions. Now TCE\_HOME\_PATH must be set before
 running make.\\

 0.3.2  & 25.12.2004 & M. Lepist� &
 Added information about assembler-generated stabs that identify processor
 resources used in program.\\

 0.3.3  & 17.01.2005 & A. Cilio &
 Corrected \LaTeX{} compilation error. Minor style changes.\\

\end{HistoryTable}

\chapter {Installing And Using Compiler}

TCE v1 Frontend Compiler does not need to be installed. One has to do two
things to make the compiler work:

\begin{itemize}
\item Set TCE\_HOME\_PATH enviroment variable to point to the frontend
      base directory.
\item Compile tools and libraries.
\end{itemize}
 
\section {Compiling TCE v1 Frontend Compiler}

Compilation should be successfull with any gcc compiler newer than 2.95.0.

First you have to set TCE\_HOME\_PATH environment variable, before you are 
able to compile anything with the TCE v1 Frontend Compiler. TCE\_HOME\_PATH 
should be set the absolute path pointing to the  same directory where you 
have copied the Frontend directory hierarchy. It's the same directory that 
contains, e.g., TCE-Changes.txt and README-gcc-tce file.

In \textit{bash} setting environment variable may be done like this:
\begin{quote}
\texttt{export TCE\_HOME\_PATH="/home/username/\textsl{frontend-path}"}
\end{quote}
%
where \emph{frontend-path} is the path where the frontend complier is
located.

Run `configure':
\begin{verbatim}
    cd $TCE_HOME_PATH
    ./configure
\end{verbatim}%$

Call `make' to compile compilers and libraries:
\begin{verbatim}
    make
\end{verbatim}

\section {Using compiler}

You have successfully compiled tools and libraries and have your
TCE\_HOME\_PATH set.  Next you might want to add the TCE v1 Frontend
Compiler to your PATH environment variable, so you don't have to invoke the
compiler always using its full path. If you do so, you should add
\${TCE\_HOME\_PATH}/bin to the END of PATH environment variable:

\begin{verbatim}
  export PATH="${PATH}:${TCE_HOME_PATH}/bin"
\end{verbatim}

If binary path is added to start of \$PATH, your normal system compiler 
probably will stop working because some GNU tools of the TCE v1 Frontend 
Compiler may interfere with your system's native compiler. The compiler 
can be run using command \textit{gcc-tce}, after the \$TCE\_HOME\_PATH 
variable is set.

\chapter {Main Changes to Original Gcc-move}

\begin{itemize}
\item All architecture dependent optimizations are removed. Only 
      optimizations that are done in higher level are allowed.
\item Streamlined instruction set. Created static instruction set, which 
      is used always. Instructions that are not supported by the target machine
      are replaced with emulation code by an another TCE application.
\item Expanded register set to 1024 floating point and 1024 integer registers.
\item Changed assembler syntax to use dot instead of underscore as a 
      register separator mark from, e.g., add\_o was replaced with add.o.
\item Cleaned up code in all files that were touched while doing the previous
      modifications. 
\item Added generation of stabs for finding name for each used register.
\end{itemize}

\textbf{Changed files:}
\begin{itemize}
\item tce-frontend/gcc-2.7.0/config/move/local.h
\item tce-frontend/gcc-2.7.0/config/move/move.c
\item tce-frontend/gas-move/move.c
\item tce-frontend/include/syscall.h
\item tce-frontend/include/annotate.h
\item tce-frontend/lib/src/libc/setjmp.c
\end{itemize}

Detailed information of changes can be found in \${TCE\_HOME\_PATH}/TCE-Changes.txt.
Test information about code generation is found from 
\${TCE\_HOME\_PATH}/test/TestInfo.txt where is told what is tested and 
what were the results. There one can found for example how to create C code 
which generates min, max or abs instructions.

\chapter{TCE A.out File Format}

A.out file contains two parts. File header and section bodies. In file header 
is stored sizes of each section and some other information about which kind
of data binary contains (does binary contain object data or something else).

In a.out binary first comes file header after that comes bodies of each
section. There is no separate section headers. Sections are organized in a way 
that program sections comes (TEXT, DATA and BSS which are the sections 
that contains program instructions, initialised data and uninitialised 
data areas that program uses when executed) comes first then comes 
rest of the sections.

\begin{table}[h]
  \centering  
  \caption{A.out binary file}
  \begin{tabular}{|c|}
    \hline
    File header \\
    \hline
    TEXT section \\
    \hline
    DATA section \\
    \hline
    Symbol section \\
    \hline
    TEXT Relocation section \\
    \hline
    DATA Relocation section \\
    \hline
    String section \\
    \hline
  \end{tabular}
\end{table}

BSS section (uninitialised data section) does not contain any data. It exists only 
in file header and does not have body stored.

\newpage

\section{A.out File Header}

File header of TCE a.out is 32 bytes long. The meaning of the bytes is 
explained in the following list. \textbf{Word} in this listing means a value 
of 4 bytes in size, and \textbf{Halfword} naturally a value of 2 bytes in size.

\begin{center}
  \begin{longtable}{|p{0.08\textwidth}|p{0.12\textwidth}|
                    p{0.16\textwidth}|p{0.50\textwidth}|}
    \caption{TCE a.out file header.}\\
    \hline
    \textbf{Offset} & \textbf{Field Size} & \textbf{Name} & \textbf{Description} \\
    \hline
    \hline
    0  & Byte & Flags & 
    Not used with TCE a.out (info about file type)? \\
    \hline
    1  & Byte & MachineType & 
    Not used with TCE a.out ?\\
    \hline
    2  & HalfWord & Magic & 
    Magic number and version information. For TCE a.out this is 0x0107. \\
    \hline
    4  & Word & TextSize & Size of the TEXT section. \\
    \hline
    8  & Word & DataSize & Size of the DATA section. \\
    \hline
    12 & Word & BssSize & Size of the BSS section. \\
    \hline
    16 & Word & SymtabSize & Size of the Symbol section. \\
    \hline
    20 & Word & EntryAddress & 
    Virtual address of the entry point of the program. TCE Frontend 
    writes zero to this. \\
    \hline
    24 & Word & TextRelocSize & Size of the TEXT Relocation section. \\
    \hline
    28 & Word & DataRelocSize & Size of the DATA Relocation section. \\
    \hline
  \end{longtable}
\end{center}

String section is the last section in the a.out file. It starts straight 
after DATA Relocation section and continues to the end of file. Size of 
String section is not stored in the file header, but it can be calculated 
from file size and other section sizes. 

\section{TEXT Section}

Instructions in the TTA a.out format are stored in a fixed size entries of 
8 bytes of width.

\begin{center}
  \begin{longtable}{|p{0.08\textwidth}|p{0.12\textwidth}|
                    p{0.16\textwidth}|p{0.50\textwidth}|}
    \caption{TEXT section entry.}\\
    \hline
    \textbf{Offset} & \textbf{Field Size} & \textbf{Name} & \textbf{Description} \\
    \hline
    \hline
    0  & Byte & Guard & Defines if the guard is conditional.\\
    \hline
    1  & Byte & Flags & Flag byte of the move. \\
    \hline
    2  & HalfWord & Destination & Type and index of destination register.\\
    \hline
    4  & Word & Source & 
    Either an immediate value or type and index of source register.\\
    \hline
  \end{longtable}
\end{center}

\subsection{Guard}

Guard is one character which tells that if move is conditional. Guard may have
three different values: 

\begin{itemize}
\item zero (0x00) means that move is not conditional and it's always executed.
\item '?' (0x3F) means that move is executed if value of boolean register is true.
\item '!' (0x21) means that move is executed if value of boolean register is false.
\end{itemize}

\subsection{Flags}

Bit 0x01 means that move contains immediate value in the source field.

\subsection{Destination}

Register that is the destination of move. Values 0..1023 are integer registers,
1024..2047 are floating point registers, 2048 is boolean register and the rest 
are operation registers.

\subsection{Source}

If Flags byte's 0x01 bit is not set, then field is coded same way that 
Destination field. If 0x01 bit is set in Flag field then the Destination 
field contains an immediate value. 

\section{DATA Section}
Contains all initialized data that is needed in TEXT section. 

\section{BSS Section}
BSS section is uninitialized data section. BSS section doesn't actually 
exist in a.out file (except in file header) i.e. BSS section doesn't have 
body, it's needed only for a.out memory image.

\section{Symbol Section}

In a.out, symbols are indexed starting from zero. Symbol entries are fixed 
size fields of 12 bytes of width. 

\begin{center}
  \begin{longtable}{|p{0.08\textwidth}|p{0.12\textwidth}|
                    p{0.16\textwidth}|p{0.50\textwidth}|}
    \caption{Symbol section entry.}\\
    \hline
    \textbf{Offset} & \textbf{Field Size} & \textbf{Name} & \textbf{Description} \\
    \hline
    \hline
    0  & Word & Name & Section offset to string table.\\
    \hline
    4  & Byte & Type & Type of the symbol.\\
    \hline
    5  & Byte & Other & Not used.\\
    \hline
    6  & HalfWord & Desc & Description of symbol. Reserved for debugger use.\\
    \hline
    8  & Word & Value & Value of symbol.\\ 
    \hline
  \end{longtable}
\end{center}

\subsection{Name}

Zero if symbol name is not available.

\subsection{Type}

N\_TYPE is used to mask all symbols that concern linker. N\_EXT mask can be 
used for masking external bit of symbol.

\subsection{Value}

For text, data and bss symbols field contains the address of the symbol or 
zero if symbol is unresolved. Meaning depends on the type of the symbol.

\subsection{Symbol Types And Symbol Bit Masks}

Symbols might have external bit set, which can be tested by masking symbol 
value with N\_EXT mask. If external bit is set, symbol is defined also outside 
from currently read file. In other words, a symbol with an external bit set is 
a global symbol.

All symbols whose id is bigger than N\_TYPE are called as stab symbols. Linker
doesn't care much about these symbols, but it does resolve also stab symbols.
Linker handles resolving stabs, same way that it resolves normal symbols, so
for example N\_PRTAB symbol is resolved same way that N\_ABS symbol because
(N\_PRTAB \& N\_TYPE) == N\_ABS.

\begin{center}
  \begin{longtable}{|p{0.15\textwidth}|p{0.10\textwidth}|
                    p{0.55\textwidth}|}
  \caption{TCE a.out symbols, symbol masks and descriptions.}\\
    \hline
    \textbf{Name} & \textbf{Value} & \textbf{Description} \\
    \hline
    \hline
    N\_UNDF & 0x0 & Undefined symbol\\
    \hline
    N\_ABS & 0x2 & Absolute\\ 
    \hline
    N\_TEXT & 0x4 & TEXT\\ 
    \hline
    N\_DATA & 0x6 & DATA\\ 
    \hline
    N\_BSS & 0x8 & BSS\\ 
    \hline
    N\_COMM & 0x12 & Common (internal to ld)\\
    \hline
    N\_FN & 0x1e & File name\\
    \hline
    N\_EXT & 0x1 & External mask\\ 
    \hline
    N\_TYPE & 0x1e & Symbol type mask\\ 
    \hline
    N\_TJ   & 0x76 & Table jump\\
    \hline
    N\_JTE  & 0x96 & Jump table entry\\
    \hline
    N\_MEM  & 0xb6 & Memory reference\\
    \hline
    N\_FC   & 0xd6 & Function call\\
    \hline
    N\_STAB & 0xe0 & Mask for stabs\\
    \hline
    N\_GSYM & 0x20 & Global\\
    \hline
    N\_FNAME& 0x22 & Procedure name (f77)\\
    \hline
    N\_FUN  & 0x24 & Procedure\\
    \hline
    N\_STSYM& 0x26 & Static\\
    \hline
    N\_LCSYM& 0x28 & .lcomm\\
    \hline
    N\_MAIN & 0x2a & Name of main rutine\\
    \hline
    N\_RSYM & 0x40 & Register\\
    \hline
    N\_SLINE& 0x44 & Source line\\
    \hline
    N\_SSYM & 0x60 & Structure\\
    \hline
    N\_SO   & 0x64 & Source file name\\
    \hline
    N\_LSYM & 0x80 & Local\\
    \hline
    N\_BINCL& 0x82 & Header file\\
    \hline
    N\_SOL  & 0x84 & \#included file name\\
    \hline
    N\_PSYM & 0xa0 & Parameter\\
    \hline
    N\_EINCL& 0xa2 & End of include file\\
    \hline
    N\_ENTRY& 0xa4 & Alternate entry\\
    \hline
    N\_LBRAC& 0xc0 & Left bracket\\
    \hline
    N\_EXCL & 0xc2 & Excluded include file\\
    \hline
    N\_RBRAC& 0xe0 & Right bracket\\
    \hline
    N\_BCOMM& 0xe2 & Begin common\\
    \hline
    N\_ECOMM& 0xe4 & End common\\
    \hline
    N\_ECOML& 0xe8 & End common (local name)\\
    \hline
    N\_PRTAB & 0xd2 & TPEF processors resource table entry\\
    \hline
    N\_LENG & 0xfe & Second symbol table entry with length\\
    \hline
    N\_PC   & 0x30 & Global pascal symbol\\
    \hline
    N\_M2C  & 0x42 & Compilation unit symbol table entry\\
    \hline
    N\_SCOPE& 0xc4 & Scope information\\
    \hline
    N\_BROWS& 0x48  & Sun source code browser\\
    \hline
  \end{longtable}
\end{center}

\subsubsection{Undefined symbol}
This symbol is used if there are unresolved references to other object files.
Value field is not used (usually zero). If this kind of symbols exist then
Symbol section should not be stripped.

\subsubsection{Absolute}
I have not found any symbols of this type from TCE a.out files.

\subsubsection{TEXT symbol}
Value of this symbol is an address to memory image. 
Address should point to TEXT section.

\subsubsection{DATA symbol}
Value of this symbol is an address to memory image. 
Address should point to DATA section.

\subsubsection{BSS symbol}
Value of this symbol is an address to memory image. 
Address should point to BSS section.

\subsubsection{Common (internal to ld)}

\subsubsection{File name}
I have not found any symbols of this type from TCE a.out files.

\subsubsection{External mask}
Mask for resolving if symbol is global or local.

\subsubsection{Symbol type mask}
Every symbol with type greater than N\_TYPE is a stab symbol. 
Stab symbols are symbols which are not needed by linker, but can 
contain useful information like info about line numbers.

\subsubsection{TPEF processors resource table entry}
Front end generates for N\_PRTAB symbol for each register that is
used in assembled a.out files. Value of symbol is id that is used
to refer register and string of symbol is name for register.

\section{Relocation Sections}

There can be up to two relocation sections in every a.out file, one for 
data and one for code. Only difference between these sections is the section
to which section Offset field of element refers. Relocation elements are not 
referred to, so there is no need to know how elements are indexed. Anyways, 
usually elements are indexed starting from zero in a.out and objdump indecies 
these entries from zero.

Every relocation entry has a fixed size of 12 bytes.

\begin{center}
  \begin{longtable}{|p{0.08\textwidth}|p{0.12\textwidth}|
                    p{0.16\textwidth}|p{0.50\textwidth}|}
    \caption{Relocation section entry.}\\
    \hline
    \textbf{Offset} & \textbf{Field Size} & \textbf{Name} & \textbf{Description}\\
    \hline
    \hline
    0  & Word & Offset & 
    Section offset to a location that contains a value to be relocated.\\
    \hline
    4  & Word & Info & 
    Contains three values, symbol index or section id, external flag 
    and relocation type.\\
    \hline
    8  & Word & Addend & 
    Contains a copy of value that is referred by the Offset field.\\
    \hline
  \end{longtable}
\end{center}

\subsection{Offset}

Section offset to a location which contains a value to be relocated.
Offset to either text section (to immediate value of move) or data section, 
depending which one of the relocation sections is read. After linking value 
of the location pointed by Offset may not be valid anymore (linker updates 
only Addend field while resolving relocations).

\subsection{Info}

First 24 bits of field is symbol index which is referred from relocation 
element or section id of the section, which is referred by Addend. If external 
flag is set then symbol index is stored. Otherwise symbol index field contains 
section id of the section that is referred by address in Addend. 

External flag is in the bit whose mask is 0x00000080. 

Relocation type given by the 5 least significant bits. It seems to be 
always RELOC\_32, a TCE-specific relocation type. See below for details. 
After full linking there is never symbols referred from relocation table.

For resolved relocations (relocations of which external bit is not set), 
symbol number tells section id of the object pointed by addend. This value 
can be used as an sanity check for the memory address in Addend field. 

\begin{center}
  \begin{longtable}{|p{0.25\textwidth}|p{0.30\textwidth}|}
    \caption{Symbol index field for resolved relocations.}\\
    \hline
    \textbf{Symbol Index} & \textbf{Section}\\
    \hline
    \hline
    0x04 & TEXT section. \\
    \hline
    0x06 & DATA section. \\
    \hline
    0x08 & BSS section. \\
    \hline
  \end{longtable}
\end{center}

\subsection{Addend}

If binary file is object file which is not linked yet then Addend  is same 
that value of element referred by Offset field. Addend contains always correct 
memory image address to element, which is destination of relocation 
(with linked binary value pointed by Offset field is not valid anymore).

It appears that TCE defines only 4 relocation types and actually uses only
two (or maybe just one) of them. The types are declared in gas-move/move.h:

\begin{verbatim}
enum reloc_type
{
        RELOC_8, RELOC_16, RELOC_32, NO_RELOC
};
\end{verbatim}

Only RELOC\_32 (=2) was ever found from test programs compiled to TCE a.out 
code. It's still good thing to implement other relocation types as well.

\subsection{Addend And The Value Referenced by Offset field}

Generally Addend field in a.out is processor dependent. In many a.out formats
there is no addend field at all. In TCE a.out addend field is used to store 
the same value that should be stored in the location pointed by Offset field 
of relocation entry. 

Addend and the value in location pointed by the Offset is same for compiled, 
but not for linked TCE a.out files. After linking, only the value in Addend field 
is valid, because the TCE linker does not update the value in a location pointed 
by Offset. The value in a location pointed by the Offset is left to hold the same 
value that there was before linking. In other words, the value in the Addend 
field of relocation entry is the one that can always be trusted and value 
referred by Offset might be wrong.

\section{String section}
String section is similair to DATA section but contains only zero ending 
strings. This section might be stripped from binary file.

\section {TCE a.out Memory Image}

There is need to refer offsets of TEXT section, DATA section and BSS section
from TEXT and DATA sections (e.g., when we load and store values to memory
in program code). So we have different memory image representation of TCE
a.out file for handling references to memory image. Also relocation sections
addend field refers this memory image (since it's copy of value in DATA or
TEXT section).

Memory image of a.out is similiar to binary file representation of file. 

Memory image starts from address defined in file header (in TCE a.out it
should be always zero) and first section in image is TEXT section. In image
all section data is represented in same form that in binary file.

\begin{center}
  \begin{longtable}{|p{0.30\textwidth}|p{0.55\textwidth}|}
    \caption{TCE a.out Memory Image.}\\
    \hline
    \textbf{Start Offset} & \textbf{Data That Memory Area Contains}\\
    \hline
    \hline
    0x0000 & 
    Nothing unless EntryAddress is zero. \\
    \hline
    EntryAddress & 
    Start Of TEXT section. \\
    \hline
    EntryAddress + TextSize & 
    Start of DATA section. \\
    \hline
    EntryAddress + TextSize + DataSize & 
    Start of BSS section. \\
    \hline
    EntryAddress + TextSize + DataSize + BssSize & 
    First memory address after memory image 
    (Should not be referred from program).\\
    \hline
  \end{longtable}
\end{center}
   
All values are gathered from a.out file header.

EntryAddress = Virtual address of the entry point of the program.

TextSize = Size of the TEXT section in bytes.

DataSize = Size of the DATA section in bytes.

BssSize = Size of the BSS section in bytes.


\chapter{Things to notice}
\section {Gcc Code Generation}
\subsection{Min And Max}
Compiler supports min and max instructions, but format of C code pattern
is quite restrictive for generating these instructions. Min and max function calls
doesn't generate code with min and max instruction used, but a function calls to 
min and max functions.

It's good idea to write macros for these patterns.

\textbf{Valid code patterns for min}
\begin{itemize}
\item minOfVariablesI1AndI2 = (i1 < i2) ? (i1) : (i2);   
\item minOfVariablesI1AndI2 = (i1 <= i2) ? (i1) : (i2); 
\end{itemize}
                                   
\textbf{Valid code patterns for max}
\begin{itemize}
\item maxOfVariablesI1AndI2 = (i1 > i2) ? (i1) : (i2);   
\item maxOfVariablesI1AndI2 = (i1 >= i2) ? (i1) : (i2); 
\end{itemize}
                                   
\subsection{Absolute}
For absolute code patterns are almost as restricted that for min and max. One
difference with abs was that abs function call actually generates abs instruction
to assembler code.

\textbf{Valid code patterns for abs}
\begin{itemize}
\item absoluteValue = (i > 0) ? (i) : -(i); 
\item absoluteValue = (i >= 0) ? (i) : -(i);
\end{itemize}

Absolute patterns for floating point values doesn't work same way. You have
to use builtin fabs() (NOT absf()) function if you want to absf instructions to
be generated.

\subsection{Multiplication, Modulus and Division}
Most of these operations are generated fine, but for example when we
are operating with small (e.g., 1000 is quite small) constant values these 
operations might be replaced with series of shifts, sub and add instructions.

\subsection{Rotate}
Rotate patterns are defined, but I was able to create C code pattern only
for rotate with constant. 

\textbf{Valid code pattern for rotate with constant}
\begin{itemize}
\item valueToRotate = (valueToRotate >> 1) | (valueToRotate << 31);
\end{itemize}

\subsection{Truncating Double to Float}
Truncating double to float is done just by register move. It probably shouldn't 
be that way?

\subsection{Parameter Passing And Return Values}
When there is not enough registers to pass arguments then stack will be 
used for rest of them. There is 4 integer and 4 floating point registers 
reserved for argument passing. If struct is passed as argument, all 
arguments that after struct in argument list are passed in stack. All 
structures, which contained one integer variable (byte, half word or word) 
inside  were returned in register. If there is more than one variable or 
table of 2-4 bytes then value is returned in memory even if it would fit 
to one register.

\chapter{Further development}

\section{Open Ended Instruction Set}
This feature is already implemented to assembler of frontend. When assembler 
faces on register string whose id is not already defined, assembler generates
new stab and id for that register name. Because of this property same 
registers may have different id, in different files. 

\section{Stabs For Save And Restore Code}

\end{document}
