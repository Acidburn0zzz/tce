\documentclass[a4paper,twoside]{tce}

\usepackage{pslatex}

\begin{document}
\author{Mikael Lepist�}
\title{Binary Handling Module}
\ver{0.6.1}
\firstday{07.01.2004}
\lastday{27.04.2006}
% id number in S- sequence
\docnum{038}
% draft/complete/committed
\state{draft}

\maketitle


\chapter*{Document History}

\begin{HistoryTable}

 0.01   & 19.01.004 & M. Lepist�  &
 Initial version of document.\\

 0.1    & 11.02.004 & M. Lepist�  &
 First version for public to read.\\

 0.2    & 19.02.004 & M. Lepist�  &
 Cleaned language and fixed some typos.\\

 0.3    & 05.03.004 & M. Lepist�  &
 Made changes after Atte's review, added one section to ``Ideas for Further
 Development'' chapter.\\

 0.4    & 12.03.004 & M. Lepist�  &
 Added two ideas for further development.\\

 0.5    & 15.07.004 & M. Lepist�  &
 Added section about Move TTA `a.out' file format.\\

 0.6    & 29.12.004 & M. Lepist�  &

 Updated to current implementation state. Added few pending issues. Removed
 description of `a.out' format, moved to TCE frontend document.\\

 0.6.1  & 27.04.2006 & A. Cilio   & 
 Format revision.\\

\end{HistoryTable}


\tableofcontents



\chapter{INTRODUCTION}

\section{Purpose and Scope}

This document describes the design and use of Binary Handling Module for
future developers to understand, how the module should be maintanced and
used. After reading this document the developer should know which are the
main internal modules of Binary Handling Module and which are the
responsibilities of each internal module. The developer should also be
familiar with the base solutions of implementation of each internal module.

\section{Developers}

\begin{tabular}[h]{ll}
  M. Lepist�      & All Binary Handling Module design,\\
                  & Binary Writer and Reader implementation,\\
                  & TPEF Reader and Writer implementation and\\
                  & Value Replacer design and implementation.\\
  P. J��skel�inen & All Binary Handling Module design and \\
                  & Reference Manager module design and implementation.\\
  A. Mets�halme   & Implementation of some sections and readers. \\
  J. Nyk�nen      & Initial implementations of a.out binary readers and \\
                  & TPEF writer.\\
\end{tabular}

\section{Definitions, Acronyms and Abbreviations}

\begin{table}[htb]
\begin{center}
\begin{tabular}{p{0.15\textwidth}p{0.85\textwidth}}
  TTA  & Transport Triggered Architecture \\
  TPEF & TTA Program Exchange Format \\
  TOR  & TPEF Object Representation \\
\end{tabular}
\end{center}
\end{table}


\chapter{ARCHITECTURE}
%% TO DECIDE THE MOST APPROPRIATE NAME

\section{Motivation and Philosophy of Design}

% Describe the principles behind this design and give an overview of
% it. Motivate briefly the design if necessary.

Binary Handling Module is based on an idea to create an object structure
which represents TPEF binary file and provide the needed interfaces for
handling those objects. This module is used by linker, simulator and some
other external clients to handle and modify a TTA binary file. Binary
Handling Module is also able to write the TPEF Object Representation (TOR)
as a TPEF binary file and read at least TCE a.out and TPEF binaries.

When binary is stored in a file, all references to for example instructions
and data blocks are made by file offsets, indexes or section offsets. In
object representation binary does not contain any binary file representation
specific data, as file offset references, but references are done by referring
real objects inside the model.

Because there might be many references to the same object inside the model,
we wanted to avoid dangling pointer problem (for example when an object is
deleted through a pointer there might be other pointers that same object left,
those other pointers would not be valid anymore). SafePointers was created to
avoid this problem. SafePointers are used always when there are references to
other objects inside TOR.

TPEF specification is still not complete, it's pretty young and very 
expandable. Because of that Binary Handling Module was designed to be easily 
expandable by adding new sections or section elements or adding support for 
reading and/or writing new binary file types. So in most parts the design is 
very flexible and that's why it's a bit more complicated that would actually 
be needed without the increased flexibility.

Data of binary file which Binary Handling Module stores should be the same
that is defined in TPEF specification. Still, binary file specific data like
section offsets, section ids and file offsets are not stored inside TPEF
Object Representation, but are generated when TPEF Object Representation is
written to file.

%% No description of the module purpose and motivation, that belongs to the
%% functional specification document.

\section{External Modules / Application Environment}

% Overview of module from an external perspective:
% - dependencies with external modules-subsystems
% - IFs of external modules that are used
% - external (file) data formats used

Module needs BinaryStream, Exceptions, ContainerTools and MapTools.

% Do not put here the interfaces provided to other modules.
\newpage
\section{Architecture of Databases}

% Describes the data types defined and used by the module and their
% relations.  The description is at type-accurate level. For example, the
% description may consist of UML class diagrams with the attributes and
% interfaces of classes of the target program. Another example, the database
% tables are defined precisely, at level of single fields.

Module doesn't have a real data base, but it has an object structure which 
contains all the data of TPEF binary file.

\begin{figure}[htb]
  \centerline{\psfig{figure=eps/TPEFObjectRepresentation.eps,width=1.0\textwidth}}
  \caption{TPEF Object Representation.}
  \label{fig:TOR}
\end{figure}

\section{Main Restrictions of Implementation}

%% This section is in design document of SW document course
%% I need to know precisely what it should contain to decide if it's useful
%% enough for us.

C++ language and state of the TPEF specification.

\chapter{Module Design}

%% This chapter is repeated for each module in case this design document
%% describes a multi-module larger module (or application).

\section{Overview}

% Overview of the module design in concrete terms.

%% Remove this section (redundant) if the design document describes a single
%% module (or an application consisting of a single module).
\subsection{TPEF Object Representation}
The most important part in Binary Handling Module is TPEF Object
Representation (TOR). That is primary data structure where a flat binary
file is stored as objects and pointers. TOR is designed to be quite similar
with TPEF binary file and it holds the same data that the file does. Main 
difference between the binary file and the object representation is that in
binary file all data is just blocks of bytes and for example references to
instructions are made by offsets in file. Inside the TOR there is real an
object for each section and section element and references are also pointers
to real objects, not just an offset to data. This helps the handling of
binary data, because anymore we don't have to care about updating offsets
for example if we remove objects from a section. All references will still
be valid.

TOR is constructed from three different type of base classes, Binary,
Section and SectionElment. Binary is a base class that contains information
of file header and the Section object for each section of binary file. Every
section contains SectionElements which can for example be MoveElements or
ImmediateElements in TextSection. Section and SectionElement classes are
abstract classes that provide a common interface for every Section and
SectionElement. For each different section type there is an actual class
inherited from Section base class.

\subsection{Binary Reader and Binary Writer}
The other two main modules in Binary Handling Module are Binary Reader
and Binary Writer. These are kind of frameworks which define abstract base
classes. Actual classes which do real reading and writing of files 
must be inherited from these framework classes. These base classes provide 
consistent interface for external clients to read and write any implented
type of binary files.

Binary Reader module contains two base classes, BinaryReader and
SectionReader. When one wants to create file reading support for example
to TPEF file format, one must first inherit TPEFReader class from
BinaryReader and then inherit SectionReader based class for each section
type to be read. For example TPEFCodeSectionReader for reading CodeSection
from TPEF binary file.

Reading binary file into TOR is always made through BinaryReader class.
First client opens BinaryStream and then it asks BinaryReader to create
TOR from it. BinaryReader can resolve by itself which actual BinaryReader
class is the right one for given binary stream (in our case there are only
TPEFReader and AOutReader implemented). 

The actual task of reading a binary file is divided between the actual
BinaryReader and SectionReader classes. With TPEF file TPEFReader
first creates a Binary object, reads file header data and stores it into
the Binary object. After that TPEFReader also reads section headers from
TPEF file. When a section header is read TPEFReader asks Section object 
with section type id from the Section base class (the type of each section 
is usually defined in section's header). Then TPEFReader asks SectionReader
base class to read the section data for the section. SectionReader knows
which actual section reader class is capable to read that section and asks
it to read the data into Section object. The actual SectionReader then 
creates the needed SectionElement objects and adds them to the Section.

After whole binary file is read and references that are made by keys are
resolved finalize() method is called for every section. In this method
may be done those things that needs data from elements referred by keys.

Binary Writer's design is pretty much the same as Binary Reader's. Main
difference between reader and writer is that client of Binary Handling 
Modul  has to set which actual Binary Writer it wants to use before writing
TOR in to BinaryStream.

\subsection{Reference Manager}
Inside TOR there might be more than one reference pointing to a same object,
so when we delete that object from the model, we have to make sure that 
every reference to it is set to null. When reading, there is also a need to
refer to objects through id, file offset, section index or section offset.
We have to convert references from binary file to real references to objects
inside TOR. 

For handling these problems SafePointer class was created and around it were
built the whole Reference Manager module. SafePointers are always used when
we make references inside TOR. SafePointers makes sure that inside the model
there won't be any dangling pointers. When SafePointed object is deleted all
references to that object are set to point to null-SafePointer. SafePointer
functionality is hidden from external clients so they can handle TOR with
normal pointers. TOR creates internally SafePointers for them.

Reference Manager module also allows creating different keys for an object
and assigning objects to a specific key. This functionality helps when we
read objects from a stream. We can create references to objects by a key,
before actual object is even created. SafePointers are must always be 
created with CREATE\_SAFEPOINTER macro. Afterwards, when the actual object
is created, it can be assigned for some keys and all references made for
those  keys will be resolved to the actual object.

For example when Binary Handling Module read SymbolSection:

1. Reader creates SymbolSection object (Section::createSection(ST\_SYMBOL))

2. Reader creates SymbolElement.

3. Reader reads data from BinaryStream for creating SymbolElement. Let's
suppose that symbol was pointing to some object which is not yet read from
the stream. 

4. Reader sets the created SymbolElement to point to that object with 
SectionOffsetKey.

5. Later on reader reaches a position where reader reads that object
which was referred in SymbolElement. When reader reads the referred element
it also adds a SectionOffsetKey for that object.

6. Now, after reading, Reference Manager is asked to resolve all 
references and after that all references are valid. Exception will be
thrown if there is still references which can't be resolved. It doesn't
matter if there are unused keys for some object, that will not emit an
exception.

\subsection{Value Replacer}
When Binary Writer writes TOR into the binary file, we encounter another
problem, with objects which are not yet written into file but which should
be referred to by an offset in binary file. When the writer writes for example
a SymbolElement object which has a pointer to InstructionElement, the writer 
has to write that section offset where instruction element will be written in 
BinaryStream. Because we do not know yet into which section offset 
InstructionElement will be written, we have to keep track where the reference
to that object was and which kind of reference it was to replace that value in
the file later. After all sections and elements are written into the file those
references which weren't able to be written before will be resolved and 
written.

Value Replacer module was created to help with writing of binary file. 
When BinaryWriter wants to write to a binary file a value which should be
replaced afterwards and the value is not known yet, writing should be done 
with ValueReplacer inherited classes. Those classes write value straight
ahead if a replacing value is already available. If value is not yet known,
ValueReplacer marks the offset where the value should be written. After the
writing of binary file is done and all replacement values should be known,
ValueReplacer can be asked to resolve those values which were left unresolved
before.

\section{Interfaces}

% High level description of the interface of this module.  Low level details
% are possible, but should be restricted to simple classes (like, for
% example, NodeDescriptor in mdf module).  Low level details of large
% interfaces, if present at all, should be stored in a separate UML
% specification file (for example, Rational Rose MDL).

\subsection{Binary Reader}
\begin{figure}[htb]
  \centerline{\psfig{figure=eps/BinaryReader.eps,width=1.2\textwidth}}
  \caption{Interface of Binary Reader.}
  \label{fig:BinaryReader}
\end{figure}

\textbf{Classes for other modules to use:}
\begin{itemize}
\item BinaryReader 
\end{itemize}

\textbf{Interfaces for other modules to use:}
\begin{itemize}
\item Binary* BinaryReader::readBinary() 
\end{itemize}

\newpage
\subsection{Binary Writer}
\begin{figure}[htb]
  \centerline{\psfig{figure=eps/BinaryWriter.eps,width=1.2\textwidth}}
  \caption{Interface of Binary Writer.}
  \label{fig:BinaryWriter}
\end{figure}

\textbf{Classes for other modules to use:}
\begin{itemize}
\item BinaryWriter 
\item TPEFWriter 
\end{itemize}

\textbf{Interfaces for other modules to use:}
\begin{itemize}
\item BinaryWriter::writeBinary() 
\item BinaryWriter::setWriter()
\item TPEFWriter::instance() 
\end{itemize}



\newpage
\subsection{Value Replacer}
\begin{figure}[htb]
  \centerline{\psfig{figure=eps/ValueReplacer.eps,width=1.2\textwidth}}
  \caption{Interface of Value Replacer.}
  \label{fig:ValueReplacer}
\end{figure}

\textbf{Classes for other modules to use:}
\begin{itemize}
\item Other modules create instances of all actual classes. 
\end{itemize}

\textbf{Interfaces for other modules to use:}
\begin{itemize}
\item ValueReplacer::initialize()
\item ValueReplacer::finalize()
\item ValueReplacer.replace()
\item SectionSizeReplacer::setSize() 
\item SectionSizeReplacer::clear() 
\end{itemize}

\newpage
\subsection{Reference Manager}
\begin{figure}[htb]
  \centerline{\psfig{figure=eps/ReferenceManager.eps,width=1.2\textwidth}}
  \caption{Interface of Reference Manager.}
  \label{fig:ReferenceManager}
\end{figure}

\textbf{Classes for other modules to use:}
\begin{itemize}
\item SafePointer
\item SafePointable is used to inherit classes which can be used with
Reference Replacer.
\item Other modules create instances of all actual ReferenceKey classes.
\end{itemize}

\textbf{Interfaces for other modules to use:}
\begin{itemize}
\item CREATE\_SAFEPOINTER()
\item SafePointer.pointer()
\item SafePointer::isReferenced()
\item SafePointer::addObjectReference()
\item SafePointer::sectionKeyFor()
\item SafePointer::sectionIndexKeyFor()
\item SafePointer::sectionOffsetKeyFor()
\item SafePointer::fileOffsetKeyFor()
\item SafePointer::resolve()
\item SafePointer::cleanupKeyTables()
\end{itemize}
\newpage
\section{Implementation}

% Describe here important low-level details. Only details about complex
% problems that require ``smart solution'' should be treated.  Do not
% describe every single detail of the implementation when that follows from
% the specification/design in a straightforward manner.
\subsection{Registration of Section Classes}
\textbf{Problem:}

Because we can't know if there is going to be new section types in TPEF 
specification, there was need to create a system which makes it possible to
create a section without knowing anything about the class which actually 
implements it. 

\textbf{Solution:}

All section objects are created by asking Section base class to give
a section of some type. 

For example when we read TPEF file and get a section\_id, we
can ask for the section object like this:

\begin{verbatim}
Section* newSection = Section::createSection(section_id);
\end{verbatim}

To ease the of maintaining we wanted that creating new sections is possible
without changing the existing code. Registeration system for section was 
created for that.

Now when actual section class is written, it must always create a static
instance of it inside class declaration. This static instance registers 
itself into Section base class when static instance's constructor is 
executed. In registration the section class gives section type id which 
type of section class implements. After registeration Section base class 
knows, which section class implements requested section type.

\begin{verbatim}
Here is a example implementation of an actual section class which contains
all needed functionality. (NOTE: intialization of section instance must 
be done in .cc file):

########### DummySection.hh ############

#include "Section.hh"

class DummySection : public Section {
public:

    virtual ~DummySection() {}

    virtual SectionType type() const {
      // NOTE: if one wants to use a enum value it must be defined in Section.hh
      return ST_DUMMY; 
    }

protected:
    DummySection(bool init) {
      if (init) {
        Section::registerSection(this);
      }
    }

    /// Creates a clone of instance.
    virtual Section* clone() const {
      return new ASpaceSection(false);
    }

private:
    // static instance which registers itself into Section base class
    static DummySection proto_;
};

########## DummySection.cc #############

#include "DummySection.hh"

/// Registers section prototype to base class.
DummySection DummySection::proto_(true);

########## end DummySection.cc #########
\end{verbatim}

\subsection{Checking File Type of Given Stream}
\textbf{Problem:}

Adding support for reading a new binary file type is possible to do 
without touching earlier code, even the clients get new support with only 
linking new objects into TPEF module. How is it possible for client to know 
if new binary format is supported? How does BinaryReader base class know that
there is a new actual BinaryReader?

\textbf{Solution:}

Basic idea is the same, that is with adding new sections. Every actual 
BinaryReader class registers itself into base class. Every BinaryReader also 
has to implement isMyStreamType(BinaryStream) method. Now, after actual
BinaryReader classes are registered to BinaryReader base class, BinaryReader
have a list of all actual BinaryReaders implementations and can ask from every
reader if they are able to read given stream.

Creating instances of actual BinaryReaders is done a bit differently than 
with Section classes. All BinaryReaders have to implement a static instance() 
method which is used to create a singleton instance of the reader and after
first call it always returns same instance of the reader. Instance() method is 
first called from SectionReader classes of BinaryReader (for example 
TPEFSectionReaders calls TPEFReader::instance() when they register themselves
into SectionReader base class).

\subsection{Registration of SectionReaders}
\textbf{Problem:}

Because there was the need to add read support for new sections without 
touching existing BinaryReaders there was again a need for a registration 
system which allows adding new actual SectionReaders for some section type.
Again, actual SectionReaders do register themselves into the section base 
class quite same way that sections do (they create static instances of 
themselves which do actual registration in their constructors). Now there are
many SectionReaders registered for the same section type (for example for 
ST\_CODE there is TPEFCodeSectionReader and AOutTextSectionReader), how does 
SectionReader know which is the right actual reader for some section type?

\textbf{Solution:}

Every actual SectionReader instance registers itself with two keys. The first
key is a reference to a BinaryReader instance that is the right reader for 
that SectionReader (for example TPEFReader for TPEFSectionReaders) and the 
second key is a section type which the actual SectionReader can read 
(ST\_CODE for TPEFCodeSectionReader). 

The actual SectionReader gets reference of its BinaryReader instance with 
BinaryReaders static instance() method (TPEFReader::instance() for 
TPEFSectionReaders). 

Now, when BinaryReader wants to read the data of some section it can do it by 
calling SectionReader::readSection() method. That method gets BinaryReader
and Section pointers as an parameter so now SectioReader can do the reading 
with the right section reader of the right BinaryReader instance.

\subsection{Reading TCE a.out Binary Format}

See TCE Frontend Notes document for complete information of TCE a.out 
file format.

TCE a.out format is quite different and much more simple format than TPEF.
A.out contains only code section, data section, uninitialized data section, 
string section, symbol section and two relocation sections. Some of a.out 
sections might be empty (for example if symboltable is stripped).

TPEF specifications have some required sections which has to exist for valid
TPEF binary. Because a.out doesn't contain these sections, required sections 
and elements must be generated during reading of a.out.

TPEF file must contain one NullSection with section index 0. There also has to
be one ResourceSection, if there exists code sections. In TPEF string sections 
must start with zero byte (probably this applies also to a.out string sections) 
and all sections must have address space defined (even if it's undefined 
address space). 

When we create ASpaceSection, we also create undefined ASpaceElement
and two other ASpaceElements. One for UDataSection and DataSection and 
one for CodeSection. These address spaces have same MAU size of 8 bits. 
So both address spaces have same MAU size that a.out binary have.

\subsubsection{Text Section}

This section is read by AOutTextSectionReader class. 

Instruction size of CodeSection that is read from a.out is set to 
8 MAU since in a.out every move is 8 bytes. Instruction encoding is
set to be undefined fixed encoding. 

A.out move contains fields described below:

\textbf{Byte guard}

Defines if guard bit is set and if guard is inverted. 
Guard register file and register index is hard coded to be
first register iof boolean register file.

\textbf{Byte flags}

If immediate bit is set, then we generate also inline encoded immediate element
with id zero to which source of move refers. If bit is not set source field 
is intepreted same way that destination.

\textbf{Half word destination}

In TCE a.out base register set is hard coded. Register numbers 0-1023 are
integer registers, numbers between 1024-2047 are floating point registers,
register number 2048 is boolean register and rest of registers are return
address register and operation registers. Values of destination and 
destination index are resolved with this information.

\textbf{Work source}

Either immediate value or type and index of source register.

TCE a.out text section contains only move instructions and no 'free' 
immediate elements exists. Immediate values are always encoded in a source 
field of move element. Immediate bit is set in flag byte to indicate that 
value of source field is actually an immediate value.

In TCE a.out every immediate value that exists is just for one move 
instruction, AOutTextSection reader generates always an immediate element
and move element when instruction with immediate is read from a.out file.
Source type of generated move is set to MVS\_IMM value to indicate that
source of this move element is id of immediate element (inline encoding)
or id of immediate register.

\begin{verbatim}
         TCE a.out format                      TPEF representation 
    | move with an immediate | -----> | Immediate element | Move element |
\end{verbatim}

When we read InstructionElement we assign it with SectionOffsetKey. Only
starting elements of instructions are assigned with keys because, only 
starting elements may have references pointing to them (location() or 
destination() field of RelocElement).

\subsubsection{Relocation Sections}

This section is read by AOutRelocationSectionReader class.

There can be up to two relocation sections in every a.out file, one for
data and one for code. Only difference between these sections is the section
to which relocations apply. In TPEF this info is stored in referenced
section field of RelocaSection. So only difference between reading
these two a.out sections is when we set referenced section value of
RelocSection class.

Relocation element contains following fields:

\textbf{Word address}

Element referred by address is used to set location of RelocElement.

This value is section offset to either data or text(a.out's code section) 
section. If value refers to code section, then we know, that the value 
of address is not section offset to start of move (location points always
to immediate element, never to move element). Address field is pointing
inside (to start of move + 4 bytes) a.out's move element to start of source 
field (where actual value, which should be relocated lays) of move. 

So AOutRelocationSectionReader fixes this value to point start of move. 
Because we have registered ImmediateElement to start section offset of move,
fixed value may be used as an section offset to immediate element of move.

Because after linking value of the location pointed by address is not valid 
anymore (linker updates only addend field while resolving relocations), we
have to fix the value that is contained by location element. This fixing 
is done in relocation reader's finalize() method which is called after all 
sections are read. 

\textbf{Word symbol, flags and type}

First 24 bits of field has two different meanings. It may be symbol index 
which is referred from  relocation element or section id of the section,
which is referred by addend.

If external flag is set, then we know that relocation have symbol and symbol
index is really index of symbol.

Relocation type is always set to be RT\_SELF or RT\_NOREL.

\textbf{Word addend}

Contains memory image address to element. This value is used to resolve 
destination field of relocation. Addend is easily modified to section 
offset (with help of knowing starting addresses and sizes of each program 
section).

\textbf {Resolving the destination of the relocation from addend}

Addend field is address to memory image of a.out binary and the value 
pointed by addend is the destination object of TPEF relocation. 

Memory image of a.out is similair to binary file representation of file. 

Memory image starts from address defined in file header (in TCE TTA a.out it 
should be always zero) and first section in image is text section. In image 
all section data is represented in same form that in binary file.

\begin{verbatim}

    A.out memory image:

    Address         
    -----------------------------------------
    0x0000         |                    
    ...            |
    -----------------------------------------
    fh_entry       | Start of text section.
    ...            |
    -----------------------------------------
    fh_entry +     | Start of data section.
    fh_textsize    |
    ...            |
    -----------------------------------------
    fh_entry +     | Start of uninitialized
    fh_textsize +  | data section.
    fh_datasize    |
    ...            |
    -----------------------------------------
    fh_entry +     | 
    fh_text_size + | First address after end
    fh_data_size + | of memory image.
    fh_bss_size    | 
    -----------------------------------------
    
    All values are gathered from a.out file header.
    fh_entry = Virtual address of the entry point of the program.
    fh_text_size = Size of text section in bytes.
    fh_data_size = Size of data section in bytes.
    fh_bss_size = Size of uninitilaized data section in bytes.

\end{verbatim}

Section offset and section id of object pointed by addend is easily resolved
with this info. For resolved relocations (relocations whose external bit
is not set) symbol number tells section id of object pointed by addend. 
This value can be used as an sanity check. 

\begin{verbatim}            

    Symbol number | Section
    -------------------------------------------   
    0x04          | Text section.
    0x06          | Data section.
    0x08          | Uninitilized data section.
  
    Table about the meaning of symbol number field (first 24 bits of second 
    word of relocation entry) of relocation entry which doesn't have external 
    bit set.

\end{verbatim}

\subsubsection{Symbol Section}

This section is read by AOutSymbolSectionReader class.

In a.out, symbols are indexed starting from zero and there is no such thing
as undefined symbol. For TPEF the undefined symbol is generated.

\textbf{Word symbol name}

Used to set name of symbol.

\textbf{Byte symbol type}

Used to resolve, how symbol is converted to TPEF. Righ now
only few of symbols are converted, most of symbols are just ignored.

\textbf{Word value}

Value of symbol is intepret differently depending on which kind of
symbol is read.

\textbf{Converting symbol types}

If symbol have external bit is set, then symbol binding is set to
STB\_GLOBAL. 

\textbf{N\_UNDF}

These symbols are converted to NoTypeSymElements and will be 
replaced with new symbol after linker resolves references. Value
of symbol is ignored.

\textbf{N\_TEXT}

Value is used to resolve element to which symbol refers. Converted to 
CodeSymElement.

\textbf{N\_DATA and N\_BSS}

Value is used to resolve data element, which to symbol refers. Symbols are 
converted to a DataSymElement.

\textbf{N\_FN}

For each of these symbols FileSymElement is created. Value is igored.

\textbf{N\_PRTAB}

If symbol contain operation register symbol, then symbol is read to 
ResourceSection and value is stored to be register id.

Rest of the symbols are ignored.

\section{Error Handling}

% How are errors handled? What is considered internal and what is considered
% external source of errors? What is handled by throwing exceptions and what
% by assertions? What are the 
\subsection{Policy}
Exceptions are thrown when the module is used in a wrong way by the clients. 
Assertions are used to verify that code inside the module is working. If fails
occurs then there is a bug, that should be fixed.

Actually right now there are lot of assertions that will fail if the module is 
used to read a broken binary file, or used wrong in some other way. Better
error handling should be done to TCE V2. 

\section{Module Communication}

%% Remove this section if the design document describes just one module (or
%% an application consistingof a single module).

% This section describes the communication between modules of the
% application or subsystem documented in this design document.

Binary Handling Module is divided in to 5 smaller modules: TPEF Object
Representation, Binary Reader, Reference Manager, Binary Writer and Value 
Replacer. Communication between these modules is restricted to each module's
public interfaces.

\begin{itemize}
\item Reference Manager is used by all other modules some way.
\item TPEF Object Representation is used by Binary Reader, Binary Writer 
and external clients.
\item Value Replacer is used by Binary Writer.
\item Binary Reader and Binary Writer is used only by external clients.
\end{itemize}

\begin{figure}[htb]
  \centerline{\psfig{figure=eps/ModuleCommunication.eps,width=1.2\textwidth}}
  \caption{Communication between modules.}
  \label{fig:Modules}
\end{figure}

% ADD NEW SECTIONS ON DESIGN HERE

\chapter{REJECTED ALTERNATIVES}

% Rejected alternatives for (parts of) the design should be listed here with
% the reasoning and date the alternative was dumped. For future reference.
%
%18.11.2003 . . .
\section{Refactoring All Registeration Functionality}
Refactoring registration functionality and refactoring reader and writer 
base functionality. There would be too much depency from the base class to the
inherited classes which didn't look very nice in test implementation. Also, one
central registeration handler was a kind of bad idea because it would cause 
one central registeration database.   

\chapter{IDEAS FOR FURTHER DEVELOPMENT}

% Ideas that are not part of the design yet but that might be added in the
% future are listed here. This is to have ideas written down somewhere.
% This list should contain the date of addition, the idea described briefly
% and the inventor of the idea.

% Example:
%
% 18.11.2003 . . . -P.J��skel�inen
\section{Default SectionReader and SectionWriter}
There could be some implementation of default section reader and writer
that would be used if there is some unknown sections inside binaryfile which
is read.

\section{Unresolved References When Reading Binary File}
If there is unresolved references while reading a binary file, exception 
will be thrown and we don't know how complete reading of the binary file was. 
If Reference Manager could set these unresolved references to point for
example to the constant SafePointer::unresolved (like SafePointer::null), 
the binary hierarchy would be ok to use even if the binary file that was 
used to create it is not.

\section{Exceptions vs. Assertions}
There is a lot inconsistencies when assertions and exceptions are 
used in module. If the module is released as a library, these inconsistencies
should be fixed.

\section{Making the Design More Symmetric}
The design can be made more symmetric if the SectionReader class would have
readHeader() method as the SectionWriter has the writeHeader(). Also offset
parameter may be removed from the SectionReader::readData method.

\chapter{PENDING ISSUES}

% Pending issues concerning this design, a sort of TODO list.
% This chapter should be empty when implementation of this module/subsystem
% is completed.

In this chapter are listed suggestions for improvements to design. 
These matters should be concerned carefully and moved to th Ideas for Further
Development or to the Rejected Alternatives chapter with reasoning of 
decission. 

\section{Restricted Interface for Clients}
Different restricted headers of module for the enduser. That would contain 
only those interfaces which should be used from the external clients. These 
headers would hide for example intarfaces which take SafePointers as an 
parameters.

\section{Removing Chunkable Interface From RawSections}
Because there is not anymore lists of elements returned from Section object
we could combine also chunk() interface behind element() interface. After 
improvement we could also ask chunks from RawSection's same way, that we
ask for normal section elements. Because Now data is stored as MAUs in 
DataSection we should not be able to ask single bytes from DataSection, 
but it would be more easy to just ask one MAU from section. This would mean 
that we need to create MAU Section element since we don't know how big values
MAU might contain. Benefits would be more consistent interfaces and code and 
it would make design easier to understand. It would also make MAU handling 
easier. Current data reading and writing functions for DataSection are
unnecessary complex.

\section{Addend Field With Section Offset and Id for Relocation}
To get rid off memory encoding stuff from TPEF file, relocation elements could
be provided with addend field, where would be section id and section offset to
destination of reloaction. This would make reading and writing of TPEF file
more easy and would remove instruction encoding depency out from tpef file.
Values of location() elements or relocations would be fixed afterwards, when
instruction encoding is actually decided.

\section{Reference to Parent for SectionElement Class}
In some cases it would be useful to know the Section object where referred
SectionElement is located.

\section{Undefined Section and Elements}
Now when there is reference to undefined symbol or to undefined address
space we have always real object at the end of reference. Should every
element type have this kind of undefined element? Should we also create
undefined section object for each type of section. It's clear that we want
to keep strong typing when ever it's possible, so we can't use NullSection
reference for all (for references that more specific type of reference)
undefined section references. These undefined elements would be more like
tpef modules internal way to handle undefined references (and to prevent
references to NULL).

\chapter{MAINTENANCE}

\section{Creating Support for Reading New Binary File Types}

When creating new binary readers the first thing is to create the actual 
BinaryReader inherited class. Actual binary reader class must implement 
virtual readData() and isMyStreamType() methods and static instance() method. 

IsMyStreamType(BinaryStream) method returns true or false, depending on if 
given stream is readable with that BinaryReader. So it mainly checks the 
magic number of binary file. Note that recognizing of binary file type must
be bullet proof or otherwise it might prevent whole Binary Handling Module
from working.

Virtual readData() does the actual reading of binary. It creates a dynamically 
allocated Binary object, reads section headers, creates needed sections 
and adds them into the Binary object. 

BinaryReaders are always singleton classes. It means that there is only one 
instance of every reader in program. So BinaryReader's constructor must be
private and the static instance() method must be implemented. The instance() 
method creates an instance of actual reader when it is executed for the first 
time. After that instance() always returns same instance of the actual 
BinaryReader.

First Binary object should be created in readData() method, then we can read
the binary file headers and store necessary information from them into the 
Binary object. If there are references to some objects in the file headers 
that we should store (for example an id to section that contains names for all 
sections) we can use Reference Manager to create those references. 
It's done by creating a SafePointer instance for that id. We can use it like a 
normal pointer.

\begin{verbatim}
Binary *newBin = new Binary();
...
//reading of file headers
...
SectionKey sectionIdKey(section_id);
bin->setStrings(CREATE_SAFEPOINTER(sectionKey));
\end{verbatim}

Now Reference Manager module knows that there should be an object for that key
after the reading. Later on when we read referred section from file, we can 
assign created section object for that key that we used before. After assigning
all references to that section id may be resolved.

After file headers are read we can start to read sections. First we read the 
section headers and create a Section instance for that. If binary file doesn't 
have same section types that the TPEF has those section types must be
converted into TPEF section types (see Section.hh). When we have created the 
Section instance, we can add it to Reference Manager with sections id. So, 
reference manager will know that when there are references to that section id it 
really means the object we just created.

\begin{verbatim}
...
// reading section header
...
SectionKey sectionKey(section_id);
Section* newSection = Section::createSection(TPEF_section_type);
bin->addSection(newSection);
SafePointer::addObjectReference(sectionKey, newSection);
\end{verbatim} 

We should also set needed fields of the newly created Section object and of 
course do references as an SafePointers needed (see TPEFReader.cc). Now we 
can ask SectionReader to read data for \emph{newSection} and continue with 
reading the next section header.

\begin{verbatim}
SectionReader::readSection(binaryStream, 
                           newSection, 
                           instance(), 
                           offsetToData, 
                           dataLength);
\end{verbatim}

After all sections are read we must call SafePointer::resolve() to resolve
all remaining references which have been done by using keys.

The most important thing to remember when reading section data is to add 
suitable keys for every created object, SafePointer::addObjectReference() 
can be called many times with different keys if there should be more than 
one key for the object. Binary file specification should tell which kind of 
references are used to refer to every object in binary file. Also exception
from SafePointer::resolve() method should be catched at least in development
version of the reader, otherwise it seems that program just crashes if an 
exception is thrown.

\section{Using ValueReplacer Class During Writing Binary}

During the writing of binary we have the same kind of problem, that was with 
reading it. Now we don't know what file offset, id, or index of referred 
object will be, because it's not yet written into the file. ValueReplacers and 
Reference Manager is used to solve this problem. New ValueReplacer classes 
can be created for any type of value that is not yet known (for example it 
is also used for writing section sizes).

When we write an object into binary file we can ask Reference Manager if it
is referenced in TOR. If it is a referenced object, we must add again the 
needed keys to Reference Manager for the object we write.

\begin{verbatim}
...
// needed keys for object are now known
...
if (SafePointer::isReferenced(objectToWrite) {
  FileOffsetKey fileOffsetKey(fileOffsetOfObject);
  SectionOffsetKey sectionOffsetKey(sectionId, sectionOffsetKey);
  SafePointer::addObjectReference(fileOffsetKey, objectToWrite);
  SafePointer::addObjectReference(sectionOffsetKey, objectToWrite);
}
\end{verbatim}

We use the replacers for writing references in a uniform way and hide
inside the ValueReplacer class the precise moment the reference is 
actually encoded and written to file.

This conditional behaviour of the replacer is hidden. The whole table of
actual ValueReplacer instances are hidden inside the ValueReplacer base class.
In other words, when you want to write reference you always do it same way:

\begin{verbatim}
// called once before any replacer instances are created
ValueReplacer::initialize(stream);
...
FileOffsetReplacer replacer(obj);
replacer.resolve();
\end{verbatim}

Now the replacer first checks if object has been already written out. If it
has been, it writes the replacement value immediately. If it hasn't been it 
adds a table entry where the file offset and the specific replacer are stored
the table entry tells that this offset should be replaced after all objects 
have been written out.

After resolve() we know that replacement is done or it will be done at last
when we call ValueReplacer::finalize() method. So after calling resolve()
method we don't need the created replacer instance anymore.

Because ValueReplacers writes their replacement values straight to BinaryStream
before creating any actual reference replacer we have to define BinaryStream 
where values are written. For that we have a static method:

\begin{verbatim}
ValueReplacer::initialize(BinaryStream& stream);
\end{verbatim}

This method has to be called before creating any concrete ValueReplacer
instances. Initialize() method must be called if one wants to start 
writing new reference replacements after running successfully also 
the method:

\begin{verbatim}
ValueReplacer::finalize();
\end{verbatim}

% This chapter treats the problem of extending the design with new
% capabilities that fit in a predefined framework.

% For example, in case of the TPEF module, this chapter could describe the
% steps that must be taken in order to add a new Section subclass and/or a
% SectionReader subclass.

% Remove this chapter no obvious and standardised way to extend the design
% with new capabilities exists.


% ------------------------------------------------------------------------

%% Remove this part if there are no references.  Usually there will be at
%% least a reference to the functional specifications of the same module.

\end{document}

%%% Local Variables: 
%%% mode: latex
%%% TeX-master: t
%%% End: 
