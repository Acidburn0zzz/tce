\documentclass[a4paper,twoside]{tce}

\usepackage{pslatex}

\begin{document}
\author{Lasse Laasonen}
\title{Program Image Generator}
\ver{0.3.1}
\firstday{01.03.2005}
\lastday{22.08.2005}
% id number in S- sequence
\docnum{021}
% draft/complete/committed
\state{draft}
\maketitle


\chapter*{Version History}

\begin{HistoryTable}

 0.1   & 01.03.2005 & L. Laasonen & 
 Draft of chapters 1,3,5. \\

 0.1.1 & 01.03.2005 & A. Cilio & 
 Complete review. Minor corrections and style changes. \\

 0.2 & 02.03.2005 & L. Laasonen &
 Improved the data flow diagram. Added the Code Compression Scheme
 section.\\

 0.3 & 07.03.2005 & L. Laasonen &
 Added a few pending issues. Renamed \emph{Program Image Model} as 
 \emph{Program Bit Vector} in the data flow diagram.\\

 0.3.1 & 22.08.2005 & L. Laasonen &
 Added a new paragraph to section 4.1.\\

\end{HistoryTable}

% Table of contents
\tableofcontents

% Document text
\chapter{INTRODUCTION}

\section{Purpose}

%  Describe the purpose of the document - example:
%  This document describes the project "TTA Codesign Environment".  Its
%  purpose is to give an overview of the proposed TTA Codesign Environment,
%  its main features and requirements, and a summary of the project
%  motivations and goals.

This document describes the Program Image Generator (PIG), an application of
the TCE toolset. The purpose of the document is to give an overview of PIG
and to define the functional specifications needed to design it.

\section{Product Idea}

% (PJ: product)

%  Describe the project motivation in few words and the context in which
%  this product should have a reason to exist - see TCE Project Plan for an
%  example

The PIG application generates a complete bit image of a TTA program
represented in scheduled TTA Program Exchange Format
(TPEF)~\cite{TPEF-specs}.  The bit image is to be uploaded into the
instruction memory of the target memory system for execution. Program image
generation is the last phase of the code generation process implemented by
the software subsystem of TCE. It finalizes the code generation process,
which aims at producing working TTA programs that run on real TTA target
processors.

\section{Product Overview}

% (PJ: product + overview)

% This section describes as briefly as possible the system that implements
% the functionality wanted.

%Only main features/requirements of the product, no details.

PIG is a standalone application driven by a command line interface. It
converts TTA programs represented in TPEF to bit images which can be
uploaded to the memory system of the target system. PIG can produce outputs
of different format for different purposes. One is, of course, the raw bit
image to be uploaded to the instruction memory. PIG can also output the
image in the formats accepted by the most popular VHDL simulators.  For
example, a common format encodes the binary data as strings of ASCII `0' and
`1' characters.

\note{DISCUSS: What kind of output formats should be supported?}

PIG can apply instruction compression algorithms to the program before
creating the binary image. The code compression algorithms are implemented
as separate modules that can be linked at run time (plugin modules).  Users
of TCE toolset can develop new compression algorithms and implement them as
plugin modules.

\section{Definitions}

\begin{description}
\item[Application]%
  A standalone executable program that provides services for users of the
  TCE toolset. It may be controlled by a command-line interface or by a
  graphical user interface, or via a driver application that integrates a
  group of related applications.
\item[Plugin (Module)]%
  A module that contains executable code and that is linked to an
  application at run time, that is, while the application is already
  running.
\end{description}

\section{Acronyms and Abbreviations}

\begin{tabular}{p{0.10\textwidth}p{0.80\textwidth}}
ASCII & American Standard Code for Information Interchange. \\
BEM   & Binary Encoding Map. \\
PIG   & Program Image Generator. \\
TCE   & TTA Codesign Environment. \\
TTA   & Transport Triggered Architectures. \\
TPEF  & TTA Program Exchange Format. \\
VHDL  & Very High Speed Integrated Circuit Hardware Description Language.\\
\end{tabular}

%\section{Document Overview}
% (PJ: summary)

\chapter{USER DESCRIPTION}

\section{User Profile}

%% Remove this section (not needed) for functional specifications of
%% modules; may be needed for applications or subsystems.

%  What's the user like? Student? Researcher? Professional? Expertise in
%  hardware or software expert? How much skilled?

% PEKKA:
% Describe who are the or users of the module/application.
% The users can be "a software engineer" or "a researcher that is developing
% new scheduling algorithms".


\chapter{PRODUCT OVERVIEW}

\section{Product Perspective}

% (external connections)

%  High-level description of the product with focus on its communication
%  with external products/modules. What external modules does it depend
%  from?  What is provided in the product and what is required and assumed
%  available as external service?

% PEKKA:
% This section describes the environment where the product belongs to, for
% example what the module is part of. Also client relationships to other
% modules or applications are described in detail.
%
% Example:
%

% Program Model
% constructs itself using this module. Applications that inspect or modify
% binary files (such as the linker) are direct clients of the module.
%
% This module uses the binary stream module to load and store binary files
% from/to disk. Uses Reference Manager to handle references.
%
% This module is used by different binary file handling tools of the
% project. Example of a such tool is TPEFDumper \cite{tpefdumper-fspecs},
% which prints useful information on the binary files.

This application is part of the software subsystem of the TCE. It represents
the final stage of the code generation process and finalises the main
product of the whole software subsystem: an executable program for a target
TTA processor.

PIG has no client applications but it is itself a client of TPEF and Binary
Encoding Map (BEM) modules. PIG reads the input TTA program to be to be
converted to a binary image from a TPEF file, and computes an instruction
encoding scheme from a BEM file. The BEM file contains the raw data
necessary to produce ``encoding rules'' for generating the bit vectors that
represent TTA instructions.  The rules defined in a BEM file refer to plain,
uncompressed TTA instructions. The instruction bit vectors may be totally
different from those implied by BEM if code compression is applied. In
addition to TPEF and BEM modules, PIG needs a description of hardware
implementation of the target instruction memory. Data flow of PIG is
depicted in Figure \ref{fig:data_flow}.

\begin{figure}[tb]
\centerline{\psfig{figure=eps/modules.eps,width=0.60\textwidth}}
\caption{Data flow diagram of the Program Image Generator.}
\label{fig:data_flow}
\end{figure}

\section{Summary of Capabilities}

% (advantages)

%  Use table format and possibly extra comments if possible.

%  Example extracted from TCE Project Plan:

\begin{center}
\begin{longtable}{p{0.45\textwidth}p{0.55\textwidth}}
  User Benefit                 &       Supporting Feature \\
% productive application programming & high-level language compiler
% high-performance code generation   & TTA instruction scheduler
% accurate application analysis      & profiling tools, simulation feedback
% application code fine-tuning       & options controlling code generation
\end{longtable}
\end{center}

\section{Assumptions and Dependencies}

% (portability)
% (PJ: design constaints)

%  Assumed environment, third-party applications or library assumed to be
%  present. External services (like OS services) that are required.

% PEKKA:
% For example "the GUI should use wxWindows GUI library to ease portability
% to other operating systems and environments" or "XML should be used for
% all clear text persistent data storage".

%  In case of functional specifications of most modules, it's sufficient to
%  specify what is required in addition to assumption/dependencies given in
%  main functional specifications document of TCE Project, and refer to it
%  for the rest.

\chapter{PRODUCT FEATURES}

% (PJ:functionality)

% PEKKA:

% This chapter describes the functionality of the product in detail. In case
% modules, the required client interfaces are defined in broad manner. These
% are the minimum interfaces the object should provide for the client. The
% format or argument list of methods should not be in detailed level, that's
% a thing that should be in the design notes instead.

% In case of an application, all command line arguments and user interface
% screens (GUI) are described in detail.  Also, all tasks the module or tool
% must be able to accomplish are described here.

% Example:
%
% This module must be able to read a.out and TPEF completely. No other
% binary reading support is required at this point. Only writing of TPEF
% format is required.

%  Organise in sections.  Dedicate one section for each main feature.
%  For example: program input/output modules (what it's usually called
%  Binary Handling Module) could be divided in four parts:
%  - common features
%  - reader features
%  - writer features
%  - reference managment
%% Some modules may be so simple that a single section is sufficient.

\section{Code Compression Scheme}

PIG supports user-defined code compression schemes. The code
compression is applied in the \emph{Compressor} module, shown in
Figure \ref{fig:data_flow}. Since it is a plugin module, user can
develop different compression algorithms and implement them in
self-made compressor modules. By default, if no compressor plugin is
defined, PIG creates a bit image which directly reflects the encodings
defined in the given BEM. In addition to code compression, the module
has address fixing in its responsibilities. Address fixing means
recalculating all the references to memory addresses in the code. The
responsibility is left to the compressor module, since only it
knows about the bit widths of the instructions and how the instructions
are set to the memory system. The compressor module creates a
\emph{Program Bit Vector}, shown in Figure \ref{fig:data_flow}. It is a C++ 
object model which represents bit image of the program as a sequence
of bits.

Most compression approaches (if not all) do not need to have a
complete and precise model of the parallel TTA program as that
provided by TPEF, but require just the binary code and the binary
mapping file (described in~\cite{ProGeSpecs}).
%
On one hand, it could be easier to ``read in'' the input TTA program from a
data structure such a POM code section or TPEF rather than the binary
code. On the other hand, a simple abstraction of the program to compress in
terms of bit vectors and bit fields (as defined by the binary encoding map)
is probably more general and simple.
%
At least dictionary-based compression methods and compression methods based
on instructions templates and Huffman coding will work with an abstraction
of the program as a sequence of bit vectors subdivided in fields.

\chapter{DATABASES}

%% Remove this whole chapter (not needed) for functional specifications of
%% applications that do not define relevant data structures and that use
%% only data from library modules.

% PEKKA:
% This chapter describes the data that is processed and used by the module
% or application in broad but still complete way.  More detailed
% descriptions like database tables or XML field names should be in the
% design document.

\section{Contents of Information}

% (PJ:data)

% PEKKA:
%
% This section describes the data types defined and used by the module and
% their relations.  The description is at a high-level, for example using
% UML class diagrams that reflect the real world data, not the classes of
% the target program.

% Example:
%
% Binary(1)--(*)Section <--TextSection(1)--(*)Instruction(1)--(1..*)InstrPart
%                       <--DataSection
%                       <--RelocationSection
%
% For exact format of the data in section, see TPEF specifications \cite{}.


\section{Intensity of Use}

\section{Files and Configuration}

% (PJ:I/O + format)

% PEKKA:

% I/O of the module or application. Examples:
%
% Input is a file that describes instructions and their behavior.
%
% Outputs information to stdout.
%
% The reader of the TPEF binary handling module takes a binary stream as
% input and outputs a binary object model. The writer takes a binary object
% model as input and outputs a TPEF format binary file to binary stream.

\subsection{Binary Encoding Map}

The binary encoding map is one of the main sources of data for PIG. It
provides data necessary to encode the program instructions into bit
vectors. The file format is specified in the document ``Processor
Generator Functional Requirements''~\cite{ProGeSpecs}.

\subsection{TTA Program Exhange Format}

The TPEF file that represents the input TTA program is one of the main
sources of data for PIG. A TPEF file provides a complete definition of the
input program that is to be translated into a memory image. The file format
is specified in~\cite{TPEF-specs}.

\subsection{Compression Data}

The modules that implement instruction compression may produce additional
files that contain auxiliary compression data (for example, a dictionary of
uncompressed instruction words for the decompression logic of the target
processor). The format of these auxiliary data files is specific to the
compression algorithm. These auxiliary data files are needed by ProGe to
generate the decompressor logic.

\subsection{Program Image}

The program image is the main output file produced by PIG. PIG can produce
the same program image in several different formats. The simplest format is
a raw bit image, where each bit of the target memory system corresponds to
one bit stored in the output file.  These bits are loaded into the target
memory system.  Other output formats are, for example, certain textual
formats accepted by HDL simulators.

\chapter{PENDING ISSUES}

% Pending issues concerning these specifications, a sort of TODO list.
% This chapter should be empty when the final product is ready.

\begin{enumerate}
\item
  How to give information about memory modules to PIG? Should there be
  some new file format which gives the information?
\item
  What information about memory modules is needed by PIG?
\item
  Does PIG create separate \emph{.bin} file for each address space
  (also data memory)?
\end{enumerate}

% ------------------------------------------------------------------------

%  References are generated with BibTeX from a bibtex file.
\bibliographystyle{alpha}
\cleardoublepage
%% Equivalent to a chapter in the table of contents
\addcontentsline{toc}{chapter}{BIBLIOGRAPHY}
\bibliography{Bibliography}

\end{document}



%%% Local Variables: 
%%% mode: latex
%%% TeX-master: t
%%% End: 
